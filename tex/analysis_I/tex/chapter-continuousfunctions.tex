\chapter{Continuous functions}
We talked about function in the previous chapter as what we have seen in basic calculus courses
which assume a function to be rule. However, it will prove fruitful to redifine functions in the
language of set theory.

\begin{Definition}[name=Functions]
    If $E,E'$ are sets, then a function $f : E \to E'$ is subset $f \subset E \times E'$ such that
    $\forall p \in E$, there is a unique $q \in E'$ denoted by $q = f(p)$ such that $(p,q) \in f$.
\end{Definition}
The set $(p,f(p))$ is called \emph{graph} of the function f. If $f : X \to Y$ and $g : Y \to Z$ are
functions then the \emph{compositon} function $g \circ f : X \to Z$ is given by $(g \circ f)(x) =
g(f(x))$, $\forall x \in X$. A function $f : X \to Y$ is an \emph{injective} or $1-1$ function whenever
$f(x_1) = f(x_2)$ then $x_1 = x_2$. $f$ is \emph{surjective} or onto function if $\forall y \in Y$, there
is a $x \in X$ such that $f(x) = y$. If $f$ is $1-1$ and $onto$ then to each $y \in Y$ there is a
unique $x \in X$ such that $y = f(x)$. We can then define a function $f^{-1} : Y \to X$ given by
$f^{-1}(y) = x$ if $x = f(y)$. Then $f^{-1}$ is called the \emph{inverse} function. Sometimes is
very useful to define the action of a function on sets. 

\begin{Definition}[name=Direct Image]
    If $f : E \to E'$ is a function and $A \subset E$, then the direct image of $A$ under $f$ is the
    subset of $E'$ given by $f(A) = \left.\lbrace f(p) : p \in A \rbrace\right.$.
\end{Definition}
Note that implicit in the definition is the existence of a $p \in A$ i.e $f(A) = \left.\lbrace q \in
E' : \exists p \in A; \ q = f(p) \rbrace\right.$ .That this is a set can be seen from the following
equation : \[ f\left(\lbrace p \rbrace \right) = \lbrace f(p) \rbrace\] $\forall p \in A$. In a
similar fashion we define the inverse image. 

\begin{Definition}[name=Inverse Image]
    If $f : E \to E'$ is a function then the inverse image of a subset $D \subset E'$ is the subset of
    $E$ given by $f^{-1}(D) = \left.\lbrace p \in E : f(p) \in D \rbrace\right.$.
\end{Definition}
Note that for a $D \subset E'$, the inverse image $f^{-1}(D)$ is just a set of all those points that
get mapped to $D$. It is not to be confused with $f^{-1}$ which is the inverse function. Only when
$f$ is injective and surjective we obtain an inverse function and then the inverse image of $D$ is
the direct image of $f^{-1}$ given by the equation : 
\[ \lbrace f^{-1}(q) \rbrace = f^{-1}\left(\lbrace q \rbrace\right) \] $\forall q \in D$. 

The following observations can be made for direct and inverse images.
\begin{Proposition}[name=Direct and iverse image under $f$]
    Let $f : E \to E'$ be a function and let $A,B \subset E$ and $C,D \subset E'$. Then,
    \begin{enumerate}
	\item $f\left(A \cup B \right) = f(A) \cup f(B)$.
	\item $f\left(A \cap B \right) \subset f(A) \cap f(B)$.
	\item $f^{-1}\left(C \cup D \right) =  f^{-1}(C) \cup f^{-1}(D)$.
	\item $f^{-1}\left(C \cap D \right) =  f^{-1}(C) \cap f^{-1}(D)$.
	\item $f^{-1}\left(f(A)\right) \supset  A$.
	\item $f\left(f^{-1}(C)\right) \subset  C$.
    \end{enumerate}


\end{Proposition}

With these definitions, let us define what it means for a function to be continuous at a point in a
metric space.
\begin{Definition}[name=Continuous function at a point]
    For metric space $(E,d)$ and $(E',d')$, let $f : E \to E'$. $f$ is continuous at $p_0 \in E$ iff
    $\forall \epsilon > 0 \in \mathbb{R}$, there is a $\delta(\epsilon) > 0 \in \mathbb{R}$ such that 
    whenever $d(p,p_0) < \delta$ then $d'(f(p),f(p_0)) < \epsilon$.

\end{Definition}
It will be useful to recast this information in terms of open balls around $f(p_0)$ and $p_0$. In
fact the following are equivalent :
\begin{Proposition}[name=Continuous functions and mappings of open balls]
    For metric space $(E,d)$ and $(E',d')$, let $f : E \to E'$. $f$ is continuous at $p_0 \in E$ then
    the following statements are equivalent :
    \begin{enumerate}
	\item $\forall \epsilon > 0$ $\exists \delta > 0$ such that $d(p,p_0) < \delta$ $\implies$
	    $d'(f(p),f(p_0)) < \epsilon$.
	\item $\forall \epsilon > 0$ $\exists \delta > 0$ such that $f\left(\mathcal{B}_{\delta}(p_0)\right)
	    \subset \mathcal{B}'_{\epsilon}(f(p_0))$.
	\item  $\forall$ neighborhood $\mathcal{N}\left[f(p)\right]$ $\exists$ $\mathcal{N}\left[p\right]$
	    such that $f\left(\mathcal{N}\left[p\right]\right) \subset \mathcal{N}\left[f(p)\right]$
    \end{enumerate}
\end{Proposition}

\begin{Definition}[name=Continuous function]
    For metric space $(E,d)$ and $(E',d')$, let $f : E \to E'$. If $\forall p \in E$, $f$ is continuous
    at $p$, then we say that $f$ is continuous.
\end{Definition}

We'll look into some examples that illustrate continuity of a function. Some of these will also show
that $\delta$ is in general depends on $\epsilon$ and hence $\delta(\epsilon)$ in the definition.
\begin{itemize}
    \item $f : E \to E'$ $\exists q \in E'$ such that $\forall p \in E$, $f(p) = q$. This is a
	\emph{costant} function. A constant function is continuous. To see this take any $\delta$, for
	example $\delta = 1$. Given any $\epsilon > 0$, $d(f(p),f(p_0)) = 0 < \epsilon$. Thus $\delta$ here
	is immaterial.
    \item Suppose $E \subset E'$. Consider the \emph{inclusion} map $i : E \to E'$ given by $i(p) = p$
	$\forall p \in E$. Then $\epsilon > 0$ be given, choose $\delta = \epsilon$. Thus $d'(f(p), f(p_0))
	= d'(p,p_0) = d(p,p_0)$ since $E \subset E'$. But $d(p,p_0) < \delta = \epsilon$. Thus $
	d'(f(p), f(p_0)) < \epsilon$. Hence, the inclusion map is continuous. If $E = E'$ then we have the
	identity map.
    \item $f : E^1 \to E^1$ where 
	\[ f(x) = \left\{ 
	    \begin{array}{l l}
		1 & \quad \text{if $x > 0$ }\\
		0 & \quad \text{if $x \leq 0$ }
	\end{array} \right.\]
	This function is not continuous at $0$. To see this take $\epsilon = 1$. For any $\delta > 0$, there
	are points lets call it $x$ in the ball $\mathcal{B}_{\delta}(0)$ such that $f(x) \not \in
	\mathcal{B}_1(f(0))$. For example $x = \frac{1}{2}\delta$, then $f(x) = 1 \not \in 
	\mathcal{B}_1(f(0)) = \mathcal{B}_1(0) $.
    \item $f : E^1 \to E^1$ such that $f(x) = x$ $\forall x \in E$. This is continuous since $f$ is the
	identity map in $E^1$.
    \item $f : E^1 \to E^1$ such that $f(x) = x^2$. To see that this is a continuous function lets
	observe that $d(f(x),f(x_0)) = \lvert x^2 - x_0^2 \rvert$. Thus $d(f(x),f(x_0)) = \lvert (x+x_0)
	\rvert \lvert (x-x_0) \rvert$. If we choose $\lvert x - x_0 \rvert < 1$, then we have to make $\lvert
	x+x_0 \rvert < \epsilon$. But $\lvert x+x_0 \rvert \leq \lvert x-x_0 \rvert + 2*\lvert x_0 \rvert$.
	Factoring out $\lvert x-x_0 \rvert$ and noting that we chose $\lvert x-x_0 \rvert < 1$, we need to
	make $\lvert x-x_0 \rvert ( 1 + 2*\lvert x_0\rvert) < \epsilon$. Thus if $\delta = \lvert x - x_0
	\rvert$ is chosen to be less that $min(1 , \frac{\epsilon}{1 + 2*\lvert x_0\rvert})$, then 
	$d(f(x),f(x_0)) < \epsilon$.
    \item $f : E^1 \to E^1$ where
	\[ f(x) = \left\{ 
	    \begin{array}{l l}
		1 & \quad \text{if $x \in \mathbb{Q}$ }\\
		0 & \quad \text{if $x \not \in \mathbb{Q}$ }
	\end{array} \right.\]
	This function is discontinuous everywhere. To see this take $\epsilon = 1$. Take any $p_0 \in E^1$.
	If $p_0 \in \mathbb{Q}$ then $f(p_0) = 1$. But a ball of radius $\delta$ given by
	$\mathcal{B}_{\delta}(p_0)$ contains both rational and irrational points. The rational points get
	mapped inside $\mathcal{B}_{1}(1)$, but the irrational points $p \in \mathcal{B}_{\delta}(p_0) $ 
	have $f(p) = 0 \not \in \mathcal{B}_{1}(1)$. We get a similar result when $p_0 \not \in \mathbb{Q}$.

    \item $f : E^1 \to E^1$ where
	\[ f(x) = \left\{ 
	    \begin{array}{l l}
		x & \quad \text{if $x \in \mathbb{Q}$ }\\
		0 & \quad \text{if $x \not \in \mathbb{Q}$ }
	\end{array} \right.\]
	This function is only continuous at $0$. To see this take $\delta = \epsilon$.

    \item Given $(E,d)$ and $q \in E$, $f : E \to \mathbb{R}$ such that $f(p) = d(p,q)$. Then $f$ is
	continuous.
\end{itemize}
We would like a characterization of continuity that is a little simpler than checking continuity at
every point in the domain. The next Theorem is sometimes called the global continuity Theorem. 
\begin{Theorem}[name=Global Continuity Theorem]
    $f : E \to E'$ is continuous iff for every open subset $\mathcal{U} \subset E'$ its inverse
    image $f^{-1}(\mathcal{U})$ is open in $E$.
\end{Theorem}
\begin{proof}
    $\Rightarrow$ Suppose $f$ is continuous and let $\mathcal{U} \subset E'$ be open in $E'$. For
    any $p \in f^{-1}(\mathcal{U})$, $f(p) \in \mathcal{U}$. But $\mathcal{U}$ is open and hence there
    is an $\epsilon > 0$ such that $\mathcal{B}'_{\epsilon}(f(p)) \subset \mathcal{U}$. Since $f$ is
    continuous there is a $\delta > 0$ such that $f(\mathcal{B}_{\delta}(p)) \subset \mathcal{B}'_
    {\epsilon}(f(p))$. Hence, $f(\mathcal{B}_{\delta}(p)) \subset \mathcal{U}$. Thus ,
    $f^{-1}\left( f(\mathcal{B}_{\delta}(p))\right) \subset f^{-1}(\mathcal{U})$. But we know that 
    $f^{-1}\left( f(\mathcal{B}_{\delta}(p))\right) \supset \mathcal{B}_{\delta}(p)$. Hence
    $\mathcal{B}_{\delta}(p) \subset f^{-1}(\mathcal{U})$. Since $p$ was arbitrary,
    $f^{-1}(\mathcal{U})$ is open.  

    $\Leftarrow$ Given any open set $\mathcal{U} \subset E'$ such that $f^{-1}(\mathcal{U})$ is open in
    $E$. Let $p_0 \in E$. Given any $\epsilon > 0$, $\mathcal{B}'_{\epsilon}(f(p_0))$ is open and by
    our hypothesis so is $f^{-1}\left( \mathcal{B}'_{\epsilon}(f(p_0))\right)$ and contains $p_0$.
    Thus since the inverse image is open, $p_0$ is an interior point, that is, 
    there is a $\delta > 0$ such that $\mathcal{B}_{\delta}(p_0) \subset 
    f^{-1}\left( \mathcal{B}'_{\epsilon}(f(p_0))\right)$. Hence, 
    $f\left(\mathcal{B}_{\delta}(p_0)\right) \subset f\left(f^{-1}\left( \mathcal{B}'_{\epsilon}
    (f(p_0))\right) \right)$. But $f\left(f^{-1}\left(\mathcal{B}'_{\epsilon}(f(p_0))\right)
    \right) \subset \mathcal{B}'_{\epsilon}(f(p_0))$. Hence, $f\left(\mathcal{B}_{\delta}(p_0)\right) 
    \subset \mathcal{B}'_{\epsilon}(f(p_0))$. Thus $f$ is continuous.
\end{proof}
\begin{Corollary}
    Suppose $ E \stackrel{f}{\longrightarrow} E' \stackrel{g}{\longrightarrow} E"$. If $f,g$ are
    continuous then so is $E \stackrel{g \circ f}{\longrightarrow} E"$.
\end{Corollary}
\begin{proof}
    Given any open set $\mathcal{U} \subset E"$. Since $g$ is continuous $g^{-1}(\mathcal{U})$ is
    open in $E'$. Since $f$ is continuos $f^{-1}(g^{-1}(\mathcal{U}))$ is open in $E$. But $(g\circ
    f)^{-1}(\mathcal{U}) = f^{-1}(g^{-1}(\mathcal{U}))$. Hence, $g \circ f : E \to E"$ is
    contiuous.
\end{proof}
A stronger form of the above corollary is particularly useful when the function is continuous at
certain points. 
\begin{Proposition}
    If $f$ is continuous at $p_0 \in E$ and $g$ is continuous at $f(p_0) \in E'$, then $g \circ f$
    is continuous at $p_0$.
\end{Proposition}
\begin{proof}
    $g$ is continuous at $f(p_0)$ then, given any $\epsilon > 0$ there is a $\gamma > 0$ such that
    $d'(q,f(p_0)) < \gamma \implies d"(g(q),g(f(p_0))) < \epsilon$. On the other hand since $f$ is
    continuous at $p_0$ there is a $\delta > 0$ such that $d(p,p_0) < \delta \implies
    d'(f(p),f(p_0)) < \gamma$. For $q = f(p)$ then we have shown that given any $\epsilon > 0$ there
    is a $\delta > 0$ such that $d(p,p_0) < \delta \implies d"(g(f(p)),g(f(p_0))) < \epsilon$. Thus,
    $g \circ f$ is continuous at $p$.
\end{proof}
Lets look at two function concepts that can be very useful. 
\begin{Definition}
    Let $f : E \to E'$ and $E' \subset E"$. Then the function $g : E \to E"$ given by $g(p) = f(p)
    \quad \forall p \in E$ is called the \emph{extension} of $f$.
\end{Definition}

\begin{Definition}
    Let $f : E \to E'$ and $E" \subset E$. Then the function $g : E" \to E'$ given by $g(p) = f(p)
    \quad \forall p \in E"$ is called the \emph{restriction} of $f$ denoted by $f|_{E"}$.
\end{Definition}
With these definitions we get an useful corollary that says that restrictions of continuous
functions are also continuous.
\begin{Corollary}
    Suppose $f : E \to E'$ is continuous at $p_0 \in S \subset E$. Then $f|_{S}$ is also continuous
    at $p_0$.
\end{Corollary}
\begin{proof}
    Let $i : S \to E$ be the inclusion map which was shown to be continuous. Then $f|_{S} = f \circ
    i$. Since $f$ is continuous at $p_0$ from the above proposition we get that $f|_{S}$ is
    continuous at $p_0$.
\end{proof}

Now that we have defined continuity and looked at its properties in a topological context as given
by the global continuity Theorem, its a good time to ask how we defined continuity in sophomore
calculus course. If we recall, continuity was defined in terms of a limiting function i.e a function
$f$ is continuous at $x_0$ if the the limit \[ \lim_{x \to x_0} f(x) = f(x_0)\]. Even though we
could have done a similar thing by defining limits of functions and then continuity, it is useful to
first understand continuity since it is a more general concept than limits. With this in mind we'll
define a limit of a function. If we have a function $f : E \to E'$ and we want to find out the limit
of the function at $p_0 \in E$ there are two thing to consider :
\begin{itemize}
    \item We don't care about $f(p_0)$ which may or may not exit. The limit is definied nonethless. 
    \item Any $p_0 \in E$ that is \textbf{not} a cluster point of $E$ is then called an \emph{isolated}
	point. Hence there is a $\delta > 0$ such that $\mathcal{B}_{\delta}(p_0) \cap E = \lbrace
	p_0 \rbrace$. Thus no matter what $\epsilon > 0$, $f(\mathcal{B}_{\delta}(p_0)) \subset
	\mathcal{B}_{\epsilon}(f(p_0))$. Hence when $p_0$ is an isolated point, any function is always
	contiuous at $p_0$. 
\end{itemize}
With this in mind we'll outline the following characterization of limit of a function at a point.
The first one is based on continuity while the second entirely on $\epsilon-\delta$.
\begin{Definition}
    Suppose $E,E'$ are metric spaces and $p_0 \in E$ is a cluster point of $E$. Let $f : E
    - \lbrace p_0 \rbrace \to E'$. Then the following equivalent characterization of  $q$ as a limit 
    of $f$ as $p \to p_0$ are : $q$ is a limit if,
    \begin{enumerate}
	\item The function, \[ g(p) = \left\{ 
		\begin{array}{l l}
		    f(p) & \quad \text{if $p \neq p_0$ }\\
		    q & \quad \text{if $p = p_0$ }
	    \end{array} \right.\] is continuous at $p_0$.
	\item $\forall \epsilon > 0$ $\exists \delta > 0$ such that whenever 
	    $ 0 \neq d(p,p_0) < \delta$, $d'(f(p),q) < \epsilon$.

    \end{enumerate}
\end{Definition}
\begin{Proposition}
    If $f$ has a limit at $p_0$ then it is unique.
\end{Proposition}
\begin{proof}
    Suppose both $q,q'$ are the limits of $f(p)$ as $p \to p_0$. Given any $\epsilon > 0$ $\exists
    \delta = min(\delta_1,\delta_2) > 0 $ such that $d(p,p_0) \implies d'(f(p),q) <
    \frac{\epsilon}{2}$ and $d'(f(p),q') < \frac{\epsilon}{2}$. Thus $d'(q,q') < \epsilon$. Since
    $\epsilon$ is arbitrary $d'(q,q') = 0$ and so $q = q'$.
\end{proof}
Hence we can talk about \emph{the} limit of a function. Thus to summarize we have 
\begin{Theorem}[name=Limit and continuity]
    Let $f : E \to E$, then $f$ is continuous at $p_0 \in E$ iff whenever $p_0$ is a cluster point
    of E then, \[\lim_{p \to p_0} f(p) = f(p_0) .\]
\end{Theorem}
We defined limit of a function based on continuity which we remarked is a much more general notion
than limits. In a similar fashion limit of a general sequence can be defined in terms of limit of
a function. Thus in topology one generally starts with notion of open sets and continuity. Closed
sets, converging sequences and limits of functions can then be derived from these definitions.
\begin{Proposition}
    Given a sequence $\left(q_n\right) \in E'$. Let $f: S \to E$ where $S = \left.\lbrace 1/n : n \in
    \mathbb{Z}_+\rbrace\right.$ such that $f(1/n) = q_n$. Then $q_n \to q \in E'$ 
    iff $f(x) \to q$ as $ x \to 0$.
\end{Proposition}
The next Theorem shows the equivalence of continuous fucntion at a point and the limit of
sequence of the function. 
\begin{Theorem}[name=Continuity and sequences]
    The function $f : E \to E'$ is continuous at $p_0 \in E$ iff for any sequence $\left(p_n\right)
    \in E$ such that $p_n \to p_0$, $f(p_n) \to f(p_0) \in E'$.
\end{Theorem}
\begin{proof}
    $\Rightarrow$ $f$ is continuous at $p_0 \in E$. Thus for any $\epsilon > 0$ there is a $\delta >
    0$ such that whenever $d(p,p_0) < \delta$ then $d'(f(p),f(p_0)) < \epsilon$. Let $\left(p_n\right)
    \in E$ be a sequence such that $p_n \to p_0$. Then for any \emph{positive} number there is a $N$ such
    that whenever $n \geq N$, $d(p_n ,p_0)$ is less than the prescibed positive number. Hence for
    $\delta$ there is a $N_{\delta}$ such that whenever $n \geq N_{\delta}$, $d(p_n,p_0) < \delta$. But
    this means that $d'(f(p_n),f(p_0)) < \epsilon$. Hence $f(p_n) \to f(p_0)$.
    \\
    $\Leftarrow$ We'll show the contrapositive. That is given $f$ is not contiuous at $p_0 \in E$, 
    we'll show that there is a sequence $\left(p_n\right) \in E$ such that $p_n \to p_0$ but $f(p_n)
    \not \to f(p_0)$. Since $f$ is not continuous at $p_0 \in E$ there is a $\epsilon > 0$ such that
    for any $\delta > 0$, $d(p,p_0) < \delta$ but $d'(f(p),f(p_0)) \geq \epsilon$. Thus for
    $\delta = 1$ there is a $p_1 \in E$  such that $d(p_1,p) < 1$ but $d'(f(p_1),f(p_0)) \geq 
    \epsilon$. Conitnuing this for all integeral $\delta$ we find that for any $n \in \mathbb{Z}_+$, 
    $d(p_n,p_0) < 1/n$ but $d'(f(p_n),f(p_0)) \geq \epsilon$. Hence we have constructed a sequence
    $\left(p_n\right)$ such that $p_n \to p_0$ but $f(p_n) \not \to f(p_0)$.
\end{proof}
This characterization will be very useful to prove that all the usual algebraic operations on 
continuous functions in $E^1$ yield a continuous functions. We proved this for sequences in $E^1$
and the use of the Theorem above yields the following corollary :
\begin{Corollary}
    Suppose $f,g : E^1 \to E^1$ are continuous at $p_0 \in E^1$. Then the following are also
    continuous at $p_0$ : 
    \begin{enumerate}
	\item $f \pm g$. 
	\item $f * g$.
	\item $f / g $ whenever $g$ is a nonzero function in $E^1$.
    \end{enumerate}
\end{Corollary}
The proof of this follows from the above Theorem and our results from sequences in $E^1$. Since
$f,g$ are continuous functions for any sequence $p_n \to p_0$, $f(p_n) \to f(p_0)$ and $g(p_n) \to
g(p_0)$. But $\left(f(p_n)\right)$ and $\left(g(p_n)\right)$ are convergent sequences in $E^1$. Thus
the usual algebraic rules hold that was proven earlier for convergent sequences in $E^1$. Using this
corrollary it is easy to obeserve that such algebraic operations hold for limits of sequences ie 
\begin{Corollary}
    Let $p_0$ be a cluster point of $E^1$ and $f,g : E^1 - \lbrace p_0 \rbrace \to E^1$. Suppose,
    \begin{displaymath}
	\begin{aligned}
	    \lim_{p \to p_0} f(p), \\
	    \lim_{p \to p_0} g(p) \\
	\end{aligned}
    \end{displaymath}
    exist. Then the following are true : 
    \begin{displaymath}
	\begin{aligned}
	    \lim_{p \to p_0} (f(p) \pm g(p))  =  \lim_{p \to p_0} f(p) \pm \lim_{p \to p_0} g(p)), \\
	    \lim_{p \to p_0} (f(p) * g(p)) = \lim_{p \to p_0} f(p) * \lim_{p \to p_0} g(p)), \\
	    \lim_{p \to p_0} (f(p) / g(p)) = \lim_{p \to p_0} f(p) / \lim_{p \to p_0} g(p)).\\
	\end{aligned}
    \end{displaymath}
    The last one being true when the limit $\lim_{p \to p_0} g(p) \neq 0 $.
\end{Corollary}
Lets look at an important proposition for functions that are mapped to $E^n$.
\begin{Proposition}
    $f : E \to E^n$, where $f = \left(f_1,f_2,\dots,f_n\right)$ such that $f_i : E \to \mathbb{R}$.
    $f$ is continuous at $p_0 \in E$ iff each $f_i$ is continuous at $p_0$. The $f_i$ are called the
    component function of $f$.
\end{Proposition}
\begin{proof}
    $\Rightarrow$ Given any $\epsilon > 0$ there is a $\delta > 0$ such that whenever $d(p,p_0) <
    \delta$, $\lvert \lvert \mathbf{f(p)} - \mathbf{f(p_0)} \rvert \rvert < \epsilon$. But we know that
    $\lvert f_i(p) - f_i(p_0) \rvert < \lvert \lvert \mathbf{f(p)} - \mathbf{f(p_0)} \rvert \rvert $ for
    all $i = 1 \dots n$. Thus each component function is continuous. 

    $\Leftarrow$ Given any $\epsilon > 0$ let $\delta = \text{ min}(\delta_1,\delta_2, \dots, \delta_n)$
    such that whenever $d(p,p_0) < \delta$, $\lvert f_i(p) - f_i(p_0) \rvert <
    \frac{\epsilon}{\sqrt{n}}$. Thus $\lvert \lvert \mathbf{f(p)} - \mathbf{f(p_0)} \rvert \rvert <
    \sqrt{n\epsilon ^2/n}$. Hence $\lvert \lvert \mathbf{f(p)} - \mathbf{f(p_0)} \rvert \rvert <
    \epsilon$. Thus $f$ is continuous at $p_0$.
\end{proof}

Note that with the above proposition we can talk about the continuity of many functions like
$f : \mathbb{R}^4 \to \mathbb{R}^2$ given by $f(w,x,y,z) = \left( \frac{x^3y - z^7}{x^2+y^2+1},
\frac{wxyz + 7w^2}{x^4+z^8+2}\right)$. Now we know that $(w,x,y,z) = id(w,x,y,z)$ is the
identity function on $\mathbb{R}^4$ which we have shown to be continuous. Thus $w$ is the the first
component function of the identity fucntion. Similarly $x,y,z$ are component functions of the
identity function and from the above proposition are continuous. Thus any rational operations like 
$\frac{x^3y - z^7}{x^2+y^2+1}$ will be contiuous. Thus we see that $f$ is continuous.

Next we'll look at the two big Theorems involving continuous functions.

\begin{Theorem}[name=Continuous function in compact sets]
    If $f : E \to E'$ is continuous and $E$ is compact then $f(E)$ is compact. 
\end{Theorem}
\begin{proof}
    Suppose $\left\lbrace \mathcal{U}_{\alpha} : \alpha \in I\rbrace\right.$ be an open cover of 
    $f(E)$ i.e $f(E) \subset \bigcup_{\alpha} \mathcal{U}_{\alpha} $. Then we can see that 
    \begin{displaymath}
	\begin{aligned}
	    f^{-1}(f(E)) \subset f^{-1}\left(\bigcup_{\alpha} \mathcal{U}_{\alpha}\right) \\
	    \Leftrightarrow E \subset \bigcup_{\alpha} f^{-1}\left(\mathcal{U}_{\alpha}\right)
	\end{aligned}
    \end{displaymath}	    
    Since $f$ is continuous each $f^{-1}\left(\mathcal{U}_{\alpha}\right)$ is open in $E$. Since
    E is compact there exist a finite subcover that is $E \subset \bigcup_{\alpha} 
    f^{-1}\left(\mathcal{U}_{\alpha}\right)$ where $\alpha \in I_N$ such that $I_N$ is a finite
    subset of $I$. Thus $f(E) \subset \bigcup_{\alpha \in I_N}
    f(f^{-1}\left(\mathcal{U}_{\alpha }\right)) \subset 
    \bigcup_{\alpha \in I_N}\left(\mathcal{U}_{\alpha}
    \right)$. Hence $f(E)$ is compact.
\end{proof}
\begin{Corollary}
    If $f : E \to E'$ is continuous and $E$ is compact then $f(E)$ is closed and bounded.
\end{Corollary}
\begin{proof}
    This follows from the above Theorem and the fact that compact sets are closed and bounded.
\end{proof}
\begin{Corollary}[name=Extreme Value Theorem]
    If $f : E \to E^1$ is continuous and $E \neq \emptyset$ is compact then $f$ has a maximum and 
    minimum value. 
\end{Corollary}
\begin{proof}
    $E$ is compact an $f$ is continuous and hence $f(E) \subset \mathbb{R}$ is compact and so is
    closed and bounded. Since a closed and bounded non-empty subset of $\mathbb{R}$ contains its 
    maximum and minimum value , lets call it $a$ and $b$ we have some $p,q \in E$ such that $f(p) =
    a$ and $f(q) = b$. Thus $f$ attains its maximum and minimum values.
\end{proof}

\begin{Theorem}[name=Continuous functions in connected sets]
    Let $f : E \to E'$ be a continuous function. Then if $E$ is connected so is $f(E)$.
\end{Theorem}
\begin{proof}
    WLOG let us take $f$ to be $f : E \to f(E)$. Hence $f$ is onto. Assume $f(E)$ is not connected.
    Then there are open disjoin sets $\mathcal{U},\mathcal{V}$ such that $f(E) = \mathcal{U} \cup
    \mathcal{V}$. Since $f$ is onto any point $p$ must be mapped either to $\mathcal{U}$ or to
    $\mathcal{V}$ but not both since they are disjoint. Thus $E = f^{-1}(\mathcal{U}) \cup
    f^{-1}(\mathcal{V})$. Since $f$ is continuous the inverse images of open sets are open. Thus $E$
    is a union of two open, non-empty and disjoint sets and so is not connected. This gives us a
    contradiction.
\end{proof}
\begin{Corollary}[name= Intermediate value Theorem]
    Let $f : \left[a,b\right] \to \mathbb{R}$ be a continuous function and $\gamma \in \mathbb{R}$ is
    between $f(a),f(b)$. Then there is a $c \in \left[a,b\right]$ such that $f(c) = \gamma$.
     
\end{Corollary}
\begin{proof}
    $\left[a,b\right]$ is connected hence $f(\left[a,b\right])$ is connected. Thus if $\gamma$ is
    between $f(a),f(b)$ then $\gamma \in f(\left[a,b\right])$. 
\end{proof}
Next we'll show two examples where the intermediate value Theorem and the extreme value Theorem fail
when the domain/co-domain of a continuous function is changed from the $\mathbb{R}$ to the set
$\mathbb{Q}$. 
\begin{itemize}
    \item $f(x) = \frac{1}{x^2-2}.$ This function is continuous on $\mathbb{R} - \lbrace \pm
	\sqrt{2} \rbrace$. But it is continuous everywhere on $\mathbb{Q}$. Now take an interval
	$\left[0,2\right]$. $f(0) = -1/2$ and $f(2) = 1/2$ but $0$ isn't contained in
	$f(\left[0,2\right])$. Similarly, there are no extrema in the inverval $\left[-2,\right]$.

    \item The function doesn't have to be pathological for the intermediate value Theorem to fail
	when the sets are $\mathbb{Q}$. For example consider $f(x) =x^2$. $f(0) = 0$ and $f(2) = 4$
	belong to $f$ but $2 \not \in f$.
\end{itemize}
Note that now we can show $E^n$ to be connected. We'll need a definition first.
\begin{Definition}
    A subset $S \subset E^n$ is star convex at $x \in S$ if for all $y \in S$ the set $\left.\lbrace
    ty + (1-t)x : 0 \leq t \leq 1 \rbrace \right. \subset S$.  
\end{Definition}
\begin{Corollary}
    $E^n$ is connected. In fact all star convex sets are connected.
\end{Corollary}
\begin{proof}
    Given any $y \in S$ define $f_y : \left[0,1\right] \to S$ by $f_y = ty + (1-t)x$. Since $f_y$ is
    continuous and $\left[0,1\right]$ is connected so is the set $\left.\lbrace	ty + (1-t)x : 0 
    \leq t \leq 1 \rbrace \right.$. But $S = \bigcup_{y \in S}f_y$ and each intersect at $x$. We
    showed that such a set is also connected. Thus $S$ is connected. Take $\mathbf{x} = \mathbf{0}$
    and then $S = E^n$ shows that $E^n$ is connected.
\end{proof}

Now we'll define a stronger version of continuity called uniform continuity. We'll see that
functions that are uniformly continuous possess some additional properties that are very useful
especially when dealing with sequences of functions.
Let us write down explicitly what is means for a function to be continuous.
A function $f : E \to E'$ is continuous iff
\begin{itemize}
    \item
	$\forall p_0 \in E$ $\forall \epsilon > 0$ $\exists \delta(\epsilon,p_0) > 0$ such that
	$\forall p \in E$ whenever $d(p,p_0) < \delta $ then $d'(f(p),f(p_0)) < \epsilon$.
\end{itemize}
Hence we see that $delta$ in general depends on both $\epsilon$ and $p_0$. But what if we move the
$p_0$ beyond the existential quantifier. This will mean that for a given $\epsilon$ only one
$\delta$ will work for all the $p_0 \in E$. 
\begin{itemize}
    \item
	$\forall \epsilon > 0$ $\exists \delta(\epsilon) > 0$ such that
	$\forall p_0 \in E$ $\forall p \in E$ whenever $d(p,p_0) < \delta $ then $d'(f(p),f(p_0)) <
	\epsilon$.
\end{itemize}
In this case our nomenclature of $p_0$ is not needed. This stronger version of continuity is called
uniform continuity.
\begin{Definition}
    $f : E \to E'$ is uniformly continuous in $E$ if for any $\epsilon > 0$ there is a $\delta > 0$
    such that for any $p,q \in E$ whenever $d(p,q) < \delta$ then $d'(f(p),f(q)) < \epsilon$.
\end{Definition} 
Note that for $f : \mathbb{R} \to \mathbb{R}$ given by $f = x^2$ we showed that $\delta =
\delta(\epsilon,p_0)$. Hence such a function is not uniformly continuous. The farther out we go in
$x$ the smaller $\delta$ we'll need to map into a given $\epsilon$ ball. Now what if we consider a
bounded function $f : \left(0,1\right] \to E^1$ given by $f(x) = sin(1/x)$. This function is not
uniformly continuous even though it is bounded. The next Theorem says when these two notions of
continuity are equivalent. 

\begin{Theorem}[name=Uniform continuity and continuity]
    $f : E \to E'$ with $E$ compact. Then $f$ is continuous iff $f$ is uniformly continuous.
\end{Theorem}
\begin{proof}
    $\Leftarrow$ from definition of uniform continuity.
    $\Rightarrow$ $f$ is continuous in $E$. Thus for any $p \in E$, given any $\epsilon > 0$ there
    is a $\delta_ p > 0$ such that $f\left(\mathcal{B}_{\delta _p}(p)\right) \subset
    \mathcal{B}_{\frac{\epsilon}{2}}\left(f(p)\right)$. We need to find a single $\delta$ that will work
    for every $p \in E$. We can take the infimum of all the $\delta _p$ but that infimum could be $0$.
    Note that $\left.\lbrace\mathcal{B}_{\frac{\delta _{p}}{2}}(p) : p \in E\rbrace\right.$ is an 
    open cover for $E$.
    However since we know $E$ is compact we know that $E$ is subset of finite such $p \in E$ i.e 
    $E \subset \bigcup_{i= 1}^n
    \mathcal{B}_{\frac{\delta_{p_i}}{2}}(p_i)$. Let $\delta = \text{ min }1/2(\delta _{p_1},\dots,
    \delta _{p_n})$. Let $p,q \in E$. Thus there is an $i \in 1 \dots n$ such that such that $p \in  \mathcal{B}_{\delta _{p_i}}
    (p_i)$. Since $d(q,p_i) \leq d(q,p) + d(p,p_i)$ we have $d(q,p_i) < \delta _{p_i}$. Thus for 
    any $p,q \in E$ such that $d(p,q) < \delta$ there is a $i \in 1 \dots n$ such that 
    $p,q \in \mathcal{B}_{\delta _{p_i}}(p_i)$. Since $f$ is continuous $f(p),f(q) \in 
    \mathcal{B}_{\frac{\epsilon}{2}}\left(f(p_i)\right)$. Since $d(f(p),f(q)) \leq d(f(p),f(p_i)) +
    d(f(q),f(p_i))$, we can see that  $d(f(p),f(q)) < \epsilon$. Hence $f$ is uniformly continuous 
    since our choice of $\delta$ was independent of $p,q \in E$.
\end{proof}

\section{Sequences of Function}
\begin{Definition} A sequence $\left(f_n\right)$ of functions $f_n : E \to E'$ \emph{converges} to
    a function $f$ at $p \in E$ if \[ f(p) = \lim_{n \to \infty} f_n(p). \] The sequence 
    $\left(f_n\right)$ converges \emph{pointwise} if it converges for all $p \in E$.
\end{Definition}
Example :
\begin{itemize}
    \item Let $f_n : \left[0 , 1 \right] \to \mathbb{R}$ be given by $f_n(x) = x^n$. Then at $p =
	1$, $f_1(1) = f_2(1) = \dots = 1$. Thus at $1$, $f_n \to 1$. But for any $p < 1$ since $x^n
	\to 0$ we have $f_n(p) \to 0$. Thus the limiting function $f$ is discontinuous at $1$. Hence
	we see that pointwise convergence doesn't preserve continiuity in the limit.
\end{itemize}
Just as we saw a stronger notion of continuity which we called uniform continuity we'll see that a
stronger notion of convergence will presevere continuity in the limit. Let us reiterate the definition
of pointwise convergence of a sequence of functions $\left(f_n\right)$. 

\begin{itemize}
    \item $\left(f_n\right)$ converges pointwise to $f$ iff $\forall p \in E$ $\forall \epsilon > 0$
	$\exists N(p,\epsilon) \in \mathbb{Z}_+$ such that whenever $n \geq N$, $d'(f_n(p),f(p)) < 
	\epsilon$.
\end{itemize}
If we move the universal quantifier in $p$ past the existential quantifier we'll see that $N$ won't
depend on our choice of $p$. Thus given an $\epsilon$ we will be able to find one $N$ such that
any $n \geq N$ will force $f_n$ closer to $f$ for all $p \in E$. This notion of convergence is 
called uniform convergence.
\begin{Definition}
    A sequence $\left(f_n\right)$ of functions $f_n : E \to E'$ \emph{converges uniformly } to
    a function $f$ at $p \in E$ if for any $\epsilon > 0$ there is a $N \in \mathbb{Z}_+$ such that 
    whenver $n \geq N$, $d'(f_n(p) ,f(p)) < \epsilon$ for all $p \in E$ . 
\end{Definition}
\begin{Theorem}[name=Uniform convergence and continuity]
    If $\left(f_n\right)$ converges uniformly to $f$ and each $f_n$ is continuous at $p_0 \in E$ 
    then $f$ is also continuous at $p_0$
\end{Theorem}
\begin{proof}
    Since $f_n \to f$ uniformly, given an $\epsilon > 0$, there is a $N \in \mathbb{Z}_+$ such that
    whenever $n \geq N$, $d'(f_n(p),f(p)) < \frac{\epsilon}{3}$ for all $p \in E$. Also since 
    each $f_n$ is continuous at
    $p_0 \in E$ there is a $\delta > 0$ such that whenever $d(p,p_0) < \delta$, we have 
    $d'(f_n(p),f(p_0)) < \frac{\epsilon}{3}$ for all
    $n$. 
    But note that $d'(f(p),f(p_0)) \leq d'(f(p),f_N(p)) + d'(f_N(p),f_N(p_0)) +
    d'(f_N(p_0),f(p_0))$. When $n \geq N$, uniform convergence guarantees that the first and last
    terms are less than $\frac{\epsilon}{3}$. Also since each $f_n$ is continuous at $p_0$ the
    second term is less than $\frac{\epsilon}{3}$. Thus whenever $d(p,p_0) < \delta$, we have 
    $d'(f(p),f(p_0)) < \epsilon$. Hence $f$ is continuous.
\end{proof}
In a similar fashion one can define \emph{uniformly Cauchy} sequences.
\begin{Definition}
    $\left(f_n\right)$ is uniformly cauchy iff for any $\epsilon > 0$ there is a $N \in
    \mathbb{Z}_+$ such that whenever $m,n \geq N$ then $d'(f_m(p),f_n(p)) < \epsilon$ for all $p \in
    E$.
\end{Definition}
\begin{Theorem}[name=Completeness of sequence of functions]
    For sequence of functions $f_n : E \to E'$ with $E'$ complete $\left(f_n\right)$ converges
    uniformly to $f$ iff $\left(f_n\right)$ is uniformly cauchy. 
\end{Theorem}
\begin{proof}
    $\Rightarrow$ Since $\left(f_n\right)$ converges uniformly to $f$, given any $\epsilon > 0$
    there is a $N \in \mathbb{Z}_+$ such that whenever $n \geq N$, $d'(f_n(p),f(p)) <
    \frac{\epsilon}{2}$ for all $p \in E$. Hence $m,n \geq N$ the triangle inequality gives
    $d'(f_m(p),f_n(p)) \leq d'(f_m(p),f(p)) + d'(f(p),f_n(p)) < \epsilon$. Thus $\left(f_n\right)$ is
    uniformly cauchy. 

    $\Leftarrow$. For a given $p \in E$ the sequence $f_1(p), f_2(p), \dots , $ is uniformly cauchy in
    $E'$. Since $E'$ is complete $f_n(p) \to $ some point in $E'$. Since this is true for all $p \in E$ let
    $f(p) = \lim_{n \to \infty}f_n(p)$ i.e $\left(f_n\right)$ converges pointwise at any $p \in E$.
    Now for any $\epsilon > 0$ there is a $N \in \mathbb{Z}_+$ such that whenver 
    $m,n \geq N$ , $d'(f_m(p),f_n(p)) < \frac{\epsilon}{2}$ for all $p \in E$. Thus for a fix $n$
    and $p$ we have $d'(f_m(p),. ) \leq \frac{\epsilon}{2}$ for all $m \geq N$. Since 
    $f_m(p) \to f(p)$ and for all $m \geq N$, $f_m(p) \in \overline{\mathcal{B}_{\epsilon /2}}(f_n(p))$,
    $f(p) \in \overline{\mathcal{B}_{\epsilon /2}}(f_n(p))$ . Thus $d'(f(p),f_n(p)) \leq \frac{\epsilon}{2} <
    \epsilon$. Hence we have shown that given any $\epsilon > 0$ there is a $N$ such that
    $d'(f_n(p),f(p)) < \epsilon$ for all $p \in E$. Therefore $\left(f_n\right)$ converges uniformly
    to $f$.  
\end{proof}

\section{Abstract metric space of continuous functions}
We defined metric space as a set with a metric defined in it. So far the metric spaces we have seen,
have been elements of $\mathbb{R},E^n$ although we have bee very careful in talking about general
spaces. We'll now look into an abstract metric space whose \emph{points} are \textit{continuous}
functions from one metric space to another. Before we elaborate on this any further we'll first have
to talk about how to measure \emph{distances} between functions.

Consider the following : \\
Let $f,g : E \to E'$ and let us take the max $\left.\lbrace d'(f(p),g(p)) : p \in E
\rbrace\right.$. The maximum may or may not exists but it gives us a way to define some sort of
distance between two functions. We'll see under what conditions we get a metric. The following Lemma
will be useful 
\begin{Lemma}
    Let $ h = d'(f(p),g(p))$ for all $p \in E$ be a function from $E \to \mathbb{R}$ .
    If $f,g$ are continuous then so is $h$.
\end{Lemma}
\begin{proof}
    Fix a $p_0 \in E$. Given any $\epsilon > 0$. Note that $d(h(p) - h(p_0)) = \lvert h(p) - h(p_0)
    \rvert$. Adding and subtracting $d'(f(p),g(p_0))$ and using triangle inequality we get $\lvert
    h(p) - h(p_0) \rvert \leq \lvert d'(f(p),g(p) - d'(f(p) , g(p_0)) \rvert + \lvert d'(f(p),g(p_0))
    - d'(f(p_0),g(p_0)) \rvert$. Using the \emph{inside out} version of triangle inequality which we
    proved earlier in metric space the first term is less than $\lvert d'(g(p),g(p_0)) \rvert$ and
    the second term is less than $\lvert d'(f(p),f(p_0)) \rvert.$ Since $f,g$ are continuous at
    $p_0$ given any $\epsilon > 0$ there is a $\delta > 0$ such that whenever $d(p,p_0) < \delta$ 
    each of two terms are less than $\epsilon / 2$ and thus $d(h(p) - h(p_0)) < \epsilon $. Hence
    $h$ is continuous at $p_0$ and since $p_0$ was arbitrary $h$ is continuous in $E$.
\end{proof}
Let $\mathcal{C}\left(E,E'\right) = \left.\lbrace f : E \to E' | f \ continuous \rbrace \right.$ and
E be compact. Let $D(f,g) = max\left.\lbrace h(p) : p \in E\rbrace\right.$ where $h$ is as defined
above. Since $E$ is compact and $h : E \to \mathbb{R}$ is continuous its maximum is defined and is
in $E$ thus $D(f,g)$ is well defined. 
\begin{Theorem}[name=The space of continuous functions $\mathcal{C}$]
    $\mathcal{C}\left(E,E'\right)$ with $D : \mathcal{C}\left(E,E'\right) \to \mathbb{R}$ is a
    metric space.
\end{Theorem}
\begin{proof}
    Let $f,g,k \in \mathcal{C}\left(E,E'\right)$. Easy to check that $D(f,f) = 0$ and $D(f,g) > 0$ 
    when $f \neq g$. Also $D(f,g) = D(g,f)$. Now choose a point $p_0 \in E$ at which the function
    $h(p) = d'(f(p),k(p))$ is maximum. As noted earlier, such a maximum exist. Then $D(f,k) =
    h(p_0) = d'(f(p_0),k(p_0))$. From the triangle inequality, 
    \begin{displaymath}
	\begin{aligned}
	    d'(f(p_0),k(p_0)) & \leq d'(f(p_0),g(p_0)) + d'(g(p_0),k(p_0)) \\
	    & \leq max\left.\lbrace d'(f(p),g(p) : p \in E \rbrace\right. + 
	    max\left.\lbrace d'(g(p),k(p) : p \in E \rbrace\right.
	\end{aligned}
    \end{displaymath}
    The last two terms are precisely $D(f,g)$ and $D(g,k)$. Thus $D$ satisfies the triangle
    inequality.
\end{proof}
When $E' = \mathbb{R}$ we omit $E'$ and we call it the space of real valued continuous functions
defined in $E$. 
It is easy to observe that a sequence of points that converges in $\mathcal{C}\left(E,E'\right)$
must converge uniformly and a sequence of points that is cauchy in $\mathcal{C}\left(E,E'\right)$
must be uniformly cauchy. As noted earlier points in $\mathcal{C}\left(E,E'\right)$ are continuous
functions from the compact set $E$ to $E'$. These observations are elaborated below.
\begin{itemize}
    \item A sequence $\left(f_n\right)$ of points in $\mathcal{C}\left(E,E'\right)$ converges to $f
	\in \mathcal{C}\left(E,E'\right)$, then by the definition of convergence in any metric space
	we must have for any $\epsilon > 0$ there is a $N \in \mathbb{Z}_+$ such that whenever $n
	\geq N$ then $D(f_n,f) < \epsilon$. This means that $max_{p \in E}
	\left.\lbrace d'(f_n(p),f(p) : \rbrace\right. < \epsilon$. Hence $d'(f_n(p),f(p)) < \epsilon$
	for all $p \in E$. This is precisely the definition of uniform convergence of 
	$\left(f_n\right)$. 
    \item Similarly if $\left(f_n\right)$ in $\mathcal{C}\left(E,E'\right)$ is Cauchy then it is
	uniformly cauchy.
\end{itemize}
The space $\mathcal{C}\left(E,E'\right)$ is a very important space dealt in mathematical analysis.
The next Theorem shows that $\mathcal{C}\left(E,E'\right)$ is complete when $E'$ is complete. Thus
for real valued functions $\mathcal{C}\left(E\right)$ is always complete.
\begin{Theorem}[name=Completeness of the space of continuous functions $\mathcal{C}$]
    For $E$ compact and $E'$ complete the metric space $\mathcal{C}\left(E,E'\right)$ is complete
    (with the metric $D$ defined above).
\end{Theorem}
\begin{proof}
    Let $\left(f_n\right)$ be a cauchy sequence in $\mathcal{C}\left(E,E'\right)$. Thus 
    $\left(f_n\right)$ is uniformly cauchy in $E'$. Since, $E'$ complete there is a $f : E \to E'$
    such that $f_n \to f \in E'$. But $\left( f_n \right)$ is a sequence in 
    $\mathcal{C}\left(E,E'\right)$ and so convergence implies uniform convergence. Since uniform
    convergence of continuous functions yield a continuous function, $f : E \to E'$ is continuous.
    Thus $f \in \mathcal{C}\left(E,E'\right)$. Hence we showed that a cauchy sequence in the metric
    space converges to \emph{point} in the same metric space. Thus $\mathcal{C}\left(E,E'\right)$ is
    complete.
\end{proof}
Note that $\mathcal{C}\left(E\right)$ is also a vector space since the sum of two real valued contiuous
function is also a real valued continuous function and a scalar multiple of a real valued
continuous function also is a real valued continuous function. We can define a \emph{norm} as
follows :
\[ \lvert\lvert p \rvert\rvert _{\mathcal{C}\left(E\right)} = \text{ max
}\left.\lbrace \lvert p \rvert, p \in E \rbrace\right. .\]
Then we have a complete vector space w.r.t to the norm above. A complete normed vector space is
called a $\textbf{Banach}$ space.
