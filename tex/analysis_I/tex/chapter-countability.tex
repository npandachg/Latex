\chapter{Countability}
In this section we will study in some detail finite sets,countable sets and uncountable sets.

We start with finite sets.
We know how to \emph{count} finite sets. We say a set has $n$ elements if we counted n distinct
elements of that set. What are we doing when we count a finite set? 
We are putting all the elements of the set in $1-1$ correspondence with a finite set of Integers.
 
\begin{Definition}[name=section]
    Let $n$ be a positive integer. We use $S_n$ to denote the set of positive integers less than $n$ and call
    it a section of positive integers. Thus $S_n = \setX{1,2,\ldots,n}$.
\end{Definition}
\begin{Definition}[name=finite sets]
    A set $A$ is said to be finite if there is a bijective correspondence of $A$ with some section of the
    positive integers, i.e.~there exists an $n \in \Zplus$ such that,
    \[\map{f}{A}{S_n},\]
    is a bijective function. We say that $A$ has cardinality $n$. If $A$ is empty we say that $A$ has
    cardinality $0$.
\end{Definition}
\begin{Remark}
    $S_n$ itself has cardinality $n$. We will show that the cardinality of a finite set is uniquely determined
    by the set.
\end{Remark}
\begin{Lemma}
    Let $n$ be a positive integer. Let $A$ be a set and let $a_0$ be any element of $A$. Then there exists a
    bijective correspondence $f$ of the set $A$ with $S_{n+1}$ if and only if there exists a bijective
    function $g$ from the set $\setDiff{A}{\setX{a_0}}$ with $S_{n}$.
\end{Lemma}
\begin{proof}
    First assume that $\map{g}{\setDiff{A}{\setX{a_0}}}{S_n}$ is bijective. Let $\map{f}{A}{S_{n+1}}$ be such
    that $\restrict{f}{\setDiff{A}{\setX{a_0}}} = g $ and $f(a_0) = n+1$. Then $f$ is bijective.

    Assume $\map{f}{A}{S_{n+1}}$ is bijective. We consider two cases.

    \textbf{CASE I}: $f(a_0) = n+1$. In this case, define $g = \restrict{f}{\setDiff{A}{\setX{a_0}}}$. Then we
    get the desired bijective map.

    \textbf{CASE II}: $f(a_0) = k$ where $k \neq n+1$. Then there is an $a_1$ in $A$ such that $f(a_1) = n+1$.
    Define $\map{\sigma}{A}{S_{n+1}}$ such that $\sigma(a_0) = n+1$ and $\sigma(a_1) = k$ and for all other
    values of $A$, $\sigma = f$. Thus $\sigma$ is a bijective mapping from $A$ onto $S_{n+1}$ such that
    $\sigma(a_0) = n+1$. Define $g = \restrict{\sigma}{\setDiff{A}{\setX{a_0}}}$. Then we
    get the desired bijective map.
\end{proof}
\begin{Theorem}
    Let $A$ be a set; suppose there exists a bijection $\map{f}{A}{S_n}$ for some $n \in\Zplus$. Let $B$ be a
    proper subset of $A$. Then there exists no bijection $\map{g}{A}{S_n}$, but if $B$ is not empty, then
    there is a bijection $\map{f}{B}{S_m}$ for some $m < n$.
\end{Theorem}
\begin{proof}
    We will prove this by induction on $n$. First note that when $A$ is empty, there is no proper subset of
    $A$ and so the theorem is trivially true. Let $\map{f}{A}{S_1}$ be a bijection. Then any proper subset $B$
    of $A$ is empty and so cardinality of $B$ = $0$.
    Assume the theorem is true for some $n$ in $\Zplus$. Let $\map{f}{A}{S_{n+1}}$ be a bijection and let $B$
    be a proper subset of $A$. Assume $B$ is not empty. There is an element $a_0$ in $B$ and note that $a_0$
    is also in $A$. By the previous Lemma, there is a map,
    \[\map{g}{A-\setX{a_0}}{S_n},\]
    such that $g$ is bijective. Note that $B - \setX{a_0}$ is a proper subset of $A - \setX{a_0}$. 
    Now we can use our inductive hypothesis to state that:
    \begin{enumerate}
	\item
	    There is no bijection $\map{h}{B-\setX{a_0}}{S_n}$.
	\item
	    Either $B - \setX{a_0}$ is empty or there is an $m < n$ such that,
	    $\map{k}{B-\setX{a_0}}{S_{m}}$ is bijective.
    \end{enumerate}
    Using the Lemma above we can state $(1)$ equivalently by stating that there is not bijection between $B$
    and $S_{n+1}$. If $B - \setX{a_0}$ is not empty, then we can use the previous lemma to state that there is
    a bijection from $B$ to $S_{m+1}$ where $m+1 < n+1$ and so we have proved the statement for $n+1$.
\end{proof}
\begin{Corollary}
    If $A$ is finite, there is no bijection of $A$ with a proper subset of itself.
\end{Corollary}
\begin{proof}
    Let $\map{f}{B}{A}$ be bijective. Since $A$ is finite there is a bijective function $\map{g}{A}{S_n}$.
    Define $h = \fog{g}{f}$. Then $\map{h}{B}{S_n}$ is bijective which contradicts the theorem above.
\end{proof}
\begin{Corollary}
    $\Zplus$ is not finite.
\end{Corollary}
\begin{proof}
    We prove this by the contrapositive of the above corollary i.e.~we construct a bijection between $\Zplus$
    and a propoer subset $\Zplus-\setX{1}$.
    Let $\map{f}{\Zplus-\setX{1}}{\Zplus}$ be given by $f(i) = i-1$. Then $f$ is bijective.
\end{proof}
\begin{Corollary}
    The cardinality of a finite set $A$ is uniquely determined by $A$ i.e.~there doesnot exist $S_n,S_m$ with
    $n \neq m$ such that $A$ is in bijective correspondence with both $S_n,S_m$.
\end{Corollary}
\begin{proof}
    Suppose there exist bijections $f,g$ from $A$ onto $S_n,S_m$ where $n \neq m$. Without loss of generality,
    assume $n < m$. Let $h = \fog{g}{f^{-1}}$. Then $\map{h}{S_n}{S_m}$ is bijective which gives a
    contradiction since $S_n$ is a proper subset of $S_m$.
\end{proof}
\begin{Corollary}
    If $B$ is a subset of the finite set $A$, then $B$ is finite. If $B$ is a proper subset of $A$, then the
    cardinality of $B$ is less than $A$.
\end{Corollary}
\begin{Corollary}
    Let $B$ be a non-empty set. Then the following are equivalent:
    \begin{enumerate}
	\item
	    $B$ is finite.
	\item
	    There is a surjective function from a section of the positive integers onto $B$.
	\item
	    There is an injective function from $B$ into a section of the positive integers.
    \end{enumerate}
\end{Corollary}
\begin{proof}
    We will prove $(1)\implies(2)$, $(2)\implies (3)$ and $(3) \implies (1)$.

    $(1)\implies (2)$. Assume $B$ is finite. Then, for some $n\in\Zplus$, there is a bijective map 
    $\map{f}{B}{S_n}$. Thus $\map{f^{-1}}{S_n}{B}$ is onto.

    $(2) \implies (3)$. Assume there is a map $\map{f}{S_n}{B}$ such that $f$ is onto. Define the function
    $\map{h}{B}{S_n}$ where for any $b \in B$, $h(b) = \min\invIm{f}{b}$. Since $f$ is onto $\invIm{f}{b}$ is
    not empty and is a subset of $S_n$ and by the well ordering property of integers, the minimum exists and
    is unique. Thus $h$ is injective.

    $(3) \implies (1)$. Assume there is an injective map $\map{f}{B}{S_n}$ for some $n \in \Zplus$. Then
    $\map{f}{B}{\dirIm{f}{B}}$ is bijective. Also, since $\dirIm{f}{B} \subset S_n$, it is finite. Hence $B$ is
    finite.
\end{proof}
\begin{Theorem}
    Finite Unions and finite cartesian products of finite sets are finite.
\end{Theorem}
\begin{proof}
    Let $\famB = \set{B_i}{1\leq i \leq n}$ be a finite collection of finite sets. We will show that
    $\finiteUnion{B_i}{i}{n}$ is also finite by induction on $n$.
    When $n = 1$, the result is trivial. Consider the case when $n = 2$. By hpyothesis, there exists bijective
    functions $f_1,f_2$ from $B_1,B_2$ onto $S_{n_1},S_{n_2}$ for some $n_1,n_2$ in $\Zplus$. Define
    $\map{h}{S_{n_1+n_2}}{B_1\cup B_2}$ as follows:
    $h(i) = f_1(i)$ for $1\leq i \leq n_1$ and $h(n_1+j) = f_2(j)$ for $1 \leq j \leq n_2$. Thus $h$ is a
    surjection from a section of integers to $B_1\cup B_2$ and hence, $B_1\cup B_2$ is finite. Note that $h$
    is note necessarily a bijection since $A,B$ may not be disjoint. The result now follows from induction.

    Note that for any $b \in B_1$, $\setX{b}\times B_2$ is finite because the function
    $\map{f}{\setX{b}\times B_2}{S_{n_2}}$ given by $f((b,c)) = f_2(c)$ for any $c \in B_2$ 
    is a bijection. Now, $B_1\times B_2$
    can be written as $\bigcup\limits_{b\in B_1}\setX{b}\times B_1$, which is a finite union of finite sets and
    so is finite. The result for arbitrary $n$ follows from induction.
\end{proof}
Just as sections of integers are the prototypes for finite sets, the set of all positive integers is
prototype for the simplest infinite set that we call countably infinite or denumerable sets.

\begin{Definition}[name=Infinite and Countably infinite]
    A set $A$ is said to be infinite if it is not finite. It is said to be countably infinite if there is a
    bijective correspondence,
    \[\map{f}{A}{\Zplus}.\]
\end{Definition}
From all our observations regarding finite sets we can immediately deduce the following about infinite sets
\begin{enumerate}
    \item
	For any proper subset $B$ of $A$, if $B$ is infinite then $A$ is also infinite.
    \item
	If there is a proper subset $B$ of $A$ and if there is a function $\map{f}{B}{A}$ which is bijective,
	then $A$ is infinite.
\end{enumerate}
\begin{Definition}[name=Countable]
    A set $A$ is said to be countable if it is either finite or if it is countably infinite. A set that is not
    countable is said to be uncountable.
\end{Definition}
We saw that $\Zplus$ was not finite. It is countably infinite by definition and hence $\Zplus$ is countable.
Is $\Z$ countable?
\begin{Proposition}
    $\Z$ is countably infinite.
\end{Proposition}
\begin{proof}
    $\Zplus$ is a proper subset of $\Z$ and is not finite. Hence we must show that there is a function
    $\map{f}{\Z}{\Zplus}$ that is bijective. Let us define,
    \begin{align*}
	f(n) = 
	\begin{cases}
	    -2n &\text{if $n < 0$},\\
	    2n + 1 &\text{if $n \geq 0$}.
	\end{cases}
    \end{align*}
    Easy to see that $f$ is a bijection.
\end{proof}
\begin{Proposition}
    The product $\Zplus \times \Zplus$ is countably infinite.
\end{Proposition}
\begin{proof}
    The proof is the famous diagonal arrangement. If we of $\Zplus \times \Zplus$ as a grid, we can start
    counting \textbf{diagonally} from right to top. Let $\map{f}{\Zplus}{\Zplus}$ be given by,
    $f(i,j) = j + \series{k}{k}{1}{i+j-2}$. Thus, for example, $f(1,1) = 1$, $f(2,1) = 2$, $f(1,2) = 3$ and so
    on. Easy to see that we have a bijection. For example to show it is surjective let $m$ be any positive
    integer, say $m = 11$. Then think of $11$ as a list $(1),(2,3),(4,5,6),(7,8,9,10),(11,12,13,14,15)$. Thus
    $11$ is on the $5^{th}$ list which means that $j$ will be such that $1\leq j \leq 5$. Similarly $i$ will
    be less than $5$. Easy to see that $i = 5$ and $j = 1$ will give $f(i,j) = 11$.
\end{proof}
%%%%%%% draw diagrams here : see pink notebook
There is a very useful criterion for showing that a set is countable.
\begin{Theorem}
    Let $B$ be a non-empty set. Then the following are equivalent.
    \begin{enumerate}
	\item
	    $B$ is countable.
	\item
	    There is a surjective function $\map{f}{\Zplus}{B}$.
	\item
	    There is an injective function $\map{g}{B}{\Zplus}$.
    \end{enumerate}
\end{Theorem}
\begin{proof}
    Assume that $B$ is countable. Hence $B$ is either finite or countably infinite. In the later case, there
    is a surjection $\map{f}{\Zplus}{B}$. If $B$ is finite, then there is a bijection $\map{g}{S_n}{B}$
    for some $n\in\Zplus$. Define $\map{f}{\Zplus}{B}$ as follows:
    $f(i) = g(i)$ for $i \leq n$ and $f(i) = g(n)$ for $i > n$. Thus $f$ is a surjection. Hence we have shown
    $(1)\implies (2)$.

    Assume that there is a surjection $\map{f}{\Zplus}{B}$. This means that for any $b$ in $B$,
    $\invIm{f}{\setX{b}}\neq\emptyset$. Moreover, for $b_1,b_2 \in B$, if $b_1\neq b_2$, then
    $\invIm{f}{\setX{b_1}} \neq \invIm{f}{\setX{b_2}}$. Let $g(b) = \min\invIm{f}{\setX{B}}$. Then $g(b)$ is
    an injective function from $B$ to $\Zplus$. Thus, we have shown that $(2) \implies (3)$.

    Assume there is an injective function $\map{g}{B}{\Zplus}$. To show that $B$ is countable we need to show
    that either $B$ is finite or $B$ is countably infinite. We can assume $B$ is not finite. 
    Thus $\dirIm{g}{B} \subset {\Zplus}$ is not finite. IF $\dirIm{g}{B}$ was countably infinite, then $B$
    would be countably infinite. Hence we need to show that any infinite subset of $\Zplus$ is countably
    infinite which would mean that $(3)\implies (1)$.
\end{proof}
\begin{Remark}
    The above theorem (part (2)) 
    states that for a countable set we can list its elements as a \textbf{sequence}. Thus if
    $A$ is countable, then $A = \setX{a_{1},a_{2},\cdots}$.
\end{Remark}
\begin{Proposition}
    Any infinite subset of $\Zplus$ is countably infinite.
\end{Proposition}
\begin{proof}
    Let $B$ be an infinite subset of $\Zplus$. 
    There is bijection $\map{f}{\Zplus}{\Zplus}$ that lists the elements of $\Zplus$ in a sequence. 
    The intuitive idea is to go along the sequence and check if it belongs to $B$. 
    We want a function $\map{g}{\Zplus}{B}$ that is bijective. 
    For example as we move along the sequence in $f$, the first occurance of an element in $B$ occured in the
    index $k$. We set $g(1) = f(k)$. Then we continue traversing along the sequence in $f$ until we find an index
    $l > k$ such that $f(l) \in B$. We set $g(2) = f(l)$. It is clear that such a function is surjective. 
    How do we define the function $g$? Let us tweak $g$ a bit. Instead of taking the first occurance
    of an element listed by $f$, we take the smallest element of $B$ listed by $f$. There must be an index $m$
    such that $f(m) = \min{B}$. We take $g(1) = f(m)$ and proceed subsequently. 

    Let $g(1)$ be the smallest element that is in
    $B$. This is possible since $B \subset \Zplus$. Let $g(2)$ be the smallest element other than $g(1)$ that
    is in $B$. This is called an \textbf{inductive} definition (or recursion). We define a function whose
    domain is the natural numbers inductively, in this case,
    \[g(n) = \min\setX{B-\setX{g(1),g(2),\ldots,g(n-1)}}.\]
    It is clear that $g$ is injective. To see that $g$ is surjective, consider an arbitrary element, say $b$, 
    of $B$. For any $m \in \Zplus$ there is an $n \in \Zplus$ such that $g(n) > S_m$. Because if not, then $B$
    would be finite. In particular for $m = b$ there is an $n$ such that $g(n) > b$. Let $C =
    \set{n\in\Zplus}{g(n)\geq b}$. Then $C \subset \Zplus$ and so a minimum $p \in \Zplus$ exists such that
    $g(p) \geq b$. This means that $g(1),g(2),\ldots,g(p-1)$ are all less than $b$. By definition of $g$,
    $g(p)\leq b$. Thus it must be the case that $g(p) = b$.
\end{proof}
We have used a key idea here called the principle of recursive definition.
\begin{Definition}[name=Principle of recursive definition]
    Let $A$ be any set. Given a formula that defines $g(1)$ as a \textbf{unique} element of $A$ and for all
    $i > 1$ defines $g(1)$ \textbf{uniquely} as an element of $A$ in terms of the values of $g$ for positive
    integers less than $i$, this formula determines a \textbf{unique function} $\map{g}{\Zplus}{A}$.
\end{Definition}
\begin{Corollary}
    A subset of a countable set is countable.
\end{Corollary}
\begin{proof}
    Let $A \subset B$ such that $B$ is countable. Thus there exists an injection $\map{f}{B}{\Zplus}$. Hence,
    $\map{\restrict{f}{A}}{A}{\Zplus}$ is also an injection.
\end{proof}
\begin{Corollary}
    The set $\Zplus\times\Zplus$ is countable.
\end{Corollary}
\begin{proof}
    Let $\map{f}{\Zplus\times\Zplus}{\Zplus}$ be given by $f(i,j) = 2^{i}3^{j}$. Then, $f$ is injective
    (because $2,3$ are prime).
\end{proof}
\begin{Corollary}
    The set of positive rational numbers is countable.
\end{Corollary}
\begin{proof}
    Let $\map{f}{\Zplus\times\Zplus}{\Q^{+}}$ be defined by $f(i,j) = \frac{i}{j}$. Then $f$ is surjective.
    We showed that there is a function $\map{g}{\Zplus}{\Zplus\times\Zplus}$ that is surjective. Thus
    $\map{\fog{f}{g}}{\Zplus}{\Q^{+}}$ is surjective.
\end{proof}
\begin{Theorem}
    A countable union of countable sets is countable.
\end{Theorem}
\begin{proof}
    Let $\famA$ be a collection of countable sets $A_1,A_2,\ldots$. Since each $A_i$ is countable there is a
    function $\map{f_i}{\Zplus}{A_i}$ that is a surjection. In other words we can list the elements of $A_i$'s,
    \begin{align*}
	A_1 &= \setX{a_{11},a_{12},a_{13},\ldots}\\
	A_2 &= \setX{a_{21},a_{22},a_{23},\ldots}\\
	&\vdots\\
	A_j &= \setX{a_{j1},a_{j2},a_{j3},\ldots}\\
	&\vdots
    \end{align*}
    Define $\map{g}{\Zplus\times\Zplus}{\countUnion{A_i}{i}}$ as,
    $g(i,j) = a_{ij} = f_i(j)$. Thus $g$ is a surjection.
\end{proof}
\begin{Theorem}
    A finite product of countable sets is countable.
\end{Theorem}
\begin{proof}
    We have shown the analogous result for finite sets. Let $\set{A_i}{1\leq i \leq n}$ be a finite collection
    of countable sets. We prove this by induction on $n$. The case $n= 1$ is trivial. When $n = 2$,
    $A_1\times A_2 = \bigcup\limits_{a\in A_1}\setX{a}\times A_2$ is a countable unioon of countable sets and
    is countable by the theorem above. We proceed by induction to prove it for any $n \in \Zplus$.
\end{proof}
\begin{Theorem}
    Let $A$ be any set. There is NO injective map \break{}$\map{f}{\powSet{A}}{A}$ and there is no surjective map
    $\map{g}{A}{\powSet{A}}$.
\end{Theorem}
\begin{proof}
    We will show that there is no surjective map which maps $A$ onto the powerset of $A$. This will then imply
    that there is no injective map that maps powerset of $A$ into $A$. This is because of the following:
    \[\thereIs{\map{f}{B}{C}}\text{$f$ is injective, then}\hspace{0.05in}\thereIs{\map{g}{C}{B}}
	\text{$g$ is surjective}.\]
    Taking the contrapositive of this statement we can just show the non-existence of a surjective map. To
    see why the above statement is true, assume there is an injection $\map{f}{B}{C}$. Thus for any $c \in
    \dirIm{f}{B}$, there is a unique $b$ such that $f(b) = c$. Define $\map{g}{C}{B}$ as follows, if $c \in C$
    is in $f(B)$, then $g(c) = b$ such that $f(b) = c$. If $c \in C$ such that $c \not\in f(B)$, then fix a
    $b_0 \in B$ and let $g(c) = b_0$. Thus $g$ is surjective. Note that $B$ has to be non-empty for all this
    to make sense.

    Now we will show the non-existence of a surjective map from $A$ onto $\powSet{A}$ by contradiction. 
    Assume, there is a $\map{g}{A}{\powSet{A}}$ such that $g$ is surjective. Thus for any $X \subset A$, there
    is a $a \in A$ such that $g(a) = X$. Let $B = \set{a\in A}{a \in g(a)}$. Then $B \subset A$ and hence by
    our assumption there is an $a_0 \in A$ such that $g(a_0) = B$.
    There are two possibilities for $a_0$. Either $a_0 \in g(a_0)$ or $a_0 \not\in g(a_0)$. In the first case,
    $a_0 \not\in B$ and in the second case $a_0 \in B$. Thus $g(a_0)\neq B$. Hence, $g$ cannot be surjective.
\end{proof}
\begin{Corollary}
    There exists an uncountable set.
\end{Corollary}
\begin{proof}
    There is no surjection from $\Zplus$ onto $\powSet{\Zplus}$. Thus, $\powSet{\Zplus}$ is uncountable.
\end{proof}
Let us now give a characterization of infinite sets. These sets are either countably infinite or uncountable.
From our discussion following the definition of infinite sets, we already gathered a few notions of what is
means to be for a set to be infinite. We elaborate on this now.
\begin{Theorem}
    Let $A$ be a set. The following statements about $A$ are equivalent.
    \begin{enumerate}
	\item
	    There is an injective function $\map{f}{\Zplus}{A}$.
	\item
	    There exists a bijection of $A$ with a proper subset of itself.
	\item
	    $A$ is infinite.
    \end{enumerate}
\end{Theorem}
\begin{proof}
    $(1)\implies(2)$
    Let $\map{f}{\Zplus}{A}$ be an injective map. Note that $\map{f}{\Zplus}{f(\Zplus)}$ is a bijective map
    that lists a subset of $A$ as a sequence $(a_1,a_2,\cdots)$. Let $B = A - \setX{a_1}$. Then $B$ is a
    proper subset of $A$. Note that $A = f(\Zplus)\cup (\setDiff{A}{f(\Zplus)})$. 
    Define $\map{g}{A}{B}$ as follows:
    \begin{equation*}
	g(a) = 
	\begin{cases}
	    a_{n+1}&\text{if $a = x_n \in f(\Zplus)$ for some $n$}\\
	    a &\text{if $a \not\in f(\Zplus)$}
	\end{cases}
    \end{equation*}

    $(2)\implies(3)$ This just follows from our discussion following the definition of infinite sets.

    $(3)\implies(1)$
    First pick \textbf{any} element of $A$ and call it $a_1$. Thus $f(1) = a_1$. Now, pick any element of $A$
    which is not equal to $a_1$ and call it $a_2$. That is $f(2) \in A - \setX{a_1}$. This process will
    continue indefinitely since $A$ is infinite. Hence $\map{f}{\Zplus}{A}$ is injective, where $f$ is
    definied recursively as follows:
    \[f(n) \in A - \setX{a_1,a_2,\ldots,a_n}.\]
    However, there is a problem in our recursive definition for $f$. Namely, $f(1)$ is not unique and $f(n)$
    is certainly not uniquely defined in terms of $\setX{a_1,a_2,\ldots,a_n}$, since we are picking an
    arbitrary element of $A$. We need an additional axiom from set theory to make this possible.
\end{proof}


\textbf{Axiom of choice} Given a collection $\famB$ of non-empty sets, there exists a function 
\[\map{c}{\famB}{\bigcup\limits_{B\in\famB}B},\]
such that $c(B)\in B$ for any $B \in \famB$.

Using the axiom of choice we can fix our recursive definition for $f$. 
Let $\famB = \powSet{A}-\emptyset$. Then there exists a choice function such that $c(B) \in B$ for any $B \in
\famB$. Pick $B = A$. Thus $c(A)$ gives an element of $A$ uniquely.
Let $f(1) = c(A)$.
Now $A - \setX{f(1)}$ is also in $\famB$. Define $f(2) = c(A - \setX{f(1)})$ which is again uniquely defined.
Thus, define
\[f(n) = c(A - \setX{f(1),f(2),\ldots,f(n-1)}).\]
This is a valid recursive function.


There is a equivalence relation which is induced by bijective functions.
Let us define $A\sim B$ whenever there is a bijective map $\map{f}{A}{B}$. It is easy to see that $\sim$ is an
equivalence relation. A set $A$ is countably infinite if $A \sim \Zplus$. Note that any two countably infinite
sets are equivalent since if $A\sim \Zplus$ and $B \sim \Zplus$, then $A \sim B$. 

This is an extremely
important idea which we will explore now in the subsequent paragraph in the form of Cantor-Schroder-Bernstein
theorem.

Suppose $A,B$ are sets and $f$ is one-to-one function from $A$ into $B$. Then $A\sim f(A)\subset B$, so it is
natural to think of $B$ as being at least as large as $A$. This suggests the following notation:
\begin{Definition}
    If $A,B$ are sets, then we will say that $B$ \textbf{dominates} $A$, and write $A \preceq B$, if there is
    an injective function $\map{f}{A}{B}$. If $A\preceq B$ and $A\not\sim B$, then we say $B$ strictly
    dominates $A$.
\end{Definition}
\begin{Theorem}[name=Schroder-Bernstein theorem]
    If $A \preceq B$ and $B \preceq A$, then $A\sim B$.
\end{Theorem}
\begin{proof}
    Let $\map{f}{A}{B}$ and $\map{g}{B}{C}$ be injective functions. Note that $\map{g}{B}{g(B)}$ is bijective.
    Let $A_0 = \setDiff{A}{g(B)}$ and let us define recursively $A_n = g(f(A_{n-1}))$.
    Let $X = \bigcup\limits_{n=0}^{\infty}A_n$. Then $X \subset A$. Define $Y = \setDiff{A}{X}$. Note that, $Y
    \subset g(B)$. This is because if $a \in Y$ then $a \in A$ and $a \not\in A_0$, that is $a \in g(B)$.
    Define,
    \begin{equation*}
	h(a) = 
	\begin{cases}
	    f(a) &\text{if $a\in X$}\\
	    g^{-1}(a) &\text{if $a\in Y$}
	\end{cases}
    \end{equation*}
    \textbf{Claim:} $h$ is injective. 
    Assume $h(a_1) = h(a_2)$ for $a_1,a_2$ in $A$. The only cases we have to check are when $a_1\in X$ and
    $a_2\in Y$ and vice-versa.
    Suppose $a_1\in X$ and $a_2\in Y$. Then $f(a_1) = g^{-1}(a_2)$. Which means that $g(f(a_1)) = a_2$. Since
    $a_1\in X$, there is an $n$ such that $a_1 \in A_n$ and hence $a_2 \in g(f(A_n))$ which means that $a_2
    \in A_{n+1}$ i.e.~$a_2 \in X$ which is a contradiction. The other case is similar.

    \textbf{Claim:} $h$ is surjective. Let $b \in B$. Then $g(b) \in A$. Thus $g(b) \in X$ or $g(b) \in Y$. If
    $g(b)$ is in $Y$, then $h(g(b)) = g^{-1}(g(b)) = b$. Thus there is an $a = g(b) \in A$ such that $h(a) =
    b$. If $g(b) \in X$, then there is an $A_n$ such that $g(b)\in A_n$. But $n$ cannot be $0$. Thus there is
    an $n > 1$ such that $g(b) \in g(f(A_{n-1}))$. This means that there is an $x_1\in f(A_{n-1})$ such that
    $g(x_1) = g(b)$. Since $g$ is injective, this means that $x_1 = b$ i.e.~there is an $x \in A_{n-1}$ such
    that $f(x) = x_1 = b$. But this means that $h(x) = b$ for some $x \in A_{n-1}\subset X$.
\end{proof}


Now we will 
show that $\mathbb{R}$ is uncountable.
\begin{Theorem}[name=Uncountability of $\mathbb{R}$]
    $\mathbb{R}$ is not countable.
\end{Theorem}
\begin{proof}
    It is sufficient to show that the set $[0,1)$ is uncountable.
    Lets suppose that $[0,1)$ is countable. Hence, we can list all its elements as $r_1, r_2, r_3,
    \ldots$. For example,  
    \begin{displaymath}
	\begin{aligned}
	    & r_1 = 0.\ \fbox{$\mathbf{9}$} \ 1 \ 3 \ 4 \ \ldots \\
	    & r_2 = 0.\ 1 \ \fbox{$\mathbf{0}$} \ 7 \ 6 \  \ldots \\
	    & r_3 = 0.\ 3 \ 0 \ \fbox{$\mathbf{1}$} \ 1 \  \ldots \\
	    & r_4 = 0.\ 0 \ 0 \ 0 \ \fbox{$\mathbf{0}$} \ \ldots \\
	    & \vdots
	\end{aligned}
    \end{displaymath}
    We can find a $r \in [0,1)$ that does not belong to any $r_i \forall i \in \mathbb{Z}_+$.
    Let r be such that its $j^{th}$ digit after the decimal is $0$ if the corresponding digit in
    $r_j$ is non zero and let it be $1$ otherwise. Then for the example above we have 
    \begin{displaymath}
	r = 0.0101\ldots
    \end{displaymath}	
    Clearly, $r \in [0,1)$ but $r$ does not belong to any list. Hence, $\mathbb{R}$ is not
    countable. 
\end{proof}
\begin{Corollary}
    The irrational numbers are uncountable.
\end{Corollary}
\begin{proof}
    $\mathbb{R} = \mathbb{R} - \mathbb{Q} \bigcup \mathbb{Q}$. We showed $\mathbb{Q}$ to be
    countable. If $\mathbb{R} - \mathbb{Q}$ is countable then $\mathbb{R}$ will be countable.
    Hence irrational numbers are uncountable. Thus we have more \emph{holes} than the rational
    numbers.
\end{proof}




