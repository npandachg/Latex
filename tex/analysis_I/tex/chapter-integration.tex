\chapter{Riemann Integration in \texorpdfstring{$\R$}{}}
In this chapter we will discuss the definition and basic properties of the Riemann integral for real
valued functions in one real variable. We will show that continuous functions are Riemann
intergrable and we will derive the fundamental Theorem of calculus that shows that differentiation
and intergration are \emph{inverse} process.

\begin{Definition}
    Given an interval $\left[a,b\right] \subset \mathbb{R}$, a partition of $\left[a,b\right]$ is a
    finite sequence given by,
    \[ a = x_0 < x_1 < x_2 < \dots < x_N = b.\]
    The \emph{width} of this partition is the $max\lbrace x_i - x_{i-1}\rbrace$ for $i = 1 \dots N$.
\end{Definition}
\begin{Definition}
    For any real number $\delta > 0$, $\delta$ - small partition of $\left[a,b\right] \subset 
    \mathbb{R}$ is a partition whose width is less than $\delta$.
\end{Definition}
\begin{Definition}
    For a function $f : \left[a,b\right] \to \mathbb{R}$ a \emph{Riemann sum} $S$ corresponding to a
    partition of $\left[a,b\right]$ is given by,
    \[ S = \sum_{i = 1}^N f(x'{_i})*(x_i - x_{i-1}),\]
    for some $x_{i-1} \leq x'{_i} \leq x_i$.
\end{Definition}
Given any partion there are many Riemann sums corressponding to different choices of $x'$. 
\begin{Definition}
    A function $f : \left[a,b\right] \to \mathbb{R}$ is Riemann integrable if there is an $A \in
    \mathbb{R}$ such that for any $\epsilon > 0$, there is a $\delta > 0$ such that for any Riemann
    sum $S$ on a $\delta$ - small partition, $\lvert S - A \rvert < \epsilon$. 
\end{Definition}
If $f$ is Riemann integrable then we call $A$ the (definite) integral of $f$ over
$\left[a,b\right]$ and we denote \[ A = \int_{a}^{b} f. \] The classical way to denote a definite
integral is the familiar notation $\int_{a}^{b}fdx$. Note that $dx$ is just a notational convenience
that doesn't mean anything.
\begin{Proposition}
    If $A$ exists, then it is unique.
\end{Proposition}
\begin{proof}
    Suppose $A,A'$ are both the integral of $f$ over $\left[a,b\right]$. Then given any $\epsilon >
    0$ there is a $\delta = \text{ min}(\delta_1,\delta_2)$ such that for any Riemann sum $S$ on a
    $\delta$ -small partition we have $\lvert S - A\rvert < \epsilon/2$,
    $\lvert S - A'\rvert < \epsilon/2$. Hence $\lvert A - A'\rvert = \lvert A - S + S - A'\rvert <
    \epsilon$. Since $\epsilon > 0$ is arbitrary, this means that $\lvert A - A'\rvert = 0$. Hence,
    $A = A'$. 
    The existence of a $\delta$ -small partition is guaranteed. To see this choose $N \in
    \mathbb{Z}_+$ such that the width of the partition is given by $\frac{b-a}{N}$. Choosing $N$
    large enough, the width can be made smaller than $\delta$.
\end{proof}
Some examples,
\begin{itemize}
    \item The constanct function $f = c$ is Riemann integrable.
	\begin{proof}
	    Any Riemann sum $S$ is given by $\sum_{i = 1}^{N}f(x'_{i})(x_i - x_{i-1})$. Since $f =
	    c$, the Riemann sum becomes $c\sum_{i = 1}^{N}(x_i - x_{i-1}) = c(x_N - x_0)$. Thus, $A
	    = c(b - a)$. To check, for any $\epsilon > 0$ , take $\delta = 1$. Then for any 
	    Riemann sum on $1$ -small partition, $\lvert S - c(b-a)\rvert$ = $\lvert c(b - a) - 
	    (b-a)\rvert$. Hence $\lvert S - A \rvert = 0 < \epsilon$. 
	\end{proof}
    \item Fix a $\xi \in \left[a,b\right]$. Define a function $f : \left[a,b\right] \to \mathbb{R}$
	given by, 
	\[ f(x) = \left\{ 
		\begin{array}{l l}
		    c & \quad \text{if } x = \xi \\
		    0 & \quad \text{if } x \neq \xi 
		\end{array} \right. .\]
	Then $\int_a^b f = 0$.
	\begin{proof}
	    For a $\delta$ -small partition, each $(x_i - x_{i-1}) < \delta$. Now whenever, $x_{i-1}
	    < x'_i = \xi < x_{i}$, i.e if $\xi$ happens to be in-between tick points then the
	    Riemann sum $S = c*(x_i - x_{i-1}) < c*\delta$. If $\xi$ happens to be one of the tick
	    marks i.e $\xi = x_i$ then at most the Riemann sum can be $S = c*(x_i - x_{i-1}) +
	    c*(x_{i+1} - x_{i}) < 2c*\delta$. This is because to the left we pick $x'_i = x_i$ and
	    to the right we pick $x'_{i+1} = x_i$. Hence, we see that $\lvert S \rvert <
	    2\lvert c \rvert *\delta$. Thus, given any $\epsilon > 0$, choose $\delta = 
	    \frac{\epsilon}{2\lvert c \rvert}$. Hence, for any $\delta$ -small partition 
	    $\lvert S - 0\rvert < \epsilon$, thus $A = 0$.
	\end{proof}
    \item Let $\alpha,\beta \in \left[a,b\right]$ with $\alpha < \beta$. Let $f : \left[a,b\right]
	\to \mathbb{R}$ be defined by, 
	\[ f(x) = \left\{ 
		\begin{array}{l l}
		    1 & \quad \text{if } x \in (\alpha,\beta) \\
		    0 & \quad \text{if } x \in \left[a,b\right], x \neq (\alpha,\beta)  
		\end{array} \right. .\]
	Then $\int_a^b f = \beta - \alpha$.
	\begin{proof}
	    Let us say we found a $\delta$ -small partition $\delta < \beta - \alpha$. Thus there
	    has to be atleast one \emph{tick} between $\alpha$ and $\beta$,  i.e we can find
	    integers $p,q < N$ such that $x_{p-1} \leq \alpha < x_p \leq x_{q-1} < \beta \leq
	    x_q$. For any Riemann sum corressponding to the partition,
	    \[ f(x'_i) = \left\{ 
		    \begin{array}{l l}
			1 & \quad \text{if } x'_i \in (\alpha,\beta) \\
			0 & \quad \text{ otherwise }   
		    \end{array} \right. .\]
	    Thus $S = \sum* (x_i - x_{i-1})$, whenever $x'_i$ is choosen in $ (\alpha,\beta)$. Lets
	    look at some cases -
	    \begin{enumerate}
		\item $x'_i \in \left[x_{i-1},x_{i}\right]$ such that $i = p-1$. Then $f(x'_i) = 0$ .
		\item $x'_i \in \left[x_{i-1},x_{i}\right]$ such that $i = p$. Then $f(x'_i) = 1$ 
		    only when $x'_i > \alpha$. Thus we are not guranteed $f(x'_i) = 1$ always. 
		\item $x'_i \in \left[x_{i-1},x_{i}\right]$ such that $i = p+1$. Then $f(x'_i) = 1$ always.
		\item $x'_i \in \left[x_{i-1},x_{i}\right]$ such that $i = q - 1$. Then $f(x'_i) = 
		    1$ always.
		\item $x'_i \in \left[x_{i-1},x_{i}\right]$ such that $i = q $. Then $f(x'_i) = 1$ 
		    only when  $x'_i < \beta$. 
		\item $x'_i \in \left[x_{i-1},x_{i}\right]$ such that $i = q + 1$. Then $f(x'_i) = 0$ .
	    \end{enumerate}
	    Hence we see that $p+1 \leq i \leq q-1$, then $S = \sum_i(x_i - x_{i-1})$. When $i \leq
	    p - 1, i \geq q +1$ then $S = 0$. Thus,
	    \[ \sum_{i = p+1}^{q-1}(x_{i} - x_{i-1}) \leq S \leq \sum_{i = p}^{q}(x_i - x_{i-1}). \]
	    Since the left and right terms are just \emph{telescoping} terms we get,
	    \[ x_{q-1} - x_p \leq S \leq x_q - x_{p-1}. \]
	    Subtracting $\beta - \alpha$ and noting the following,
	    \begin{enumerate}
		\item $\lvert x_{q-1} - \beta \rvert < \delta$.
		\item $\lvert x_{p} - \alpha \rvert < \delta$.
		\item $\lvert x_{q} - \beta \rvert < \delta$.
		\item $\lvert x_{p-1} - \alpha \rvert < \delta$.
	    \end{enumerate}
	    Thus,
	    \[ \lvert S - (\beta - \alpha) \rvert < 2\delta.\]
	    Hence, given any $\epsilon > 0$, choose $\delta = \text{ min}(\beta - \alpha,
	    \epsilon/2)$ such that for any $\delta$ -small partition $\lvert S - (\beta - \alpha) 
	    \rvert < 2\delta < \epsilon$. Hence $A = \beta - \alpha$.
	\end{proof}
    \item Let $f : \left[a,b\right] \to \mathbb{R}$ be defined in the following way,
	\[ f(x) = \left\{ 
		\begin{array}{l l}
		    1 & \quad \text{if } x  \in \mathbb{Q}\\
		    0 & \quad \text{otherwise }    
		\end{array} \right. .\]
	$f$ is NOT (Reimann) integrable.
	\begin{proof}
	    For a given $\epsilon > 0$ no matter how small we choose our $\delta$ if $x'_i \in
	    \mathbb{Q}$ then $ S  = b-a $ . But for the same partition if $x'_i \not
	    \in \mathbb{Q}$ then $S = 0$.
	\end{proof}
\end{itemize}
In view of the last example let us reiterate the definition of Riemann integral which we will from
now on just replace with the word integral keeping in mind that other integrals exists.
\begin{Definition}
    Let $f :\left[a,b\right]$. The (Riemann) integral $\int_{a}^{b}f$, if it exists, is the unique
    number such that for any $\epsilon > 0$ there is a $\delta > 0$ such that for all Riemann sums
    $S$ on a $\delta$ -small partition, \[\lvert S - \int_{a}^{b}f \rvert < \epsilon .\]
\end{Definition}
Now we'll look at some properties of the integral.

\begin{Theorem}[name=Properties of Riemann integral]
 Let $f,g : \left[a,b\right] \to \mathbb{R}$.
    \begin{enumerate}
	\item (LINEAR property) If $f,g : \left[a,b\right] \to \mathbb{R}$ are integrable and $c \in
	    \mathbb{R}$ then $(f+g), c*f$ are integrable. These are given by,
	    \begin{displaymath}
		\begin{aligned}
		    &\int_a^b (f+g) = \int_a^b f + \int_a^b g \\
		    & \int_a^b c*f = c*\int_a^b f
		\end{aligned}
	    \end{displaymath}
	\item (Order property) If $f,g : \left[a,b\right] \to \mathbb{R}$ are integrable and for all
	    $x \in \left[a,b\right]$, $f \leq g$ then $\int_a^b f \leq \int_a^b g$.
    \end{enumerate}
\end{Theorem}
\begin{proof}
    \begin{enumerate}
	\item Given any $\epsilon > 0$ there is a $\delta_f > 0$ such that for all Riemann sum $S_f$
	    on the $\delta_{f}$ -small partition $\lvert S_f - \int_a^b f\rvert < \epsilon/2$.
	    Similarly, there is a $\delta_{g} > 0$ such that for all Riemann sum $S_g$ on the
	    $\delta_{g}$ -small partition $\lvert S_g - \int_a^b g\rvert < \epsilon/2$. Let $\delta
	    = \text{ min}(\delta_{f},\delta_{g})$. Let $S_{f+g}$ be the Riemann sum on the $\delta$
	    -small partition. Note that  $S_{f+g} = \sum_{i}(f+g)(x'_i)*(x_{i} - x_{i-1}) = S_f +
	    S_g$. Thus,
	    \begin{displaymath}
		\begin{aligned}
		     \lvert S_{f+g}-\int_a^b(f+g) \rvert & =  \lvert S_{f+g} - (\int_a^b f+\int_a^b
		     g) \\
		     & = \lvert (S_f - \int_a^b f) + (S_g - \int_a^b g) \rvert \\
		     &\leq \lvert S_f - \int_a^b f \rvert + \lvert S_g - \int_a^b g \rvert \\
		     & \quad = \epsilon.
		\end{aligned}
	    \end{displaymath}	
	    Hence, $\int_a^b (f+g)$ is integrable.
	    To show that $c*f$ is integrable, given any $\epsilon > 0$ there is $\delta > 0$ such
	    that for any $\delta$ -small partition, $\lvert S - \int_a^b f \rvert <
	    \frac{\epsilon}{\lvert c \rvert}$.  Let $S_{cf}$ be the
	    Riemann sum on the $\delta$ -small partition for the function $c*f$. Then $S_{cf} =
	    \sum_i c*f(x'_i)*(x_i - x_{i-1}) = c*S$. Easy to check that $\lvert S_{cf} - c\int_a^b
	    f\rvert < \epsilon$.
	\item First let us show that if $f \geq 0$ for all $x \in \left[a,b\right]$ then $\int_a^b f
	\geq 0$. Given any $\epsilon > 0$ there is a $\delta > 0$ such that $\lvert S - \int_a^b f
	\rvert < \epsilon$. Thus $\int_a^bf \geq S - \epsilon \geq -\epsilon$ since $S \geq 0$.
	$\int_a^b f \in \mathbb{R}$ which is greater than $-\epsilon $ for any arbitrary $\epsilon >
	0$. Thus $\int_a^b f \geq 0$. Now since $f \leq g$ for all $x$,$(g-f) \geq 0$ and so
	$\int_a^b(g-f) \geq 0$ and using the linearity we get the proof.
    \end{enumerate}
\end{proof}
\begin{Corollary}
If $f$ is integrable real valued function on the interval $\left[a,b\right]$ and $m,M \in
\mathbb{R}$ are such that $m \leq f(x) \leq M$ for all $x \in \left[a,b\right]$ then 
\[ m(b-a) \leq \int_a^b f \leq M(b-a). \]
\end{Corollary}
\begin{proof}
    We can consider $m,M$ as constant functions then the first inequality gives $m \leq f$ and so
    from the Theorem above $int_a^b m \leq int_a^b f$. But we know that integral of the constant 
    function $m$ is given by $m(b-a)$. Thus we get the first inequality. Similarly the other.
\end{proof}
We have shown the properties of the Riemann integral but have not yet talked about what functions
are Riemann integrable. In the following section we'll talk about the existence of Riemann integral
for a certain class of functions.

\section{Existence of the Riemann integral}
We'll start with the cauchy criterion for the integral.
\begin{Lemma}
    A real valued function $f$ on the interval $\left[a,b\right]$ is integrable on
    $\left[a,b\right]$ iff given an $\epsilon > 0$ there is a $\delta > 0$ such that whenever 
    $S_1,S_2$ are Riemann sums on a $\delta$ -small partition then $\lvert S_1-S_2\rvert <
    \epsilon$. 
\end{Lemma}
\begin{proof}
    $\Rightarrow$ If $f$ is integrable then given any $\epsilon > 0$ there is a $\delta$ such that
for any Riemann sums on $\delta$ -small partitions $\lvert S - \int_a^b f \rvert < \epsilon/2$. Thus
$S_1,S_2$ are Riemann sums on $\delta$ small partitions then $\lvert S_1 - S_2 \rvert = \lvert S_1 -
\int_a^b f + \int_a^b f - S_2 \rvert < \epsilon $ from the triangle inequality.

$\Leftarrow$ Given an $\epsilon > 0$ there is a $\delta > 0$ such that whenever $S_1,S_2$ are
    Riemann sums on $\delta$ -small partition then $\lvert S_1 - S_2 \rvert < \epsilon$. Let
    $S^{(1)}, S^{(2)}, S^{(3)} , \dots $ be Riemann sums where each $S^{(k)}$ is a Riemann sum on a 
    partition of width less than $1/k$. Choose a $N \in \mathbb{Z_+}$ such that $1/N < \delta$. 
    Then for any $m,n \geq N$, $S^{(m)},S^{(n)}$ are Riemann sums on a $\delta$ -small partition and
    so by the hypothesis $\lvert S^{(n)} - S^{(m)} \rvert < \epsilon$. Hence $S^{(k)}$ is a cauchy
    sequence in $\mathbb{R}$. Since $\mathbb{R}$ is complete $S^{(k)} \to A$ for some $A \in
    \mathbb{R}$. Hence given any $\epsilon > 0$ there is a $N > 0$ such that whenever $n \geq N$,
    $\lvert S^{(n)} - A \rvert < \epsilon/2$. Choose $n > N $ such that $1/n < \delta$. 
    Let $S^{(m)} = S$ be any Riemann sum on a $\delta$ - small partition. 
    Then $\lvert S - A \rvert = \lvert S - S^{(n)} + S^{(n)} - A \rvert < \epsilon$
    using the triangle inequality. 
\end{proof}

Next we'll define step functions.
\begin{Definition}
    A real-valued function on the interval $\left[a,b\right]$ is called a step function if there
    exists a partition $x_0,x_1,\dots,x_N$ of $\left[a,b\right]$ such that $f$ is constant on each
    open subinterval $(x_0,x_1),(x_1,x_2),\dots,(x_{N-1},x_N)$.
\end{Definition}
\begin{Lemma}
    A step function is integrable. In particular, if $x_0,x_1,\dots,x_N$ is a partition 
    of $\left[a,b\right]$, if $c_1,c_2,\dots,c_N \in \mathbb{R}$ and if $f : \left[a,b\right]$ is
    such that $f(x) = c_i$ if $x_{i-1} < x < x_i$ for $i = 1,\dots,N$ then,
    \[ \int_a^b f = \sum_{i = 1}^N c_i(x_i - x_{i-1}). \]
\end{Lemma}
\begin{proof}
    
    We'll use example 2,3 following the definition of Riemann integral and the linear property to
    prove this Lemma. For $i = 1, \dots, N$ define a function $\phi_{i} : \left[a,b\right] \to
    \mathbb{R}$ by,
    \[ \phi_{i}(x) = \left\{ 
	    \begin{array}{l l}
		1 & \quad \text{if } x \in (x_{i-1},x_i) \\
		0 & \quad \text{if } x \in \left[a,b\right], x \not \in (x_{i-1},x_i) 
	    \end{array} \right. .\]
    Then $(f - \sum_{i=1}^N c*\phi_{i}(x)) = 0$ at all points except at the partition points
    $x_1,\dots,x_N$. It is the sum of finite functions of type 2 seen in the example following
    definition of Riemann integral. Thus $\int_a^b ( f - \sum_{i=1}^N c*\phi_{i}(x) ) = 0$. Also
    note that each $\phi_{i}(x)$ is the type 3 function in the example and by linearity $\int_a^b 
    \sum_{i=1}^N c*\phi_{i}(x) = c*\sum_{i}\int_a^b\phi_{i}$ which is given by $\sum_{i=1}^N c*(x_i
    - x_{i-1})$. Noting that $f = (f - \sum_{i=1}^N c*\phi_{i}(x) ) + \sum_{i=1}^N c*\phi_{i}(x)$,
    taking the integral we get the proof.
\end{proof}
The next proposition is an extremely important one stating that an integrable function can be
arbitrarily sandwhiched between two step functions. This will be useful in showing that all
continuous functions on a closed and bounded subset of $\mathbb{R}$ are Riemann integrable thereby
giving us a class of functions that satisfy integrability.

\begin{Proposition}
    The real-valued function f on the interval $\left[a,b\right]$ is integrable on
    $\left[a,b\right]$ iff for each $\epsilon > 0$ there exists step functions $f_1,f_2$ on
    $\left[a,b\right]$ such that 
    \begin{displaymath}
	\begin{aligned}
	    &f_1(x) \leq f(x) \leq f_2(x) \quad \text{for each } x \in \left[a,b\right], \\
	    &\int_a^b(f_2(x) - f_1(x)) < \epsilon .
	\end{aligned}
    \end{displaymath}
\end{Proposition}
\begin{proof}
    $\Rightarrow$. We'll carry out the proof in 3 steps. Given $f :\left[a,b\right]$ is integrable
and given any $\epsilon > 0$. 

\textbf{Step 1} : Show that $f$ is bounded on $\left[a,b\right]$
Using the Cauchy criteria, we have a $\delta > 0$ and can find a partition $x_0,x_1,\dots,x_N$ of
width less that $\delta$ such that any two Riemann
sums $S_1,S_2$ for $f$ corresponding to this partition, $\lvert S_1 - S_2 \rvert < \epsilon/2$. That
is, for arbitrary $x'_i,x"_i \in \left[x_{i-1},x_{i}\right]$, $i = 1,\dots,N$ we have,
\[ \sum_{i = 1}^N \lvert (f(x'_i) - f(x"_i))*(x_i - x_{i-1}) \rvert < \epsilon/2 .\] Now, for any $j
= 1, \dots, N$, pick $x'_i \neq x"_i$ but for all other $i$ pick $x'_i = x"_i$. Thus for this $j$ we
have 
\[ \lvert (f(x'_j) - f(x"_j))*(x_j - x_{j-1}) \rvert < \epsilon/2 .\] Since our choice of $x',x"$ are
arbitrary pick $x"_j = x_j \neq x'_j$. Thus,
\[ \lvert (f(x'_j) - f(x_j))*(x_j - x_{j-1}) < \lvert \epsilon/2 , \] and thus 
\[ \lvert f(x'_j) \rvert < \lvert f(x_j) \rvert + \frac{\epsilon}{2(x_j - x_{j-1})} .\]
Thus $f(x)$ is bounded in $\left[x_{j-1},x_{j}\right]$ 
. We can do this for all $j \in 1, \dots, N$
and so $f(x)$ is bounded in $\left[a,b\right]$.

\textbf{Step 2} : Find the existence of step functions $f_1,f_2$ such that $f_1(x) \leq f(x) \leq 
f_2(x)$.
Since $f(x)$ is bounded on each $\left[x_{i-1},x_{i}\right]$, let us define the following :
\begin{displaymath}
    \begin{aligned}
	&m_i = \text{ inf}\lbrace f(x'_i) : x'_i \in \left[x_{i-1},x_i\right]\rbrace \\
	&m = \text{ min}(m_i : i = 1 \dots N) \\
	&M_i = \text{ sup}\lbrace f(x'_i) : x'_i \in \left[x_{i-1},x_i\right]\rbrace \\
	&M = \text{ max}(M_i : i = 1 \dots N)
    \end{aligned}
\end{displaymath}
With this we can define step functions $f_1,f_2$ as follows : 
    \[ f_{1}(x) = \left\{ 
	    \begin{array}{l l}
		m_i & \quad \text{if } x \in (x_{i-1},x_i) \quad i = 1,\dots,N \\
		m & \quad \text{if } x = x_i \quad i = 0,\dots,N 
	    \end{array} \right. ,\]
    \[ f_{2}(x) = \left\{ 
	    \begin{array}{l l}
		M_i & \quad \text{if } x \in (x_{i-1},x_i) \quad i = 1,\dots,N \\
		M & \quad \text{if } x = x_i \quad i = 0,\dots,N 
	    \end{array} \right. .\]
    Hence $f_1(x) \leq f(x) \leq f_2(x)$ for all $x \in \left[a,b\right]$. From the lemma concerning
    step functions we know that $\int_a^b f_1 = \sum_{i = 1}^N m_i*(x_i - x_{i-1})$ and similarly
    $\int_a^b f_2(x) = \sum_{i = 1}^N M_i*(x_i - x_{i-1})$.

\textbf{Step 3} : Show that $\int_a^b (f_2(x) - f_1(x)) < \epsilon$.
Since $m_i$ is the infimum of $f(x)$ for $x \in \left[x_{i-1},x_i\right]$, given an $\eta > 0$ there
is some $\xi'_i \in \left[x_{i-1},x_i\right]$ such that $f(\xi'_i) < m_i + \eta$. Similarly there is
some $\xi"_i \in \left[x_{i-1},x_i\right]$ such that $f(x"_i) > M_i + \eta$. Hence, 
\[ \sum_{i = 1}^{N}(f(\xi"_i) - f(\xi'_i))*(x_i - x_{i-1}) > \sum_{i = 1}^{N}(M_i - \eta - m_i -
    \eta)(x_i - x_{i-1}) .\]
The left hand term is less than $\epsilon/2$ from the Cauchy criteria. The right hand term contain
the integral of $f_1$ and $f_2$ as stated towards the end of step 2. Thus collecting the terms of
$\eta$ which have telescoping sums we get :
\[ \int_a^b (f_2 - f_1) -2\eta(b-a) < \epsilon /2 ,\] and so moving terms containing $\eta$ we have
\[ \int_a^b (f_2 - f_1) < \epsilon/2 + 2\eta(b-a) .\] Since, $\eta > 0 $ was arbitrary we get the
proof i.e 
\[ \int_a^b (f_2 - f_1) \leq \epsilon/2 < \epsilon .\]

$\Leftarrow$. Given step functions $f_1,f_2$ such that $f_1(x) \leq f(x) \leq f_2(x)$ for each 
   $ x \in \left[a,b\right]$ and $\int_a^b(f_2(x) - f_1(x)) < \epsilon/3$. Since $f_1,f_2$ are
   integrable (they are step functions), there is a $\delta > 0$ such that for a partition
   $x_0,x_2,\dots,N$ of width less than delta, 
   \begin{displaymath}
       \begin{aligned}
	   &\lvert \sum_{i=1}^N f_1(x'_i)(x_i - x_{i-1}) - \int_a^b f_1 \rvert < \epsilon/3 \\
	   &\lvert \sum_{i=1}^N f_2(x'_i)(x_i - x_{i-1}) - \int_a^b f_2 \rvert < \epsilon/3 \\
       \end{aligned}
   \end{displaymath}
   Let $S = \sum_{i = 1}^Nf(x'_i)(x_i - x_{i-1})$ be a riemann sum of $f$ on the partition. Since,
   $f_1(x) \leq f(x) \leq f_2(x)$ for each $x$ we have
   \[\sum_{i=1}^N f_1(x'_i)(x_i - x_{i-1}) \leq S \leq \sum_{i=1}^N f_2(x'_i)(x_i - x_{i-1}) .\]
   Using the inequality above for the integrability of $f_1,f_2$ we have
   \[\int_a^b f_1 - \frac{\epsilon}{3} \leq S \leq \int_a^b f_2 + \frac{\epsilon}{3}. \]Since this
   is true for any Riemann sum of $f$ on a $\delta$ -small partition, let $S_1,S_2$ be two Riemann
   sums of $f$ on the $\delta$ small partition then 
   \[\int_a^b f_1 - \frac{\epsilon}{3} \leq S_1 \leq \int_a^b f_2 + \frac{\epsilon}{3}. \]
   \[-\int_a^b f_1 + \frac{\epsilon}{3} \geq -S_2 \geq -\int_a^b f_2 - \frac{\epsilon}{3}. \]
   Note that we have multiplied by $-1$ to get the inequality for $S_2$. Thus,
   $S_1 - S_2 \leq 2\epsilon/3 + \int_a^b(f_2 - f_1)$. But using the hypothesis $
   \int_a^b(f_2 - f_1) < \epsilon/3$ and so $S_1 - S_2 < \epsilon$. Similarly $S_1 - S_2 >
   -\epsilon$. Hence,
   \[ \lvert S_1 - S_2 \rvert < \epsilon .\] Thus we have established the cauchy criterion for $f$.
   Hence $f$ is integrable.
\end{proof}
The next corollary was already proved in Step 1 of the proposition.
\begin{Corollary}
    If $f : \left[a,b\right]$ is integrable then $f$ is bounded.
\end{Corollary}
\begin{Theorem}[name=Continuous real valued function on compact sets are integrable]
    If $f$ is a continuous real valued function on the interval $\left[a,b\right]$ then $f$ is
    integrable.
\end{Theorem}
\begin{proof}
    We will prove the Theorem by showing the existence of step functions $f_1,f_2$ such that for all
    $x \in \left[a,b\right]$, $f_1(x) \leq f(x) \leq f_2(x)$ and given any $\epsilon > 0$ 
    $\int_a^b(f_1(x),f_2(x)) <\epsilon$. Since $f$ is continuous on a compact subset of 
    $\mathbb{R}$ $f$ is uniformly continuous. Thus for $\epsilon > 0 $ there is a $\delta > 0$ such
    that for any $x',x" \in \left[a,b\right]$ whenever $\lvert x' - x" \rvert < \delta $ then
    $\lvert f(x') - f(x") \rvert < \epsilon$. Choose a partition $x_0 < x_1 < \dots , < x_N$ such
    that the width is less than $\delta$. Since $f$ is continuous on $\left[a,b\right]$, it is bounded in
    $\left[a,b\right]$ and so it is bounded in each $\left[x_{i-1},x_i\right]$ for $i = 1,\dots,N$
    and hence $f$ restricted to each $\left[x_{i-1},x_i\right]$ attains its maxima and mimima. Let
    $x'_i,x"_i$ be the points where $f$ attains its minima and maxima when restricted to 
    $\left[x_{i-1},x_i\right]$
    Define $f_1,f_2$ as follows :
    \[ f_{1}(x) = \left\{ 
	    \begin{array}{l l}
		f(x'_i) & \quad \text{if } x \in \left(x_{i-1},x_i\right) \quad i = 1,\dots,N \\
		f(x) & \quad \text{if } x = x_i \quad i = 0,\dots,N 
	    \end{array} \right. ,\]
    \[ f_{2}(x) = \left\{ 
	    \begin{array}{l l}
		f(x"_i) & \quad \text{if } x \in (x_{i-1},x_i) \quad i = 1,\dots,N \\
		f(x) & \quad \text{if } x = x_i \quad i = 0,\dots,N 
	    \end{array} \right. .\]
    Hence $f_1(x) \leq f(x) \leq f_2(x)$. Since $x'_i,x"_i \in \left[x_{i-1},x_i\right]$ and we have
    a $\delta$ -small partition we get $\lvert x'_i - x"_i \rvert < \delta$ for each $i =
    0,\dots,N$. Hence by uniform continuity, $\lvert f(x'_i) - f(x"_i) \rvert <
    \frac{\epsilon}{b-a}$ for each $x'_i,x"_i \in \left[x_{i-1},x_i\right]$ and therfore 
    for any $x \in \left[a,b\right]$ $\lvert f_1(x) - f_2(x) \rvert < \frac{\epsilon}{b-a}$. Thus, 
    since $f_2 \geq f_1$ we can remove the absolute value to say 
    $ f_2(x) - f_1(x) < \frac{\epsilon}{b-a}$. Hence max$\lbrace (f_2(x) - f_1(x)) : x \in 
    \left[a,b\right]\rbrace < \frac{\epsilon}{b-a}$ . Thus,
    \[ \int_a^b(f_2(x) - f_1(x)) \leq \text{ max} \lbrace (f_2(x) - f_1(x)) : x \in
	\left[a,b\right]\rbrace .(b-a) .\] Hence $\int_a^b(f_2(x) - f_1(x)) < \epsilon$.

\end{proof}
Note that the existence of integrability based on continuity used the cauchy criteria in its proof
which uses the complenetess on $\mathbb{R}$. Thus when our field is the rational numbers this
existence is not guaranteed.   

\begin{Theorem}
    Let $a,b,c \in \mathbb{R}$, such that $a < b < c $ and let $f$ be a real valued function on
    $\left[a,c\right]$. Then $f$ is integrable on $\left[a,c\right]$ iff $f$ is integrable on 
    $\left[a,b\right]$ and $\left[b,c\right]$, in which case 
    \[ \int_a^b f + \int_b^c f = \int_a^c f .\]
\end{Theorem}
\begin{proof}
    $\Leftarrow$
    Since $f$ is integrable on $\left[a,b\right]$ and $\left[b,c\right]$ we can find step functions
    $h_1(x) \leq f(x) \leq h_2(x)$ for each $x \in \left[a,b\right]$ and $k_1(x) \leq f(x) \leq
    k_2(x)$ on each $x \in \left[b,c\right]$ such that $\int_a^b (h_2(x) - h_1(x)) < \epsilon /2$
    and $\int_b^c(k_2(x) - k_1(x)) < \epsilon/2$. Define $f_1,f_2$ as :
    \[ f_{1}(x) = \left\{ 
	    \begin{array}{l l}
		h_1 & \quad \text{if } a \leq x \leq b \\
		k_1 & \quad \text{if } b < x \leq c. 
	    \end{array} \right. ,\]
    \[ f_{2}(x) = \left\{ 
	    \begin{array}{l l}
		h_2 & \quad \text{if } a \leq x \leq b \\
		k_2 & \quad \text{if } b < x \leq c. 
	    \end{array} \right. ,\]
    Then, $f_1,f_2$ are step functions such that $f_1(x) \leq f(x) \leq f_2(x)$ for all $x \in
    \left[a,b\right]$. Since $f_1,f_2$ are step functions we can calculate $\int_a^c f_1$ and
    $\int_a^c f_2$ and so $\int_a^c (f_2(x) - f_2(x))$. Moreover this is the same as $\int_a^b (f_2
    -f_1) + \int_b^c (f_2 - f_1)$. Thus $\int_a^c (f_2 - f_1) < \epsilon$ and hence $f$ is integrable
    in $\left[a,c\right]$.
    
    $\Rightarrow$
    $f$ is integrable on $\left[a,c\right]$ then for any $\epsilon > 0$ there are step functions
    $f_1,f_2$ on $\left[a,c\right]$ such that $f_1(x) \leq f(x) \leq f_2(x)$ and $\int_a^c (f_2(x) -
    f_1(x)) < \epsilon$. Since $f_2(x) - f_1(x) \geq 0$ on $\left[a,b\right]$, $\int_a^b(f_2(x) -
    f_1(x)) \geq 0$. Similarly $\int_b^c (f_2(x) - f_1(x)) \geq 0 $ on $\left[b,c\right]$ Moreover,
    since $f_1,f_2$ are step functions,
    \[\int_a^c (f_2(x) - f_1(x)) = \int_a^b (f_2(x) - f_1(x)) + \int_b^c (f_2(x) - f_1(x)).\]. Since
    we noted that both integrals on the right hand side are positive and since the integral on the
    left hand side is less than $\epsilon$ each integral on the right is less than $\epsilon$. But
    that is the criteria of integrability of $f$ on $\left[a,b\right]$ and $\left[b,c\right]$.
    Now we'll prove that if the Theorem holds then \[  \int_a^b f + \int_b^c f = \int_a^c f . \]
    Given an $\epsilon > 0$, since $f$ is integrable on $\left[a,b\right]$ there is $\delta_{1} > 0$
    such that if $S_1$ is a Riemann sum for any $\delta_{1}$ small partition 
    $\lvert S_1 - \int_a^b f \rvert < \epsilon /3 $. Similarly there is $\delta_{2} > 0 $ such that
    $\lvert S_2 - \int_b^c f \rvert < \epsilon /3$ and a $\delta_{3} > 0$ such that 
    $\lvert S_3 - \int_a^c f \rvert < \epsilon /3$. Take $\delta = \text{ min
    }(\delta_1,\delta_2,\delta_3)$ and note that for any $\delta$ -small partition  $S_3 = S_1 +
    S_2$. Now,
    \begin{displaymath}
	\begin{aligned}
	    \lvert \int_a^b f +  \int_b^c f -  \int_a^c f \rvert & = \lvert \int_a^b f - S_1 + 
	    \int_b^c f - S_2  +  (S_1 + S_2) - \int_a^c f \rvert  \\
	    & \leq \lvert S_1 - \int_a^b f \rvert + \lvert S_2 - \int_b^c f \rvert + 
	    \lvert S_3 - \int_a^c f \rvert , \\
	    & < \epsilon . 
	\end{aligned}
    \end{displaymath}
    Since $\epsilon$ was arbitrary,
    \[ \lvert \int_a^b f +  \int_b^c f -  \int_a^c f \rvert = 0. \] Thus the quantity inside the
    absolute value is $0$ which completes the proof.
\end{proof}
\begin{Definition} 
    If $f : \left[a,b\right] \to \mathbb{R}$ is integrable then,
    \[ \int_a^b f = - \int_b^a f, \] and for any $a \leq c \leq b$,
    \[ \int_c^c f = 0.\]
\end{Definition}
With this definition and the Theorem preceeding it, we can easily show the following corollary
\begin{Corollary}
    If $f$ is a real valued function on an interval in $\mathbb{R}$ which contains the points
    $a,b,c$ and if two of the quantities $\int_a^b f, \int_b^c f, \int_a^c f$ exists, then the third
    exists and 
    \[ \int_a^b f + \int_b^c f = \int_a^c f .\]
\end{Corollary}
\begin{Corollary}
    If $f$ is integrable on a closed interval in $\mathbb{R}$ then for any $a,b$ in the interval
    $\int_a^b f$ exists. If $\lvert f(x) \rvert \leq M$ for all $x$ in the interval then for any
    $a,b$ 
    \[\lvert \int_a^b f \rvert \leq M \lvert b - a \rvert . \]
\end{Corollary}

\section{Fundamental Theorem of Calculus}
\begin{Theorem}[name=Fundamental Theorem of calculus]
    Let $f$ be a continuous real valued function on an open interval $\mathcal{U}$ in $\mathbb{R}$
    and let $a \in \mathcal{U}$. Let the function $F$ on $\mathcal{U}$ be defined by 
    \[F(x) = \int_a^x f,\] for all $x \in \mathcal{U}$. Then $F$ is differentiable and $F' =
    f$. 
\end{Theorem}
\begin{proof}
 We have to show that for any $x_0 \in
    \mathcal{U}$,
    \[ \lim_{x \to x_0} \frac{F(x) - F(x_0)}{x - x_0} = f(x_0) .\] 
    Now for any $x \in \mathcal{U}$ and $x \neq x_0$ we have
    \begin{displaymath}
	\left.\lvert \frac{F(x) - F(x_0)}{x - x_0} - f(x_0)\rvert\right. = \left.\lvert \frac{\int_a^x f -
	    \int_a^{x_0}f}{x - x_0} - f(x_0) \rvert\right.
    \end{displaymath}
    Now from the corollary of the previous Theorem for any $a,x,x_0 \in \mathcal{U}$,
    \[ \int_a^x f = \int_a^{x_0} f + \int_{x_0}^{x} f ,\]
    and since $f(x_0)$ is just a number we can write $f(x_0)$ as 
    \[ f(x_0)*(x - x_0) = \int_{x_0}^x f(x_0) .\] Hence
    \[\left.\lvert \frac{F(x) - F(x_0)}{x - x_0} - f(x_0)\rvert \right. = \left.\lvert
	    \frac{\int_{x_0}^x (f - f(x_0))}{x -x_0}\rvert \right. .\]
     Since $f$ is continuous at $x_0$, for a given $\epsilon > 0 $ there is a $\delta > 0$ such that 
     whenever $\lvert x - x_0 \rvert < \delta$ then $\lvert f(x) - f(x_0) \rvert < \epsilon$. Thus
     we see that $\epsilon $ is an upper bound for $\lbrace \lvert f(\xi) - f(x_0) \rvert : \text{
	 $\xi$ in between $x$ and $x_0$ } \rbrace.$ Thus from the corollary 2 of the previous
     Theorem $\lvert \int_{x_0}^x (f - f(x_0)) \rvert < \epsilon \lvert x - x_0 \rvert$. Hence,
     \[\left.\lvert \frac{F(x) - F(x_0)}{x - x_0} - f(x_0)\rvert\right. < \epsilon .\] Since
     $\epsilon > 0$ was arbitrary we get the proof.
\end{proof}
\begin{Corollary}
    If $f$ is a real valued function on an open interval in $\mathbb{R}$, then there exists a real
    valued function $F$ on the same interval whose derivative is $f$. 
\end{Corollary}
\begin{proof}
    Pick an $a \in \mathcal{U}$ and for any $x \in \mathcal{U}$ let $F(x) = \int_a^x f $. Then from
    the above Theorem $F' = f$.
\end{proof}
Note that $F$ is called the $\emph{anti-derivative}$ of $f$. Also the antiderivative is not unique.
For if $F$ is the anti-derivative then so is $F + c$.
\begin{Corollary}
    If the real valued function $F$ an open interval $\mathcal{U}$ in $\mathbb{R}$ has the 
    continuous derivative $f$ and $a,b \in \mathcal{U} $ then
    \[ \int_a^b f = F(b) - F(a) .\]
\end{Corollary}
\begin{proof}
    $f(x) = \frac{d}{dx}(F(x)) = \frac{d}{dx}( \int_a^x f)$. Since the derivatives of $F(x)$ and
    $\int_a^b f$ are equal the functions must differ by a constant i.e 
    \[ F(x) = \int_a^x f + c ,\] for all $x \in \mathcal{U}$. Now when $x = a$ we get $F(a) = 0 +
    c$. Thus when $x = b$ we have $F(b) = \int_a^b f + F(a)$ and so we get the proof. 
\end{proof}
\begin{Corollary}[name=Change of variable Theorem]
Let $\mathcal{U},\mathcal{V}$ be open intervals in $\mathbb{R}$ and 
$\mathcal{U} \stackrel{\phi}{\longrightarrow} \mathcal{V}$ such that $\phi$ is a differentiable
function with continuous derivative. Given a continuous function $f : \mathcal{V} \to \mathbb{R}$, 
for any $a,b \in \mathcal{U}$
\[ \int_{\phi(a)}^{\phi(b)} f(v)dv = \int_a^b f(\phi(u))\phi^{\prime}(u)du .\] 
\end{Corollary}
\begin{proof}
    

    We'll complete the proof in 2 steps.

    \textbf{Step 1} : Find the anti-derivative of $f$ on $\mathcal{V}$.
    Fix a $v = \phi(a)$ for $a \in \mathcal{U}$. Let $F : \mathcal{V} \to \mathbb{R}$ be the
    function defined by $F(y) = \int_{\phi(a)}^y f(v)dv$. Since $f$ is continuous in $\mathcal{V}$,
    $F$ is defined and by the fundamental Theorem of calculus $F' = f$. 

    \textbf{Step 2} : Construct a composite function $G = F \circ \phi$ such that $G$ is an
    antiderivative of $f$ in $\mathcal{U}$.
    Let $G : \mathcal{U} \to \mathbb{R}$ be defined by $\int_{\phi(a)}^{\phi(x)} f(v)dv$. Then $G =
    F \circ \phi$ and is the composite of two differentiable functions and hence by chain rule 
    $G'(x) = F'(\phi(x))\phi^{\prime}(x) = f(\phi(x))\phi^{\prime}(x)$ . 
    Hence by the fundamental Theorem of calculus $G(x) = \int_a^x f(\phi(x))\phi^{\prime}(x) + c$. 
    Setting $x = a$ we get $c = 0$ and setting $ x = b$ we get the proof.

\end{proof}

