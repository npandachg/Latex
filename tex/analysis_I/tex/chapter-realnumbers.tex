\chapter{Real Numbers}


\begin{Theorem}[name=Existence of irrational numbers]
	$\sqrt{2}$ is irrational.
\end{Theorem}
\begin{proof}
    
    Assume $\sqrt{2} \in \mathbb{Q}$. Then, there exists $p$, $q$ $\in
    \mathbb{Z}$ such that $ \frac{p^2}{q^2} = 2 $ and $q \neq 0$. 
    Without loss of generality, assume $p,q$ are relatively prime.
    Hence, $p^2 = 2q^2$. Therefore, $p^2$ and hence $p$ is even.
    Let $p = 2r$ for some $r \in \mathbb{Z}$. Hence, $2q^2 = 4r^2$ which
    implies $q$ is even i.e $q = 2k$ for some $k \in \mathbb{Z}$. But, this
    means that $p,q$ have $2$ in common and contradicts our assumption of
    them being relatively prime. Hence, $\sqrt{2}$ is irrational.
\end{proof}

Hence, we are faced with two important questions.
\begin{enumerate}
	\item How do we fill holes?
	\item How do we know we have filled all the holes?
\end{enumerate}
The answers to these is the construction of the \textbf{real} numbers set
$\mathbb{R}$. We give an axiomatic definition of the real numbers as an ordered field having a completeness
property. In the Appendix, we construct the real numbers first from the rational numbers and also as a limit
of cauchy sequences and show that both these constructions are isomorphic.

First, we define what a field means.

\begin{Definition}[name=Field]
	A field is a set F together with an ordered pair of functions 
	$ + , * : F \times F \rightarrow F$ such that 
\begin{enumerate}
    \item $ + , *$ are Commutative 
    \item $ + , *$ are Associative
    \item $ * $ is Distributive over $ + $
    \item There are distinct elements $0_F$ and $1_F$ s.t $\forall$ 
	$a \in F$, $ a + 0_F = a$ and $ a * 1_F = a$
    \item There exists Additive and multiplicative inverse $ \forall a \in
	F$ denoted by $-a$ and $a^{-1}$ respectively such that 
	$ a + (-a) = 0_F $ and $ a * a^{-1} = 1_F $.
    \item $1_{F} \neq 0_{F}$.
\end{enumerate}
\end{Definition}
What happens in a ring when we multiply element in $G$ with $0$ and with additive inverses i.e.~negatives.
\begin{Proposition}
    Let $\F$ be a field and let
    $a,b$ be any element in $F$. Then,
    \begin{enumerate}
	\item
	    $a0 = 0a = 0$,
	\item
	    $a(-b) = -(ab) = -(a)b$, 
	\item
	    $(-a)(-b) = ab$.
    \end{enumerate}
\end{Proposition}
\begin{proof}
    We prove in order.
    \begin{enumerate}
	\item
	    First, observe that
	    \[0 + a0 = a0 = a(0 + 0) = a0 + a0.\]
	    Thus, from cancellation law in $0 + a0 = a0 + a0$
	    \[0 = a0.\]
	\item
	    This follows from,
	    \[a(-b) + ab = a(-b + b) = a0 = 0.\]

	\item
	    Using the above twice,
	    \[(-a)(-b) = -(a)\left[-b\right] = -\left[-(ab)\right].\]
	    Since $(-(-a)) = a$ we get the result.
    \end{enumerate}
\end{proof}
Next are the order axioms. 
\begin{Definition}
	A field $F$ is ordered  
	(total order) if there exists a set $F_{+} 
	\subset F$ s.t 
	\begin{enumerate}
		\item if $a,b$ $\in F_{+}$ then $a + b \in F_{+}$ and $a*b \in
			F_{+}$.
		\item for any $a \in F$, one and only one of the following 
		statements is true $ a \in F_{+} $ or $ a = 0_F $ or $ -a \in
		F_{+}$. 	
	\end{enumerate}
\end{Definition}
With these we can define $ > $ and $ < $ that mean greater than and less than
respectively. For example, if $a,b \in \F$ then $ a > b$ means that $a - b \in \F_{+}$. 
Thus $a \in \F_{+}$ if and only if $a > 0$.
Some consequences of the order property.
\begin{Corollary}[name=Trichotomy] 
	\begin{displaymath}
		\forall a,b \ \in F \ \text{exactly one is true } :
	       a > b \ or \ a < b \ or \ a = b.	
	\end{displaymath}
\end{Corollary}
\begin{proof}
	Note that $a > b$ means $ a - b \in F_{+}$. Similarly, $a < b$ means
	$b - a \in F_{+}$ whereas, $a = b$ means $a-b = 0_F$. Let, $c = a - b$.
	Clearly $c \in F$. Now, since $F$ is ordered we know from prop 2 of the
	order axioms that either $c \in F_{+}$ or $c = 0_F$ or $ -c \in F_{+}$.
	Hence, this proves the trichotomy corollary.
\end{proof}
\begin{Corollary}[name = Transitivity]
	if $ a > b$ and $ b > c $ then $ a > c$ $\forall a,b,c$ $\in F$
\end{Corollary}
\begin{Corollary}
	if $ a > b$ and $ c \geq d$ then $ a + c > b + d$. Additionally,
	if $b,d \in F_{+}$ then $ a*c > b*d$.
\end{Corollary}
\begin{Corollary}
    $\forall$ $a \in F$ if $a \neq 0_{\F}$ then $a^2 \in F_{+}$. 
\end{Corollary}
\begin{Corollary}
    $1_{F} > 0_{F}$.
\end{Corollary}
\begin{proof}
    We need to show that $1_{\F} \in \F_{+}$. Since $1_{\F}\neq 0$, let us suppose that $-1\F \in \F_{+}$.
    Then $(-1_{\F})(-1_{\F}) = 1_{\F} \in \F_{+}$. This is a contradiction. Thus, $1_{\F}$ is in $\F_{+}$.
\end{proof}
An ordered field comes equipped with an \emph{absolute} value function.
\begin{Definition}
    Let $\F$ be an ordered field. For any $a\in\F$ we define,
    \begin{equation*}
	\abs{a} = 
	\begin{cases}
	    a &\text{if $a \geq 0$}\\
	    -a &\text{if $a \leq 0$}
	\end{cases}
    \end{equation*}
    as the absolute value of $a$.
\end{Definition}
\begin{Proposition}
    We gather some important properties of the absolute value function.
    \begin{enumerate}
	\item
	    Let $\epsilon > 0$. Then $\abs{a} \leq \epsilon$ if and only if $-\epsilon \leq a \leq \epsilon$.
	\item
	    For any $a\in \F$, $a \leq \abs{a}$.
	\item
	    For any $a,b \in \F$, $\abs{ab} = \abs{a}\abs{b}$.
	\item
	    For any $a,b \in \F$, $\abs{a + b} \leq \abs{a} + \abs{b}$.
    \end{enumerate}
\end{Proposition}
\begin{proof}
    We prove in order.
    \begin{enumerate}
	\item
	    Let $\abs{a} < \epsilon$. If $a > 0$, then $a < \epsilon$. Note that $-\epsilon < a$. If $a < 0$,
	    then $-a < \epsilon$ and thus $a > -\epsilon$. Note that $a < \epsilon$. If $a = 0$ then $0 <
	    \epsilon$. Thus, $-\epsilon < a < \epsilon$. If $\abs{a} = \epsilon$, then either $a = \epsilon$
	    or $-a = \epsilon$. 

	    If $-\epsilon < a < \epsilon$, then $a < \epsilon$ and $ -a < \epsilon$. Thus, $ \abs{a} <
	    \epsilon$.
	\item
	    If $a > 0$, then $ a = a = \abs{a}$. If $a < 0$, then $a < -a = \abs{a}$. If $a = 0$, then
	    $\abs{a} = 0$.
	\item
	    There are four cases to consider. $a < 0, b < 0$, $a < 0 , b \geq 0$, $a \geq 0,b < 0$ and 
	    $a \geq 0, b\geq 0$. In each case we can check that $\abs{ab} = \abs{a}\abs{b}$.
	\item
	    If $a + b > 0$, then $\abs{a + b} = a + b$. But $a \leq \abs{a}$ and $b\leq \abs{b}$. Thus $a + b
	    \leq \abs{a} + \abs{b}$. Similarly, for the other case.
    \end{enumerate}
\end{proof}
An ordered field cannot contain finite elements.
To prove this we'll use the following Theorem:
\begin{Theorem}[name=Ordered fields contain Integers]\label{th:orf_1to1}
    For any ordered field F there exists a unique function $\varphi : \mathbb{Z} \rightarrow F$
    such that: 
\begin{enumerate}
	\item $\varphi$ is injective.
	\item $\varphi(m+n) = \varphi(m) \ + \ \varphi(n)$ and
	    $\varphi(m*n) = \varphi(m) * \varphi(n)$ $\forall m,n \in \mathbb{Z}$ 
\end{enumerate}
\end{Theorem}
\begin{proof}
     
    Let $\varphi : \mathbb{Z} \rightarrow F$ be such that:
\begin{displaymath}
    \begin{aligned}
	&\varphi(0) = 0_F. \\
	&\forall n > 0 \ \varphi(n) = 1_F + \cdots + 1_F \ (n \ times) \text{and} \\
	&\varphi(-n) = -\varphi(n).
    \end{aligned}
\end{displaymath}
By way of construction $\varphi$ 
satisfies the 2 \textit{homomorphic} properties and hence we only need to show 
its injectiveness. Consider, $m,n \in \mathbb{Z}$. WLOG, let $m > n$. Then $m-n > 0$ and from our 
formula $\varphi(m-n) = 1_F + \cdots + 1_F \ (m-n) \text{times} $. Thus $\varphi(m-n) \in F_{+}$. 
But, $\varphi(m-n) = \varphi(m) + -\varphi(n)$. Hence, $\varphi(m) - \varphi(n) \in F_{+}$. Thus for
$m \neq n$ we have $\varphi(m) \neq \varphi(n)$. 
\end{proof}

\begin{Theorem}[name=Finite fields cannot be ordered]
    No finite field can be ordered.
\end{Theorem}
\begin{proof}
    Using the earlier Theorem we can see that a finite field cannot be ordered (no injective function
    from $\mathbb{Z}$ to a finite field $F$ can exist).  
\end{proof}

This shows that there can be lots of ordered fields. To completely characterize the real numbers $\mathbb{R}$,
i.e, to completely fill all the holes in $\mathbb{Q}$ we need an additional property. This property is
called the \textit{Least Upper Bound} or L.U.B property. 

\begin{Definition}
    In an ordered field $\F$ an element $a \in \F$ is an upper bound of a subset $S \subset F$ if
    $ \forall s \in S$ $ a \geq s$.
\end{Definition}
\begin{Definition}
    $a$ is the \textit{least upper bound} called l.u.b $S$ or sup $S$ if $\forall b \in F$ such that
    $b$ is an upper bound of $S$ then $ a \leq b$.  
\end{Definition}
In a similar way we have lower bounds and the greatest lower bound g.l.b $S$ or inf $S$. 
For a given subset $S \subset F$, the sup $S$ (or inf $S$) may or may not be in S. It 
also may or may not belong to F. For example let us take $ F = \mathbb{Q} $. 
\begin{itemize}
    \item $ S = \mathbb{Z} $. No sup $S$ exist since $\Z$ is not bounded from above.
    \item $ S = \left \lbrace q \in \mathbb{Q} : q < 0 \right \rbrace$. $\sup{S} = 0$.
    \item $ S = \mathbb{Q} - \mathbb{Q}_+ $. $\sup{S}  = 0$.
    \item $ S = \left \lbrace q \in \mathbb{Q} : q^2 < 2 \right \rbrace $. What is $\sup{S}$? We will 
	show that $\sup{S}$ does not exist in $\F$. 
\end{itemize}
Next, we'll look at the L.U.B property. 
\begin{Definition}
    An ordered field $F$ has the L.U.B property if every non-empty subset $ S \subset F$ that is bounded
    from above (i.e has an upper bound) has the least upper bound.  
\end{Definition}

From the above examples, it is easy to observe that $\mathbb{Q}$ doesn't have the L.U.B property.
We now have the axioms to construct a Field that will fill all the \textit{holes} in $\mathbb{Q}$.
Our goal is to show an existence of a unique ordered field with the L.U.B property. 
\begin{Theorem}[name=Existence of $\mathbb{R}$]
    There is a unique ordered field with the L.U.B property called the real numbers, denoted by $\R$, and
    containing the rational numbers $\Q$.
\end{Theorem} 
We will not prove this theorem here. Instead we assume its existence. Thus, the real numbers $\R$ is an
ordered field having the L.U.B property.

\begin{Corollary}
    If $S \subset \R$ has a least upper bound then it is unique. 
\end{Corollary}
\begin{proof}
    Suppose $a,b$ are the least upper bounds of $S$. Then $b$ is an upper bound of $S$.
    Since, $a$ is sup $S$, it means that $ a \leq b$. Similary $a$ is an upper bound of $S$.
    Because $b$ is also the sup $S$ we get $ b \leq a $. Therefore, $a = b$.
\end{proof}

\begin{Corollary}
    $\R$ has the G.L.B property, i.e.~for any nonempty subset $S$ of $\R$ that is bounded from below, the
    greatest lower bound of $S$ exists in $\R$.	
\end{Corollary}

\begin{proof}
    Let $L = \set{x \in \R}{\forEv{s\in S} x \leq s}$. Thus $L$ is the set of all lower bounds for $S$. Since,
    $S$ is bounded from below, $L$ is not empty. Moreover, $L \subset \R$ and hence by the L.U.B property,
    $\sup{S}$ exists in $\R$. Let $b = \sup{L}$. We will show that $b$ is also $\inf{S}$. For any $l \in L$,
    $l \leq b$. If there was an $s \in S$ such that $s < b$, then $s+\frac{b-s}{2} < b$ will be any upper
    bound for $L$ which is not possible. Hence, $b$ is a lower bound for $S$. If $x$ is any other lower bound
    for $S$, then $x \in L$ and hence $x \leq \sup{L} = b$. Thus, $b$ is $\inf{S}$. 
\end{proof}
The L.U.B property has some very important consequences.
\begin{Proposition}
    Let $\F$ be an ordered field satisfying the L.U.B property. Then,
    \begin{enumerate}
	\item
	    For any $x$ in $\F$, there is an integer $n$ such that $x < n$. This is called the Archemedian
	    property.
	\item
	    For any $\epsilon > 0$ in $\F$ no matter how small, there is a positive integer $n$ such 
	    that $\frac{1}{n} < \epsilon$.
	\item
	    For any $x\in \F$, there is an integer $n$ such that $n\leq x < n+1$. Such an integer $n$ is
	    called the greatest integer of $x$.
	\item
	    For any $x \in \F$ and any positive integer $N$, there is an integer $n$ such that 
	    \[\frac{n}{N} \leq x < \frac{n+1}{N}.\]
	\item
	    For any $x \in \F$ and any $\epsilon > 0 \in \F$, there is a rational number $p$ such that
	    \[0 \leq x - p < \epsilon.\]
	\item
	    For any $x,y \in \F$, there is a rational number \textbf{in between} $x$ and $y$.
    \end{enumerate}
\end{Proposition}
\begin{proof}
    We prove in order. 
    \begin{enumerate}
	\item
	    Assume no such $n$ exists. Thus for any integer $n$ we must have that $n \leq x$. Thus $x$ is an
	    upper bound for the set of integers $\Z$. Since, we saw that $\Z$ is contained in an ordered
	    field, $\F$ satisfies the L.U.B property, there is a $b \in \F$ such that $b = \sup{S}$. For any
	    $n \in \Z$, $n+1$ is also in $\Z$, this means that $b \geq n+1$ and so $b - 1 \geq n$. Hence,
	    $b-1$ is also an upper bound for $\Z$. But $b -1 < b$ and so this gives us a contradiction.
	\item
	    By the Archemedian property, for $x = \frac{1}{\epsilon}$, there is an integer $n$ such that $x <
	    n$. Thus $\frac{1}{n} < \epsilon$.
	\item
	    There is an integer $m$ such that $\abs{x} < m$, which means that $-m < x < m$. Pick an integer
	    $n$ such that $n \leq x$. Thus $x < n+1 \leq m$. 
	\item
	    Using the above result on $Nx$ we get an integer $n$ such that $n\leq Nx < n+1$. Hence, dividing
	    by $N$, we get the result.
	\item
	    Using the above result, there is an $n$ such that $ \frac{n}{N} \leq x < \frac{n+1}{N}$ for any
	    positive integer $N$. Thus, $0 \leq x - \frac{n}{N} < \frac{1}{N}$. Pick $N$ such that
	    $\frac{1}{N} < \epsilon$. Since $n,N$ are in $\Z$, $\frac{n}{N} = p \in \Q$. Thus we found a
	    rational number $p$ such that,
	    \[0 \leq x - p < \epsilon.\]
	\item
	    For any positive integer,
	    \[\frac{n}{N} \leq x < \frac{n+1}{N}.\]
	    Thus,
	    \[\frac{n}{N} \leq x < \frac{n}{N} + \frac{1}{N} \leq x + \frac{1}{N}.\]
	    Assume $y > x$. Thus $y-x > 0$. Pick $N$ such that $\frac{1}{N} < y-x$. Hence,
	    \[x < \frac{n+1}{N} < x + y-x = y.\]
	    Take $p = \frac{n+1}{N}$. Hence, there is a $p \in \Q$ such that $x < p < y$.
    \end{enumerate}
\end{proof}
We can now show that L.U.B property is not shared by the set of rational numbers.
Consider the set  $S = \set{q \in \mathbb{Q}}{q^2 < 2}$. Note that $S \subset \Q \subset \R$. We will show
that no matter what $q$ we take in $S$ we will find a $p > q$ such that $p \in S$. $S$ is not empty and is
bounded above by $2$ which is a rational number. Similarly, we will show that if we take a  $q \in \Q - S$, we
can find a $p < q$ such that $p \in \Q - S$. Thus, the rational numbers don't have the L.U.B property. 

First let us pick a rational number $q \in S$. Let $n$ be an integer such that,
\[\frac{1}{n} < \max{(\frac{2-q^2}{1+2q},1)}.\]
Let $p = q + \frac{1}{n}$. Then, $q < p$ and,
\begin{align*}
    p^2 &= {(q + \frac{1}{n})}^2 \\
    & = q^2 + \frac{1}{n^2} + \frac{2q}{n}\\
    & < q^2 + \frac{1}{n} + \frac{2q}{n} \\
    &\quad = q^2 + \frac{1}{n}(1 + 2q)\\ 
    &\quad < q^2 + 2 - q^2\\
    &\quad\quad = 2.
\end{align*}
We will now show that $\sup{S} = 2$, when we regard $S \subset \R$. Infact we will show a much general
statement. We will show the existence of $n^{th}$ roots of non-negative real numbers. The idea is that the
$n^{th}$ root a non-negative real number $a$ is the supremum of all such real numbers $b$ such that 
$b^n \leq a$.
\begin{Theorem}
    If $a$ is a non-negative real number and $n$ is any positive integer, then there exists a number $b \geq
    0$ such that $b^n = a$.
\end{Theorem}
\begin{proof}
    First note that the set $A = \set{x \geq 0 \in \R}{x^n \leq a}$ is not empty since $0$ is in the set.
    Moreover, the set is bounded by $a+1$ because ${(a+1)}^n \geq 1 + na > a$. Hence supremum of the set exists
    in $\R$. Let $b$ be the supremum of this set. 
    Assume $b^n < a$. We will show that a $\delta > 0$ exists such that ${(b + \delta)}^n < a$ and hence $b + 
    \delta$ must be an element of $A$. But since $b + \delta > b$ and $b$ is the supremum of $A$ we get a
    contradiction.
    How do we choose a suitable $\delta$. As we did for the case above, let $\delta = \frac{1}{m}$
    We know that,
    \begin{align*}
	{(b + \frac{1}{m})}^n &= \series{\binom{n}{k}b^{k}{\frac{1}{m}}^{(n-k)}}{k}{0}{n},\\
	&= b^n + \series{\binom{n}{k}b^{k}{\frac{1}{m}}^{(n-k)}}{k}{0}{n-1}.\\
    \end{align*}
    The sum $\series{\binom{n}{k}b^{k}{\frac{1}{m}}^{(n-k)}}{k}{0}{n-1}$ is a real number which can be made
    smaller than $\frac{(a - b^n)}{n}$ by a suitble choice of $m$. That is for each $k$ we select an $m_k$
    such that $\binom{n}{k}b^k {\frac{1}{m}}^{(n-k)} < \frac{(a - b^n)}{n}$ and take $\frac{1}{m} =
    \min\set{m_k}{1\leq k \leq n}$.
    Thus,
    ${(b + \frac{1}{m})}^n < b^n + (a - b^n) = a$. Hence, we get the contradiction. Similarly, we can show
    that $b^n \not > a$.
\end{proof}
We can extend this to arbitrary real numbers for the case when $n$ is an odd positive integer.
\begin{Corollary}
    If $a$ is a real number and $n$ is an odd positive integer, then there exists a real number $b$ such that 
    $b^n = a$.
\end{Corollary}
\begin{proof}
    Use the theorem on $\abs{a}$ to get a $c \geq 0$ such that $c^n = a$. If $a \geq 0$, then $b = c$ else 
    $b = -c$.
\end{proof}
\begin{Definition}[name=root]
    Let $x$ be a non-negative real number, and let $n$ be a positive integer. We define $x^{\frac{1}{n}}$ to
    the be the non-negative real number $y$ such that $y^n = x$ and we say that the $n^{th}$ root of $x$ is
    $y$. If $x$ is an arbitrary real number and $n$ is an odd positive integer, we define $x^{\frac{1}{n}}$ to
    be the real number $y$ called the $n^{th}$ root of $x$, such that $y^n = x$.
\end{Definition}
If $x$ is a real number and $n$ is a positive integer, we define
$x^{-\frac{1}{n}} = \frac{1}{x^{\frac{1}{n}}}$ whenever $x^{\frac{1}{n}}$ is defined and not $0$.
In this way for any $p \in \Q$, where $p = \frac{m}{n}$ in its lowest term we define,
\[x^{p} = {(x^{\frac{1}{n}})}^{m},\]
whenever $x^{\frac{1}{n}}$ is defined. Easy to check, that is definition obeys all the usual laws of
exponents. When $p \in \R$ but $p \not\in Q$, we do not yet have a notion of exponentiation. We will define it
when we have understood limits.


Now that we have filled all the holes in $\mathbb{Q}$, an appropriate question to ask would be
\emph{how many holes did we fill}. Are there more holes than the $\mathbb{Q}$? We'll explore this
when we define what is means to count infinite sets. To close this section we will describe the Complex
numbers.
\begin{Definition}[name=Complex numbers]
    A complex number is an ordered pair $(a,b)$ of Real numbers. The set of Complex numbers is denoted by
    $\C$.
\end{Definition}
Let $z_1,z_2$ be complex numbers $(a,b)$ and $(c,d)$ respectively. We define the binary operation $+$
as:
\[z_1 + z_2 := (a+c,b+d).\]
The zero complex number $0_{\C} : (0,0)$. We usually omit $\C$ while denoting the zero complex number.
We define the binary operation $*$ as:
\[z_1 * z_2 = (ac - bd, ad + bc).\]
We define the $(1,0)$ as the one of Complex number.
\begin{Proposition}
    $\left(\C,+,(0,0),*,(1,0)\right)$ is a field.
\end{Proposition}
There is a natural injection from $\R$ into $\C$ which maps $x \in \R$ to $(x,0) \in \C$. Infact, such an
injection is also a homomorphism since $x + y$ maps to $(x+y,0) = (x,0) + (y,0)$ and similary $x*y$ maps to
$(x*y,0) = (x,0)*(y,0)$. Thus we denote the complex number  $(x,0)$ as $x$, i.e we say $(x,0) = x$.
\begin{Definition}
    The complex number $(0,1)$ is denoted by $i$.
\end{Definition}
\begin{Theorem}
    $i^2 = -1$.
\end{Theorem}
\begin{proof}
    $i^2 = (0,1)*(0,1) = (-1,0) = -1$.
\end{proof}
From now on, we will omit the $*$ to denote multiplication and just write $z_1z_2$ to denote $z_1 * z_2$ for
any complex number $z_1,z_2$.
\begin{Theorem}
    Complex numbers cannot be ordered. 
\end{Theorem}
\begin{proof}
    Assume the Complex number are ordered. Then for any non-zero complex number $a$, $a^2$ must be a
    \emph{positive} complex number, where if $z_1,z_2$ are positive then $z_1z_2$ and $z_1 + z_2$ are positive
    and for any $z\in \C$, either $z = (0,0)$, or $z \in \C$ or $-z \in \C$. Since real numbers are embedded
    in $\C$, it is easy to see the $(-1,0)$ is not positive. However, $(-1,0) = i^2$ and hence it must be
    positve. Thus, we get a contradiction.
\end{proof}

\begin{Theorem}
    Any complex number $(a,b)$ is equal to $a + ib$.
\end{Theorem}
\begin{proof}
    $a + ib = (a,0) + (0,1)(b,0) = (a,0) + (0,b) = (a,b)$.
\end{proof}
The above theorem enables us to write a complex number $z = (a,b)$ as $a + ib$.
\begin{Definition}
    If $a,b$ are real numbers and $z = a + ib$, then the complex number $\bar{z} := a - ib$ is called the
    conjugate of $z$. The numbers $a,b$ are called the real and imaginary parts of $z$ respectively and we
    denote them by $a = \Re(z)$ and $b = \Im(z)$.
\end{Definition}
\begin{Theorem}
    If $z,w$ are complex numbers, then
    \begin{enumerate}
	\item
	    $\overline{z + w} = \overline{z} + \overline{w}$.
	\item
	    $\overline{zw} = \overline{z}\overline{w}$.
	\item
	    $z + \overline{z} = 2\Re{(z)}$, $z - \overline{z} = 2\Im{(z)}$. 
	\item
	    $z\overline{z}$ is real and positive (except when $z = 0$).
    \end{enumerate}
\end{Theorem}
\begin{Definition}
    If $z$ is a complex number, then its absolute value, $\abs{z}$ is the non-negative square root of
    $z\overline{z}$; that is $\abs{z} = {(z\overline{z})}^{\frac{1}{2}}$.
\end{Definition}
Note that, $z = a + ib$, then ${\abs{z}}^{2} = (a + ib)(a- ib) = a^2 + b^2$. The absolute value function has similar
properties for the case when $z$ is real.
\begin{Theorem}
    Let $z,w$ be complex numbers. Then 
    \begin{enumerate}
	\item
	    $\abs{z} > 0$, unless $z = 0$, in which case $\abs{z} = 0$.
	\item
	    $\abs{\overline{z}} = \abs{z}$.
	\item
	    $\abs{zw} = \abs{z}\abs{w}$.
	\item
	    $\abs{\Re(z)} \leq \abs{z}$.
	\item
	    $\abs{z + w} \leq \abs{z} + \abs{w}$.
    \end{enumerate}
\end{Theorem}
\begin{proof}
    We prove only $(5)$. Note that $\overline{z\overline{w}} = \overline{z}w$.
    \begin{align*}
	{\abs{z + w}}^{2} &= (z + w)\overline{(z + w)} \\
	&= (z + w)(\overline{z} + \overline{w}) \\
	&= z\overline{z} + w\overline{w} + w\overline{z} + z\overline{w}\\
	&= {\abs{z}}^{2} + {\abs{w}}^{2} + \overline{z\overline{w}} + z\overline{w}  \\
	&= {\abs{z}}^{2} + {\abs{w}}^{2} + 2\Re(z\overline{w})\\
	&\leq {\abs{z}}^{2} + {\abs{w}}^{2} + 2\abs{z\overline{w}}\\
	&\quad = {\abs{z}}^{2} + {\abs{w}}^{2} + 2\abs{z}\abs{\overline{w}}\\
	&\quad = {\abs{z}}^{2} + {\abs{w}}^{2} + 2\abs{z}\abs{w}\\
	&\quad = {(\abs{z} + \abs{w})}^{2}.
    \end{align*}
\end{proof}
\begin{Definition}
    If $\listV{z}{n}$ are complex numbers, we write:
    \[z_1 + z_2 + \cdots + z_n = \finiteSum{z}{j}{n}.\]
\end{Definition}
The next theorem is extremely important and is known as the Schwarz inequality.
\begin{Theorem}
    If $\listV{z}{n}$ and $\listV{w}{n}$ are complex numbers, then,
    \[{\abs{\finiteSum{z_j\overline{w_j}}{j}{n}}}^{2} \leq
	\finiteSum{{\abs{z_j}}^{2}}{j}{n}\finiteSum{{\abs{w_j}}^{2}}{j}{n}.\]
\end{Theorem}
\begin{proof}
    Let $A = \finiteSum{{\abs{z_j}}^{2}}{j}{n}$, $B = \finiteSum{{\abs{w_j}}^{2}}{j}{n}$ and 
    $C = \finiteSum{z_j\overline{w_j}}{j}{n}$. Assume $B \neq 0$. Note that $\overline{B} = B$ and
    $\overline{A} = A$.
    Then,
    \begin{align*}
	\finiteSum{{\abs{Bz_j - Cw_j}}^{2}}{j}{n} &= \finiteSum{(Bz_j - Cw_j)(B\overline{z_j} -
	    \overline{C}\overline{w_j})}{j}{n}\\
	&= B^2\finiteSum{{\abs{z_j}}^{2}}{j}{n} - B\overline{C}\finiteSum{z_j\overline{w_j}}{j}{n} - 
	    BC\finiteSum{\overline{z_j}w_j}{j}{n} + {\abs{C}}^2 \finiteSum{{\abs{w_j}}^{2}}{j}{n}\\
	&= B^2 - B{\abs{C}}^{2}\\
	&= B(AB - {\abs{C}}^{2}).
    \end{align*}
    Since the LHS is greater than or equal to $0$, we must have $B(AB - {\abs{C}}^{2}) \geq 0$. Since we
    assume $B \neq 0$, we must have $AB \geq {\abs{C}}^{2}$ which proves the theorem.
    Note that, if $B = 0$, then the result is trivial.
\end{proof}
