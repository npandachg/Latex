\chapter{Metric Spaces}
% put smiley
Before beginning our study of limits, continuity and differentiation of functions it is important
to understand the domains and ranges of functions and notions like distance between elements
in a set. This notion of studying the \emph{closeness} of elements in a set is undertaken in the 
field of \emph{topology}. 
We will not study the ideas of topology in anywhere near their full generality. 
Instead, we will study one particular way to define a notion of closeness by defining an explicit 
distance among the elements of our set. The elements of such a set are usually called \emph{points}
and the set with the associated distance function is called a metric space.

We have seen functions like these :
\begin{displaymath}
    \begin{aligned}
	& f : \mathbb{R} \rightarrow \mathbb{R} \\
	& f : plane \rightarrow \mathbb{R} \\
	& f : space \rightarrow \mathbb{R}
    \end{aligned}
\end{displaymath}
Also we have seen \emph{curves} which are functions $f : \mathbb{R} \rightarrow \text{plane or
space} $. In general we can define a \emph{n dimensional space} 
 of reals to be 
\begin{Definition}
    \begin{displaymath}
	\mathbb{R}^n := \left.\lbrace (x_1, x_2, \dots, x_n) : x_i \in \mathbb{R} \forall i =
	    1\dots n \rbrace.\right.
    \end{displaymath}
\end{Definition}  
We define a \emph{scalar} multiple of $\mathbf{x} \in \mathbb{R}^n$ to be 
\begin{Definition}
    For any $\alpha \in \mathbb{R}$,
    \begin{displaymath}
	\alpha\mathbf{x} : (\alpha*x_1, \alpha*x_2, \dots, \alpha*x_n)
    \end{displaymath}
\end{Definition}
This enables us to talk about functions $f :  \mathbb{R}^n \rightarrow  \mathbb{R}^m $. Also as
a notational candy we'll refer to elements of $ \mathbb{R}^n$ with boldface letters. We would also 
need to calculate the \emph{distance} between two points of  $\mathbb{R}^n$. For any two elements
$ \mathbf{x},\mathbf{y} \in \mathbb{R}^n$ we know about the \emph{dot product} defined as :
\begin{Definition}
    \begin{displaymath}
	\text{dot}: \mathbf{x} . \mathbf{y} := \sum_{i=1}^{n} x_i*y_i
    \end{displaymath}
\end{Definition}
With this we can define the \emph{length} of $\mathbf{x}$ to be the norm as follows :
\begin{Definition}
    \begin{displaymath}
	\text{norm}: \lvert \lvert \mathbf{x} \rvert \rvert := \sqrt{\mathbf{x} . \mathbf{x}}
    \end{displaymath}
\end{Definition}
Then the distance between two points $\mathbf{x},\mathbf{y} \in \mathbb{R}^n$ is 
$\lvert \lvert \mathbf{x} -\mathbf{y} \rvert \rvert$.
Even though we haven't defined the meaning of $\sqrt{}$, we can take a leap of faith and and just
assume it has the same properties that we know of. Note that for $\mathbb{R}$, we can define
$\lvert \lvert x \rvert \rvert$ to be just $\lvert x \rvert$. In more general sets, we'll need a 
more general notion of distance. Knowing the properties of norm in $\mathbb{R}^n $ for $ n \leq 3$
we can define a distance function in an abstract set as follows :
\begin{Definition}
    A metric on a set $E$ is the function $d : E \times E \rightarrow \mathbb{R}$ such that $\forall
    p,q,r \in E$
    \begin{enumerate}
	\item $d(p,q) \geq 0$.
	\item $d(p,q) = 0 \Leftrightarrow p = q$.
	\item $d(p,q) = d(q,p)$. 
	\item $d(p,r) \leq d(p,q) + d(q,r)$.
    \end{enumerate}
\end{Definition}
Now we can define a \emph{metric space}. 
\begin{Definition}
    A metric space is a set $E$ with a metric $d$ defined on $E$.
\end{Definition}
Euclidean space is $\mathbb{R}^n$ with the euclidean distance. It is denoted by 
$E^n :=$ $\left(\mathbb{R}^n, d_{\lvert \lvert . \rvert \rvert} \right)$.

\begin{Theorem}[name=Euclidean metric space]\label{th:euclidean_space}
$E^n$ is a metric space. 
\end{Theorem}
In order to prove this we'll be needing the following Lemma.
\begin{Lemma}[name=Cauchy Schwarz]{\emph{Cauchy Schwarz :}}
    \label{th:cauchy_schwarz}
    $\lvert \mathbf{x} . \mathbf{y} \rvert \leq \lvert \lvert \mathbf{x} \rvert \rvert * 
    \lvert \lvert \mathbf{y} \rvert \rvert$ $\forall \mathbf{x,y} \in \mathbb{R}^n$.
\end{Lemma}
\begin{proof}
    $\forall \alpha,\beta \in \mathbb{R}$ we have the following :
    \begin{displaymath}
	\begin{aligned}
	    0 \leq  &\lvert \lvert \alpha\mathbf{x} - \beta\mathbf{y} \rvert \rvert \\
	    & = \left(\alpha\mathbf{x} - \beta\mathbf{y}\right). 
	    \left(\alpha\mathbf{x} - \beta\mathbf{y}\right) \\
	    & = \alpha^2 \mathbf{x}.\mathbf{x} + \beta^2 \mathbf{y}.\mathbf{y} -
	    2*\alpha*\beta\mathbf{x}.\mathbf{y} \\
	    & = \alpha^2 \lvert\lvert \mathbf{x} \rvert\rvert ^2 + 
	    \beta^2 \lvert\lvert \mathbf{y} \rvert\rvert ^2 - 2*\alpha *\beta\mathbf{x}.\mathbf{y} \\
	    2*\alpha*\beta\mathbf{x}.\mathbf{y} \leq & \ \alpha^2 \lvert\lvert \mathbf{x}
	    \rvert\rvert ^2 + \beta^2\lvert\lvert \mathbf{y} \rvert\rvert ^2
	\end{aligned}
    \end{displaymath}	
    Now choose $\alpha = \lvert\lvert \mathbf{y} \rvert\rvert$ and $\beta = \pm 
    \lvert\lvert \mathbf{x} \rvert\rvert$. Hence proved.
\end{proof}
\begin{Corollary}
    $\lvert\lvert \mathbf{x} + \mathbf{y} \rvert\rvert \leq \lvert\lvert \mathbf{x} \rvert\rvert +
    \lvert\lvert \mathbf{y}\rvert\rvert$
\end{Corollary}
\begin{proof}
    \begin{displaymath}
	\begin{aligned}
	    \lvert\lvert \mathbf{x} + \mathbf{y}\rvert\rvert ^2  = & \left(\mathbf{x} +
		\mathbf{y}\right) .  \left(\mathbf{x} + \mathbf{y}\right) \\
            = & \lvert\lvert \mathbf{x} \rvert\rvert ^2 + \lvert\lvert \mathbf{y} \rvert\rvert ^2 +
	    2\mathbf{x}.\mathbf{y} \\
            \leq & \lvert\lvert \mathbf{x} \rvert\rvert ^2 + \lvert\lvert \mathbf{y} \rvert\rvert ^2 +
	    2\lvert \lvert \mathbf{x} \rvert\rvert * \lvert \lvert \mathbf{y} \rvert \rvert \\
            & =\left(\lvert\lvert \mathbf{x} \rvert\rvert + \lvert\lvert \mathbf{y} \rvert\rvert
	    \right)^2 	    
	\end{aligned}
    \end{displaymath}
    Hence proved.
\end{proof}
Now we can prove \ref{th:euclidean_space}.
\begin{proof}
    In order to show $E^n$ is a metric space we have to show that the euclidean distance function
    is indeed a metric. Properties 1-3 can be easily verified. To check the triangle inequality (i.e
    property 4), we can see that 

    \begin{displaymath}
	\begin{aligned}
	    d_{\lvert\lvert . \rvert\rvert}(\mathbf{x},\mathbf{z}) = & \lvert\lvert \mathbf{x}
	    - \mathbf{y} \rvert\rvert \\
	    = & \lvert\lvert (\mathbf{x} - \mathbf{y}) + (\mathbf{y} - \mathbf{z}) \rvert\rvert \\
	    \leq & \lvert\lvert \mathbf{x} - \mathbf{y} \rvert\rvert + \lvert\lvert \mathbf{y} -
	    \mathbf{z} \rvert\rvert \\
	    & = d_{\lvert\lvert . \rvert\rvert}(\mathbf{x},\mathbf{y}) + 
	    d_{\lvert\lvert . \rvert\rvert}(\mathbf{y},\mathbf{z})
	\end{aligned}
    \end{displaymath}
\end{proof}

\begin{Definition}
    Suppose $(E,d)$ is a metric space and $E^{\prime} \subset E$. Then $(E^{\prime},d^{\prime})$ is
    called a \emph{subspace} of $E$ where $d^{\prime}$ is a restriction of $d$ on $E$ i.e
    \begin{displaymath}
	\begin{aligned}
	    & d^{\prime} : E^{\prime} \times E^{\prime} \rightarrow \mathbb{R} \\
	& \forall p,q \in E^{\prime} \ d^{\prime}(p,q) = d(p,q)
	\end{aligned}
    \end{displaymath}	
\end{Definition}

Some examples
\begin{itemize}
    \item Subsets of $E^n$.
    \item Let $E$ be any set. Define $d: E \times E \rightarrow \mathbb{R}$ by 
    \begin{displaymath}
	d(p,q) = \left\{ 
	    \begin{array}{l l}
		0 & \quad \text{if $p = q$}\\
		1 & \quad \text{if $p \neq q$}
	    \end{array} \right.
    \end{displaymath}
    \begin{proof}
	We only need to verify the triangle inequality $d(p,r) \leq d(p,q) + d(q,r)$. Note that
	$d(p,r) \leq 1$ always. However, can we have the right hand side to be both zero ? if
	$d(p,q) = 0$ then $p = q$ and if $d(q,r) = 0$ then $q = r$. Hence, we have $p = r$ and hence
	$d(p,r) = 0$. Hence it satisfies triangle inequality.
    \end{proof}
    Note that if $E$ has only 3 points, then those points are a \emph{triangle} in $E^2$. That is 
    if $E$ is finite then it is a subspace of Euclidean metric space $E^{n+1}$. But if $E$ is
    infinite it is not a subspace of an $E^n$.
\end{itemize}
Some consequences of a metric in a metric space $E$.
\begin{itemize}
    \item $d(p_1,p_n) \leq d(p_1,p_2) + \dots + d(p_{n-1},p_{n})$.
    \item $\forall p,q,r \in E$ we have $\lvert d(p,r) - d(q,r) \rvert \leq d(p,q)$.
	\begin{proof}
	    From triangle inequality we have,
	    \begin{displaymath}
		\begin{aligned}
		    & d(p,r) \leq d(p,q) + d(q,r) \\
		    & \implies d(p,r) - d(q,r) \leq d(p,q)
		\end{aligned}
	    \end{displaymath}
	    Similarly,
	    \begin{displaymath}
		\begin{aligned}
		    & d(q,r) \leq d(q,p) + d(p,r) \\
		    & \implies d(q,r) - d(p,r) \leq d(p,q)
		\end{aligned}
	    \end{displaymath}
	    Hence proved.
	\end{proof}	    
\end{itemize}

\section{Open, closed and bounded sets}
\begin{Definition}{\emph{Open Ball:}}\label{def:open_ball}
    In a metric space $(E,d)$, consider $p \in E$ and $\epsilon \in \mathbb{R} > 0$. Then the
    open ball centered at $p$ of \emph{radius} $\epsilon$ is given by :
    \begin{displaymath}
	\mathcal{B}_{\epsilon}(p) := \left.\lbrace q \in E : d(q,p) < \epsilon \rbrace\right.
    \end{displaymath}
\end{Definition}

\begin{Definition}{\emph{Closed Ball:}}\label{def:closed_ball}
    In a metric space $(E,d)$, consider $p \in E$ and $\epsilon \in \mathbb{R} > 0$. Then the
    closed ball centered at $p$ of \emph{radius} $\epsilon$ is given by :
    \begin{displaymath}
	\overline{\mathcal{B}_{\epsilon}(p)} := \left.\lbrace q \in E : d(q,p) \leq \epsilon 
	    \rbrace\right.
    \end{displaymath}
\end{Definition}

Some Examples : 
\begin{itemize}
    \item In $E^1$, for any $x \in \mathbb{R}$ we have an open ball :
	\begin{displaymath}
	    \mathcal{B}_{\epsilon}(x) := \left.\lbrace y \in \mathbb{R}  : \lvert y - x \rvert 
		< \epsilon \rbrace\right. .
	\end{displaymath}
	It is the same as $\left.\lbrace y \in \mathbb{R} : x -\epsilon < y < x + \epsilon 
	    \rbrace\right.$. 
    \item In $E^2$,	
	\begin{displaymath}
	    \mathcal{B}_{\epsilon}(\mathbf{x}) := \left.\lbrace \mathbf{y} \in \mathbb{R}^2  : 
		\lvert\lvert \mathbf{y} - \mathbf{x} \rvert\rvert 
		< \epsilon \rbrace\right. .
	\end{displaymath}

	
    \item Consider an \emph{unusual} metric space where
	open balls are a bunch of different pieces.
	
    \item Let $E = (\mathbb{R}^2,d)$ where $d(\mathbf{x},\mathbf{y}) = \lvert x_1 - y_1 \rvert + 
	\lvert x_2 - y_2 \rvert$. Then $\mathcal{B}_{\epsilon}(\mathbf{0}) = \left.\lbrace
	    \mathbf{y} \in \mathbb{R}^2 : \lvert y_1 \rvert + \lvert y_2 \rvert < \epsilon 
	    \rbrace \right. $. The unit ball then looks like :

	


\end{itemize}
Now that we have seen open balls, we can define what we mean by \emph{open sets} in a metric space.
\begin{Definition}
    A set $\mathcal{U} \subset E$ is an open set if 

    \begin{displaymath}
	\forall \left(p \in \mathcal{U}\right) \ \exists\left(\epsilon(p)  > 0 \in \mathbb{R}\right) \
       	\mathcal{B}_{\epsilon}(p) \subset \mathcal{U}.    \end{displaymath}
\end{Definition}
\begin{Theorem}[name=Open ball is an open set]\label{th:op_ballset}
    In any metric space $(E,d)$, every open ball is an open set.
\end{Theorem}
\begin{proof}
    
    Given a $p \in E$ and an $\epsilon > 0 \in \mathbb{R}$, let $q$ be an arbitrary point of the 
    set $\mathcal{B}_{\epsilon}(p)$. Let $\delta = \epsilon - d(q,p)$. Then we have $\delta > 0 \in
    \mathbb{R}$. For any $r \in \mathcal{B}_{\delta}(q)$, we have $d(r,q) < \delta$. Now, 
    \begin{displaymath}
	\begin{aligned}
	    d(r,p) \leq & \ d(r,q) + d(q,p) \\
	    d(r,p) < & \ \delta + d(q,p) \\
	    & = \epsilon - d(q,p) + d(q,p) 
	\end{aligned}
    \end{displaymath}	
    Hence, $d(r,p) < \epsilon$. Thus, $r \in \mathcal{B}_{\epsilon}(p)$. Hence proved.
    \\
    
\end{proof}
Examples,
\begin{itemize}
    \item
	In $E^n$, the closed ball $\overline{\mathcal{B}_{\epsilon}(\mathbf{0})}$ is not open.
	Note that to show that this ball is not open we have to find a point in the ball 
	such that there exist a point in that ball that is not contained in 
	$\overline{\mathcal{B}_{\epsilon}(\mathbf{0})}$.
	Let $\mathbf{p} = (1,0,\dots,0) \in E^n$. Then $\mathbf{p} \in 
	\overline{\mathcal{B}_{\epsilon}(\mathbf{0})}$. Now take a point $\mathbf{q} \in
	\mathcal{B}_{\epsilon}(\mathbf{p})$ such that $\mathbf{q} = (1+\frac{\epsilon}{2},0,\dots,0)$.
	Then $d_{\lvert\lvert . \rvert\rvert}(\mathbf{q},\mathbf{p}) = \frac{\epsilon}{2}$, but
	$d_{\lvert\lvert . \rvert\rvert}(\mathbf{q},\mathbf{0}) = 1 + \frac{\epsilon}{2}$.

    \item In $E^n$ choose $a \in \mathbb{R}$ and let $\mathcal{H} = \left.\lbrace 
	    (x_1, x_2, \dots, x_n) \in \mathbb{R^n} : x_1 > a \rbrace\right.$ 
	    We can show that $\mathcal{H}$ is open. Given any $\mathbf{x} \in \mathbb{R}^n$ we
	    know that $x_1 > a$. Therfore, $\epsilon = x_1 - a > 0 \in \mathbb{R}$. Now, we need
	   to show that $\mathcal{B}_{\epsilon}(\mathbf{x}) \subset \mathcal{H}$. For any 
	   $\mathbf{y} \in \mathcal{B}_{\epsilon}(\mathbf{x})$ we have $\lvert\lvert \mathbf{y} -
	   \mathbf{x} \rvert\rvert < \epsilon$. That is 
	   \begin{displaymath}
	       \begin{aligned}
		   \epsilon > & \lvert\lvert \mathbf{y} - \mathbf{x} \rvert\rvert \\
		    & = \sqrt{\sum_{i=1}^n(x_i-y_i)^2} \\
		    & \geq \sqrt{(x_1-y_1)^2} \\
		    &\quad = \lvert x_1 - y_1 \rvert \\
		    &\quad \geq x_1 - y_1
	       \end{aligned}
	   \end{displaymath}   
	   Hence, $ \epsilon > x_1 - y_1$ and therfore $y_1 > a$ and so $\mathbf{y} \in
	   \mathcal{H}$. Below we show $\mathcal{H}$ in $E^2$.

	   

\end{itemize}

\begin{Theorem}[name=Properties of open sets]
    In any metric space $(E,d)$ the following are true.    
    \begin{enumerate}
	\item $\emptyset$ is open.
	\item $E$ is open.
	\item For any collection of open sets in $E$, their union is open in $E$.
	\item For any \emph{finite} collection of open sets in $E$, their intersection is open in 
	    $E$.
    \end{enumerate}
\end{Theorem}
\begin{proof}

    
    

    \begin{enumerate}
	\item From the logical statement this is vacuously true.
	\item Similar to above.
	\item Let $G = \left.\lbrace \mathcal{U}_i \in E : i \in I \rbrace\right.$ be a 
	    collection of open sets indexed by some possibly uncountable index set $I$. Let
	    $p$ be an arbitrary point of $\bigcup_{i\in I} \mathcal{U}_i$. Then,
	    \begin{displaymath}
		\begin{aligned}
		   & p \in \bigcup_{i\in I} \mathcal{U}_i \\
		   & \Leftrightarrow \exists j \in I \ s.t \ p \in \mathcal{U}_j \\
		   & \Leftrightarrow \exists \epsilon_j > 0 \in \mathbb{R} \ s.t \ 
		       \mathcal{B}_{\epsilon_j}(p) \subset \mathcal{U}_j \\
		\end{aligned}
	    \end{displaymath}
	    But, $\mathcal{U}_j \subset \bigcup_{i\in I} \mathcal{U}_i$. Hence, 
	    $\mathcal{B}_{\epsilon_j}(p) \subset \bigcup_{i\in I} \mathcal{U}_i$. Hence, the union
	    is open.
	\item Let $G = \left.\lbrace \mathcal{U}_i : i \in \lbrace 1, \dots, n\rbrace
		\rbrace\right.$ be a finite collection of $n \in \mathbb{Z}_+$ open sets in $E$.
	    Then,
	    \begin{displaymath}
		\begin{aligned}
		   & p \in \bigcap_{i = 1}^n \mathcal{U}_i \\
		   & \Leftrightarrow \forall i \in \lbrace 1,\dots,n \rbrace \ s.t \ p 
		       \in \mathcal{U}_i \\
		   & \Leftrightarrow \forall i  \exists \epsilon_i > 0 \in \mathbb{R} \ s.t \ 
		       \mathcal{B}_{\epsilon_i}(p) \subset \mathcal{U}_i \\
		\end{aligned}
	    \end{displaymath}
	    Since there are only \emph{finite} such $i^{'s}$ there exists a minimum and since 
	    each $\epsilon_i > 0$, the minimum is also greater than $0$. That is, let
	    $\epsilon = min \left( \epsilon_i : i \in \lbrace 1, \dots, n \rbrace \right)$. Then,
	   $ \forall i \ \mathcal{B}_{\epsilon}(p) \subset \mathcal{B}_{\epsilon_i}(p) \subset
	   \mathcal{U}_i$. Therfore $\mathcal{B}_{\epsilon}(p) \subset 
	   \bigcap_{i = 1}^n \mathcal{U}_i$.
	   
    \end{enumerate}
\end{proof}
With this we can look at the following equivalance. The obove properties are quite general and is 
the definition of open sets in abstract topological spaces where choosing a metric may become 
cumbersome. The following Theorem shows that if we have a metric space, the usual metric space 
definition of open sets and topological notion are equivalent.
\begin{Theorem}[name=Topological and metric space equivalence of open sets]
    A set $\mathcal{U}$ is open iff it is a union of open balls in $E$.
\end{Theorem}
\begin{proof}
    $\Leftarrow $ 
	Since, an open ball in $E$ is an open set \ref{th:op_ballset}, 
	for any collection its union is open by the Theorem above.
	$\Rightarrow$
    Given any point $p \in \mathcal{U}$ such that $\mathcal{U}$ is open, let
    $\mathcal{B}_{\epsilon_{p}}(p)$ be an open ball around $p$. Each $\mathcal{B}_{\epsilon_{p}}(p)$
    is a subset of $\mathcal{U}$ and hence $\bigcup_{p \in \mathcal{U}}\mathcal{B}_{\epsilon_{p}}(p)
    \subset \mathcal{U}$. Similarly $\forall p \in \mathcal{U}$, there is an $\epsilon_{p}$ such
    that $p \in \mathcal{B}_{\epsilon_{p}}(p)$. And hence $ p \in 
    \bigcup_{p \in \mathcal{U}}\mathcal{B}_{\epsilon_{p}}(p)$. Therfore $\mathcal{U} \subset 
    \bigcup_{p \in \mathcal{U}}\mathcal{B}_{\epsilon_{p}}(p)$. Hence proved.
    
\end{proof}
\begin{Definition}
    A subset $\mathcal{C} \subset E$ is closed if its complement is open.
\end{Definition}
\begin{Theorem}[name= Closed ball is a closed set]
    A closed ball in $(E,d)$ is a closed set.
\end{Theorem}
\begin{proof}
    Let $p \in E$ be an arbitray point in $E$ and $\epsilon > 0 \in \mathbb{R}$. 
    Take any $q \in E - \overline{\mathcal{B}_{\epsilon}(p)}$ and let $\delta = d(q,p) - \epsilon$.
    Then since $q \in  E - \overline{\mathcal{B}_{\epsilon}(p)}$
    we have $d(q,p) > \epsilon$ and hence $\delta > 0 \in \mathbb{R}$. Now for any point $r \in
    \mathcal{B}_{\delta}(q)$ we have,
    \begin{displaymath}
	\begin{aligned}
	    d(r,p) \geq & \ d(q,p) - d(q,r) \\
	    > & \ d(q,p) - \delta \\
	    & = \epsilon
	\end{aligned}
    \end{displaymath}
    Hence, $d(r,p) > \epsilon$ Therfore, $r \in E - \overline{\mathcal{B}_{\epsilon}(p)}$. Hence,
    $\mathcal{B}_{\delta}(q) \subset E - \overline{\mathcal{B}_{\epsilon}(p)}$. Hence, the
    complement of $\overline{\mathcal{B}_{\epsilon}(p)}$ is open. Hence, 
    $\overline{\mathcal{B}_{\epsilon}(p)}$ is closed.
\end{proof}

Note that the following are equivalent. In general, a set $\mathcal{C} \subset E$ is closed if
\begin{enumerate}
    \item $E - \mathcal{C}$ is open.
    \item $\forall \left(q \in E\right) \ \left[ q \not \in \mathcal{C} \implies \exists 
	    \left(\delta > 0 \in \mathbb{R} \right) \ \mathcal{B}_{\delta}(q) \cap 
	    \mathcal{C} = \emptyset \right] $.
    \item $\forall \left(q \in E\right) \ \left[ \forall \left( \delta > 0 \in \mathbb{R}\right) \
	    \mathcal{B}_{\delta}(q) \cap \mathcal{C} \neq \emptyset \implies q \in
	    \mathcal{C}\right]$.
\end{enumerate}
The last one enables us to look at closed set only in terms of itself without using complements. For
example to show that the open ball in $E^n$ is not closed all we have to do is find a point $q \in
E$ such that  $\forall \left( \delta > 0 \in \mathbb{R}\right) \ \mathcal{B}_{\delta}(q) \cap 
\mathcal{C} \neq \emptyset$ is true but $q \not \in \mathcal{C}$. Such a $q$ is shown in the figure
below for $E^2$,



Closed sets and open sets are not complements of each other. As these examples show, a set can be 
both closed and open and also neither !!
\begin{itemize}
    \item Closed and open half spaces in $E^n$. A closed half space is defined as 
	$\mathcal{C} = \left\lbrace x \in \mathbb{R}^n : x_1 \leq a \right\rbrace$. Note that the
	complement of a closed half space is the open half space $\mathcal{H}$ which we showed to be
	open. Hence, closed half space is closed. So half spaces defined like these are either 
	closed or open.
    \item Consider the set $\left[ 0 , 1\right)$ $\subset E^1$. This set is neither open nor closed.
	To see why its not open we can look at the left point. To see why it is not closed we can
	look at the right.
    \item Let $E = E^1 - \lbrace 0 \rbrace$. Note that $\mathbb{R}_+ $ is open and so is 
	$\mathbb{R}_{-}$. But $E = \mathbb{R}_+ \cup \mathbb{R}_{-}$. Since $\mathbb{R}_{-}$ is open
	its complement is closed. But its complement is $\mathbb{R}_+$. Hence, $\mathbb{R}_+$ is
	both closed and open. Similarly $\mathbb{R}_{-}$.
\end{itemize}
From the properties of open sets we can infer the properties of a closed set.
\begin{Theorem}[name=Properties of closed sets]
    In any metric space $(E,d)$ the following are true.    
    \begin{enumerate}
	\item $E$ is closed.
	\item $\emptyset$ is open.
	\item For any collection of closed sets in $E$, their intersection is open in $E$.
	\item For any \emph{finite} collection of closed sets in $E$, their union is open in 
	    $E$.
    \end{enumerate}
\end{Theorem}
\begin{proof}
    \begin{enumerate}
	\item $E$ complement $\emptyset$ is open.
	\item $\emptyset$ complement $E$ is open.
	\item Let $G = \left.\lbrace \mathcal{C}_i \in E : \forall i \in I \rbrace\right.$ be a
	    collection of closed sets $\mathcal{C}_i$ indexed by possibly uncountable index set $I$.
	    Then note that $E - \bigcap_{i \in I}\mathcal{C}_i = \bigcup_{i \in I}\left(E - 
		\mathcal{C}_i\right)$. But the right hand side is the union of an arbitrary
	    collection of open sets and hence is open. Thus, we have shown that the complement is
	    open. Hence, $\bigcap_{i \in I}\mathcal{C}_i$ is closed. 
	\item Similar to above.
    \end{enumerate}
\end{proof}
In a topological space we usually start with the open sets that satisfy properties 1-4 of the open
sets and then infer the properties of closed sets. Hence, one gives us the other. Before looking at
a few examples where we'll use these properties let us define \emph{open boxes}  in $E^n$. 

\begin{Definition}
    For any $\mathbf{a}, \mathbf{b}$ $\in \mathbb{R}^n$ with $a_i < b_i \ \forall i \in \lbrace 1
    \dots n \rbrace$, an open box is the set $ \left.\lbrace \left(x_1,\dots,x_n\right) \in
	\mathbb{R}^n : a_i < x_i < b_i \rbrace\right.$
\end{Definition}
The open box in $E^1$ is just the open interval $(a,b)$. In $E^2$ the open box looks like,


While in $E^3$ it looks like,


Now let us look at some examples where we'll use the properties of closed and open sets in 
deciding whether the sets are closed or open.
\begin{itemize}
    \item In any metric space $(E,d)$, for any $p \in E$, the single point set $\lbrace p \rbrace$
	is a closed set.
	\begin{proof}
	    We'll show that $E - \lbrace p \rbrace$ is open. Take any $q \in E - \lbrace p \rbrace$.
	    Hence, $p \neq q$ and so let $\epsilon = d(q,p) > 0$. Thus the ball
	    $\mathcal{B}_{\epsilon}(q)$ doesn't contain $p$. Hence $\mathcal{B}_{\epsilon}(q)
	    \subset E - \lbrace p \rbrace$. Thus the complement is open and hence $\lbrace p
	    \rbrace$ is closed.
	\end{proof}
    \item Every finite subset of $E$ is closed.
	\begin{proof}
	    Any finite set is the finite union of one point sets in $E$. By property 4 of closed
	    sets such a union is closed.
	\end{proof}
    \item \emph{Spheres} are defined as the set $\mathcal{S} = \left.\lbrace q \in E : d(q,p) =
	    \epsilon \rbrace\right.$ for a given $\epsilon > 0 \in \mathbb{R}$ and 
	    a point $p \in E$. $\mathcal{S}$ is closed. Spheres are shown here in $E^2$ and $E^3$.
	    
	\begin{proof}
	    Given an $\epsilon > 0 \in \mathbb{R}$ and a $p \in E$, $\mathcal{S} =
	    \overline{\mathcal{B}_{\epsilon}(p)} - \mathcal{B}_{\epsilon}(p)$. Closed balls are
	    closed sets while open balls are open sets whose complements are closed. Hence,
	    rewriting $\mathcal{S} = \overline{\mathcal{B}_{\epsilon}(p)} \bigcap \left( E - 
		\mathcal{B}_{\epsilon}(p) \right)$, we see that $\mathcal{S}$ is an intersection
	    of two closed sets and hence by property 3 of closed sets, $\mathcal{S}$ is closed.	       
	\end{proof}
    \item Every open box in $E^n$ is an open set.
	\begin{proof}
	    An open box in $E^n$ is an intersection of $2n$ half open space $\mathcal{H}$. Hence by
	    property 4 of open sets, open box is open.
	\end{proof}
\end{itemize}
Now that we have seen some of the topological properties of open/closed sets, let us get back
to something that is specific to metric spaces. We'll define what it means to say that a set is
bounded in $(E,d)$.
\begin{Definition}
    A set $\mathcal{G} \subset E$ in a metric space $(E,d)$ is bounded if it lies in some open
    ball  i.e,
    \begin{displaymath}
	\exists \left(p \in E\right) \exists \left(R > 0 \in \mathbb{R}\right) \ s.t \ \mathcal{G}
	\subset \mathcal{B}_{R}(p).
    \end{displaymath}
\end{Definition}
Let us look at some examples of bounded (or unbounded ) sets in $(E,d)$.
\begin{itemize}
    \item Every open (or closed ) ball is bounded.
    \item Every open (or closed ) box is bounded in $E^n$.
    \item The discrete metric space is bounded (take $\mathcal{B}_2(p)$ for any $p \in E$).
    \item Half open (or closed ) spaces are not bounded.
\end{itemize}

Next we'll look at Theorem which says that for bounded sets the choice of center and radius isn't
crucial.
\begin{Theorem}[name= Bounded sets in metric space]
    For any $p \in E$, a set $\mathcal{G} \subset E$ is bounded iff there is a radius $R > 0$ such
    that $\mathcal{G} \subset \mathcal{B}_{R}(p)$.
\end{Theorem}
\begin{proof}
    $\Leftarrow$ Follows from the definition of bounded sets. $\Rightarrow$ Since $\mathcal{G}$ is
    bounded $\exists q \in E$ and $R' > 0$ such that $\mathcal{G} \subset \mathcal{B}_{R'}(q)$. Now
    consider any point $p \in E$ such that $p \neq q$. Let $R = d(q,p) + R'$. Then for any 
    $r \in \mathcal{B}_{R'}(q)$ we have $d(r,q) < R'$. Now $d(r,p) \leq d(r,q) + d(q,p)$. Hence
    $d(r,p) < R$. This means that ${B}_{R'}(q) \subset {B}_{R}(p)$. Hence $\mathcal{G} \subset
    {B}_{R}(p)$. The following picture depicts the mechanisms of the proof.
    

\end{proof}
\begin{Corollary}
    If $\mathcal{G}_1, \dots, \mathcal{G}_n$ are bounded then $\bigcup_{i=1}^{n}\mathcal{G}_i$ is
    bounded.
\end{Corollary}
\begin{proof}
    Fix a $p \in E$. From the above Theorem since each $\mathcal{G}_i$ is bounded there is a $R_i >
    0$ such that $\mathcal{G}_i \subset \mathcal{B}_{R_i}(p)$. Let $R = $ max$(R_1,\dots,R_n)$. Then
    for each $i$, $\mathcal{B}_{R_i} \subset \mathcal{B}_{R}$. Hence, $\bigcup_{i=1}^{n}
    \mathcal{G}_i \subset \mathcal{B}_{R}$. Hence, the union is bounded.  
\end{proof}
Next we'll focus on \emph{sequences} in a metric space. However, before we begin talking 
about sequences in a metric space we'll show that a closed and bounded subset 
of $\mathbb{R}$ contains its supremum and infimum.

\begin{Theorem}[name=Bounded and closed non-empty sets in $\mathbb{R}$]
    If $\mathcal{C} \subset \mathbb{R}$ is nonempty, closed and bounded then $\mathcal{C}$ has a
    greatest and least element.
\end{Theorem}
\begin{proof}
    $\mathcal{C}$ is closed and bounded. Hence it is bounded from above.  Hence, there is a supremum $a \in \mathbb{R}$. If $a \in \mathbb{R}$ then we are 
    done. $a$ cannot be in $\mathbb{R} - \mathcal{C}$. To show this, lets assume that $a \not \in 
    \mathcal{C}$. Since $\mathcal{C}$ is closed, $\mathbb{R} - \mathcal{C}$ is open and there is 
    $\epsilon > 0$ such that $\mathcal{B}_{\epsilon}(a) \cap \mathcal{C} = \emptyset$. But in 
    $\mathbb{R}$, the $\mathcal{B}_{\epsilon}(a)$ is an open interval $ a - \epsilon < x < a +
    \epsilon $. Hence for any $x \in \mathbb{R}$, if $x > a - \epsilon$ then 
    $x \not \in \mathcal{C}$. But this means that $a - \epsilon$ is also the supremum of 
    $\mathcal{C}$ giving us a contradiction. Hence $a$ must belong to $\mathcal{C}$. 
    Similar argument can be used to show $\mathcal{C}$ contains its infimum.
\end{proof}

\section{Sequences in a metric space}
\begin{Definition}
    In a metric space $E$, a sequence denoted by $\left(p_i\right)$ is a function from $\mathbb{Z}_+
\rightarrow E$.
\end{Definition}
We know from our intuition that if a sequence is getting closer and closer to a point, it must be
converging to that point. We'll establish this intuition with the following concrete definition of
the convergence of a sequence in a metric space.
\begin{Definition}
    A sequence $\left(p_i\right)$ converges to a point $p \in E$ if $\forall \epsilon > 0 \in
    \mathbb{R}$ $\exists N \in \mathbb{Z}_+$ such that whenever $n \geq N$ then $d(p_n,p) <
    \epsilon$ (i.e$p_n \in \mathcal{B}_{\epsilon}(p)$).
\end{Definition}
The following proposition will tell us that if a sequence converges it must converge to a unique
point. Hence, we can talk about $\underline{the}$ \emph{limit} of a sequence.

\begin{Proposition}[name= Uniqueness of limit of a convergent sequence]
    A sequence $\left( p_n \right)$ in $(E,d)$ converges to \emph{atmost} one point.
\end{Proposition}
\begin{proof}
    Suppose $\left( p_n \right)$ converges to $p,q \in E$ with $p \neq q$. Then $d(p,q) > 0$. Let
    $\epsilon = d(p,q)$. Since we have a convergent sequence $\exists N_1 \in \mathbb{Z}_+$ such
    that whenever $n \geq N_1$ we have $d(p_n,p) < \frac{\epsilon}{2}$. Similary there is another
    $N_2 \in \mathbb{Z}_+$ such that whenever $n \geq N_2$ we have $d(p_n,q) < \frac{\epsilon}{2}$.
    Let $N =$ max$(N_1,N_2)$. Then for $n \geq N$ both $d(p_n,p) < \frac{\epsilon}{2}$ and 
    $d(p_n,q) < \frac{\epsilon}{2}$ are true. But this means, $d(p,q) \leq d(p,p_n) + d(p_n,q)$ and
    hence $d(p,q) < \epsilon$ which is absurd since $\epsilon \text{ is } d(p,q)$. Hence we reach a
    contradiction. Thus $p = q$.
\end{proof}

Thus if a sequence $\left( p_n \right)$ converges to a point $p$ we call $p$ the limit of the
sequence $\left( p_n \right)$. Following are some equivalent notations for limits of a sequence.
\begin{displaymath}
    \begin{aligned}
	p_n \rightarrow p. \\
	\lim_{n \to +\infty}p_n = p.
    \end{aligned}
\end{displaymath}
The choice of metric space is important in deciding whether a sequence converges or not. 
Consider the following example
\begin{itemize}
    \item $\frac{1}{n} \rightarrow 0$ in $E^1$.
    By Archimedian property we know that $\forall \epsilon > 0$ there is a $N \in \mathbb{Z}_+$ such
    that $\frac{1}{N} < \epsilon$. Hence for any $ n \geq N$ we have $\frac{1}{n} \leq \frac{1}{N} <
    \epsilon$. Thus $d(p_n,0) < \epsilon$.
\item Take the sequence $\left(1/n\right)$ in $\mathbb{R}_+$ with the Euclidean metric. This
    sequence doesn't converge. ($0 \not \in \mathbb{R}_+$)
\end{itemize}
Next we'll define a subsequence. 
\begin{Definition}
    A subsequence of a sequence $\left( p_n \right)$ is a sequence $\left( p_{n_i}
    \right)$ where $ n_1 \ < \ n_2 \ < \ n_3 \ < \ \dots $.
\end{Definition}
For example $\left( 1/2n \right)$ is a subsequence of $\left( 1/n \right)$ with $n_1 = 2$, $n_2 = 4$
and so on. 

Intuition says that if a sequence converges to a point in a metric space, its subsequence 
should also converge to the same point. We'll prove this in the next proposition. 
\begin{Proposition}[name=Convergence of a subsequence]
   A subsequence of a convergent sequence in a metric space converges to the same limit. 
\end{Proposition}
\begin{proof}
    Given a sequence $\left(p_i\right) \rightarrow p$ . Hence given any $\epsilon > 0 \in
\mathbb{R}$ there is a $N \in \mathbb{Z}_+$ such that whenever $n \geq N$ we have $d(p_n,p) <
\epsilon$. But $\forall i \in \mathbb{Z}_+$ $n_i \geq i$.  
Hence for $i \geq N$, $n_i \geq i \geq N$ so $d(p_{n_i},p) < \epsilon$. 
\end{proof}
We define the \emph{range} of a sequence $\left(p_i\right)$ to be the \textbf{set} $P = \left\lbrace
    p_i : i \in \mathbb{Z}_+ \right\rbrace$. Note that a sequence is always infinite whereas the
range of the sequence could be finite (e.g constant sequences). With this definition we can 
define what it means for a sequence to be bounded in $(E,d)$.
\begin{Definition}
    A sequence $\left(p_i\right)$ is bounded if its range is bounded.
\end{Definition}
The next proposition shows that convergent sequences are bounded.
\begin{Proposition}[name=Boundedness of convergent sequences]
    Every convergent sequence is a metric space is bounded.
\end{Proposition}
\begin{proof}
    Let $\left(p_n\right)$ be a convergent sequence in $E$. Then for $\epsilon = 1$ there is a
    $N \in \mathbb{Z}_+$ such that whenever $n \geq N$, $d(p_n,p) < 1$. This means that there
   are only finitely many points $p_1,p_2,\dots,p_{N-1}$ that are not in $\mathcal{B}_1(p)$. Choose
  $R \in \mathbb{R}_+$ such that 
  $R > max(d(p_1,p),\dots,d(p_{N-1},p)) + 1$. Then
 $\forall n < N$, $d(p_n,p) < R$ and $\forall n \geq N$, $d(p_n,p) < 1 < R$. Hence, $\forall n \in
 \mathbb{Z}_+$, $d(p_n,p) < R$. This means that for any element $p_n$ in the range of
 $\left(p_n\right)$; $p_n \in \mathcal{B}_R(p)$. Hence, the sequence is bounded. 
\end{proof}
A closed subset of a metric space was defined to be the complement of an open set. Thus the
knowledge of all closed subsets is equivalent to the knowledge of all open subsets. Now, we'll see
that the knowledge of all open sets determines which sequences of points are convergent and to which
limit. And knowledge of convergent sequences of points in the metric space determines which sets of
a metric space are closed. Hence any statement concerning the open subsets of metric space can be
translated to statements concerning closed subsets which itself can be translated into statements
concerning convergent sequences of points and their limits. The next proposition will show the
connection of open sets and convergent sequences. Before doing that let us define what a
\emph{neighborhood} is. 
\begin{Definition}
    A neighborhood of a point $p \in E$ denoted by $\mathcal{N}[p]$ is an open set containing $p$. 
\end{Definition}
Note that an open ball of radius $r$ centered at $p$ is obviously a neighborhood of $p$, but a
neighborhood of $p$ may or may not be an open ball.
\begin{Proposition}[name=Open sets and convergent sequences]
   $p_n \rightarrow p$ iff for any neighborhood $\mathcal{N}[p]$ 
   there is a $N \in \mathbb{Z}_+$ such that whenever $n \geq N$, $p_n \in \mathcal{N}[p]$.
\end{Proposition}
\begin{proof}
    $\Leftarrow$ since a ball is a neighborhood. 
    $\Rightarrow$ Given $p_n \rightarrow p$. Let $\mathcal{N}[p]$ be any neighborhood of $p$. Then
since $\mathcal{N}[p]$ is open and $p \in \mathcal{N}[p]$ there is an $\epsilon > 0$ such that
$\mathcal{B}_{\epsilon}(p) \subset \mathcal{N}[p]$. Since $p_n \rightarrow p$, there is a $N \in
\mathbb{Z}_+$ such that whenever $n \geq N$, $d(p_n,p) < \epsilon$ i.e $p_n \in
\mathcal{B}_{\epsilon}(p)$. Hence, $p_n \in \mathcal{N}[p]$.
\end{proof}
Now we'll show the connection of closed sets and convergent sequences. 
\begin{Theorem}[name=Closed sets and convergent sequences]
    A set $\mathcal{C} \subset E$ is closed iff for every sequence $\left(p_i\right) \in
    \mathcal{C}$ whenever $p_i \rightarrow p \in E$ then $p \in \mathcal{C}$.
\end{Theorem}
\begin{proof}
    $\Rightarrow$ 
     Let $\mathcal{C}$ be closed and let $\left(p_i\right) \in \mathcal{C}$ such that $p_i
 \rightarrow p \in E$. Since $\left(p_i\right)$ is a convergent sequence, for any $\epsilon > 0$
 there is a $N \in \mathbb{Z}_+$ such that whenever $n \geq N$, $d(p_n,p) < \epsilon$ i.e 
$p_n \in \mathcal{B}_{\epsilon}(p)$. But $p_n \in \mathcal{C}$. Hence, $\mathcal{B}_{\epsilon}(p) \cap
\mathcal{C} \neq \emptyset$. Since $\epsilon > 0$ was arbitrary and $\mathcal{C}$ is closed, $p \in
\mathcal{C}$(from characterization of closed sets \#3). \\
$\Leftarrow$ 
Suppose $\mathcal{C}$ is not closed. Then there is a point $p \in E$ such that $\forall \epsilon >
0$, $\mathcal{B}_{\epsilon}(p) \cap \mathcal{C} \neq \emptyset$ but $p \neq \mathcal{C}$. Since for 
any $\epsilon >0$ we get a ball that intersects $\mathcal{C}$, let $\epsilon = 1/n$, there is a 
$p_n \in \mathcal{C}$ such that $p_n \in \mathcal{B}_{1/n}(p)$. Its easy to see that $\left(p_n\right)$
converges to $p$ since, let $\epsilon > 0$ choose $N \in \mathbb{Z}_+$ such that $1/N < \epsilon$.
Then $n \geq N \implies 1/n \leq 1/N$. Thus $\mathcal{B}_{1/n}(p) \subset \mathcal{B}_{\epsilon}(p)$.
But $p_n \in \mathcal{B}_{1/n}(p)$ hence, $d(p_n,p) < \epsilon$. Thus $p_n \rightarrow p$ but $p
\not \in \mathcal{C}$.
\end{proof}
Some very important concepts in metric spaces are defined below. One very important concept is the
idea of a cluster point or a limit point or accumulation ptof a set in a metric space. 
Such a point is \emph{approximated} by points of the set.
\begin{Definition}
    A point $p \in E$ is a cluster point (or limit point) of a subset $S \subset E$ if \emph{every}
    open ball about $p$ intersects $S$ in infinitely many points of $S$.
\end{Definition}
Note that the above definition is logically equivalent to saying that if every open ball about $p$
contains a point of $S$ other than maybe $p$ then $p$ is the cluster point of $S$.
Some examples :
\begin{itemize}
    \item $S = \left(0,1\right)$. The set of cluster points of $S = \left[0,1\right]$.
    \item $S = \mathbb{Z}$ and $E = \mathbb{R}$. Then $S$ has no cluster points.
    \item $S = \left.\lbrace 1/n : n \in \mathbb{Z}_+\rbrace\right.$. Then $S$ has only one cluster
	point namely $0$.
\end{itemize}
\begin{Definition}
    A point $p \in S$ in metric space $(E,d)$ is an interior point of $S$ if there is an open ball
    around $p$ contained entirely in $S$.
\end{Definition}
Note that for a set to be open all its points must be interior points.
\begin{Definition}
    A point $p \in S$ in metric space $(E,d)$ is an isolated point if there is an open ball around
    $p$ that contains not other points of $S$.
\end{Definition}
The set of integers $\mathbb{Z}$ in $\mathbb{R}$ contains only isolated points. For example consider
the ball of radius $1/2$ about every integer.
\begin{Definition}
    A point $p \in E$ is the boundary point of the set $S \subset E$ in the metric space $(E,d)$ if
    every open ball around $p$ contains points of $S$ and $S-E$. 
\end{Definition}
Note that every real number is a boundary point of $\mathbb{Q}$. We have seen a characterization of
closed sets and converging sequences. A different characterization is based on cluster points.
\begin{Theorem}
    A set $S$ in metric space $(E,d)$ is closed if and only if it contains all its cluster points.
\end{Theorem}
\begin{proof}
    Let $p$ be a cluster point of $S$ and assume $p \not \in S$. Hence $p \in E - S$. Since $S$ is
    closed, $E-S$ is open and so there is a $r > 0$ such that $\mathcal{B}_r(p) \subset S^c$. But
    this means that $p$ cannot be a cluster point of $S$ since for this $r$, 
    $\mathcal{B}_r(p) \bigcap S = \emptyset$. Hence we have arrived at a contradiction. Thus $p \in
    S$.

    Let $p \in E$ be an arbitrary point such that $p \not \in S$. Hence $p \in E - S$ and so $p$ is
    not a limit point of $S$. Thus there is an $r > 0$ such that $\mathcal{B}_r(p) \bigcap S =
    \emptyset$. Hence for this $r$, $\mathcal{B}_r(p) \subset S^{c}$. Thus $S^c$ is open and so $S$
    is closed.  
\end{proof}
Closed sets are very important in metric space as far as converging sequences are concerned. Given
any set $S$ in metric space $(E,d)$ we can derive a closed set called the closure of $S$ by
adjoining all its cluster points. 
\begin{Definition}
    Let $S \subset E$ in metric space $(E,d)$ and let $S'$ be the set of all the cluster points of
    $S$, then the closure of $S$ is the set $\overline{S} = S \bigcup S'$.
\end{Definition}
\begin{Theorem}[name=Properties of closure]
    If $E$ is a metric space and $X \subset E$, then
    \begin{enumerate}
	\item $\overline{X}$ is closed.
	\item $X = \overline{X}$ iff $X$ is closed.
	\item If $X \subset Y$ then $X' \subset Y'$.
	\item If for every closed set $F$ in metric space $E$, $X \subset F$ then $\overline{X}
	    \subset F$. Thus, the closure of a set is the smallest closed set containing the set.
    \end{enumerate}
\end{Theorem}

\begin{proof}
    \begin{enumerate}
	\item Let $p$ be a limit point of $\overline{X}$. Assume $p \not \in X$. 
	    Let $r > 0$ be an arbitrary positive real number. There is a $q \neq p$ in $\overline{X}$
	    such that $q$ is in the open ball $\mathcal{B}_r(p)$. If $q$ belongs to $X$ then we have
	    shown that $p$ is a limit point of $X$ and so belongs to $X'$ and hence $\overline{X}$.
	    If $q \not \in X$, then $q$ must be a limit point of $X$. Let $\delta = r - d(p,q)$.
	    Then there is an open ball $\mathcal{B}_{\delta}(q)$ which intersects $X$ at points
	    other than $q$. Since $\mathcal{B}_{\delta}(q) \subset \mathcal{B}_r(p) $, we see that
	    $\mathcal{B}_r(p)$ contains points of $X$ other than $p$. Hence $p$ is a limit point of
	    $X$ and so is contained in $\overline{X}$.
	\item Since $\overline{X}$ is closed $X$ is closed. Now let $X$ be closed and hence $X$
	    contains all its limit points and so $X' \subset X$ and so $\overline{X} = X\cup X' =
	    X$.
	\item Let $X \subset Y$. Let $p$ be a limit point of $X$ and hence for any $r > 0$ 
	    there is a $q \neq p$ in $X$ $q$ is in the ball $\mathcal{B}_r(p)$. But $q \in X$
	    implies $q \in Y$ and so $p$ is a limit point of $Y$ and hence belongs to $Y'$.
	\item Let $F$ be an arbitrarily closed subset in $E$ and so $F \supset F'$. Since $X \subset
	    F$ we have $X' \subset F'$. Thus $X \cup X' \subset X \cup F' \subset F \cup F'$ and so
	    $\overline{X} \subset F \cup F'$ i.e $\overline{X} \subset F$. 
    \end{enumerate}
\end{proof}

Next we will show that the set containing all the subsequential limits of a sequence is closed. Note
that finite sets are closed and hence we can assume that the set of subsequential limits is infinite
to prove this assertion.

\begin{Theorem}
    The subsequential limits of a sequence $(p_n)$ in a metric space $(E,d)$ is a closed subset of
    $E$.
\end{Theorem}
\begin{proof}
    Let $p$ be a limit point of the set of all subsequential limits of $(p_n)$. We will show that
    $p$ belongs to the set i.e $p$ is a subsequential limit. Since $p$ is a limit point, for any $
    \epsilon > 0$, there is a subsequential limit $q$ such that $q \neq p$ and $q \in
    \mathcal{B}_{\epsilon / 2}(p)$. Since $q$ is a subsequential limit we can form a sequence of
    $(p_{n_i})$, such that for all $i \in \mathbb{Z}^{+}$, $d(p_{n_i},q) < 1/i$. Choosing $i$ large
    enough we can make $d(p_{n_i},q) < \epsilon/2$. Thus $d(p_{n_i},p) < d(p_{n_i},q) + d(q,p)$ and
    so $d(p_{n_i},p) < \epsilon$. Hence $p$ is a subsequential limit.  
\end{proof}


Armed with these properties of a convergent sequence in a metric space, we can infer many other 
when the metric space is the Euclidean metric space. Since $\mathbb{R}$ is an ordered field with
the $L.U.B$ property many rational operation on convergent sequences follow. In the following 
paragraphs we'll focus on convergent sequences in $\mathbb{R}$.
\begin{Theorem}[name=Convergent sequences in $\mathbb{R}$]
    In $E^1$, if $a_n \rightarrow a$ and $b_n \rightarrow b$, then the following are true :
    \begin{enumerate}
	\item $a_n + b_n \rightarrow a+b$ 
	\item $a_n - b_n \rightarrow a-b $
	\item $a_n * b_n \rightarrow a*b $
	\item if $b \neq 0 $ and $\forall n \in \mathbb{Z}_+ b_n \neq 0$ then $\frac{a_n}{b_n} \rightarrow
	    \frac{a}{b}$
    \end{enumerate}
\end{Theorem}

\begin{proof}
    \begin{enumerate}
	\item Let $\epsilon > 0 \in \mathbb{R}$ be arbitrary. Since $a_n \rightarrow a$, 
	there is a $N_1 \in \mathbb{Z}_+$ such that whenever $n\geq N$, $d(a_n,a) < \epsilon/2$.
	Similarly there is $N_2$ for $\left(b_n\right)$. Let $N = \text{ max}(N_1,N_2)$. Note,
	$d(a_n + b_n, a+b) = \lvert a_n + b_n - (a+b) \rvert$. But the right hand side can be
	written as $\lvert (a_n - a) + (b_n - b)\rvert$ which from the triangle inequality gives us
	\begin{displaymath}
	    \begin{aligned}
		d(a_n + b_n, a + b) = & \lvert (a_n - a) + (b_n - b) \\
		\leq & \lvert a_n - a \rvert + \lvert b_n - b \rvert \\
		& = d(a_n,a) + d(b_n,b) \\
		& < \epsilon/2 + \epsilon/2 
	    \end{aligned}
	\end{displaymath}
	Hence $d(a_n + b_n, a+b) < \epsilon$. Thus $a_n + b_n \rightarrow a+b$.
	\item Similar to above. An alternate proof is using $3$ and noting that the constant
	    sequence $a_n = -1$ converges to $-1$, we get $-b_n \rightarrow -b$. Hence, 
	using $1$ we get $2$.
	\item Since convergent sequences are bounded, consider the \emph{closed} ball of radius
	    $M_1$ about $0$ containing the range of $\left(a_n\right)$ and closed ball of radius 
	    $M_2$ containing the range of $\left(b_n\right)$. Since a closed ball is also a closed
	    set and we have shown that the limit of a convergent sequence that is contained in a
	    closed ball is itself in a closed ball we can conclude that :
	    $\forall n \in \mathbb{Z}_+ \lvert a_n \rvert < M_1$ and $a < M_1$. Similarly for $b_n$
	    and $b$. Taking $M = max(M_1,M_2)$ we have $\forall n \ \lvert a_n \rvert < M $ and
	    $\lvert b_n \rvert < b$. Moreover, $a,b < M$. Now, let $\epsilon > 0 \in \mathbb{R}$ be
	    arbitrary. There is a $N = max(N_1,N_2)$ such that whenever $n \geq N$, $d(a_n,a) <
	    \frac{\epsilon}{2*M}$ and $d(b_n,b) < \frac{\epsilon}{2*M}$. Hence,
	    \begin{displaymath}
		\begin{aligned}
		    d(an*b_n, a*b) & =  \lvert a_n*b_n - a*b +\text{\fbox{$a*b_n$}} -
		    \text{\fbox{$a*b_n$}} \ \rvert \\
		    & =  \lvert (a_n - a)*b_n + (b_n - b)*a \rvert \\
		    & \leq \lvert b_n \rvert * \lvert a_n - a \rvert + \lvert a \rvert * \lvert b_n
		    - b \rvert \\
		    & < M*\frac{\epsilon}{2*M} +   M*\frac{\epsilon}{2*M} \\
		    & \ \ = \epsilon
		\end{aligned}
	    \end{displaymath}	
	\item We'll prove for a special case when $a_n = 1$. Note that $d(\frac{1}{b_n},\frac{1}{b})
	    = \lvert \frac{1}{b_n} - \frac{1}{b} \rvert$. Hence, we want $b_n$ to be bounded away 
	    from $0$. If we take $n$ large so that $\lvert b_n - b \rvert < b/2$, then $b_n > \lvert
	    b/2 \rvert$. Given an $\epsilon > 0 \in \mathbb{R}_+$, choose $N \in \mathbb{Z}_+$ such
	    that $\lvert b_n - b \rvert < \text{ min}\left.\lbrace \frac{\lvert b \rvert}{2},
		b^2*\epsilon/2 \rbrace\right.$. Then $n \geq N$ we get $d(\frac{1}{b_n},\frac{1}{b})
	    < \epsilon$
    \end{enumerate}
\end{proof}
An important corollary of the above Theorem is :
\begin{Corollary}
    In $E^1$ if $a_n \rightarrow a$ and $b_n \rightarrow b$ and if $\forall n \in \mathbb{Z}_+$,
$a_n \leq b_n$ the $a \leq b$. 
\end{Corollary}
\begin{proof}
    From the Theorem above we know that $b_n - a_n \rightarrow b - a$. Now, each $b_n - a_n \geq 0$
and they belong in $[0,\infty)$. But $[0,\infty)$ is a closed set since its the complement of
$(-\infty,0)$. Hence the limit $b-a \in [0,\infty)$. Thus $b-a \geq 0$ which means $a \leq b$.
\end{proof}
Next, we'll look into an important class of sequencese in $E^1$ called \emph{monotone} sequences.
\begin{Definition}
    A sequence $\left(a_n\right)$ in $\mathbb{R}$ is an increasing sequence if $a_1 \leq a_2 \leq
    a_3 \leq \dots$.
\end{Definition}
\begin{Definition}
    A sequence $\left(a_n\right)$ in $\mathbb{R}$ is an decreasing sequence if $a_1 \geq a_2 \geq
    a_3 \geq \dots$.
\end{Definition}
\begin{Definition}
    A sequence  in $\mathbb{R}$ is a monotone sequence if it is either increasing or decreasing.
\end{Definition}
The next Theorem guarantees the convergence of monotone sequences in $\mathbb{R}$ 
if they are bounded. This, as we shall see, is possible because $\mathbb{R}$ has the $L.U.B$ 
property. 
\begin{Theorem}[name=Bounded monotone sequences]
    Every bounded monotonic sequence converges.
\end{Theorem}
\begin{proof}
    Let $\left(a_n\right)$ be a bounded monotone sequence in $\mathbb{R}$. Without loss of generality, 
    lets assume that $\left(a_n\right)$ is increasing. To show the convergence of
    $\left(a_n\right)$, we need to find $a \in \mathbb{R}$ such that $\forall \epsilon > 0 \in
    \mathbb{R}$ there is a $N \in \mathbb{Z}_+$ such that whenever $n \geq N$, $a_n \in
    \mathcal{B}_{\epsilon}(a)$. Since $\left(a_n\right)$ is bounded, its range $A = \left.\lbrace a_n
	: n \in \mathbb{Z}_+ \rbrace\right.$ is bounded. But $A \subset \mathbb{R}$ and hence its
  supremum, lets call it a, exist. For any $\epsilon > 0$, $a_n < a + \epsilon$. Also, since $a -
  \epsilon$ is not the supremum of $A$, there is a $N \in \mathbb{Z}_+$ such that $a - \epsilon <
  a_N$. Now, whenever $n \geq N$, since $\left(a_n\right)$ is increasing we have, $a - \epsilon <
  a_N \leq a_n < a + \epsilon$. Thus $a_n \in \mathcal{B}_{\epsilon}(a)$.  
\end{proof}
An example of where the above Theorem is used is illustrated in the following example. Before we
use that it will be useful to note that for any $\left(b_n\right)$ in $E^1$, $\lvert b_n \rvert
\rightarrow 0$ $\Leftrightarrow$ $b_n \rightarrow 0$. It is important that the limit is $0$ for this
observation to be correct. With this, lets look at the following example.
\begin{itemize}
    \item Suppose $a \in \mathbb{R}$ such that $\lvert a \rvert < 1$. Then $a^n \rightarrow 0$.
    \begin{proof}
	Since $a^n \rightarrow 0$ is the same as $\lvert a^n \rvert \rightarrow 0$, we can assume
that $a > 0$ (because $\lvert a^n \rvert = \lvert a \rvert ^n$. Then $a_{n+1} - a_n = a_n(1 - a)$.
Hence, $\left(a_n\right)$ is a decreasing sequence. Since, the range is bounded by $(0,1)$, using
the Theorem above $\left(a_n\right)$ converges. Let $a_n \rightarrow x$. Then $a*a^{n-1} \rightarrow
a*x$. Since $a^n = a*a^{n-1}$, this means that $a*x = x$ or $(a-1)*x = 0$. Since, 
$\lvert a \rvert < 1$, $x = 0$.
    \end{proof}
\end{itemize}

In $E^1$, it is useful to look at upper and lower limits that are very useful in dealing with
infinite series. For any sequence any finite terms have no bearing on the convergence. As such if
$(p_n)$ is a sequence in $E^1$ it is important to look at the terms $n > N$ where we can vary $N$. 
Let $u_N = inf\lbrace p_n : n \geq N\rbrace$ and $v_N = sup \lbrace p_n : n \geq N \rbrace$. Then as
$N$ increases we have $u_1 \leq u_2 \ldots$ and $v_1 \geq v_2 \ldots$ and so we have monotonic
sequences. 

\begin{Definition}
    Let $(p_n)$ be a sequence in $E^1$. The limit superior i.e
    \begin{equation*}
	p^{*} = \lim \, sup \, p_N =
	\lim_{N\to\infty}sup\lbrace p_n : n\geq N \rbrace.
    \end{equation*}
\end{Definition}

\begin{Definition}
    Let $(p_n)$ be a sequence in $E^1$. The limit inferior i.e
    \begin{equation*}
	p_{*} = \lim \, inf \, p_N =
	\lim_{N\to\infty}inf\lbrace p_n : n\geq N \rbrace.
    \end{equation*}
\end{Definition}
Note that for any $\epsilon$ there is a $N$ such that $p^{*} - \epsilon < sup\lbrace p_n : n \geq
N\rbrace < p^{*} + \epsilon$. Thus for any $x > p^{*}$, we have $p_n < x$ for infinitely many
$n$. Analogous observation can be made for $p_{*}$. The following Theorem states some useful facts
about limit superior and limit inferior.
\begin{Theorem}
    Let $(p_n)$ be a sequence in $E^1$. Then
    \begin{enumerate}
	\item If $\lim \, p_n = p $ is defined then $p^{*} = p = p_{*}$.
	\item If $p^{*} = p_{*} = p$ then $\lim_{n \to \infty} \, p_n = p$.
    \end{enumerate}
\end{Theorem}
\begin{proof}
    \begin{enumerate}
	\item When $p^{*} = \infty$ or $p_{*} = - \infty$ it is easy to check. Let $p$ be a real
	    number. We know that for all $N$, $u_N \leq p_N \leq v_N$ and so $p_{*} \leq p \leq
	    p^{*}$. We need to show the other inequality. For any $\epsilon > 0$, we know that there
	    is a $N$ such that  $p - \epsilon < p_n  < p + \epsilon$. Thus 
	    $v_N = sup\lbrace p_n : n \geq N\rbrace \leq p + \epsilon$. For any $M \geq N$, $v_M
	    \leq v_N$ and so for all $M \geq N$, we have $v_M \leq p + \epsilon$. Hence in the
	    limit, $p^{*} \leq p + \epsilon$. Since $\epsilon > 0$ is arbitrary we get $p^{*} \leq
	    p$. A similar argument holds for $p_{*}$.  
	\item Again disregarding the case when either limit is $\pm \infty$, WLOG let $p^{*} =
	    p_{*}$ be a real number. Then for any $\epsilon > 0$ there is a $N_1$ such that $p^{*} -
	    \epsilon < v_N < p^{*} + \epsilon$ and an $N_2$ such that $p_{*} - \epsilon < u_N <
	    p_{*} + \epsilon$. Taking $N \geq max(N_1,N_2)$, we have
	    \begin{equation*}
		p -\epsilon < p_N < p + \epsilon
	    \end{equation*}
	    Thus, $p_n \to p$.
    \end{enumerate}
\end{proof}

A different characterization of the limit superior and limit inferior is given below,
\begin{Theorem}
    Let $(p_n)$ be a sequence in $E^1$. Let $S$ be the set of all subsequential limits of the
    sequence $(p_n)$. Then $p_{*} = inf\, S$ and $p^{*} = sup \, S$.
\end{Theorem}
\begin{proof}
    If $S$ is bounded then note that $p_{*}$ and $p^{*}$ belong to the set $S$ since $S$ is closed 
    and so it must contain its superemum and infimum. Let $t \in S$ be arbitrary. Then there is a 
    subsequence $p_{n_i}$ such that $p_{n_i} \to t$. From the above Theorem we know that,
    \begin{equation*}
	\lim_{N \to \infty} inf \, p_{n_N} = t = \lim_{N \to \infty} sup \, p_{n_N}
    \end{equation*}
    However, the set $\lbrace p_n : n > N \rbrace$ contains the the set $\lbrace p_{n_i} : n_i > N
    \rbrace$ and so we get the inequality,
    \begin{equation*}
	p_{*} = \lim_{n \to \infty} inf \, p_{n} \leq \lim_{N \to \infty} inf \, p_{n_N} = t = \lim_{N \to
	    \infty} sup \, p_{n_N} \leq \lim_{n \to \infty} sup \, p_{n} = p^{*}
    \end{equation*}
    Thus $p^{*}$ is an upper bound for the set $S$ and so $sup \, S \leq p^{*} $ 
    and similarly $inf \, S \geq p_{*}$. Since $p_{*}$ and $p^{*}$ belong to $S$ we get the desired
    equality.
\end{proof}

The definition of convergence of a sequence requires the apriori knowledge of the limit. This can be
cumbersome since the limit of a general sequence is not known beforehand. With this motivation in
mind we'll define \emph{Cauchy} sequences.

\begin{Definition}
    A sequence $\left(p_n\right)$ in any metric space $(E,d)$ is a cauchy sequence if $\forall
    \epsilon > 0 \in \mathbb{R}$, there is a $N \in \mathbb{Z}_+$ st whenever $m,n \geq N$,
    $d(p_m,p_n) < \epsilon$.
\end{Definition}
The next Theorem shows that if a sequence is convergent it is also cauchy.
\begin{Theorem}[name=Convergent sequences are cauchy]
    Every convergent sequence is a cauchy sequence.
\end{Theorem}
\begin{proof}
    Suppose $p_n \rightarrow p$. Then given any $\epsilon > 0$ there is a $N \in \mathbb{Z}_+$ such
that whenver $n \geq N$, $d(p_n,p) < \epsilon/2$. Hence, $m \geq N \implies d(p_m,p) < \epsilon/2$. 
Now, $d(p_m,p_n) \leq d(p_m,p) + d(p,p_n)$. Thus, $d(p_m,p) < \epsilon$.
\end{proof}
Note that the reverse implication is not always true. Consider the following example
\begin{itemize}
    \item In $E^1$, $\left(1/n\right)$ converges. Clearly the sequence is cauchy. However, in
	$\mathbb{R}_+ \subset E^1$, the sequence doesn't converge even though it is still cauchy.
\end{itemize}
There are some spaces in which these notions are equivalent. Such spaces are called \emph{complete}
spaces.
\begin{Definition}
    A metric space $(E,d)$ is complete iff every cauchy sequence in it converges. 
\end{Definition}
As we saw from the earlier example not all metric spaces are complete. Fortunately, many metric
spaces are complete. Before we show that $E^n$ is a complete metric space, we should see what are
the properties of a cauchy sequence.
\begin{Proposition}[name=Properties of a cauchy sequence]

    \begin{enumerate}
	\item Every subsequence of a cauchy sequence is cauchy.
	\item Every cauchy sequence is bounded.
	\item A cauchy sequence with a convergent subsequence must converge.
    \end{enumerate}
\end{Proposition}

\begin{proof}
    \begin{enumerate}
	\item since any $n_i \geq i$, for $n,m \geq N$ we have $n_n,n_m \geq N$. Thus
	    $d(p_{n_m},p_{n_n}) < \epsilon$.
	\item Let $\epsilon = 1$. $\left(p_n\right)$ is cauchy and hence there is a $N \in
	    \mathbb{Z}_+$ such that whenever $m,n \geq N$, $d(p_m,p_n) < 1$. Let $R = \text{
		max}(d(p_1,p_N),\dots,d(p_{N-1},p_N)) + 1$. Then $\forall n \in \mathbb{Z}_+$,
	    $p_n \in \mathcal{B}_{R}(p_N)$.
	\item Since $\left(p_n\right)$ is cauchy, given an $\epsilon > 0$ there is a $N_1$ such
	    that whenever, $m,n \geq N_1$, $d(p_n,p_m) < \epsilon/2$. Also let
	    $\left(p_{n_i}\right)$ be a subsequence such that $p_{n_i} \rightarrow p$. Then there is 
	a $N_2$ such that whenever for some $i \in \mathbb{Z}_+$, $n_i \geq N_2$, $d(p_{n_i},p) <
	\epsilon/2$. Take, $N = \text{ max}(N_1,N_2)$ and pick an $m > N$ such that $p_m$ is in the
	subsequence. Then $d(p_m,p) < \epsilon/2$. Using the triangle inequality we get $d(p_n,p) < 
	\epsilon$ whenver $n \geq N$.
    \end{enumerate}
\end{proof}
Next, we'll prove a very important Theorem that says that $E^1$ is complete.
\begin{Theorem}[name=Completeness of $E^1$]
    $E^1$ is complete.
\end{Theorem}
\begin{proof}
    Let $\left(a_n\right)$ be a cauchy sequence in $E^1$. Let $S = \left.\lbrace x \in \mathbb{R} :
	x \leq a_n \text{ for infinitely many } n \rbrace\right.$. Then, $S$ is a subset of the 
   range of $\left(a_n\right)$ which is bounded. Thus its supremum exists, lets call it $a$. Given
   any $\epsilon > 0$ there is a $N \in \mathbb{Z}_+$ such that whenever $m,n \geq N$, $d(a_m,a_n) <
   \epsilon$. Since, $sup S = a$, $a + \epsilon/2 \not \in S$, but $a - \epsilon/2 \in S$. Hence,
   $a - \epsilon/2 \leq a_n$ for infinitely many $n$. Choose $m \geq N$ such that $a_m \in
   (a-\epsilon/2 , a)$.  Hence, $\forall n \geq N$, $d(a_n,a) \leq d(a_n,a_m) + d(a_m,a)$. Thus,
   $d(a_n,a) < \epsilon$.
\end{proof}
The choice of $S$ is illustrated in the following figure. Our first guess of $a$ is denoted by 
$+_1$. Note that there are only finitely many points in the sequence that are greater than this
choice. There will be a point
during which we will \emph{cross} from finitely many points to inifinitely many points. Our set $S$
is then all the points to the left of that \emph{crossing} point. A supremum
is guaranteed since we are in $\mathbb{R}$. Also note that sup $S$ is the limit inferior of the
sequence. The sequence of $u_N$ form a monotonic increasing sequences and should converge to the
superemum. The supremum is guaranteed since the sequence is cauchy and so the range is bounded.
\newline
\newline



\begin{Corollary}
    $E^n$ is complete.
\end{Corollary}
\begin{proof}
    Let $\mathbf{p_i} = \left(x_{1i},x_{2i},\dots,x_{ni}\right)$ such that $\left(\mathbf{p_i}\right)$ 
    is cauchy. Hence, given an $\epsilon > 0 \in \mathbb{R}$, there is a $N \in \mathbb{Z}_+$ such 
    that whenver $i,j \geq N$, $d(\mathbf{p_i},\mathbf{p_j}) < \epsilon$. Now
    $d(\mathbf{p_i},\mathbf{p_j}) = $
    \begin{displaymath}
	\begin{aligned}
	    = & \sqrt{\left(\sum_{k = 1}^n (x_{ki} - x_{kj})^2 \right)}  \\
            \geq & \sqrt{\left(x_{ki} - x_{kj}\right)} \ \forall k = 1 \dots n \\
	     & = \lvert x_{ki} - x_{kj} \rvert \forall k = 1 \dots n
	\end{aligned}
    \end{displaymath}
    Hence, $\lvert x_{1i} - x_{1j} \rvert < \epsilon$. Thus $\left(x_{1i}\right)$ is cauchy.
    Similary $\left(x_{2i}\right)$ and so on. Since each $\left(x_{ki}\right) \in \mathbb{R}$, and
    $E^1$ is complete $x_{1i} \rightarrow x_1$ and so on. Let $\mathbf{p} = \left(x_1,x_2,\dots,
    x_k\right)$. Then since each $x_{ki}$ converges let $N = max(N_1,N_2,\dots,N_n)$ such that given
an $\epsilon > 0$, $d(x_{ki},x_k) < \epsilon/\sqrt(k)$ whenver $i \geq N$. Hence $\mathbf{p_i}
\rightarrow \mathbf{p}$.
\end{proof}
A good question to ask will be are all subspaces of complete metric spaces complete? The answer is
no since we saw that $\mathbb{R}_+$ was not complete. It was seen that $\left(1/n\right)$ did not 
converge since $0 \not \in \mathbb{R}_+$. What if we add all the \emph{boundary points} to the 
subspaces of complete metric space?

\begin{Proposition}[name=Closed subspace of complete spaces]
    Every closed subspace of a complete metric space is complete.
\end{Proposition}

\begin{proof}
    Let $\mathcal{C}$ be a closed subset of a complete metric space $E$. Given any cauchy sequence
    $\left(p_n\right) \in \mathcal{C}$, since $E$ is complete there is a $p \in E$ such that $p_n
\rightarrow p$. But since $\mathcal{C}$ is closed and $\left(p_n\right) \in \mathcal{C}$, since
$\left(p_n\right)$ converges, $p$ must belong to $\mathcal{C}$.
\end{proof}


Thus we see that finding complete metric spaces aren't cumbersome. Any closed subset of $E^n$ will
just do. However, we have seen that closed and bounded sets in $E^n$ possess some additional 
properties that just closed sets (for example closed rays in $E^1$) lack. For example, 
extreme value Theorem hold in closed and bounded sets of $E^1$ and fails in open sets and closed
rays. In exploring these nice properties we'll define an important concept called
\emph{compactness}.
\section{Compactness}
\begin{Definition}
    Given $S \subset E$, a collection $\left.\lbrace \mathcal{U}_i : i \in I \rbrace\right.$ of
    subsets of $E$ is called a cover of $S$ if $S \subset \bigcup_{i \in I} \mathcal{U}_i$. If each
    $\mathcal{U}_i$ is an open set in $E$ then we have an open cover of $S$.
\end{Definition}
\begin{Definition}
    A subset $K \subset E$ is compact if for every open cover of $K$ there is a finite subset $I_0
    \subset I$ such that the collection $\left.\lbrace \mathcal{U}_i : i \in I_0 \rbrace\right.$
    covers $K$.
\end{Definition}
Let us look at some examples :
\begin{itemize}
    \item Every finite set is compact.
	\begin{proof}
	    Given any open cover $\left.\lbrace \mathcal{U}_i : i \in I \rbrace\right.$ of a finite
	    set $K = \left.\lbrace p_j : j \in I_0\rbrace\right.$, each $p_j$ belongs to atleast one
	    $\mathcal{U}_i$. Hence, picking those gives us a finite subcovering.
	\end{proof}
    \item $E^1$ is not compact .
	\begin{proof}
	    To show that a set is not compact we need to find an open conver such that no finite
	    sub-collection of the open cover is a cover for the set. Let $\mathcal{U}_i =
	    \left(-\infty, i\right)$ where $i \in \mathbb{Z}_+$.  Then $\bigcup_{i \in \mathbb{Z}_+}\mathcal{U}_i = \mathbb{R}$.
	     Therefore $\left.\lbrace \mathcal{U}_i : i \in \mathbb{Z}_+ \rbrace\right.$ is an open
	    cover.However, no finite cover exists. To see this let $I_0$ be a finite subcollection 
	    of $\mathbb{Z}_+$ such that $\left.\lbrace \mathcal{U}_i : i \in I_0 \rbrace\right.$. 
	    Since $I_0 \subset \mathbb{Z}_+$, it maximum exist, lets call it $N$. Then $
	    \bigcup_{i \in I_0}\mathcal{U}_i = \left(-\infty,N\right)$. Hence, any $x \geq N$ is not
	    covered in this collection.
	\end{proof}
    \item $K = \left.\lbrace 1/n : n \in \mathbb{Z}_+\rbrace\right. \bigcup \left.\lbrace 0
	    \rbrace\right. \subset E^1$ is compact.  
	\begin{proof}
	    Given any open cover $\left.\lbrace \mathcal{U}_i : i \in I \rbrace\right.$ of K, there
	    is an $i \in I$ such that $0 \in \mathcal{U}_i$. Since $1/n \rightarrow 0$,
	$\mathcal{U}_i$ contains all but finitely many $1/n$. Hence, we have only finitely many
	points to cover.
	\end{proof}

    \item $S = \left.\lbrace 1/n : n \in \mathbb{Z}_+\rbrace\right. \subset E^1$ is not compact.
	\begin{proof}
	    For any $i \in \mathbb{Z}_+$, let $\mathcal{U}_i = \left(1/i+1 , 1/i-1\right)$. For any
	    $i$, $\mathcal{U}_i \bigcap S = S$, hence $\left.\lbrace \mathcal{U}_i : i \in
		\mathbb{Z}_+ \rbrace\right.$ is an open conver of $S$, but no finite subcover exist.
	\end{proof}

\end{itemize}
\begin{Definition}
    A metric space $(E,d)$ is compact if it is a compact subset of itself.
\end{Definition}
Before, we show that compactness is an inherent property we need to show that open\emph{ness} of a
set is not inherent to the set. To see this note that a set $E \subset X$ is open (w.r.t $X$) if
for any point $p \in E$ there is an $r > 0$ such that $\mathcal{B}_r(p) \subset E$. Now, if we have
a subspace $Y \subset X$ such that $E \subset Y$, then $E$ is open w.r.t $Y$ whenever $
\mathcal{B}_r(p) \bigcap Y \subset E$. It is easy to show that that $ E \subset Y \subset X$ is open
relative to Y iff $E = Y \bigcap \mathcal{U}$ for $\mathcal{U}$ open in $X$. Thus the open interval
$E = (a,b)$ is open w.r.t $E^1$ but not w.r.t $E^2$.


\begin{Theorem}[name=relative open sets]
    Suppose $Y \subset X$. A subset $E \subset Y$ is open relative to $Y$ iff $E  = Y \cap
    \mathcal{U}$ for some $\mathcal{U}$ open in $X$.
\end{Theorem}
\begin{proof}
    Suppose $E \subset Y$ be such that $E$ is open relative to $Y$. Let $p \in E$ be an arbitrary
    point in $E$. Then there is an $r(p) > 0$ such that $\mathcal{B}_{r(p)}(p) \cap Y \subset E$.
    Let $\mathcal{U} = \bigcup_{p \in E} \mathcal{B}_{r(p)}(p)$. Then $\mathcal{U}$ is open in $X$.
    For any $p \in E$, $p \in \mathcal{U} \cap Y$ and vice versa. Thus $E = Y \cap \mathcal{U}$.

    Let $E = Y \cap \mathcal{U}$ for some $\mathcal{U}$ open in $X$. Let $p \in Y$ be an arbitrary 
    element of $\mathcal{U}$ i.e $p \in Y$ and $p \in \mathcal{U}$. 
    Since $\mathcal{U}$ is open there is an $r(p) >
    0$ such that $\mathcal{B}_{r(p)}(p) \subset \mathcal{U}$. Hence 
    $\mathcal{B}_{r(p)}(p) \cap Y \subset E$.
\end{proof}

\begin{Theorem}[name=Compactness is inherent]
    A subset $K \subset Y \subset X$ is compact relative to $Y$ iff it is compact relative to $X$. 
\end{Theorem}
\begin{Corollary}
    A subset $K \subset E$ is compact iff it is compact as a metric space.
\end{Corollary}
Hence we can talk about compactness of a set without ambiguity. 
Before we look into some of the
interseting proporties of compact sets, let us define a few other important concepts in metric space
Now, lets look at some properties of compact sets (or compact metric spaces).
\begin{Proposition}[name=Properties of compact sets]
    Let $(E,d)$ be any metric space. Then,
    \begin{enumerate}
	\item If $E$ is compact then every closed subset of $E$ is compact.
	\item Every compact subset of $E$ is bounded.
	\item (Finite Intersection property.) If $E$ is compact and $\left.\lbrace 
		\mathcal{C}_{\alpha} : 
		\alpha \in I \rbrace\right.$ be a collection of non-empty closed sets in $E$ such
	    that any finite instersection of the collection is non-empty then \[ \bigcap_{\alpha \in
		    I}\mathcal{C}_{\alpha} \neq \emptyset.\] 
	\item If $E$ is compact then every infinite subset of $E$ has a cluster point in $E$.
	    \begin{itemize}
		\item
		    A \textbf{corollary} of this gives us \emph{sequential} compactness ie if $E$ is
		    compact then every sequence has a convergent subsequence.
	    \end{itemize}
	\item Every compact subset of $E$ is closed.
    \end{enumerate}
\end{Proposition}
\begin{proof}
    \begin{enumerate}
	\item Given $E$ compact, let $\mathcal{C} \subset E$ such that $\mathcal{C}$ is closed in
	    $E$. Let $\left.\lbrace \mathcal{U}_i : i \in I \rbrace\right.$ be an open cover of
	    $\mathcal{C}$. Since $\mathcal{C}$ is closed $E - \mathcal{C}$ is open. Moreover, 
	    $\left.\lbrace \mathcal{U}_i : i \in I \rbrace\right. \bigcup E - \mathcal{C} $ is an
	    open cover of $E$. Since $E$ is compact, let $I_0 \subset I$ be a finite set then
	   $\left.\lbrace \mathcal{U}_i : i \in I_0 \rbrace\right. \bigcup E - \mathcal{C} $ also 
	   covers $E$ and hence $\mathcal{C}$. But, $E - \mathcal{C}$ doesn't cover $\mathcal{C}$. 
	   Thus, \[\mathcal{C} \subset \bigcup_{i \in I_0}\mathcal{U}_i . \] Therefore, 
	   $\mathcal{C}$ is compact.  
       \item Fix any $p \in E$. Let, $\mathcal{U}_n = \mathcal{B}_n(p)$ such that $n \in
	   \mathbb{Z}_+$. Hence, $\left.\lbrace \mathcal{U}_n : n \in \mathbb{Z}_+ \rbrace\right.$
	   is an open cover of $E$.  Let $K \subset E$ such that
	   $K$ is compact. Then, $\left.\lbrace \mathcal{U}_n : n \in \mathbb{Z}_+ \rbrace\right.$
	   also covers $K$ and hence there is a $I_0 \subset \mathbb{Z}_+$ with $I_0$ finite, such
	   that \[ K \subset \bigcup_{i \in I_0}\mathcal{B}_i(p) .\] Since, $I_0$ is a finite subset
	   of poisitive integers, a maximum exists. Let $N = \text{ max}(I_0)$. Then, 
	   $\bigcup_{i \in I_0}\mathcal{B}_i(p) = \mathcal{B}_N(p)$. Therefore, $K \subset 
	   \mathcal{B}_N(p)$ and hence $K$ is bounded. 
       \item Suppose,  \[ \bigcap_{\alpha \in I}\mathcal{C}_{\alpha} = \emptyset .\] Then 
	   $\bigcup_{\alpha \in I}\left(E - \mathcal{C}_{\alpha}\right) = E - \bigcap_{\alpha \in I}
	   \mathcal{C}_{\alpha} = E$. Thus $\left.\lbrace E - \mathcal{C}_{\alpha} : \alpha \in I
	       \rbrace\right.$ is an open cover for $E$. Since, $E$ is compact, let $I_0 \subset I$
	   be a finite set such that 
	   \begin{displaymath}
	       \begin{aligned}
		   & E \subset \bigcup_{\alpha \in I_0}E - \mathcal{C}_{\alpha} \\
		   & \Leftrightarrow E \subset E - \bigcap_{\alpha \in I_0} \mathcal{C}_{\alpha}. 
	       \end{aligned}
	   \end{displaymath} 
	   But, this means $\bigcap_{\alpha \in I_0} \mathcal{C}_{\alpha} = \emptyset$ which
	   contradicts the hypothesis that intersection of any finite subcollection is non-empty.
       \item $E$ is compact, let $S \subset E$ such that $S$ is infinite but it doesn't have any
	   cluster point in $E$ i.e no point of $E$ is a cluster point of $S$. Hence, $\forall p \in
	   E$, there is an $\epsilon(p) > 0$ such that $\mathcal{B}_{\epsilon(p)}\left(p\right)
	   \bigcap S$ contains only finite points of $S$. Since any $p \in E$ also belongs to 
	   $\mathcal{B}_{\epsilon(p)}\left(p\right)$, it is easy to see that \[ E \subset \bigcup_{p
		   \in E} \mathcal{B}_{\epsilon(p)}\left(p\right) .\] Since open balls are open sets,
	       the set $\left.\lbrace \mathcal{B}_{\epsilon(p)}\left(p\right) : p \in E
		   \rbrace\right.$ is an open cover of $E$. But since $E$ is compact, $E$ is also
	       covered by a finite subcover (dropping parenthesis around $\epsilon$) 
	       \begin{displaymath}
		   \begin{aligned}
		       & S \subset E \subset \bigcup_{i = 1}^n
		       \mathcal{B}_{\epsilon }\left(p_i\right) \\
		       & \implies S = \bigcup_{i = 1}^n \left( \mathcal{B}_{\epsilon}
			   \left(p_i \right) \bigcap S \right)
		   \end{aligned}
	       \end{displaymath} 
	        But each
	       $\left( \mathcal{B}_{\epsilon }\left(p_i \right) \bigcap S \right)$ is finite
	       and hence the finite union is a finite set. Thus $S$ is finite hence giving us a
	       contradiction.
	       \begin{itemize}
		   \item Lets proof sequential compactness. 
		       \begin{proof}
			   $E$ is compact and let $\left(p_n\right)$ be any sequence in $E$. If the
			   range of $\left(p_n\right)$ i.e the set $\left.\lbrace p_n : n \in
			       \mathbb{Z}_+ \rbrace\right.$ is finite then we pick any $p_k$ in the
			   set and we get a constant subsequence that converges to $p_k$. If the
			   range is infinite, then from the above proof it must have a cluster
			   point, let us say $p \in E$. Hence, we can construct a sequence as
			   follows. Since any ball about $p$ must intersect the set $\left.\lbrace 
			       p_n : n \in \mathbb{Z}_+ \rbrace\right.$ let $n_1 \in \mathbb{Z}_+$ such
			   that $p_{n_1} \in \mathcal{B}_1(p)$. Since the range is infinite we can
			   get a $n_2 > n_1$ such that $p_{n_2} \in \mathcal{B}_{1/2}(p)$. Given any
			   $\epsilon > 0$ there is a $N \in \mathbb{Z}_+$ such that $1/N <
			   \epsilon$. Thus for any $n_i \geq N$ we get $d(p_{n_i},p) < 1/n_i <
			   \epsilon$. Hence $p_{n_i} \rightarrow p$.
		       \end{proof}
	       \end{itemize}
	   \item Let $K \subset E$ such that $K$ is compact. Let $\left(p_n\right)$ be a sequence in
	       K such that $p_n \rightarrow p$ where $p \in E$. Since the sequence converges to $p$
	   every subsequence must also converge to $p$. But we know that $K$ is sequentially compact
	   and so atleast one subsequence must converge to a point in $K$. From uniqueness of limits
	   this point has to be $p$. Thus $p \in K$. Hence, $K$ is closed.
   \end{enumerate}
\end{proof}
Two important corollaries of the above proposition can be obeserved.
\begin{Corollary}
    If $E$ is a compact metric space, then
    \begin{enumerate}
	\item (Nested set property) Let $\left.\lbrace\mathcal{C}_i : i \in \mathbb{Z}_+
		\rbrace\right.$ be a collection of
	    non-empty closed sets in $E$ such that $\mathcal{C}_1 \supset \mathcal{C}_2 \dots$
	    then \[\bigcap_{i = 1}^{\infty} \mathcal{C}_i \neq \emptyset .\]
	    \begin{proof}
		Follows from the finite intersection property of the above proposition.
	    \end{proof}
	\item $E$ is complete.
	    \begin{proof}
		Let $\left(p_n\right)$ be a cauchy sequence in E. There is a subsequence that
		converges to a point in $E$. But we showed that a cauchy sequence with a convergent
		subsequence is itself convergent. Hence $E$ is complete.
	    \end{proof}
    \end{enumerate}
\end{Corollary}
Hence, we have seen that compact sets possess additional properties that complete sets don't. We
have shown that compact sets are closed and bounded. It turns out that the reverse implication is
also true for $E^n$. To show that, we'll define what is means to be a \emph{totally} bounded set in
a metric space.

\begin{Definition}
    A set $S \in E$ is totally bounded if for any $\epsilon > 0$, $S$ can be covered by finitely
    many \emph{closed} $\epsilon -$ balls.
\end{Definition}
To show that any bounded subset of $E^n$ is also totally bounded we'll first need an important lemma
that $E^n$ is covered by a countable collection of closed $\epsilon -$ balls. Note that if $\epsilon
> 0$ is given then $\bigcup_{\mathbf{p} \in E^n}\overline{\mathcal{B}_{\epsilon}}(\mathbf{p})$ is
also a cover of $E^n$ although it is an uncountable collection. The next lemma shows that $E^n$ can
be covered by a countable collection of closed balls.
\begin{Lemma}[name=Covering $E^n$ by a countable collection of closed balls]
    For any $\epsilon > 0 \in \mathbb{R}$, $E^n$ is contained in a countable collection of closed
    balls.
\end{Lemma}
\begin{proof}
    We know that for any $x \in \mathbb{R}$ and for any $N \in \mathbb{Z}_+$ there is a $k \in
    \mathbb{Z}$ such that \[ \frac{k}{N} \leq x < \frac{k+1}{N} .\] Thus, given any $\mathbf{x} =
    \left(x_1,x_2,\dots,x_n\right) \in E^n$, we can find $a_i \in \mathbb{Z}$ such that $\forall i =
    1 \dots n$, \[ \frac{a_i}{N} \leq x_i < \frac{a_i+1}{N} .\] Let $\mathbf{a} =
    \left(a_1/N,a_2/N,\dots,a_n/N\right)$. Then, $d(\mathbf{x},\mathbf{a}) = \lvert\lvert \mathbf{x} -
    \mathbf{a} \rvert\rvert$ which is given by 
    \[\sqrt{\left(x_1 - a_1/N\right)^2 + \dots + \left(x_n - a_n/N\right)^2} .\] Since each $x_i -
    a_i/N < 1/N$, $d(\mathbf{x},\mathbf{a}) < \frac{\sqrt{n}}{N}$. Since $N$ is also arbitrary, for a
    given $\epsilon$ we can choose $N$ such that $1/N < \epsilon/ \sqrt{n}$. Then $
    d(\mathbf{x},\mathbf{a}) < \epsilon$. Thus $\mathbf{x} \in \overline{B}_{\epsilon}(\mathbf{a})$. 
    By varying $\mathbf{a} \in \mathbb{Z}^n$ we can cover any $\mathbf{x} \in E^n$. Thus we get a
    countable collection (since  $\mathbb{Z}^n$ is countable)  of closed balls that covers $E^n$.
\end{proof}
\begin{Lemma}[name=Bounded sets in $E^n$ are totally bounded]
    Every bounded subset of $E^n$ is totally bounded.
\end{Lemma}
\begin{proof}
    Let $S \subset E^n$ such that $S$ is bounded. Using the proof above we need to show that 
    only finite choices of $\mathbf{a}$ exists such that $S \subset \bigcup_{i = 1}^M
    \overline{\mathcal{B}_{\epsilon}}(\mathbf{a}_i)$. Since $S$ is bounded, there is a $R > 0$ such
    that $S \subset \mathcal{B}_{R}(\mathbf{0})$. This means for any $\mathbf{x} =
    \left(x_1,x_2,\dots,x_n\right) \in S$, $d(\mathbf{x},\mathbf{0}) < R$. Hence, \[\sqrt{x_1^2 +
	    x_2^2 + \dots + x_n^2} < R .\] Therefore for all $i = 1 \dots n$, $\lvert x_i \rvert <
    R$. But for this $i$ there is a $a_i \in \mathbb{Z}$ such that $\lvert a_i/N - x_i \rvert <
    1/N$. Thus $\lvert a_i \rvert < N*R + 1$ for all $i = 1 \dots n$. Thus there are only finite 
    choices for these $a_i$ and hence only finite such $\mathbf{a}$ are possible. 
      
\end{proof}

Now we can show that for sets in $E^n$ closed and boundedness togather implies compactness. Note
that this is in general not true. The following examples illustrate this counterpoint.
\begin{itemize}
    \item
	$E$ be the discrete metric space. Then any $S \subset E$ is always closed and bounded. But
	$S$ is compact only when it is finite.
    \item We need both closed and boundedness in $E^n$ to have compactness. For example $(a,b)$ is
	bounded but not closed and $[a,\infty)$ is closed but not bounded. Both of these are not
	compact. 	
\end{itemize}

\begin{Theorem}[name=Hiene Borel]
    A subset $S \subset E^n$ is compact iff it is closed and bounded.
\end{Theorem}
\begin{proof}
    $\Rightarrow$ proved as a property of compact sets in general metric space. 

    $\Leftarrow$
	We'll prove by contradiction. Suppose $S \subset E^n$ is closed and bounded but not compact.
	Hence there is an open cover $\left.\lbrace \mathcal{U}_i : i \in I \rbrace\right.$ such
	that no finite subcover exists. Since $S$ is bounded in $E^n$, it is also totally bounded.
	Thus for $\epsilon = 1/2$,
	\begin{displaymath}
	    \begin{aligned}
		& S \subset \bigcup_{j = 1}^M \overline{\mathcal{B}_{1/2}}(\mathbf{a}_j) \\
		& \implies S = \bigcup_{j = 1}^M \overline{\mathcal{B}_{1/2}}(\mathbf{a}_j) 
		\bigcap S \\
		& \implies S = \bigcup_{j = 1}^M \left(\overline{\mathcal{B}_{1/2}}(\mathbf{a}_j) 
		    \bigcap S\right)
	    \end{aligned}
	\end{displaymath}
	for $\mathbf{a}_j \in \mathbb{Z}^n$. Since $S$ is closed each $\left(\overline{\mathcal{B}_{1/2}}
	    (\mathbf{a}_j) \bigcap S\right)$ is closed. Aslo $S$ is not covered by any finite
	sub-collection of $\mathcal{U}_i$ hence atleast one such $\left(\overline{\mathcal{B}_{1/2}}
	    (\mathbf{a}_j) \bigcap S\right)$, lets call it $S_1$ is not covered by any finite
	sub-collection of $\mathcal{U}_i$. For any $\mathbf{p},\mathbf{q} \in \overline{\mathcal{B}_{1/2}}
	(\mathbf{a}_j)$, $d(\mathbf{p},\mathbf{q}) < d(\mathbf{p},\mathbf{a}_j) +
	d(\mathbf{a}_j, \mathbf{q})$. Hence $d(\mathbf{p},\mathbf{q}) < 1$. Thus we can obeserve
	the following about $S_1$ :
	\begin{enumerate}
	    \item $S_1 \subset S$ is a closed and bounded subset of $E^n$.
	    \item $S_1$ is covered by $\left.\lbrace\mathcal{U}_i\rbrace\right.$ but not by any of 
		its finite sub-collection.
	    \item For any $\mathbf{p},\mathbf{q} \in S_1$ we have $d(\mathbf{p},\mathbf{q}) < 1$.
	\end{enumerate}

	Thus we see from 1-2 $S_1$ has the same properties as $S$. Thus we can do the similar thing
	to $S_1$ and get for $\epsilon = 1/4$ an $S_2$ such that :

	\begin{enumerate}
	    \item $S_2 \subset S_1 \subset E^n$ is a closed and bounded subset of $E^n$.
	    \item $S_2$ is covered by $\left.\lbrace\mathcal{U}_i\rbrace\right.$ but not by any of 
		its finite sub-collection.
	    \item For any $\mathbf{p},\mathbf{q} \in S_2$ we have $d(\mathbf{p},\mathbf{q}) < 1/2$.
	\end{enumerate}

	After repeated application of the same procedures we get :

	\begin{enumerate}
	    \item $S \supset S_1 \supset S_2 \dots $ closed and bounded subsets of $E^n$.
	    \item Each $S_N$ is covered by $\left.\lbrace\mathcal{U}_i\rbrace\right.$ but not by 
		any of its finite sub-collection.
	    \item For any $\mathbf{p},\mathbf{q} \in S_N$ we have $d(\mathbf{p},\mathbf{q}) < 1/N$.
	\end{enumerate}
	No $S_N$ is emplty since \# 2 will be vacuously false, hence let $\mathbf{p}_N \in S_N$ for
	$N = 1 \dots $. The sequence $\left(\mathbf{p}_N\right)$ is then cauchy since for any $\alpha,\beta
	> N$ , $\mathbf{p}_{\alpha},\mathbf{p}_{\beta} \in S_N$ (because $S_N \supset S_{\alpha}$ 
	and similarly $S_N \supset S_{\beta}$). Thus $d(\mathbf{p}_{\alpha},\mathbf{p}_{\beta}) <
	1/N$ which can be made arbitrarily small. $E^n$ is complete hence $\mathbf{p}_N \rightarrow
        \mathbf{p}$. But $\left(\mathbf{p}_N\right) \in S$ and $S$ closed therfore $\mathbf{p} \in S$.
	This contradicts \# 2 that a finite subcover of S from $\left.\lbrace\mathcal{U}_i\rbrace\right.$
	cannot exists. To see this since $\mathbf{p} \in S$, it must be in some $\mathcal{U}_0$ in
	the collection. But since $\mathcal{U}_0$ is open there is an $\epsilon > 0$ such that
	$\mathcal{B}_{\epsilon}(\mathbf{p}) \subset \mathcal{U}_0$. We can find an $N$ such that
	$1/N < \epsilon$ which would mean $\mathcal{B}_{1/N}(\mathbf{p}) \subset \mathcal{U}_0$. But
	any $\mathbf{p}_N \in \mathcal{B}_{1/N}(\mathbf{p})$, $p_N \in S_N$. Thus
	$S_N$ is covered by $\mathcal{U}_0$, which is a finite sub-collection of $\left.
	    \lbrace\mathcal{U}_i\rbrace\right.$ 

\end{proof}
\section{Connectedness}
We saw how $E = E^1 - \lbrace 0 \rbrace $ was a union of two non-empty sets that were both closed
and open. The space seemed to be split or unconnected. It turn out that connectedness is an
important property. While the extrem value Theorem served as a motivation for introducing
compactness, the intermediate value Theorem relies on the connectedness of the metric space. 

\begin{Definition}
    A metric space $(E,d)$ is connected if the only subsets of $E$ that are both open and closed
    are $\emptyset$ and $E$. A subset $S \subset E$ is connected if it is connected as a metric
    space.
\end{Definition}

\begin{Theorem}[name=Connectedness of a metric space]
    $E$ is not connected (separated) iff $E = A \cup B$ such that $A,B$ are disjoint,open and
    non-empty.\end{Theorem}
\begin{proof}
     

    $\Rightarrow$ $E$ is not connected, thus there is a $A \subset E$ which is not $\emptyset$ and 
    $E$ but is both open an closed. Let $B = E - A$. Since $A \neq \emptyset$, $B \neq E$. Since $A
    \neq E$, $B \neq \emptyset$. Moreover since $A$ close $B$ is open and $A \cup B = E$.
    $\Leftarrow$ Similar argument
\end{proof}
The notion of connectedness in $\mathbb{R}$ is fairly simple. Before we show that $\mathbb{R}$ is
connected we need to know when a real number is \emph{between} 2 real numbers. If $a,b,c \in 
\mathbb{R}$, we say that $c$ is between $a,b$ if $a < c < b$ or $b < c < a$.
\begin{Proposition}[name=Connectedness of open segments in $\mathbb{R}$]
    Any subset of $\mathbb{R}$ which contains two distinct points $a,b$ but doesn't contain all the
    points between $a,b$ is not connected.
\end{Proposition}
\begin{proof}
    First let us note that for any $A \subset S \subset E$ , we stated earlier that $A$ is open
    relative to $S$ iff $A = S \cap \mathcal{U}$ for some $\mathcal{U}$ open in $E$. Now $S$
    contains two distinct points $a,b$ but not all the points in between them. WLOG, let $a < b$.
    Hence, there is a $c \in \mathbb{R}$, $a < c < b$ such that $c \not \in S$. Clearly $S \subset \mathbb{R} -
    \lbrace c \rbrace$. Thus $S = S \cap \left( \mathbb{R} - \lbrace c \rbrace \right)$. Since
    $\mathbb{R} - \lbrace c \rbrace = \left(-\infty,c\right) \cup \left(c,\infty\right)$, we can
    write \[ S = \left(S \cap \left(-\infty,c\right) \right) \cup \left(S \cap
	    \left(c,\infty\right)\right).\]
    Let $A = S \cap \left(-\infty,c\right)$. Since $\left(-\infty,c\right)$ open in $E^1$ 
    ($\mathbb{R}$), $A$ is open relative to $S$. Similarly $B = S \cap \left(c,\infty\right)$ is
    open relative to $S$. Since $S$ is non-empty $A,B$ are non-empty. Moreover $A,B$ are disjoint
    and $S = A \cup B$. Thus $S$ is not connected.
\end{proof}
The next proposition shows how to obtain connected set from a collection of connected sets.
\begin{Proposition}[name=Collection of connected sets]
    Suppose $\left.\lbrace S_i : i \in I \rbrace\right.$ be a collection of connected subsets of $E$
    such that there is a $i_0 \in I$, such that $\forall i \in I$, $S_i \cap S_{i_0} \neq
    \emptyset$. Then $\bigcup_{i \in I}S_{i}$ is connected.
\end{Proposition}
\begin{proof}
    It is always easy to show a set is not disconnected by assuming it is. Hence, we'll do a proof
    by contradiction. Suppose $\bigcup_{i \in I}S_{i}$ is disconnected. Then $\bigcup_{i \in
	I}S_{i} = A \bigcup B$ where $A,B$ are disjoint,non-empty open sets. Hence, $\forall i \in
    I$ we have that $S_i \subset A \bigcup B$. Thus $S_i = \left(S_i \cap A\right) \bigcup \left(S_i
	\cap B\right)$. Since, each $S_i$ is separated and we assumed $A,B$ are disjoint either
    $\left(S_i \cap A\right) = \emptyset$ or $\left(S_i \cap B\right) = \emptyset$. WLOG, assume
    $\left(S_{i_0} \cap B\right) = \emptyset$. Thus $S_{i_0} \subset A$. But since $\forall i \in
    I$, $S_i \cap S_{i_0} \neq \emptyset$, $S_i \subset A$ and thus $S_i \cap B = \emptyset$.
    Hence, $\bigcup_{i \in I}S_{i} \subset A$. This means that $A \bigcup B \subset A$. Since $A,B$
    are disjoint $B = \emptyset$. This is a contradiction. 

\end{proof}
