\documentclass{article}
\usepackage{standalone}
\usepackage{tikz}
\usetikzlibrary{shapes,backgrounds}
\begin{document}

\def\firstcircle{(0,0) circle (1.5cm)} %A
\def\secondcircle{(1.75,1.25) circle (1.5cm)} % E
\def\thirdcircle{(8,0) circle (1.5cm)} % E
\def\fourthcircle{(8,0) circle (2.5cm)} % A

% Now we can draw the sets:
\begin{tikzpicture}
    \begin{scope}[shift={(3cm,0cm)},fill opacity=0.65]
	% clip secondcircle and fill the first circle
	% clip E and then fill the first circle will result in A - E
	% clip against a rectangle that contains both the circle.
	\begin{scope}[even odd rule]
	    \clip \secondcircle (-3,-3) rectangle (3,3);
	    \fill[red] \firstcircle;
	\end{scope}
	% clip A and then fill E but dont use even odd rule. Will hightlight the intersection.
	\begin{scope}
	    \clip \firstcircle;
	    \fill[green] \secondcircle;
	\end{scope}
	% Now draw the outline
	\draw \firstcircle node {$A\cap E^{\mathsf{c}}$};
	\draw[green] \secondcircle node{$E$}; 
	\node at (0,-2) {$A$};
	% A - E
	\begin{scope}[even odd rule]
	    \clip \thirdcircle (2,-6) rectangle (12,6);
	    \fill[red] \fourthcircle;
	\end{scope}
	% fill A \cap E
	\begin{scope}
	    \clip \thirdcircle;
	    \fill[green] \fourthcircle;
	\end{scope}
	% draw the outline
	\draw \fourthcircle;
	\draw \thirdcircle node {$E$};
	\node at (8,-3) {$A$};
    \end{scope}
\end{tikzpicture}
\end{document}
