\chapter{Preliminary Concepts: Set theory}
\section{Foundations of set theory}
In this section, we will cover the basics of axiomatic set theory and will conclude with the 
Axiom of Choice. Most proofs will be omitted.
The axiomatic theory of set is just based two (undefined) notions \textit{class} and the 
\textit{membership relation} denoted by $\in$. All objects are classes. However there are two kinds
of classes:
\begin{itemize}
    \item Sets
    \item Proper Classes
\end{itemize}
If $x,A$ are classes then the expression $x \in A$ means that $x$ is an element of $A$. This leads
to our first definition
\begin{Definition}
    Let $x$ be a class. If $x$ is belongs to some class $A$ then $x$ is called an element.
\end{Definition}
All elements are denoted by lower case letters. Hence whenever we write $x,y,z$ we mean classes that
belong to some class. Whenever we denote classes by capital letters $A,B,C$ then such a class may be
an element of some other class or may not be an element at all.

\begin{Definition}
    Let $A,B$ be classes. We define $A = B$ to mean that every class that has $A$ as its element
    must have $B$ as an element and vice versa. Logically,
    \begin{equation*}
	A = B \, \text{iff} \, (\forall X)\left[ A \in X \implies B \in X \land B \in X \implies A \in X
	\right].
    \end{equation*}
\end{Definition}

The first axiom is called the \textbf{axiom of Extent} and is an equivalent statement of the above
definition.
\begin{enumerate}[label=\bfseries Axiom 1:]
    \item $A = B$ iff $x \in A \iff x \in B$.  
\end{enumerate}

\begin{Definition}
    Let $A,B$ be classes; we define $A \subseteq B$ to mean that every element of $A$ is an element of
    $B$. $A$ is called a subclass of $B$.
\end{Definition}

The second axiom is called the \textbf{axiom of class construction} and defines way to construct 
sets from elements. For this we define a \textit{property} as,
\begin{Definition}
    A property $P(x)$ is a mathematical statement involving an element $x$ such that it can be
    expressed entirely in terms of the logical symbols $\in, \land, \lor, \lnot, \exists, \forall$
    and variables $x,y,z,A,B \dots$. 
\end{Definition}

\begin{enumerate}[label=\bfseries Axiom 2:]
    \item If $P(x)$ is a property then there exists a class $C$ whose elements are precisely those
	that satisfy $P(x)$. Logicall we denote $C$ as,
	\begin{equation*}
	    C = \left.\lbrace x : P(x) \rbrace\right..
	\end{equation*}
\end{enumerate}

If $A$ and $B$ are classes then the following properties gives very important classes,
\begin{enumerate}
    \item $P(x)$ is $x \in A \lor x \in B$.
    \item $P(x)$ is $x \in A \land x \in B$.
\end{enumerate}

The class satisfying the first property is called the union of $A,B$ and is denoted by $A \cup B$.
The class satisfying the second property is called the intersection of $A,B$ and is denoted by 
$A \cap B$.

\begin{Definition}
    The universal class $\mathcal{U}$ is the class of all the elements. Thus it contains classes
    that belong to some class. Thus, 
    \begin{equation*}
	\mathcal{U} = \left.\lbrace x : x = x \rbrace\right.
    \end{equation*}
\end{Definition}

\begin{Definition}
    The empty class $\emptyset$ is the class that has no elements. Thus,
    \begin{equation*}
	\emptyset = \left.\lbrace x : x \neq x \rbrace\right.
    \end{equation*}
\end{Definition}

\begin{Definition}
    If two classes $A,B$ have no elements in common, they are said to be disjoint. Thus, $A,B$ are
    disjoint if,
    \begin{equation*}
	A \cap B = \emptyset.
    \end{equation*}
\end{Definition}

\begin{Definition}
    The complement of a class $A$, $\comp{A}$ is the class of all elements that do not belong to $A$.
    Thus,
    \begin{equation*}
	\comp{A} =\left.\lbrace x : x \not\in A \rbrace\right.
    \end{equation*}
\end{Definition}

Note that for any class $A$, $\emptyset \subseteq A$, $A \subseteq \mathcal{U}$ and $A \cup
\comp{A} = \mathcal{U}$ and $A \cap \comp{A} = \emptyset$.
The Demorgan Laws provide a duality about union and intersection,
\begin{itemize}
    \item $\comp{(A \cup B)} = \comp{A} \cap \comp{B}$.
    \item $\comp{(A \cap B)} = \comp{A} \cup \comp{B}$.
\end{itemize}

It is very important fact that union, intersection and complement describe an algebra of classes
which is summarized below (easily proven),

\begin{itemize}
    \item Identity Laws:
	\begin{itemize}
	    \item $A \cup A = A$.
	    \item $A \cap A = A$.
	\end{itemize}
    \item Associative Laws:
	\begin{itemize}
	    \item $A \cup ( B \cup C) = (A \cup B)\cup C $.
	    \item $A \cap ( B \cap C) = (A \cap B)\cap C $.
	\end{itemize}
    \item Commutative Laws:
	\begin{itemize}
	    \item $A \cup B = B \cup A$.
	    \item $A \cap B = B \cap A$.
	\end{itemize}
    \item Distributive Laws:
	\begin{itemize}
	    \item $A \cup (B \cap C) = (A \cup B) \cap (A \cup C)$.
	    \item $A \cap (B \cup C) = (A \cap B) \cup (A \cap C)$.
	\end{itemize}
\end{itemize}

\begin{Definition}
    The difference of two classes $A,B$, $A - B$ is the class of those elements that are elements of $A$ but
    not of $B$. Thus,
    \begin{equation*}
	A - B = \left.\lbrace x : x \in A \land x \not\in B \rbrace\right.
    \end{equation*}
\end{Definition}
Note that $A - B = A \cap \comp{B}$.

If $a$ is an element then from Axiom $2$ we can construct the class that contains only $a$. Such a
class is called a \textit{singleton} and is given by:
\begin{equation*}
    \lbrace a \rbrace =\lbrace x : x = a \rbrace.
\end{equation*}

Similarly we can create the \textit{un-ordered} pair $\lbrace a,b \rbrace$ which is,
\begin{equation*}
    \lbrace a,b \rbrace = \lbrace x : x = a \lor x = b \rbrace.
\end{equation*}

It is easy to see that $\lbrace a,b \rbrace = \lbrace c,d \rbrace$ 
when $(a = c) \lor (b = d)$ or $ (a = d) \lor (b = c)$. Something that is very important is the notion of
ordered pair which we denote by $(a,b)$. What is important about ordered pairs is that if 
$(a,b) = (c,d)$ the $a = c$ and $b = d$. Thus the \textit{order} in which they appear in the set is
important.

\begin{Definition}
    If $a,b$ are elements, then the ordered pair is the class given by,
    \begin{equation*}
	(a,b) = \left.\lbrace \lbrace a \rbrace, \lbrace a,b \rbrace \rbrace\right.
    \end{equation*}
\end{Definition}

\begin{Definition}
    The Cartesian product of two classes $A$ and $B$ denoted by $A \times B$ 
    is the class of all ordered pairs $(x,y)$ where $x \in A$ and $y \in B$. Thus,
    \begin{equation*}
	A \times B = \left.\lbrace (x,y) : x \in A \land y \in B \right.\rbrace.
    \end{equation*}
\end{Definition}

A class of ordered pairs is called a \textit{Graph}. Thus any subclass of $\mathcal{U} \times
\mathcal{U}$ is a graph. If $G$ is a graph, we denote $G^{-1}$ to be the inverse graph given by,
\begin{equation*}
    G^{-1} = \left.\lbrace (y,x) : (x,y) \in G \right.\rbrace.
\end{equation*}

\begin{Definition}
    If $G,H$ are graphs, then $G \circ H$ is the graph defined as follows:
    \begin{equation*}
	G \circ H = \set{(x,y)}{\exists z \ni (x,z) \in H \, \land \, (z,y) \in G}
    \end{equation*}
\end{Definition}

\begin{Theorem}
    If $G,H,J$ are graphs, then the following hold:
    \begin{enumerate}
	\item $(G \circ H)\circ J = G \circ (H \circ J)$.
	\item ${(G^{-1})}^{-1} = G $.
	\item ${(G \circ H)}^{-1} = H^{-1} \circ G^{-1}$.
    \end{enumerate}
\end{Theorem}
\begin{proof}
    We prove in order,
    \begin{enumerate}
	\item let $(x,y) \in (G \circ H)\circ J$. Then there is a $ z \ni (x,z) \in J$ and 
	    $(z,y) \in (G \circ H) $. Thus there is a $u \ni (z,u) \in H$ and $(u,y) \in G$. Thus $(x,u) \in 
	    H \circ J$. And so $(x,y) \in G \circ (H \circ J)$. The argument can be reversed and 
	    so we get the equality of classes. 
	\item Let $(x,y) \in {(G^{-1})}^{-1}$. Hence $(y,x) \in G^{-1}$. Thus $(x,y) \in G$. The
	    other direction is similar.
	\item Let $(x,y) \in {(G \circ H)}^{-1}$. Hence $(y,x) \in (G \circ H)$. Thus there is a 
	    $z \ni (y,z) \in H$ and $(z,x) \in G$. Thus $(z,y) \in H^{-1}$ and $(x,z) \in G^{-1}$, 
	    which is just $(x,z) \in G^{-1}$ and $(z,y) \in H^{-1}$. Hence 
	    $(x,y) \in H^{-1} \circ G^{-1}$.
    \end{enumerate}
\end{proof}

\begin{Definition}
    Let $G$ be a graph. By the domain of $G$ we mean the class
    \begin{equation*}
	dom\,G = \set{x}{\exists y \ni (x,y) \in G}.
    \end{equation*}
\end{Definition}

\begin{Definition}
    Let $G$ be a graph. By the range of $G$ we mean the class
    \begin{equation*}
	range\,G = \set{y}{\exists x \ni (x,y) \in G}.
    \end{equation*}
\end{Definition}

\begin{Theorem}
    IF $G,H$ are graphs then
    \begin{enumerate}
	\item $dom\,G = range\,G^{-1}$.
	\item $dom\,G^{-1} = range\,G$.
	\item $dom\,(G \circ H) \subseteq dom\,H$.
	\item $range\,(G \circ H) \subseteq range\,G$.
    \end{enumerate}
\end{Theorem}

The proof of the third statement is as follows,
\begin{proof}
    Let $x \in dom\,(G \circ H)$. Thus there is a $y \ni (x,y) \in G \circ H$. Thus there
    is a $z \ni (x,z) \in H$ and $(z,y) \in G$. Thus the existence of $z \ni (x,z) \in H$
    means that $x \in dom\,H$.
\end{proof}

An important corollary of the above theorem is that if $range\,H = dom\,G$ then $dom\,G\circ H =
dom\,H$. To prove the equality we just need to show that $dom\, H \subseteq dom\,G\circ H$. Consider
an element $x \in dom\,H$. Thus there is a $z \ni (x,z) \in H$. Since range of $H$ equal to
$dom\,G$, this means that $z \in dom\,G$. Hence there is a $y \ni (z,y) \in G$ and so 
$(x,y) \in G\circ H$. Thus $x \in dom\,G\circ H$.

\begin{Definition}
    An indexed class is the class denoted by $\lbrace A_i : i \in I \rbrace$ where $I$ is the class
    whose elements are called indices.
\end{Definition}
Formally an indexed class is a graph $G$ and each $A_i = \set{x}{(i,x) \in G}$. Thus if $I =
\lbrace 1,2\rbrace$ and $A_1 = \lbrace a,b \rbrace $ and $A_2 = \lbrace e , f \rbrace$ then the
indexed class $\lbrace A_i : i \in I \rbrace$ is the graph $G = \lbrace (1,a) , (1,b) , (2,e), (2,f)
\rbrace$.

\begin{Definition}
    Let $\lbrace A_i : i \in I \rbrace$ be an indexed family. Then,
    \begin{enumerate}
	\item The union of the classes $A_i$ consists of all those elements $x$ that are contained
	    in atleast one $A_i$. 
	    \begin{equation*}
		\bigcup_{i \in I} A_i = \set{x}{\exists j \ni x \in A_j}.
	    \end{equation*}
	\item The intersection of the classes $A_i$ consists of all those elements $x$ that are 
	    contained in each $A_i$. 
	    \begin{equation*}
		\bigcap_{i \in I} A_i = \set{x}{\forall j, x \in A_j}.
	    \end{equation*}
    \end{enumerate}
\end{Definition}

\begin{Theorem}
    Let $\lbrace A_i : i \in I \rbrace$ be an indexed class and $B$ be any class. Then,
    \begin{enumerate}
	\item If $B \subseteq A_i$ for every $i \in I$ then $B \subseteq \bigcap_{i \in I} A_i$.
	\item If $A_i \subseteq B$ for every $i \in I$ then $\bigcup_{i \in I} A_i \subseteq B$.
    \end{enumerate}
\end{Theorem}

The Generalized DeMorgan's Laws and Distributive laws can be restated as:
\begin{Theorem}[name=DeMorgan's Laws]
    Let $\lbrace A_i : i \in I \rbrace$ be an indexed class. Then,
    \begin{enumerate}
	\item ${\comp{(\bigcup_{i \in I} A_i)}} = \bigcap_{i \in I} \comp{A_{i}}$
	\item ${\comp{(\bigcap_{i \in I} A_i)}} = \bigcup_{i \in I} \comp{A_{i}}$
    \end{enumerate}
\end{Theorem}

\begin{Theorem}[name=Distributive Laws]
    Let $\lbrace A_i : i \in I \rbrace$ and $\lbrace B_j : j \in J \rbrace$  be indexed classes. 
    Then,
    \begin{enumerate}
	\item ${(\bigcup_{i \in I} A_i) \bigcap (\bigcup_{j \in J} B_j) } = \bigcup_{(i,j) \in I
	    \times J} A_{i} \cap B_{j}$.
	\item ${(\bigcap_{i \in I} A_i) \bigcup (\bigcap_{j \in J} B_j) } = \bigcap_{(i,j) \in I
	    \times J} A_{i} \cup B_{j}$.
    \end{enumerate}
\end{Theorem}

\begin{Theorem}
    Let $\lbrace G_i : i \in I \rbrace$ be a family of graphs. Then,
    \begin{enumerate}
	\item $dom\,(\bigcup_{i \in I} G_i) = \bigcup_{i \in I} (dom\,G_i)$.
	\item $range\,(\bigcup_{i \in I} G_i) = \bigcup_{i \in I} (range\,G_i)$.
    \end{enumerate}
\end{Theorem}

In the begining we noted that there were two kinds of classes. We now define the most important kind
of class called \textit{Set}.
\begin{Definition}
    A class $X$ is called a set if there is a class $Y$ such that $X \in Y$.
\end{Definition}
If for all class $Y$, $X \not\in Y$ then $X$ is called a proper class. The remaining axioms all
concern sets. 

\begin{enumerate}[label=\bfseries Axiom 3:]
    \item Every subclass of a set is itself a set.  
\end{enumerate}
Such a subclass is called a \textit{subset}. Note that for any class $B$, if $A$ is a set then
$A\cap B \subseteq A$. Thus \textit{intersections} are sets. 
The next axiom gives the existence of sets.

\begin{enumerate}[label=\bfseries Axiom 4:]
    \item The empty class $\emptyset$ is a set.
\end{enumerate}

\begin{enumerate}[label=\bfseries Axiom 5:]
    \item If $a,b$ are sets then the un-ordered pair $\lbrace a , b \rbrace$ is a set. 
\end{enumerate}

Note that $\emptyset$ is a set and the set containing the emptyset $\lbrace{\emptyset}\rbrace$ is
also a set i.e we used $b = a$ in the above axiom to construct this set. 
Thus we can form a new set $\lbrace \emptyset,\lbrace \emptyset \rbrace \rbrace$. We can
continue this forever.

\begin{Definition}
    Let $A$ be a set. By the power set of $A$ we mean the class which contains all subsets of $A$.
    Thus,
    \begin{equation*}
	\powSet{A} = \set{X}{X \subseteq A}.
    \end{equation*}
\end{Definition}

The next two axioms concerns sets of sets. 

\begin{enumerate}[label=\bfseries Axiom 6:]
    \item If $\mathcal{A}$ is a set of sets then $\bigcup \mathcal{A}$ is also a set.
\end{enumerate}
Note that $\bigcup{\mathcal{A}}$ is the set $\set{x}{\exists A \in \mathcal{A} \ni x \in A}$.

\begin{enumerate}[label=\bfseries Axiom 7:]
    \item If $A$ is a set then $\powSet{A}$ is also a set.
\end{enumerate}

The following theorem shows that the cartesian product of two sets is also a set.
\begin{Theorem}
    If $A,B$ are sets then $A \times B$ is also a set.
\end{Theorem}
\begin{proof}
    We will show that $A \times B$ is an subset of $\powSet{\powSet{A\cup B}}$
    and thus by Axiom $3$ is a set.
    Let $(x,y) \in A \times B$. Note that $(x,y) = \lbrace \lbrace x \rbrace, \lbrace x,y \rbrace
    \rbrace$. But $\lbrace x \rbrace \in \powSet{A \cup B}$ and $\lbrace y \rbrace \in
    \powSet{A \cup B}$.
    Thus $\lbrace \lbrace x \rbrace, \lbrace x,y \rbrace \rbrace$ is a subset of $\powSet{A\cup B}$. 
    That is $(x,y) \in \powSet{\powSet{A\cup B}}$. Hence $A \times B$ is a subset of 
    $\powSet{\powSet{A\cup B}}$.
\end{proof}

Now we will look into the set-theoretic definition of functions.
A function $f$ is a \emph{triple} $\left(A,B,f\right)$ where $A,B$ are sets and $f \subseteq A
\times B$ is a graph satisfying the following conditions:
\begin{enumerate}[label=\bfseries F \arabic*:]
    \item For every $x \in A$ there is a $y \in B$ such that $(x,y) \in f$.
    \item For every $x \in A$, if $y_1,y_2 \in B$ such that $(x,y_1) \in f$ and $(x,y_2) \in f$ then
	$y_1 = y_2$.
\end{enumerate}

We usually denote $\left(A,B,f\right)$ as $ f : A \to B$ and write $(x,y) \in f$ as $f(x)= y$.
Thus the above conditions become:
\begin{enumerate}[label=\bfseries F \arabic*:]
    \item For every $x \in A$ there is a $y \in B$ such that $f(x) = y$.
    \item For every $x \in A$, $f(x) = y_1 $ and $f(x) = y_2$ implies $y_1 = y_2$. 
\end{enumerate}

We note that if $A,B$ are sets then any graph $f \subseteq A \times B$ is a function iff
\begin{enumerate}
    \item $dom\,f = A$.
    \item $range\,f \subseteq B$.
    \item $F2$ is satisfied.
\end{enumerate}

Some important functions are:
\begin{enumerate}
    \item INJECTIVE, A function $f : A \to B$ is said to be injective iff for $x_1,x_2 \in A$,
	$f(x_1) = f(x_2)$ implies that $x_1 = x_2$.
    \item SURJECTIVE, A function $f : A \to B$ is surjective iff for every $y \in B$ there is a $x
	\in A$ such that $y = f(x)$, i.e range $f$ $ = B$.
    \item BIJECTIVE, A function $f : A \to B$ is bijective iff it is both surjective and injective.
\end{enumerate}

Some examples are as follows:
\begin{enumerate}
    \item Identity function. The function $i_A : A \to A$ given by $i_A(x) = x$ 
	for every $ x\in A$ is called the identity function.  
    \item Inclusion function. Let $A,B$ be sets such that $B \subseteq A$. The function $i_B : B \to
	A$ is called the inclusion function. Note that $i_B(x) = x$ for every $x \in B$.
    \item Characteristic function. Let $2$ designate the class of all functions with two elements, say the
	class $\lbrace 0,1\rbrace$. If $B \subset A$, then the characteristic function of $B$ is
	given by $\chi_B : B \to 2$ such that whenever $x \in B$ then $\chi_B(x) = 1$, otherwise
	$\chi_B(x) = 0$. 
    \item Restriction function. Let $C \subseteq A$ and $f : A \to B$. Then the restriction of $f$
	to $C$ is the function given by $f_{|_C} : C \to B$ such that $f_{|_C}(x) = f(x)$ for every
	$x \in C$.
\end{enumerate}

The next theorem concerns composition.
\begin{Theorem}
    If $f : A \to B$ and $g : B \to C$ are functions then $g \circ f : A \to C $ is a function and
    $(g\circ f)(x) = g(f(x))$ for every $x \in A$.
\end{Theorem}
The proof is easy once we notice that since $g$ is a fuction $dom\,g = B$ and so range of $f$ is
subset of domain of $g$ thus $dom\,g\circ f = A$. Also we observe that $range\, g\circ f \subseteq
range\,g$ for any graph. That $F2$ is satisfied is easy.

\begin{Definition}
    A function $f : A \to B$ is invertible if the inverse graph is a function $f^{-1} : B \to A$.
\end{Definition}

\begin{Theorem}
    A function is invertible iff it is bijective. Furthermore, if a function is invertible then
    the inverse is a bijective function.
\end{Theorem}

A useful characterization of invertible function is the following:
\begin{Theorem}
    A function $f : A \to B$ is invertible iff there is a function $g : B \to A$ such that $g \circ
    f = i_A$ and $f\circ g = i_B$. If such a function $g$ exists then $g = f^{-1}$.
\end{Theorem}

\begin{Theorem}
    A function $f : A \to B$ is injective iff there is a function $g : B \to A$ such that $g \circ
    f = i_A$.
\end{Theorem}
\begin{proof}
    Let a function $g : B \to A$ be such that $g\circ f (x) = x $ for every $x \in A$. Consider
    $x_1,x_2$ such that $f(x_1) = f(x_2)$. Thus $x_1 = (g \circ f)(x_1) = g(f(x_1)) = g(f(x_2)) = 
    x_2$. Hence $f$ is injective. Consider an injective function $f$ and fix an element $a \in A$.
    Construct $g : B \to A$ as follows. If $y \in range\,f$ let $g(y) = f(x)$. If $y \not \in
    range\,f$ then $g(y) = a$. Easy to see that $g\circ f = i_A$. 
\end{proof}

\begin{Theorem}
    Let $f : A \to B$ and $g : B \to C$ be functions. Then we can state the following about the
    composition $g \circ f$,
    \begin{enumerate}
	\item If $f,g$ are injective then so is $g \circ f$.
	\item If $f,g$ are surjective then so is $g \circ f$.
	\item If $f,g$ are bijective then so is $g \circ f$.
    \end{enumerate}
\end{Theorem}

Now we will define Direct and Inverse images of sets under a function.
\begin{Definition}
    Let $f : A \to B$ be a function and consider a set $C \subseteq A$. Then the direct image of $C$
    under $f$ is the set of all images of elements of $C$,
    \begin{equation*}
	f(C) = \set{y \in B}{\exists x \in C \ni f(x) = y}.
    \end{equation*}
\end{Definition}

\begin{Definition}
    Let $f : A \to B$ be a function and consider a set $D \subseteq B$. Then the inverse image of $D$
    under $f$ is the set of all elements in $A$ whose images are elements of $D$,
    \begin{equation*}
	\invIm{f}{D} = \set{x \in A}{\exists y \in D \ni f(x) = y}.
    \end{equation*}
\end{Definition}
We can alternatively write the inverse image of $D$ as the set 
\[\invIm{f}{D} = \set{x \in A}{f(x) \in D}.\]
It is important to see how direct images and inverse images act on generalized unions and
intersections. The next theorem summarizes there actions.
\begin{Theorem}
    Let $f : A \to B$ and let ${\lbrace C_i \rbrace}_{i \in I}$ and 
    ${\lbrace D_i \rbrace}_{i \in I}$ be sub-families in $A$ and $B$ respectively. Then,
    \begin{enumerate}
	\item $\dirIm{f}{\indexUnion{C_i}{i}{I}} = \indexUnion{\dirIm{f}{C_i}}{i}{I}$,
	\item $\dirIm{f}{\indexIntersection{C_i}{i}{I}} \subseteq \indexIntersection{\dirIm{f}{C_i}}{i}{I}$,
	\item $\invIm{f}{\indexUnion{D_i}{i}{I}} = \indexUnion{\invIm{f}{D_i}}{i}{I}$,
	\item $\invIm{f}{\indexIntersection{D_i}{i}{I}} = \indexIntersection{\invIm{f}{D_i}}{i}{I}$,
    \end{enumerate}
\end{Theorem}
Thus, we see that the inverse image is well behaved w.r.t unions and intersections (and
complements). Note that the inverse and direct image are functions that maps powersets, i.e
$C(\in \powSet{A}) \mapsto \dirIm{f}{C}(\in \powSet{B})$ and $D (\in \powSet{B}) 
\mapsto \invIm{f}{D} (\in \powSet{A})$. This is easy to see if we observe that if $C_1 = C_2$ then
$\dirIm{f}{C_1} = \dirIm{f}{C_2}$. Similarly for inverse images. However, the converse 
is not true in general.

We have defined the \emph{product} of two classes as $A \times B$ as the class of all the ordered
pairs in $A$ and $B$. We can extend this idea to the product of finite classes $A_1,A_2,\dots,A_n$
as the class of all \emph{n-tuple} $(a_1,a_2,\dots,a_n)$ such that $a_i \in A_i$ for all $1\leq i
\leq n$. However, we have a potential problem if we an arbitrary indexed family, 
$\set{A_i}{i\in I}$. In such a case we have to redefine what a product of classes mean.

\begin{Definition}
    Let $\set{A_i}{i\in I}$ be an indexed family of classes; let
    \[A = \indexUnion{A_i}{i}{I}.\]
    The product of the classes $A_i$ is defined to be the class
    \[\indexProduct{A_i}{i}{I} = \set{f}{\map{f}{I}{A}\,\text{and $f(i)\in A_i$}\,\forEv{i\in I}}.\]
\end{Definition}
We will adopt the following notational convention: we designate elements of a product
$\indexProduct{A_i}{i}{I}$ with boldface letters $\vect{a},\vect{b}$ etc. If $\vect{a}$ is an
element of $\indexProduct{A_i}{i}{I}$, we will denote by $a_j$ as $\vect{a}(j)$. We call $a_j$ as
the $j^{th}$ co-ordinate of $\vect{a}$.
Let $A = \indexProduct{A_i}{i}{I}$, corresponding to each index we define a function
$\map{\Pi_{i}}{A}{A_i}$ by
$\Pi_i(\vect{a}) = a_i$. We call $\Pi_i$ as the $i^{th}-projection$ of $A$ to $A_i$.

\begin{Definition}
    If $A,B$ are classes, we denote by $B^A$ as the class of all functions whose domain is $A$ and
    whose co-domain is $B$.
\end{Definition}
In particular if $2 = \lbrace 0,1\rbrace $ 
denotes the class of two elements, then $2^A$ is the class of all functions 
from $A$ to $\lbrace 0,1\rbrace$.
\begin{Theorem}
    If $A$ is a set, then  $\powSet{A}$ and $2^A$ are in $1-1$ correspondence.
\end{Theorem}
\begin{proof}
We will show that there is a function $\map{f}{\powSet{A}}{2^A}$, such that $f$ is injective.
For any $B \in \powSet{A}$ define $f(B) = \charFunc{B}$. Easy to see that $f$ is injective. Infact
there is a bijection. Let $g \in 2^A$. Define $B = \invIm{g}{\lbrace 1 \rbrace }$. 
Then $g = \charFunc{B}$.
\end{proof}

Let us list a couple more axioms that will \emph{almost} complete the set construction axiom. 

\begin{enumerate}[label=\bfseries Axiom 9:]
    \item If $A$ is a non-empty set, there is an element $a \in A$ such that $a\cap {A} =
	\emptyset$.
\end{enumerate}
The above axiom states that a set is disjoint from its elements. Hence if $A$ is a set, the
singleton $\lbrace A \rbrace \neq A$.
\begin{enumerate}[label=\bfseries Axiom 10:]
    \item If $A$ is a set and $\map{f}{A}{B}$ is a surjective function, then $B$ is a set.
\end{enumerate}

Next we define relations on sets.
\begin{Definition}
    Let $A$ be a class, by a relation $R$ in $A$ we mean an arbitrary subclass of $A \times A$.
\end{Definition}
Let $R$ be a relation in $A$, then
\begin{enumerate}
    \item
	(Reflexive) $R$ is reflexive if for every $a \in A$, $(a,a) \in R$. 
    \item
	(Irreflexive) $R$ is irreflexive if for every $a \in A$, $(a,a) \not \in R$. 
    \item
	(Symmetric) $R$ is symmetric if $(a,b) \in R \implies (b,a) \in R$. 
    \item
	(Asymmetric) $R$ is asymmetric if $(a,b) \in R \implies (b,a) \not \in R$. 
    \item
	(Anti-symmetric) $R$ is anti-symmetric if $(a,b),(b,a) \in R \implies a=b $. 
    \item
	(Transitive) $R$ is transitive if $(a,b) ,(b,c) \in R \implies (a,c) \in R$. 
\end{enumerate}
\begin{Definition}
    A relation $R$ in $A$ is called an equivalence relation if it is Reflexive, Transitive and
    Symmetric.
\end{Definition}
\begin{Definition}
    A relation $R$ in $A$ is called a partial order relation if it is Reflexive, Transitive and
    Anti-symmetric.
\end{Definition}
\begin{Definition}
    A relation $R$ in $A$ is called a strict order relation if it is Irreflexive, Transitive and
    Asymmetric.
\end{Definition}

\section{Countability}
In this section we will study in some detail finite sets,countable sets and uncountable sets.
We start with finite sets.
\begin{Definition}[name=section]
    Let $n$ be a positive integer. We use $S_n$ to denote the set of positive integers less than $n$ and call
    it a section of positive integers. Thus $S_n = \setX{1,2,\ldots,n}$.
\end{Definition}
\begin{Definition}[name=finite sets]
    A set $A$ is said to be finite if there is a bijective correspondence of $A$ with some section of the
    positive integers, i.e.~there exists an $n \in \Zplus$ such that,
    \[\map{f}{A}{S_n},\]
    is a bijective function. We say that $A$ has cardinality $n$. If $A$ is empty we say that $A$ has
    cardinality $0$.
\end{Definition}
\begin{Remark}
    $S_n$ itself has cardinality $n$. We will show that the cardinality of a finite set is uniquely determined
    by the set.
\end{Remark}
\begin{Lemma}
    Let $n$ be a positive integer. Let $A$ be a set and let $a_0$ be any element of $A$. Then there exists a
    bijective correspondence $f$ of the set $A$ with $S_{n+1}$ if and only if there exists a bijective
    function $g$ from the set $\setDiff{A}{\setX{a_0}}$ with $S_{n}$.
\end{Lemma}
\begin{proof}
    First assume that $\map{g}{\setDiff{A}{\setX{a_0}}}{S_n}$ is bijective. Let $\map{f}{A}{S_{n+1}}$ be such
    that $\restrict{f}{\setDiff{A}{\setX{a_0}}} = g $ and $f(a_0) = n+1$. Then $f$ is bijective.

    Assume $\map{f}{A}{S_{n+1}}$ is bijective. We consider two cases.

    \textbf{CASE I}: $f(a_0) = n+1$. In this case, define $g = \restrict{f}{\setDiff{A}{\setX{a_0}}}$. Then we
    get the desired bijective map.

    \textbf{CASE II}: $f(a_0) = k$ where $k \neq n+1$. Then there is an $a_1$ in $A$ such that $f(a_1) = n+1$.
    Define $\map{\sigma}{A}{S_{n+1}}$ such that $\sigma(a_0) = n+1$ and $\sigma(a_1) = k$ and for all other
    values of $A$, $\sigma = f$. Thus $\sigma$ is a bijective mapping from $A$ onto $S_{n+1}$ such that
    $\sigma(a_0) = n+1$. Define $g = \restrict{\sigma}{\setDiff{A}{\setX{a_0}}}$. Then we
    get the desired bijective map.
\end{proof}
\begin{Theorem}
    Let $A$ be a set; suppose there exists a bijection $\map{f}{A}{S_n}$ for some $n \in\Zplus$. Let $B$ be a
    proper subset of $A$. Then there exists no bijection $\map{g}{A}{S_n}$, but if $B$ is not empty, then
    there is a bijection $\map{f}{B}{S_m}$ for some $m < n$.
\end{Theorem}
\begin{proof}
    We will prove this by induction on $n$. First note that when $A$ is empty, there is no proper subset of
    $A$ and so the theorem is trivially true. Let $\map{f}{A}{S_1}$ be a bijection. Then any proper subset $B$
    of $A$ is empty and so cardinality of $B$ = $0$.
    Assume the theorem is true for some $n$ in $\Zplus$. Let $\map{f}{A}{S_{n+1}}$ be a bijection and let $B$
    be a proper subset of $A$. Assume $B$ is not empty. There is an element $a_0$ in $B$ and note that $a_0$
    is also in $A$. By the previous Lemma, there is a map,
    \[\map{g}{A-\setX{a_0}}{S_n},\]
    such that $g$ is bijective. Note that $B - \setX{a_0}$ is a proper subset of $A - \setX{a_0}$. 
    Now we can use our inductive hypothesis to state that:
    \begin{enumerate}
	\item
	    There is no bijection $\map{h}{B-\setX{a_0}}{S_n}$.
	\item
	    Either $B - \setX{a_0}$ is empty or there is an $m < n$ such that,
	    $\map{k}{B-\setX{a_0}}{S_{m}}$ is bijective.
    \end{enumerate}
    Using the Lemma above we can state $(1)$ equivalently by stating that there is not bijection between $B$
    and $S_{n+1}$. If $B - \setX{a_0}$ is not empty, then we can use the previous lemma to state that there is
    a bijection from $B$ to $S_{m+1}$ where $m+1 < n+1$ and so we have proved the statement for $n+1$.
\end{proof}
\begin{Corollary}
    If $A$ is finite, there is no bijection of $A$ with a proper subset of itself.
\end{Corollary}
\begin{proof}
    Let $\map{f}{B}{A}$ be bijective. Since $A$ is finite there is a bijective function $\map{g}{A}{S_n}$.
    Define $h = \fog{g}{f}$. Then $\map{h}{B}{S_n}$ is bijective which contradicts the theorem above.
\end{proof}
\begin{Corollary}
    $\Zplus$ is not finite.
\end{Corollary}
\begin{proof}
    We prove this by the contrapositive of the above corollary i.e.~we construct a bijection between $\Zplus$
    and a propoer subset $\Zplus-\setX{1}$.
    Let $\map{f}{\Zplus-\setX{1}}{\Zplus}$ be given by $f(i) = i-1$. Then $f$ is bijective.
\end{proof}
\begin{Corollary}
    The cardinality of a finite set $A$ is uniquely determined by $A$ i.e.~there doesnot exist $S_n,S_m$ with
    $n \neq m$ such that $A$ is in bijective correspondence with both $S_n,S_m$.
\end{Corollary}
\begin{proof}
    Suppose there exist bijections $f,g$ from $A$ onto $S_n,S_m$ where $n \neq m$. Without loss of generality,
    assume $n < m$. Let $h = \fog{g}{f^{-1}}$. Then $\map{h}{S_n}{S_m}$ is bijective which gives a
    contradiction since $S_n$ is a proper subset of $S_m$.
\end{proof}
\begin{Corollary}
    If $B$ is a subset of the finite set $A$, then $B$ is finite. If $B$ is a proper subset of $A$, then the
    cardinality of $B$ is less than $A$.
\end{Corollary}
\begin{Corollary}
    Let $B$ be a non-empty set. Then the following are equivalent:
    \begin{enumerate}
	\item
	    $B$ is finite.
	\item
	    There is a surjective function from a section of the positive integers onto $B$.
	\item
	    There is an injective function from $B$ into a section of the positive integers.
    \end{enumerate}
\end{Corollary}
\begin{proof}
    We will prove $(1)\implies(2)$, $(2)\implies (3)$ and $(3) \implies (1)$.

    $(1)\implies (2)$. Assume $B$ is finite. Then, for some $n\in\Zplus$, there is a bijective map 
    $\map{f}{B}{S_n}$. Thus $\map{f^{-1}}{S_n}{B}$ is onto.

    $(2) \implies (3)$. Assume there is a map $\map{f}{S_n}{B}$ such that $f$ is onto. Define the function
    $\map{h}{B}{S_n}$ where for any $b \in B$, $h(b) = \min\invIm{f}{b}$. Since $f$ is onto $\invIm{f}{b}$ is
    not empty and is a subset of $S_n$ and by the well ordering property of integers, the minimum exists and
    is unique. Thus $h$ is injective.

    $(3) \implies (1)$. Assume there is an injective map $\map{f}{B}{S_n}$ for some $n \in \Zplus$. Then
    $\map{f}{B}{\dirIm{f}{B}}$ is bijective. Also, since $\dirIm{f}{B} \subset S_n$, it is finite. Hence $B$ is
    finite.
\end{proof}
\begin{Theorem}
    Finite Unions and finite cartesian products of finite sets are finite.
\end{Theorem}
\begin{proof}
    Let $\famB = \set{B_i}{1\leq i \leq n}$ be a finite collection of finite sets. We will show that
    $\finiteUnion{B_i}{i}{n}$ is also finite by induction on $n$.
    When $n = 1$, the result is trivial. Consider the case when $n = 2$. By hpyothesis, there exists bijective
    functions $f_1,f_2$ from $B_1,B_2$ onto $S_{n_1},S_{n_2}$ for some $n_1,n_2$ in $\Zplus$. Define
    $\map{h}{S_{n_1+n_2}}{B_1\cup B_2}$ as follows:
    $h(i) = f_1(i)$ for $1\leq i \leq n_1$ and $h(n_1+j) = f_2(j)$ for $1 \leq j \leq n_2$. Thus $h$ is a
    surjection from a section of integers to $B_1\cup B_2$ and hence, $B_1\cup B_2$ is finite. Note that $h$
    is note necessarily a bijection since $A,B$ may not be disjoint. The result now follows from induction.

    Note that for any $b \in B_1$, $\setX{b}\times B_2$ is finite because the function
    $\map{f}{\setX{b}\times B_2}{S_{n_2}}$ given by $f((b,c)) = f_2(c)$ for any $c \in B_2$ 
    is a bijection. Now, $B_1\times B_2$
    can be written as $\bigcup\limits_{b\in B_1}\setX{b}\times B_1$, which is a finite union of finite sets and
    so is finite. The result for arbitrary $n$ follows from induction.
\end{proof}
Just as sections of integers are the prototypes for finite sets, the set of all positive integers is
prototype for the simplest infinite set that we call countably infinite or denumerable sets.

\begin{Definition}[name=Infinite and Countably infinite]
    A set $A$ is said to be infinite if it is not finite. It is said to be countably infinite if there is a
    bijective correspondence,
    \[\map{f}{A}{\Zplus}.\]
\end{Definition}
From all our observations regarding finite sets we can immediately deduce the following about infinite sets
\begin{enumerate}
    \item
	For any proper subset $B$ of $A$, if $B$ is infinite then $A$ is also infinite.
    \item
	If there is a proper subset $B$ of $A$ and if there is a function $\map{f}{B}{A}$ which is bijective,
	then $A$ is infinite.
\end{enumerate}
\begin{Definition}[name=Countable]
    A set $A$ is said to be countable if it is either finite or if it is countably infinite. A set that is not
    countable is said to be uncountable.
\end{Definition}
We saw that $\Zplus$ was not finite. It is countably infinite by definition and hence $\Zplus$ is countable.
Is $\Z$ countable?
\begin{Proposition}
    $\Z$ is countably infinite.
\end{Proposition}
\begin{proof}
    $\Zplus$ is a proper subset of $\Z$ and is not finite. Hence we must show that there is a function
    $\map{f}{\Z}{\Zplus}$ that is bijective. Let us define,
    \begin{align*}
	f(n) = 
	\begin{cases}
	    -2n &\text{if $n < 0$},\\
	    2n + 1 &\text{if $n \geq 0$}.
	\end{cases}
    \end{align*}
    Easy to see that $f$ is a bijection.
\end{proof}
\begin{Proposition}
    The product $\Zplus \times \Zplus$ is countably infinite.
\end{Proposition}
\begin{proof}
    The proof is the famous diagonal arrangement. If we of $\Zplus \times \Zplus$ as a grid, we can start
    counting \textbf{diagonally} from right to top. Let $\map{f}{\Zplus}{\Zplus}$ be given by,
    $f(i,j) = j + \series{k}{k}{1}{i+j-2}$. Thus, for example, $f(1,1) = 1$, $f(2,1) = 2$, $f(1,2) = 3$ and so
    on. Easy to see that we have a bijection. For example to show it is surjective let $m$ be any positive
    integer, say $m = 11$. Then think of $11$ as a list $(1),(2,3),(4,5,6),(7,8,9,10),(11,12,13,14,15)$. Thus
    $11$ is on the $5^{th}$ list which means that $j$ will be such that $1\leq j \leq 5$. Similarly $i$ will
    be less than $5$. Easy to see that $i = 5$ and $j = 1$ will give $f(i,j) = 11$.
\end{proof}
%%%%%%% draw diagrams here : see pink notebook
There is a very useful criterion for showing that a set is countable.
\begin{Theorem}
    Let $B$ be a non-empty set. Then the following are equivalent.
    \begin{enumerate}
	\item
	    $B$ is countable.
	\item
	    There is a surjective function $\map{f}{\Zplus}{B}$.
	\item
	    There is an injective function $\map{g}{B}{\Zplus}$.
    \end{enumerate}
\end{Theorem}
\begin{proof}
    Assume that $B$ is countable. Hence $B$ is either finite or countably infinite. In the later case, there
    is a surjection $\map{f}{\Zplus}{B}$. If $B$ is finite, then there is a bijection $\map{g}{S_n}{B}$
    for some $n\in\Zplus$. Define $\map{f}{\Zplus}{B}$ as follows:
    $f(i) = g(i)$ for $i \leq n$ and $f(i) = g(n)$ for $i > n$. Thus $f$ is a surjection. Hence we have shown
    $(1)\implies (2)$.

    Assume that there is a surjection $\map{f}{\Zplus}{B}$. This means that for any $b$ in $B$,
    $\invIm{f}{\setX{b}}\neq\emptyset$. Moreover, for $b_1,b_2 \in B$, if $b_1\neq b_2$, then
    $\invIm{f}{\setX{b_1}} \neq \invIm{f}{\setX{b_2}}$. Let $g(b) = \min\invIm{f}{\setX{B}}$. Then $g(b)$ is
    an injective function from $B$ to $\Zplus$. Thus, we have shown that $(2) \implies (3)$.

    Assume there is an injective function $\map{g}{B}{\Zplus}$. To show that $B$ is countable we need to show
    that either $B$ is finite or $B$ is countably infinite. We can assume $B$ is not finite. 
    Thus $\dirIm{g}{B} \subset {\Zplus}$ is not finite. IF $\dirIm{g}{B}$ was countably infinite, then $B$
    would be countably infinite. Hence we need to show that any infinite subset of $\Zplus$ is countably
    infinite which would mean that $(3)\implies (1)$.
\end{proof}
\begin{Remark}
    The above theorem (part (2)) 
    states that for a countable set we can list its elements as a \textbf{sequence}. Thus if
    $A$ is countable, then $A = \setX{a_{1},a_{2},\cdots}$.
\end{Remark}
\begin{Proposition}
    Any infinite subset of $\Zplus$ is countably infinite.
\end{Proposition}
\begin{proof}
    Let $B$ be an infinite subset of $\Zplus$. 
    There is bijection $\map{f}{\Zplus}{\Zplus}$ that lists the elements of $\Zplus$ in a sequence. 
    The intuitive idea is to go along the sequence and check if it belongs to $B$. 
    We want a function $\map{g}{\Zplus}{B}$ that is bijective. 
    For example as we move along the sequence in $f$, the first occurance of an element in $B$ occured in the
    index $k$. We set $g(1) = f(k)$. Then we continue traversing along the sequence in $f$ until we find an index
    $l > k$ such that $f(l) \in B$. We set $g(2) = f(l)$. It is clear that such a function is surjective. 
    How do we define the function $g$? Let us tweak $g$ a bit. Instead of taking the first occurance
    of an element listed by $f$, we take the smallest element of $B$ listed by $f$. There must be an index $m$
    such that $f(m) = \min{B}$. We take $g(1) = f(m)$ and proceed subsequently. 

    Let $g(1)$ be the smallest element that is in
    $B$. This is possible since $B \subset \Zplus$. Let $g(2)$ be the smallest element other than $g(1)$ that
    is in $B$. This is called an \textbf{inductive} definition (or recursion). We define a function whose
    domain is the natural numbers inductively, in this case,
    \[g(n) = \min\setX{B-\setX{g(1),g(2),\ldots,g(n-1)}}.\]
    It is clear that $g$ is injective. To see that $g$ is surjective, consider an arbitrary element, say $b$, 
    of $B$. For any $m \in \Zplus$ there is an $n \in \Zplus$ such that $g(n) > S_m$. Because if not, then $B$
    would be finite. In particular for $m = b$ there is an $n$ such that $g(n) > b$. Let $C =
    \set{n\in\Zplus}{g(n)\geq b}$. Then $C \subset \Zplus$ and so a minimum $p \in \Zplus$ exists such that
    $g(p) \geq b$. This means that $g(1),g(2),\ldots,g(p-1)$ are all less than $b$. By definition of $g$,
    $g(p)\leq b$. Thus it must be the case that $g(p) = b$.
\end{proof}
We have used a key idea here called the principle of recursive definition.
\begin{Definition}[name=Principle of recursive definition]
    Let $A$ be any set. Given a formula that defines $g(1)$ as a \textbf{unique} element of $A$ and for all
    $i > 1$ defines $g(1)$ \textbf{uniquely} as an element of $A$ in terms of the values of $g$ for positive
    integers less than $i$, this formula determines a \textbf{unique function} $\map{g}{\Zplus}{A}$.
\end{Definition}
\begin{Corollary}
    A subset of a countable set is countable.
\end{Corollary}
\begin{proof}
    Let $A \subset B$ such that $B$ is countable. Thus there exists an injection $\map{f}{B}{\Zplus}$. Hence,
    $\map{\restrict{f}{A}}{A}{\Zplus}$ is also an injection.
\end{proof}
\begin{Corollary}
    The set $\Zplus\times\Zplus$ is countable.
\end{Corollary}
\begin{proof}
    Let $\map{f}{\Zplus\times\Zplus}{\Zplus}$ be given by $f(i,j) = 2^{i}3^{j}$. Then, $f$ is injective
    (because $2,3$ are prime).
\end{proof}
\begin{Corollary}
    The set of positive rational numbers is countable.
\end{Corollary}
\begin{proof}
    Let $\map{f}{\Zplus\times\Zplus}{\Q^{+}}$ be defined by $f(i,j) = \frac{i}{j}$. Then $f$ is surjective.
    We showed that there is a function $\map{g}{\Zplus}{\Zplus\times\Zplus}$ that is surjective. Thus
    $\map{\fog{f}{g}}{\Zplus}{\Q^{+}}$ is surjective.
\end{proof}
\begin{Theorem}
    A countable union of countable sets is countable.
\end{Theorem}
\begin{proof}
    Let $\famA$ be a collection of countable sets $A_1,A_2,\ldots$. Since each $A_i$ is countable there is a
    function $\map{f_i}{\Zplus}{A_i}$ that is a surjection. In other words we can list the elements of $A_i$'s,
    \begin{align*}
	A_1 &= \setX{a_{11},a_{12},a_{13},\ldots}\\
	A_2 &= \setX{a_{21},a_{22},a_{23},\ldots}\\
	&\vdots\\
	A_j &= \setX{a_{j1},a_{j2},a_{j3},\ldots}\\
	&\vdots
    \end{align*}
    Define $\map{g}{\Zplus\times\Zplus}{\countUnion{A_i}{i}}$ as,
    $g(i,j) = a_{ij} = f_i(j)$. Thus $g$ is a surjection.
\end{proof}
\begin{Theorem}
    A finite product of countable sets is countable.
\end{Theorem}
\begin{proof}
    We have shown the analogous result for finite sets. Let $\set{A_i}{1\leq i \leq n}$ be a finite collection
    of countable sets. We prove this by induction on $n$. The case $n= 1$ is trivial. When $n = 2$,
    $A_1\times A_2 = \bigcup\limits_{a\in A_1}\setX{a}\times A_2$ is a countable unioon of countable sets and
    is countable by the theorem above. We proceed by induction to prove it for any $n \in \Zplus$.
\end{proof}
\begin{Theorem}
    Let $A$ be any set. There is NO injective map \break{}$\map{f}{\powSet{A}}{A}$ and there is no surjective map
    $\map{g}{A}{\powSet{A}}$.
\end{Theorem}
\begin{proof}
    We will show that there is no surjective map which maps $A$ onto the powerset of $A$. This will then imply
    that there is no injective map that maps powerset of $A$ into $A$. This is because of the following:
    \[\thereIs{\map{f}{B}{C}}\text{$f$ is injective, then}\hspace{0.05in}\thereIs{\map{g}{C}{B}}
	\text{$g$ is surjective}.\]
    Taking the contrapositive of this statement we can just show the non-existence of a surjective map. To
    see why the above statement is true, assume there is an injection $\map{f}{B}{C}$. Thus for any $c \in
    \dirIm{f}{B}$, there is a unique $b$ such that $f(b) = c$. Define $\map{g}{C}{B}$ as follows, if $c \in C$
    is in $f(B)$, then $g(c) = b$ such that $f(b) = c$. If $c \in C$ such that $c \not\in f(B)$, then fix a
    $b_0 \in B$ and let $g(c) = b_0$. Thus $g$ is surjective. Note that $B$ has to be non-empty for all this
    to make sense.

    Now we will show the non-existence of a surjective map from $A$ onto $\powSet{A}$ by contradiction. 
    Assume, there is a $\map{g}{A}{\powSet{A}}$ such that $g$ is surjective. Thus for any $X \subset A$, there
    is a $a \in A$ such that $g(a) = X$. Let $B = \set{a\in A}{a \in g(a)}$. Then $B \subset A$ and hence by
    our assumption there is an $a_0 \in A$ such that $g(a_0) = B$.
    There are two possibilities for $a_0$. Either $a_0 \in g(a_0)$ or $a_0 \not\in g(a_0)$. In the first case,
    $a_0 \not\in B$ and in the second case $a_0 \in B$. Thus $g(a_0)\neq B$. Hence, $g$ cannot be surjective.
\end{proof}
\begin{Corollary}
    There exists an uncountable set.
\end{Corollary}
\begin{proof}
    There is no surjection from $\Zplus$ onto $\powSet{\Zplus}$. Thus, $\powSet{\Zplus}$ is uncountable.
\end{proof}
Let us now give a characterization of infinite sets. These sets are either countably infinite or uncountable.
From our discussion following the definition of infinite sets, we already gathered a few notions of what is
means to be for a set to be infinite. We elaborate on this now.
\begin{Theorem}
    Let $A$ be a set. The following statements about $A$ are equivalent.
    \begin{enumerate}
	\item
	    There is an injective function $\map{f}{\Zplus}{A}$.
	\item
	    There exists a bijection of $A$ with a proper subset of itself.
	\item
	    $A$ is infinite.
    \end{enumerate}
\end{Theorem}
\begin{proof}
    $(1)\implies(2)$
    Let $\map{f}{\Zplus}{A}$ be an injective map. Note that $\map{f}{\Zplus}{f(\Zplus)}$ is a bijective map
    that lists a subset of $A$ as a sequence $(a_1,a_2,\cdots)$. Let $B = A - \setX{a_1}$. Then $B$ is a
    proper subset of $A$. Note that $A = f(\Zplus)\cup (\setDiff{A}{f(\Zplus)})$. 
    Define $\map{g}{A}{B}$ as follows:
    \begin{equation*}
	g(a) = 
	\begin{cases}
	    a_{n+1}&\text{if $a = x_n \in f(\Zplus)$ for some $n$}\\
	    a &\text{if $a \not\in f(\Zplus)$}
	\end{cases}
    \end{equation*}

    $(2)\implies(3)$ This just follows from our discussion following the definition of infinite sets.

    $(3)\implies(1)$
    First pick \textbf{any} element of $A$ and call it $a_1$. Thus $f(1) = a_1$. Now, pick any element of $A$
    which is not equal to $a_1$ and call it $a_2$. That is $f(2) \in A - \setX{a_1}$. This process will
    continue indefinitely since $A$ is infinite. Hence $\map{f}{\Zplus}{A}$ is injective, where $f$ is
    definied recursively as follows:
    \[f(n) \in A - \setX{a_1,a_2,\ldots,a_n}.\]
    However, there is a problem in our recursive definition for $f$. Namely, $f(1)$ is not unique and $f(n)$
    is certainly not uniquely defined in terms of $\setX{a_1,a_2,\ldots,a_n}$, since we are picking an
    arbitrary element of $A$. We need an additional axiom from set theory to make this possible.
\end{proof}


\begin{enumerate}[label=\bfseries Axiom 11:]
    \item (Axiom of choice) Given a collection $\famB$ of non-empty sets, there exists a function 
	\[\map{c}{\famB}{\bigcup\limits_{B\in\famB}B},\]
	such that $c(B)\in B$ for any $B \in \famB$.
\end{enumerate}
Using the axiom of choice we can fix our recursive definition for $f$. 
Let $\famB = \powSet{A}-\emptyset$. Then there exists a choice function such that $c(B) \in B$ for any $B \in
\famB$. Pick $B = A$. Thus $c(A)$ gives an element of $A$ uniquely.
Let $f(1) = c(A)$.
Now $A - \setX{f(1)}$ is also in $\famB$. Define $f(2) = c(A - \setX{f(1)})$ which is again uniquely defined.
Thus, define
\[f(n) = c(A - \setX{f(1),f(2),\ldots,f(n-1)}).\]
This is a valid recursive function.


There is a equivalence relation which is induced by bijective functions.
Let us define $A\sim B$ whenever there is a bijective map $\map{f}{A}{B}$. It is easy to see that $\sim$ is an
equivalence relation. A set $A$ is countably infinite if $A \sim \Zplus$. Note that any two countably infinite
sets are equivalent since if $A\sim \Zplus$ and $B \sim \Zplus$, then $A \sim B$. 

This is an extremely
important idea which we will explore now in the subsequent paragraph in the form of Cantor-Schroder-Bernstein
theorem.

Suppose $A,B$ are sets and $f$ is one-to-one function from $A$ into $B$. Then $A\sim f(A)\subset B$, so it is
natural to think of $B$ as being at least as large as $A$. This suggests the following notation:
\begin{Definition}
    If $A,B$ are sets, then we will say that $B$ \textbf{dominates} $A$, and write $A \preceq B$, if there is
    an injective function $\map{f}{A}{B}$. If $A\preceq B$ and $A\not\sim B$, then we say $B$ strictly
    dominates $A$.
\end{Definition}
\begin{Theorem}[name=Schroder-Bernstein theorem]
    If $A \preceq B$ and $B \preceq A$, then $A\sim B$.
\end{Theorem}
\begin{proof}
    Let $\map{f}{A}{B}$ and $\map{g}{B}{C}$ be injective functions. Note that $\map{g}{B}{g(B)}$ is bijective.
    Let $A_0 = \setDiff{A}{g(B)}$ and let us define recursively $A_n = g(f(A_{n-1}))$.
    Let $X = \bigcup\limits_{n=0}^{\infty}A_n$. Then $X \subset A$. Define $Y = \setDiff{A}{X}$. Note that, $Y
    \subset g(B)$. This is because if $a \in Y$ then $a \in A$ and $a \not\in A_0$, that is $a \in g(B)$.
    Define,
    \begin{equation*}
	h(a) = 
	\begin{cases}
	    f(a) &\text{if $a\in X$}\\
	    g^{-1}(a) &\text{if $a\in Y$}
	\end{cases}
    \end{equation*}
    \textbf{Claim:} $h$ is injective. 
    Assume $h(a_1) = h(a_2)$ for $a_1,a_2$ in $A$. The only cases we have to check are when $a_1\in X$ and
    $a_2\in Y$ and vice-versa.
    Suppose $a_1\in X$ and $a_2\in Y$. Then $f(a_1) = g^{-1}(a_2)$. Which means that $g(f(a_1)) = a_2$. Since
    $a_1\in X$, there is an $n$ such that $a_1 \in A_n$ and hence $a_2 \in g(f(A_n))$ which means that $a_2
    \in A_{n+1}$ i.e.~$a_2 \in X$ which is a contradiction. The other case is similar.

    \textbf{Claim:} $h$ is surjective. Let $b \in B$. Then $g(b) \in A$. Thus $g(b) \in X$ or $g(b) \in Y$. If
    $g(b)$ is in $Y$, then $h(g(b)) = g^{-1}(g(b)) = b$. Thus there is an $a = g(b) \in A$ such that $h(a) =
    b$. If $g(b) \in X$, then there is an $A_n$ such that $g(b)\in A_n$. But $n$ cannot be $0$. Thus there is
    an $n > 1$ such that $g(b) \in g(f(A_{n-1}))$. This means that there is an $x_1\in f(A_{n-1})$ such that
    $g(x_1) = g(b)$. Since $g$ is injective, this means that $x_1 = b$ i.e.~there is an $x \in A_{n-1}$ such
    that $f(x) = x_1 = b$. But this means that $h(x) = b$ for some $x \in A_{n-1}\subset X$.
\end{proof}

