\chapter{Elements of measure theory: measurable functions}
In this chapter we will look at mappings from one measurable space to another. This chapter, along with the 
preceding one, contains the basic elements of measure theory. 
This will then enable us to study calculus on a general setting.
\section{Measurable functions}
In topology, a continuous function is one for which the inverse image of open sets are open sets. In similar
vein, we will define measurable functions between two measure spaces. The definition is motivated by the
construction used by Lebesgue to extend Riemann integration. Integration will be dealt in the next chapter.
\begin{Definition}[name=Measurable functions]
    Let $\metricS{X}{\algebra{M}}$, $\metricS{Y}{\algebra{N}}$ be two measure spaces. A function,
    \[\map{f}{X}{Y}\]
    is $\measMap{\algebra{M}}{\algebra{N}}$ if,
    \[\invIm{f}{A} \in \algebra{M}\quad\text{for every}\,{A\in\algebra{N}}.\]
\end{Definition}
Consider the collection $\invIm{f}{\algebra{N}} = \set{\invIm{f}{A}}{A\in\algebra{N}}$. This is a sigma algebra
by~\ref{thm:pre_img_sigma}. We call this the induced sigma algebra on $X$. Thus, we see that if $f$ is
$\measMap{\algebra{M}}{\algebra{N}}$ then $\invIm{f}{\algebra{N}} \subset \algebra{M}$. Note that the inverse
image in general is a function $\map{f^{-1}}{\powSet{Y}}{\powSet{X}}$. The definition above states that, if
$f$ is measurable then the
restriction of the inverse image, $\restrict{f^{-1}}{\algebra{N}}$, 
is a function that maps $\algebra{N}$ to $\algebra{M}$.

Since, measurable sets are generally \emph{huge}, the above condition is not very useful in figuring which
functions are measurable. The following theorem is hence very useful.
\begin{Theorem}[name=Criteria for measurable functions]\label{thm:crit_meas_func}
    Let $\metricS{X}{\algebra{M}}$, $\metricS{Y}{\algebra{N}}$ be two measure spaces and let $\algebra{N} =
    \sigmaGen{\family{E}}$. A function,
    \[\map{f}{X}{Y}\]
    is $\measMap{\algebra{M}}{\algebra{N}}$ if and only if,
    \[\invIm{f}{A} \in \algebra{M}\quad\text{for every}\,{A\in\family{E}}.\]
\end{Theorem}
\begin{proof}
    \emph{only if} part is immediate since, if $f$ is measurable, the statement is true for all $A \in
    \algebra{N}$ and hence certainly true for $A \in \family{E}$. 

    For the \emph{if} part, consider the set
    \[F = \set{A \subset Y}{\invIm{f}{A}} \in \algebra{M}.\]
    This set is not empty by our assumption i.e $\algebra{F} \supset \algebra{E}$. 
    Easy to see that it is a $\sigmaAlgebra$. See~\ref{thm:prop_sigmaA}. 
    Hence, $F \supset \sigmaGen{\family{E}} = \algebra{N}$.
\end{proof}
\begin{Corollary}\label{thm:continuous_func_meas}
    If $X,Y$ are metric spaces (or topological spaces), every continuous function $\map{f}{X}{Y}$ is
    $\measMap{\borelS{X}}{\borelS{Y}}$.
\end{Corollary}
\begin{proof}
    Let $\family{G}_{Y}$,$\family{G}_{X}$ be the class of open sets in $Y,X$ respectively. 
    We know that $\borelS{Y} = \sigmaGen{\family{G}_{Y}}$. Since $f$ is continuous, 
    for any open set $G \in \family{G}_Y$,
    $\invIm{f}{G}$ is open in $\family{G}_X \subset \borelS{X}$. Hence, by~\ref{thm:crit_meas_func}
    $f$ is $\measMap{\borelS{X}}{\borelS{Y}}$.
\end{proof}
\begin{Definition}[name=Real or Complex valued measurable functions]
    If $\metricS{X}{\algebra{M}}$ is a measurable space, a real or complexed valued function is called
    $\algebra{M}$ measurable (or just measurable if $\algebra{M}$ is understood from the context) if it is
    $\measMap{\algebra{M}}{\borelS{\R}}$ or $\measMap{\algebra{M}}{\borelS{\C}}$ respectively. 
    Thus the co-domain $\sigmaAlgebra$ is always the Borel sigma algebra for real or complex valued functions.
\end{Definition}
\begin{Definition}[name=Lebesgue measurable functions]
    A real (or complex) valued function,
    \[\map{f}{\Rn}{\R}\,(\text{or}\,\map{f}{\Rn}{\C}),\]
    is Lebesgue measurable if it $\measMap{\family{L}}{\borelS{R}}$ (or $\measMap{\family{L}}{\borelS{C}}$
    resp.).
\end{Definition}

We often want to discuss measurability relative to a subset of a measure space.
\begin{Definition}[name=Restriction of measurable functions]
    Let $\metricS{X}{\algebra{M}}$ and $\metricS{Y}{\algebra{N}}$ 
    be measurable spaces and $E \in \algebra{M}$ be any measurable set and $f$ be a
    function defined on $X$ given by $\map{f}{X}{Y}$. We say that $f$ is measurable on $E$ if 
    $\invIm{f}{A}\cap E \in \algebra{M}$ for any $A \in \algebra{N}$. We say that $\restrict{f}{E}$ is
    $\measMap{\algebra{M}_{E}}{N}$, where $\algebra{M}_{E}$ is the restricted sigma algebra on $E$ as
    described in~\ref{ex:type_of_sigma_alg} (5).
\end{Definition}

\begin{Proposition}\label{prop:real_val_meas_funct}
    Let $\metricS{X}{\algebra{M}}$ be a measurable space and let $\map{f}{X}{R}$ be a real valued function.
    Then, the following are equivalent:
    \begin{enumerate}
	\item
	    $f$ is $\algebra{M}$ measurable.
	\item
	    $\invIm{f}{(a,\infty)} \in \algebra{M}$ for all $a\in\R$.
	\item
	    $\invIm{f}{\interval[open right]{a}{\infty}} \in \algebra{M}$ for all $a\in\R$.
	\item
	    $\invIm{f}{(-\infty,a)} \in \algebra{M}$ for all $a\in\R$.
	\item
	    $\invIm{f}{\interval[open left]{-\infty}{a}} \in \algebra{M}$ for all $a\in\R$.
    \end{enumerate}
\end{Proposition}
\begin{proof}
    We will show $(1)\iff(2)$ using~\ref{thm:crit_meas_func} and~\ref{thm:gen_borel_R}. The rest are similar.
    Consider the family of open rays $\family{E}_5$ as described in~\ref{thm:gen_borel_R}. Since
    $\sigmaGen{\family{E}_5} = \borelS{\R}$, by~\ref{thm:crit_meas_func} we get the result.
\end{proof}
\begin{Remark}\label{rmk:extended_reals_meas_func}
    Sometimes, it will be convenient to consider functions with values in the extended reals 
    $\closure{\R} = \extReals$. In that case we can define the Borel sets in $\closure{\R}$ by,
    \[\borelS{\closure{\R}} = \set{E\subset\closure{\R}}{E\cap\R \in \borelS{\R}}.\] Then,
    \[\borelS{\closure{\R}} = \sigmaGen{\lbrace\hInt{a}{\infty}\rbrace} = 
	\sigmaGen{\lbrace \interval[open right]{-\infty}{a}\rbrace}.\]
    Thus in the proposition above, a function $\map{f}{X}{\closure{\R}}$ is measurable if we include
    $-\infty,\infty$, where it is understood that we take $\borelS{\closure{\R}}$ as the co-domain sigma algebra.
\end{Remark}
For functions that take values in the extended reals, note that the set 
$\invIm{f}{\interval[open right]{-\infty}{a}} = \set{x \in X}{f(x) < a}$. It is a common
practice to denote this set as $\lbrace f < a \rbrace$. Thus from~\ref{prop:real_val_meas_funct} 
and~\ref{rmk:extended_reals_meas_func}, a real valued function is measurable iff $\lbrace f < a \rbrace$,
$\lbrace f \leq a \rbrace$, $\lbrace f > a \rbrace$ and $\lbrace f \geq a \rbrace$ are measurable. This
notation is used in probability and is also preferrable if we do not want to show explicitly a real valued
function or a function that takes values on the extended reals. 

\begin{Example}\label{ex:measurable_func}
    We give a few examples of real valued measurable functions.
    \begin{enumerate}
	\item
	    Consider the counting measure on $\R$,
	    \begin{equation*}
		\measure{E} = 
		\begin{cases}
		    \text{number of elements in E}\, &\text{if $E$ is finite}\\
		    \infty\, &\text{if $E$ is infinite}
		\end{cases}
	    \end{equation*}
	    Then each set is measurable and hence any function is measurable.
	\item
	    Define an outer measure $\outMeas$ on $\R$ by 
	    \begin{equation*}
		\outMeasure{E} = 
		\begin{cases}
		    0\, &\text{if $E = \emptyset$ }\\
		    1\, &\text{otherwise}
		\end{cases}
	    \end{equation*}
	    Then only $\emptyset$ and $\R$ are measurable. Indeed for any $E\subset \R$ where $E \neq \R,E\neq
	    \emptyset$ the $\carathCrit$, 
	    \begin{align*}
		\carathGeq{\R}{E}
	    \end{align*}
	    is NOT satisfied. Consider the set $f(x) = x$. Since $\fCompA{f}{>}{0}$ is not measurable, $f$ is
	    not measurable. 
	\item
	    Consider a measurable space $\metricS{X}{\algebra{M}}$. Let $\map{f}{X}{\R}$ be the
	    function given by $f(x) = k > 0$ whenever $x \in E$ for some $E\subset X$, else $f(x) = 0$. Hence the
	    range of $f$ is just $\lbrace 0,k\rbrace$ Then,
	    \begin{equation*}
		\fCompA{f}{>}{c} = 
		\begin{cases}
		    \emptyset &\text{if $c \geq k$}\\
		    E &\text{if $0\leq c < k$ }\\
		    X &\text{if $c < 0$}
		\end{cases}
	    \end{equation*}
	    Hence, $f$ is measurable iff $E$ is measurable.
	\item
	    Consider the Lebesgue measure space on $\R$, i.e the measurable space $\metricS{\R}{\family{L}}$ 
	    and the following function $f$
	    \begin{equation*}
		f(x) = 
		\begin{cases}
		    x^2, & x \leq 1 \\
		    2, & x = 1 \\
		    -2-x, & x > 1
		\end{cases}
	    \end{equation*}
	    Then,
	    \begin{equation*}
		\fCompA{f}{>}{c} = 
		\begin{cases}
		    (-\infty,-2-c) & c \leq -3 \\
		    \interval[open left]{-\infty}{1} & -3 < c < 0  \\
		    (-\infty,-\sqrt{c})\cup\hInt{\sqrt{c}}{1} & 0 \leq c < 1  \\
		    (-\infty,-\sqrt{c}) \cup \lbrace 1 \rbrace & 1\leq  c < 2 \\
		    (-\infty,-\sqrt{c}) & c \geq 2.
		\end{cases}
	    \end{equation*}
	    Hence, $f$ is measurable.
    \end{enumerate}
\end{Example}
Appropriate measurable functions agree with compostion. This is proved in the following Proposition.
\begin{Proposition}[name=Composing measurable functions]\label{prop:composition_meas_func}
    If $\metricS{X}{\algebra{M}}$, $\metricS{Y}{\algebra{N}}$ and $\metricS{Z}{\algebra{O}}$ are 
    measurable spaces and $\map{f}{X}{Y}$ is
    $\measMap{\algebra{M}}{\algebra{N}}$, $\map{g}{Y}{Z}$ is $\measMap{\algebra{N}}{\algebra{O}}$, then the
    compositon map $\map{g\circ f}{X}{Z}$ is $\measMap{\algebra{M}}{\algebra{O}}$.
\end{Proposition}
\begin{proof}
    Consider an arbitrary set $C \in \algebra{O}$. Then $B = \invIm{g}{C} \in \algebra{N}$ and so 
    $\invIm{f}{B} \in \algebra{M}$. Thus $\invIm{\fog{g}{f}}{C} = \invIm{f}{\invIm{g}{C}} \in \algebra{M}$.
\end{proof}
\begin{Remark}
    Note,~\ref{prop:composition_meas_func} does not imply that two Lebesgue measurable functions are measurable. 
    If $f$ is
    $\measMap{\algebra{L}}{\borelS{\R}}$ and $g$ is $\measMap{\algebra{L}}{\borelS{\R}}$, then the theorem
    above doesn't gurantee the Lebesgue measurability of $\fog{g}{f}$ or $\fog{f}{g}$, since there is no
    guarantee that for a set $C \in \borelS{\R}$, $\invIm{g}{C} \in {\algebra{L}}$. If $g$ is Borel measurable
    then the issue is resolved. See~\ref{fig:tikz:measurable_maps_comp}.
\end{Remark}
\begin{figure}
  \includestandalone[width=0.5\textwidth]{tex/tikz_figures/commutative_diag}
  \caption{Composition of measurable functions}\label{fig:tikz:measurable_maps_comp}
\end{figure}
A measurable function also induces a measure in the co-domain which in some regards is the \emph{natural}
measure. To make this idea concrete, we provide the next proposition.
\begin{Proposition}[name=Induced measure]\label{prop:induced_meas}
    Let $f$ be $\measMap{\algebra{M}}{\algebra{N}}$ function that maps from a measure space
    $\measureS{X}{\algebra{M}}{\mu}$ in to a measure space $\metricS{Y}{\algebra{N}}$. For any $B\in
    \algebra{N}$, set
    \[\mu_{f}(B) = \mu(\invIm{f}{B}).\]
    Then $\measureS{Y}{\algebra{N}}{\mu_f}$ is a measure space.
\end{Proposition}
\begin{proof}
    Since $\invIm{f}{\emptyset} = \emptyset$, $\mu_f(\emptyset) = 0$. Let $\lbrace B_i \rbrace \subset
    \algebra{N}$ be a sequence of pairwise disjoint sets. Then,
    \begin{align*}
	\mu_f(\countDisjUnion{B_i}{i}) &= \measure{\invIm{f}{\countDisjUnion{B_i}{i}}},\\ 
	&= \measure{\countDisjUnion{\invIm{f}{B_i}}{i}}, \\
	&= \infiniteSum{\measure{\invIm{f}{B_i}}}{i}, \\
	&= \infiniteSum{\mu_f(B_i)}{i}.
    \end{align*}
    Hence, $\measureS{Y}{\algebra{N}}{\mu_f}$ is a measure space. 
\end{proof}

\begin{figure}
  \includestandalone[width=0.5\textwidth]{tex/tikz_figures/induced_meas}
  \caption{Induced measure. Since $f$ is measurable, the inverse image when restricted to $\algebra{N}$ is a
      function that maps to $\algebra{M}$. 
      It is also called the \textbf{pull back} measure.}\label{fig:tikz:induced_measure}
\end{figure}
Even though we have included complex valued functions, we haven't defined yet the borel sigma algebra
$\borelS{\C}$. Note that $\borelS{\C} = \borelS{\R^{2}}$. Since we deal with a complex valued function by
dealing with its real and imaginary parts, it will be useful to describe a product sigma
algebra.

\begin{Remark}
    If we have a set $\metricS{X}{\algebra{M}}$ and a collection of measurable spaces 
    ${\lbrace\metricS{Y_{\alpha}}{\algebra{N}_{\alpha}} \rbrace }_{\alpha \in A}$, what is the smallest
    $\sigmaAlgebra$ $\algebra{M}$ on $X$ with respect to which each map $\map{f_{\alpha}}{X}{Y_{\alpha}}$ is
    $\measMap{\algebra{M}}{\algebra{N}_{\alpha}}$? 
    Note that $\invIm{f_{\alpha}}{\algebra{N}_{\alpha}}$ is a sigma algebra on $X$. Let 
    \[\family{E} = \indexUnion{\invIm{f_{\alpha}}{\algebra{N}_{\alpha}}}{\alpha}{A}.\]
    The unions of $\sigmaAlgebra$ is not necessarily a $\sigmaAlgebra$. The smallest sigma algebra containing a
    family is the one generated by it. Hence, $\algebra{M} = \sigmaGen{\algebra{E}}$ is the smallest sigma algebra
    on $X$ on which all $f_{\alpha}$ are measurable. In particular, if $X = \indexProduct{Y_{\alpha}}{\alpha}{A}$,
    then the co-ordinate maps $\map{\projFunc{\alpha}}{X}{Y_{\alpha}}$ will be
    $\measMap{\algebra{M}}{\algebra{N}_{\alpha}}$ for suitably defined $\algebra{N}_{\alpha}$. This leads us to
    the definition of a product $\sigmaAlgebra$.
\end{Remark}

\begin{Definition}[name=Product sigma algebra]
    Let $\set{X_{\alpha}}{\alpha \in A}$ be an indexed collection of non-empty sets. Set,
    \[X = \indexProduct{X_{\alpha}}{\alpha}{A},\quad\text{and}\quad\map{\projFunc{\alpha}}{X}{X_{\alpha}},\]
    as the co-ordinate map. If $\algebra{M}_{\alpha}$ is a $\sigmaAlgebra$ on $X_{\alpha}$ for each $\alpha
    \in A$, then we define the product sigma algebra on $X$ as the $\sigmaAlgebra$ generated by
    \[\family{E} = \set{\invIm{\projFunc{\alpha}}{E_{\alpha}}}{E_{\alpha}\in\algebra{M}_{\alpha},\alpha\in A}.\]
    We denote this $\sigmaAlgebra$ by $\iProdSigma{\algebra{M}_{\alpha}}{\alpha}{A}$.
\end{Definition}
Again, as with measurable functions we want to only check for sets in the family that generates
$\algebra{M}_{\alpha}$. This leads to the next proposition.
\begin{Proposition}\label{prop:prod_sigma_gen}
    Suppose that $\algebra{M}_{\alpha}$ is generated by $\family{E}_{\alpha}$. Then
    $\iProdSigma{\algebra{M}_{\alpha}}{\alpha}{A}$ is generated by,
    \[\family{F}_{1} = 
	\set{\invIm{\projFunc{\alpha}}{E_{\alpha}}}{E_{\alpha} \in \family{E}_{\alpha}\,\alpha\in A}.\]
\end{Proposition}
\begin{proof}
    Clearly $\family{F}_{1} \subset \family{E}$, hence $\sigmaGen{\family{F}_{1}} \subset \sigmaGen{\family{E}}$. We
    will use~\ref{rmk:prin_good_sets} to prove the other inclusion. Consider the set,
    \[\set{E\subset X_{\alpha}}{\invIm{\projFunc{\alpha}}{E}\in\sigmaGen{\family{F}_{1}}}.\]
    Clearly, the set is a $\sigmaAlgebra$ containing $\family{E}$ and hence contains $\sigmaGen{\family{E}}$.
\end{proof}
The next theorem enables us to use product sets if we have a countable collection. See Appendix for properties
of coordinate maps.
\begin{Proposition}\label{prop:prod_sigma_gen_countable}
    If $A$ is a countable collection of index sets, then $\iProdSigma{\algebra{M}_{\alpha}}{\alpha}{A}$ is the
    sigma algebra generated by.
    \[\family{F}_{2} = \set{\indexProduct{E_{\alpha}}{\alpha}{A}}{E_{\alpha} \in \family{E}_{\alpha}},\]
    where $\algebra{M}_{\alpha} = \sigmaGen{\family{E}_{\alpha}}$.
\end{Proposition}
\begin{proof}
    First we will show that $\iProdSigma{\algebra{M}_{\alpha}}{\alpha}{A}$ is generated by,
    \[\family{F} = \set{\indexProduct{E_{\alpha}}{\alpha}{A}}{E_{\alpha} \in \family{M}_{\alpha}},\]
    and then use~\ref{prop:prod_sigma_gen} to complete the proof.
    Note that for any $x \in \family{E}$, there is an $E_{\alpha} \in \algebra{M}_{\alpha}$ such that $x =
    \invIm{\projFunc{\alpha}}{E_{\alpha}}$. Now,
    \[\invIm{\projFunc{\alpha}}{E_{\alpha}} = \indexProduct{Y_{\beta}}{\beta}{A},\]
    where $Y_{\beta} = X_{\beta}$ whenever $\beta \neq \alpha$, and $Y_{\beta} = E_{\alpha}$ whenever 
    $\beta = \alpha$. Thus $x \in \family{F}$ i.e, $\family{E} \subset \family{F}$ and hence
    $\sigmaGen{\family{E}}\subset \sigmaGen{\family{F}}$. To show the other inclusion, let $x \in
    \family{F}$,
    then $x = \indexProduct{E_{\alpha}}{\alpha}{A}$, where $E_{\alpha} \in \algebra{M}_{\alpha}$. Now,
    \[\indexProduct{E_{\alpha}}{\alpha}{A} =
	\indexIntersection{\invIm{\projFunc{\alpha}}{E_{\alpha}}}{\alpha}{A} \in \sigmaGen{\family{E}}.\]
    Thus, $\sigmaGen{\family{F}} \subset \sigmaGen{\family{E}}$. Now using~\ref{prop:prod_sigma_gen}, we get
    the result since $\sigmaGen{\family{F}_1} = \sigmaGen{\family{E}} = \sigmaGen{\family{F}}$.
\end{proof}
\begin{Theorem}[name=Sigma Algebra in product metric space]
    Let $X_1,\ldots,X_n$ be metric spaces and let $X = \finiteProduct{X_j}{j}{n}$,equipped with the product
    metric. Then $\fProdSigma{\borelS{X_j}}{j}{n} \subset \borelS{X}$. If $X_j$'s are \textbf{separable}, then
    $\fProdSigma{\borelS{X_j}}{j}{n} = \borelS{X}$.
\end{Theorem}
\begin{proof}
\end{proof}
\begin{Corollary}\label{thm:prod_sigma_algeb_Rn}
    $\borelS{\Rn} = \fProdSigma{\borelS{\R}}{j}{n}$.
\end{Corollary}
The following propositon is useful in considering the measurability of vector valued functions.
\begin{Theorem}\label{thm:meas_component_func}
    Let $\metricS{X}{\algebra{M}}$ and $\metricS{{Y}_{\alpha}}{\algebra{N}_{\alpha}}$ ($\alpha \in A$)
    be measurable spaces and let $Y = \indexProduct{Y_{\alpha}}{\alpha}{A}$. Define $\family{N}$ to be the product
    sigma algebra in $Y$ and $\projFunc{\alpha}$ to be the $\alpha^{th}$ co-ordinate map. Then,
    \begin{align*}
	&\map{f}{X}{Y}\quad\text{is}\measMap{\algebra{M}}{\algebra{N}}\quad\text{iff} \\
	&\quad 
	f_{\alpha} = \fog{\projFunc{\alpha}}{f}\quad \text{is} \quad \measMap{\algebra{M}}{\algebra{N}_{\alpha}} 
	\,\text{for all $\alpha$}.
    \end{align*}
\end{Theorem}
\begin{proof}
    By definition, each $\projFunc{\alpha}$ is $\measMap{\algebra{N}}{\algebra{N}_{\alpha}}$. If $f$ 
    is $\measMap{\algebra{M}}{\algebra{N}}$ then by~\ref{prop:composition_meas_func}, $f_{\alpha}$ is 
    $\measMap{\algebra{M}}{\algebra{N}_{\alpha}}$.

    To show that $f$ is measurable given that each $f_{\alpha}$ is measurable we need to show that for each
    $E\in \family{E}$, $\invIm{f}{E}\in \algebra{M}$ where $\family{E}$ is the generating set $\algebra{N}$.
    Let $E$ be an arbitrary element of $\family{E}$. Then $E = \invIm{\projFunc{\alpha}}{E_{\alpha}}$ for some
    $\alpha \in A$ and $E_{\alpha} \in \algebra{N}_{\alpha}$. Thus,
    \[\invIm{f}{\invIm{\projFunc{\alpha}}{E_{\alpha}}} = {(\fog{\projFunc{\alpha}}{f})}^{-1}(E_{\alpha}) =
	\invIm{f_{\alpha}}{E_{\alpha}} \in \algebra{M}\]
    since $f_{\alpha}$ is $\measMap{\algebra{M}}{\algebra{N}_{\alpha}}$.
\end{proof}
\begin{Corollary}
    Let $\metricS{X}{\algebra{M}}$ be a measurable space. A function $\map{f}{X}{\Rm}$ is 
    $\algebra{M}$-measurable 
    iff each co-ordinate function $\map{f_{i}}{X}{\R}$, for $1\leq i \leq m$, is $\algebra{M}$-measurable.
\end{Corollary}
\begin{proof}
     The function $\map{f}{X}{\Rm}$ is $\algebra{M}$-measurable whenever it is
    $\measMap{\algebra{M}}{\borelS{\Rm}}$. By~\ref{thm:prod_sigma_algeb_Rn}, $\borelS{\Rm} =
    \fProdSigma{\borelS{\R}}{i}{m}$, hence by~\ref{thm:meas_component_func}, $f$ is $\algebra{M}$-measurable iff
    $\map{f_{i}}{X}{\R}$, for $1\leq i \leq m$, is $\algebra{M}$-measurable.
\end{proof}
\begin{Corollary}
    Let $\metricS{X}{\algebra{M}}$ be a measurable space. A function $\map{f}{X}{\C}$ is
    $\algebra{M}$-measurable iff $\Rea f$ and $\Ima f$ are $\algebra{M}$-measurable.
\end{Corollary}
\begin{proof}
    By~\ref{thm:meas_component_func} and the fact that $\borelS{\C} = \borelS{\R}\bigotimes\borelS{\R}$.
\end{proof}
Intuition suggests measurability should not be affected by behavior on sets of measure $0$. This is indeed
true for complete measures.
\begin{Proposition}\label{prop:ae_equality_meas_func}
    Let $\metricS{X}{\algebra{M}}$, $\metricS{Y}{\algebra{N}}$ be measure spaces. Let $\metricS{X}{\algebra{M}}$
    be a complete measure space. If $f$ is $\measMap{\algebra{M}}{\algebra{N}}$ and $f=g$ a.e., then $g$ is
    also $\measMap{\algebra{M}}{\algebra{N}}$.
\end{Proposition}
\begin{proof}
    Let $A = \set{x \in X}{f(x)\neq g(x)}$. By our assumption, measure of $A$ is $0$. Fix any $E \subset
    \algebra{N}$ and consider the set
    $\invIm{g}{E} = \set{x\in X}{g(x) \in E}$. Then we can write this set a union of disjoint sets as follows,
    \begin{align*}
	\set{x}{g(x) \in E} &= \set{x}{g(x) \in E}\cap A \disjU \set{x}{g(x) \in E}\cap
	\comp{A} \\
	&\set{x}{g(x) \in E}\cap A \disjU \set{x}{f(x) \in E}, 
    \end{align*}
    where the second set after the equality is due to the fact that $f=g$ in $\setDiff{X}{A}$. The first set
    after the equality is a subset of $A$ which has measure $0$, since $\algebra{M}$ is complete and thus is
    in $\algebra{M}$. The second set is in $\algebra{M}$ by our assumption. Thus, $\invIm{g}{E}$ is the union
    of two measurable sets in $\algebra{M}$ and hence, $g$ is $\measMap{\algebra{M}}{\algebra{N}}$.
\end{proof}
Before we observe some important properties of measurable functions, let us define the characteristic function
of a set. Such a function is the building block for approximating any measurable functions. 
\begin{Definition}[name=Characteristic function]
    Let $\metricS{X}{\algebra{M}}$ be a measure space and let $E \subset X$ be any set. The characteristic
    function $\charFunc{E}$ is given by,
    \begin{equation}\label{eq:char_func}
	\charFunc{E}(x) = 
	\begin{cases}
	    1 &\text{if $x\in E$}\\
	    0 &\text{if $x\in \comp{E}$}\\
	\end{cases}
    \end{equation}
    The characteristic function of a set $E$ is also called the \textbf{indicator} function of the set $E$ and
    is denoted by $\boldsymbol{I}_{E}$ or $\boldsymbol{1}_{E}$.
\end{Definition}
It is easy to see (~\ref{ex:measurable_func} (3)) that $\charFunc{E}$ is measurable if and only if $E \in
\algebra{M}$. 
\begin{Remark}
    Note that if $A,B$ are disjoint sets then $\charFunc{A\cup B} = \charFunc{A} + \charFunc{B}$. This means
    that $\charFunc{A} + \charFunc{\comp{A}} = 1$. For any
    $A,B$ we can observe that $\charFunc{A}\charFunc{B} = \charFunc{A\cap B}$. These observations are very
    useful.
\end{Remark}
\section{Properties of measurable functions}
Let $\metricS{X}{\algebra{M}}$ be a measure space.

\begin{Proposition}\label{prop:prop0_mfunc_composition_cont}
    If $\map{f}{X}{\C}$ is $\algebra{M}$-measurable and $\map{\Phi}{X}{C}$ is continuous then,
    \begin{properties}[series=mfunc]
    \item 
	$\fog{\Phi}{f}$ is $\algebra{M}$-measurable.
    \end{properties}
\end{Proposition}
\begin{Proposition}\label{prop:prop1_mfunc_additivity}
    If $\map{f,g}{X}{\C}$ are $\algebra{M}$-measurable then,
    \begin{properties}[resume*=mfunc]
    \item 
	$f+g$ and $fg$ are $\algebra{M}$-measurable.
    \end{properties}
\end{Proposition}
\begin{proof}
    Define $\map{F}{X}{\C \times \C}$ by $F(x) = (f(x),g(x))$. Then, since $\borelS{\C\times\C} =
    \borelS{\C}\bigotimes\borelS{\C}$, $F$ is measurable by~\ref{thm:meas_component_func}. Now, consider the
    following functions,
    \begin{itemize}
	\item $\map{\phi}{\C\times\C}{\C}$ given by, $\phi(z,w) = z + w$,
	\item $\map{\psi}{\C\times\C}{\C}$ given by, $\psi(z,w) = zw$.
    \end{itemize}
    Both $\phi,\psi$ are $\measMap{\borelS{\C\times\C}}{\borelS{\C}}$, since they are continuous
    (see~\ref{thm:continuous_func_meas}). Hence, $f + g = \fog{\phi}{F}$ and 
    $fg = \fog{\psi}{F}$ are $\mMeas$.
\end{proof}
By taking $g = c$ a constant we see that $cf$ is also $\mMeas$. If $g$ is function that is not $0$, then by
using $1/g$ above we see $f/g$ is also $\mMeas$. Hence, we preserve measurability by doing arithmetic
operations.
\begin{Proposition}\label{prop:prop2_mfunc_limits}
    If $\seq{f}{j}$ is a sequence of $\closure{\R}$ valued $\mMeas$ functions then the functions,
    \begin{properties}[resume*=mfunc]
    \item
	$\sup\limits_{j}f_j,\inf\limits_{j}f_j,\limsup f_j, \liminf f_j$ are $\mMeas$. If
	$f = \limit{f_j}{n}{\infty}$ exists for all $x\in X$, then $f$ is $\mMeas$.
    \end{properties}
\end{Proposition}
\begin{proof}
    Let $g_1 = \sup\limits_{j}f_j$. For each $j$, $f_j(x) \leq g_1(x)$ for all $x \in X$. Let $g > g_1$.
    Hence, $\fCompA{f_j}{\geq}{a} \subset \fCompA{g}{\geq}{a}$. Thus for each $j$,
    \[\fCompA{f_j}{\geq}{a} \subset \countUnion{\fCompA{f_j}{\geq}{a}}{j}\subset \fCompA{g}{\geq}{a}.\]
    Hence,
    \[\fCompA{g_1}{\geq}{a} = \countUnion{\fCompA{f_j}{\geq}{a}}{j}.\]
    Thus $g_1$ is $\mMeas$ since it is a countable union of $\mMeas$ sets.
    Let $g_2 = \inf\limits_j f_j$. A similar argument yields,
    \[\fCompA{g_2}{\leq}{a} = \countUnion{\fCompA{f_j}{\leq}{a}}{j}.\]
    Now, $\liminf f_j = \sup\limits_{k}\inf\set{f_j}{j\geq k}$ and $\limsup f_j =
    \inf\limits_{k}\sup\set{f_j}{j\geq k}$ are $\mMeas$ since $\inf,\sup$ are $\mMeas$.
    Finally, if $f = \limit{f_j}{j}{\infty}$ exists for all $x$ then $f = \limsup f_j$ and hence $f$ is
    $\mMeas$.
\end{proof}
\begin{Definition}[name=Decomposition of real valued functions]
    Let $\map{f}{X}{\closure{R}}$. 
    Consider the sets, \[A^{+} = \fCompA{f}{>}{0} \quad \text{and} \quad A^{-} = \fCompA{f}{<}{0}.\]
    Then $f$ can be \textbf{decomposed} into is positive and negative parts, 
    \[f = \fPlus{f} - \fMinus{f},\]
    where $\fPlus{f} = f\charFunc{A^{+}} = \max(f,0)$ and 
    $\fMinus{f} = -f\charFunc{A^{-}} = \max(-f,0)$.
\end{Definition}
Note that each $\fPlus{f}$ and $\fMinus{f}$ is non-negative.
\begin{figure}
  \includestandalone[width=0.35\textwidth]{tex/tikz_figures/f_decomposition}
  \caption{Decomposition of a real valued function. Here $\fPlus{f}$ is drawn in blue,
  while $\fMinus{f}$ is drawn in red.}\label{fig:tikz:decomposition_f}
\end{figure}
\begin{Proposition}\label{prop:prop3_mfunc_decomposition}
    Let $\map{f}{X}{\closure{R}}$.
    \begin{properties}[resume*=mfunc]
    \item
	$f$ is $\mMeas$ iff $\fPlus{f},\fMinus{f}$ are $\mMeas$.
    \end{properties}
\end{Proposition}
\begin{proof}
    If $f$ is $\mMeas$ then $A^{+},A^{-}$ are measurable (sets in $\algebra{M}$) and hence
    $\fPlus{f},\fMinus{f}$ are $\mMeas$. 
    If $\fPlus{f},\fMinus{f}$ are $\mMeas$ then $f$ is the sum of two
    $\mMeas$ functions and so is $\mMeas$. 
\end{proof}
Note that $\abs{f} = \fPlus{f} + \fMinus{f}$. Hence, if $f$ is $\mMeas$ then $\abs{f}$ is also $\mMeas$. 
\begin{Proposition}\label{prop:prop4_mfunc_ae_limit}
    Let $\seq{f}{j}$ be a sequence of $\closure{R}$ valued $\mMeas$ functions. Let $g_1$ be a $\mMeas$
    function. Let $\algebra{M}$ be a complete
    measure space. 
    \begin{properties}[resume*=mfunc]
    \item
	If $g_1 = g_2$ a.e., then $g_2$ is $\mMeas$.
    \item
	If $\limit{f_j}{j}{\infty} = f$ a.e., then $f$ is $\mMeas$.
    \end{properties}
\end{Proposition}
\begin{proof}
    By~\ref{prop:ae_equality_meas_func}, with $f=g_1$ and $g=g_2$ we get the result.

    Let $g = \limsup f_j$. Then by~\ref{prop:prop2_mfunc_limits}, $g$ is $\mMeas$. Also $g=f$ a.e., hence
    using the result from above $f$ is $\mMeas$.
\end{proof}
\section{Approximation by simple functions}
In this section, we will see that any $\mMeas$ function $\map{f}{X}{\C}$ can be approximated by simpler
functions named appropriately as \textbf{simple function}. 
\begin{Definition}
    A simple function $\map{s}{X}{\C}$ is a finite sum,
    \[s = \finiteSum{a_k\charFunc{E_k}}{k}{N},\]
    where each $E_k$ is a measurable set in $\algebra{M}$ of finite measure and $a_k \in \C$ are constants. We
    require that $E_k$'s be disjoint and $X = \finiteDisjUnion{E_k}{k}{N}$.
\end{Definition}
Since the characteristics functions are measurable, a simple function is measurable
by~\ref{prop:prop1_mfunc_additivity}. Note that simple functions do not have unique representation as seen in
the following example.
\begin{Example}
    Let $X = \interval{0}{3}$ and define 
    \begin{equation*}
	s(x) = 
	\begin{cases}
	    -1, &0\leq x \leq 1 \\
	    2, & 1< x \leq 2 \\
	    0, & 2< x \leq 3 \\
	\end{cases}
    \end{equation*}
    Let $E_1 = \interval{0}{1}, E_2 = \hInt{1}{2} , E_3 = \hInt{2}{3}$. Then,
    \[s = -1\charFunc{E_1} + 2\charFunc{E_2} + 0\charFunc{E_3}.\]
    However, if we define $D_1 = \interval[open right]{0}{1/2}, D_2 = \interval{1/2}{1}, D_3 =\hInt{1}{2},
    D_4=\hInt{2}{3}$, then
    \[s = -1\charFunc{D_1} - 1\charFunc{D_2} + 2\charFunc{D_3} + 0\charFunc{D_4}.\]
\end{Example}
A key result in measure theory is that any function can be approximated by a sequence of monotonic increasing
simple functions. We give an alternate characterization for simple functions,
\begin{Proposition}\label{prop:characterization_simple_func}
    Let $\metricS{X}{\algebra{M}}$ be a measurable space. A function \break$\map{s}{X}{\C}$ is a simple 
    function if and only if $s$ is measurable and the range of $s$ is a finite set of points in $\C$. 
\end{Proposition}
\begin{proof}
    If $s$ is measurable function, then by definition its range is finite. For the other implication, assume
    range of $s = \set{z_i}{1\leq i\leq n}$. Let $E_i = \invIm{s}{\lbrace z_i\rbrace}$. For any $x \in X$, if
    $x\in E_i$, then $s(x) = z_i$. Thus,
    \[s = \finiteSum{z_i\charFunc{E_i}}{i}{n},\]
    is a simple function. Note that $\finiteUnion{E_i}{i}{n} = X$ and $E_i$'s are pairwise disjoint.
\end{proof}
In the proof above we used the \textbf{standard} representation of $s$. It exhibits $s$ as a linear 
combination, with distinct coefficient, of characteristic functions of disjoint measurable sets whose union 
is $X$. A key observation is that the standard representation of a simple function is unique. The proposition
that follows shows that the space of simple functions defined on a measure space is a Vector space.
\begin{Proposition}\label{prop:prop_simple_func}
    Let $\metricS{X}{\algebra{M}}$ be a measure space. 
    \begin{properties}
    \item
	If $s$ is a simple function defined on $X$ then $cs$ is also a simple function.
    \item
	If $s_1,s_2$ are  simple functions defined on $X$ then $s_1 + s_2$ is also a simple function.
    \end{properties}
\end{Proposition}
\begin{proof}
    We will show $(2)$. Let $s_1 = \finiteSum{a_k\charFunc{E_k}}{k}{N}$ and $s_2 =
    \finiteSum{b_j\charFunc{F_j}}{j}{M}$ be two simple functions in standard representation. 
    First assume $M=1$. Note that $E_k$'s and $F_1$ all partition $X$ and hence $F_1 = X$. Let $G_i = E_i\cap
    F_1$ for all $1\leq i \leq N$. Then, $\finiteDisjUnion{G_i}{i}{N} = X$ and 
    \[s_1 + s_2 = \finiteSum{c_l\charFunc{G_l}}{l}{N},\]
    where $c_l = a_l + b_1$ for $1\leq l \leq N$. This idea can be extended to all $M \geq 1$ by the using the
    collection,
    \[\set{E_k\cap F_j}{1\leq k \leq N,\, 1\leq j \leq M},\]
    and using the sum $a_k + b_j$ in the corresponding set. See~\ref{fig:tikz:simple_func_add}. Thus we see
    that,
    \[s_1 + s_2 = \series{(a_k + b_j)\charFunc{E_k\cap F_j}}{i,j}{1}{}.\]
\end{proof}

\begin{figure}
  \includestandalone[width=0.5\textwidth]{tex/tikz_figures/simple_func_add}
  \caption{Addition of two simple functions.}\label{fig:tikz:simple_func_add}
\end{figure}
We now come to the most important result in this chapter.
\begin{Theorem}[name=Approximation by simple functions]\label{thm:approx_by_simple_func}
    Let $\metricS{X}{\algebra{M}}$ be a measure space.
    \begin{enumerate}
	\item
	    If $\map{f}{X}{\extRealsPos}$ is measurable, there is a sequence $\seq{s}{n}$ of simple functions
	    such that,
	    \[0\leq s_1 \leq s_2 \leq s_3 \cdots \leq f\quad\text{and}\quad\atob{s_n}{f},\]
	    pointwise. The convergence is uniform on any set on which $f$ is bounded.
	\item
	    If $\map{f}{X}{\C}$ is measurable, there is a sequence $\seq{s}{n}$ of simple functions such that,
	    \[0\leq \abs{s_1} \leq \abs{s_2} \leq \abs{s_3} \cdots \leq \abs{f} 
		\quad\text{and}\quad\atob{s_n}{f},\]
	    pointwise. The convergence is uniform on any set on which $f$ is bounded.

    \end{enumerate}
\end{Theorem}
\begin{proof}
    We prove in order. See~\ref{fig:tikz:approx_simple_func}. 
    \begin{enumerate}
	\item
	    The key idea will be to \emph{partition the range} of $f$. For each $n$, we partition the
	    co-domain of $f$ by two sets,
	    $\interval{0}{n}$ and $\hInt{n}{\infty}$. We partition $\interval{0}{n}$ further and then treat
	    the other set as a \emph{remainder}. In order to get an approximation from below, we need the
	    partition of $\interval{0}{n}$ to increase with $n$ which can be done by decreasing the 
	    partition length. 
	    Thus for each $n$, 
	    we fix a uniform partition length of $\Delta_n = \frac{1}{2^n}$, hence as $\atob{n}{\infty}$,
	    $\atob{\Delta_n}{0}$. Let us index the partition as follows,
	    \[I_k^n = \interval[open right]{k\Delta_n}{(k+1)\Delta_n}\quad,0\leq k < \frac{n}{\Delta_n}.\]
	    Thus,
	    \begin{align*}\invIm{f}{\interval{0}{\infty}} &= \invIm{f}{\interval{0}{n}} 
		\bigcup \invIm{f}{\hInt{n}{\infty}} \\
		& = \bigcup\limits_{k=0}^{\frac{n}{\Delta_n}-1}(\invIm{f}{I_k^n}) \bigcup 
		\invIm{f}{\hInt{n}{\infty}}. \\ 
	    \end{align*}
	    Let us set,
	    \begin{align*}
		& E_k^n = \invIm{f}{I_k^n},\quad 0\leq k < \frac{n}{\Delta_n} = n2^n\\
		& F_n = \invIm{f}{\hInt{n}{\infty}}
	    \end{align*}
	    We define the simple function $s_n$ as,
	    \[s_n = \series{k2^{-n}\charFunc{E_k^n}}{k}{0}{n2^n-1} + n\charFunc{F_n}.\]
	    Each $I_k^n$ splits into $2$ intervals at the level $n+1$ and hence when $x \in E_k^n$, $s_n(x)
	    \leq s_{n+1}(x)$. When $x \in \hInt{n}{n+1}$, then $s_n(x) = n$ whereas, $s_{n+1}(x) \geq n$. When
	    $x\in \hInt{n+1}{\infty}$, then $s_n(x) = n$ while $s_{n+1}(x) = n+1$ and thus $s_n \leq s_{n+1}$
	    for all $x$,$n$. Also, whenever $f(x) \leq N$, then $f(x) \in I_k^n$ for some $k \in
	    1,\dots,n2^n$. Thus $f(x) - s_n(x) \leq \frac{k+1}{2^n} - \frac{k}{2^n} = \frac{1}{2^n}$, which
	    goes to $0$ as $\atob{n}{\infty}$. 

	    If $f$ is bounded in some set, then there is a $N$ such that $f(x) < N$ for
	    all $x$ in the set. 
	    But, for any $x$, $f(x) - s_N(x) < \frac{1}{2^N}$ and taking $N$ larger and using the
	    fact that $s_n \geq s_N$ for $n\geq N$ we can make $f(x) - s_n(x) < \epsilon$ for any $\epsilon$.
	    Thus, convergence is uniform.
	\item
	    We can decompose $\Rea f$, $\Ima f$ as $\fPlus{f},\fMinus{f}$ and use the preceeding theorem.
    \end{enumerate}
\end{proof}

\begin{figure}
  \includestandalone[width=0.75\textwidth]{tex/tikz_figures/approx_by_simple_func}
  \caption{Approximation by simple functions}\label{fig:tikz:approx_simple_func}
\end{figure}
\break{}
%\section{\texorpdfstring{$\ast$}{}Random Variables}
\section{{$\ast$} Random Variables}
Let $\probS$ be a probability space and let $\metricS{\mbbY}{\algebra{N}}$ be a measure space. A
$\measMap{\family{F}}{\algebra{N}}$ function $X$ is called a $\mbbY$ valued random variable.
In particular for $\mbbY = \R$ or $\mbbY = \Rn$ we have the following definition,
\begin{Definition}[name=Random variable]
    A random variable (r.v) $X$ is a real valued 
    $\measMap{\algebra{F}}{\borelS{\R}}$ measurable function that maps from $\probS$ into
    $\metricS{\R}{\borelS{\R}}$. A random vector $X$ is a vector valued $\measMap{\family{F}}{\borelS{\Rn}}$
    function that maps from $\probS$ into $\metricS{\Rn}{\borelS{\Rn}}$.
\end{Definition}
As always, we wish to discard the importance of behavior of sets of measure zero.
\begin{Definition}[name=Almost surely]
    Suppose $X,Y$ are two random variables on the same probability space. Then $X = Y$ a.s.~or $X = Y$ almost
    surely, means that,
    $\probMeasure{\set{\omega\in\Omega}{X(\omega) \neq Y(\omega)} }= 0$.
\end{Definition}
Note, almost surely is precisely what is meant by almost everywhere that was defined for a general measure
space.
\begin{Example}
    Consider the space $\mcalB$ of Bernoulli sequence and the corresponding measure space $\Omega = \hInt{0}{1}$
    with the Lebesgue measure $\mu$. Any Rademacher function $R_n$ which maps from
    $\measureS{\hInt{0}{1}}{\family{L}}{\mu}$ into $\metricS{\R}{\borelS{\R}}$ 
    is a random variable because for any $E\in\borelS{\R}$, 
    $\invImIndx{R}{n}{E} = \set{\omega\in\hInt{0}{1}}{R_n(\omega)\in E}$ is a finite union of
    intervals. Similarly $W_N$ is a random variable.
\end{Example}
\begin{Example}
    Consider the space $\mcalB$ of Bernoulli sequence and let $X$ be a function from
    $\setDiff{\mcalB}{\mcalB_{\text{neg}}}$ in to $\hInt{0}{1}$ defined by,
    \[X(b_1,b_2,b_3,\dots) = 0.\omega_1\omega_2\dots,\]
    where $\omega_i = 1$ if $b_i$ is a head $H$, $\omega_i = 0$ if $b_i$ is a tails $T$. Then $X$ is a random
    variable. Thus a random variable \emph{transforms} a probabilistic process to a measure space.
\end{Example}
\begin{Proposition}
    Let $X$ be a r.v.~from a probability space $\probS$ to a measure space $\metricS{\mbbY}{\algebra{N}}$.
    Let $\indMeas{\probMeas}{X}$ be the induced measure in $\metricS{\mbbY}{\algebra{N}}$. Then, 
    $\measureS{\mbbY}{\algebra{N}}{\indMeas{\probMeas}{X}}$ is a probability space.  
\end{Proposition}
\begin{proof}
    By~\ref{prop:induced_meas}, we know that $\measureS{\mbbY}{\algebra{N}}{\indMeas{\probMeas}{X}}$ is a
    measure space. Hence, we only need to check if $\indMeasure{\probMeas}{X}{\mbbY} = 1$. Indeed
    $\invIm{X}{\mbbY} = \Omega$ and $\probMeasure{\Omega} =1$, and thus we see that
    $\measureS{\mbbY}{\algebra{N}}{\indMeas{\probMeas}{X}}$ is a probability space.
\end{proof}
\begin{Definition}[name=Distribution]
    The probability measure $\indMeas{\probMeas}{X}$ on $\metricS{\mbbY}{\algebra{N}}$ is called the
    distribution of the r.v.~X and is said to be induced by $X$. 
\end{Definition}
\begin{Theorem}\label{thm:rv_equal_ae_distribution}
    Suppose $X,Y$ are two r.v.~mapping a (complete) probability space $\probS$ to a measurable space
    $\metricS{\mbbY}{\famN}$, 
    such that $X=Y$ a.s.~Then $X,Y$ have the same distribution.
\end{Theorem}
\begin{proof}
    Let $N = \set{\omega\in\Omega}{X(\omega) \neq Y(\omega)}$. Then, by our assumption, $\probMeasure{N} = 0$.
    Let $B \in \famN$ be arbitrary. Then \[\invIm{X}{B} = (\invIm{X}{B}\cap N) \disjU
    (\invIm{X}{B}\cap\comp{N}).\]
    Thus,
    \begin{align*}
	\indMeasure{\probMeas}{X}{B} &= \probMeasure{\invIm{X}{B}} \\  
	& = \probMeasure{\invIm{X}{B}\cap N} + \probMeasure{\invIm{X}{B}\cap\comp{N}} \\
	& \leq 0 + \probMeasure{\invIm{Y}{B}}\\
	&\quad = \indMeasure{\probMeas}{Y}{B}
    \end{align*}
    Similarly, $\indMeasure{\probMeas}{Y}{B} \leq \indMeasure{\probMeas}{X}{B}$.
\end{proof}
Even when we don't have a complete probability space, we can easily complete it by~\ref{thm:comp_of_meas}.
Thus we can easily forget if the probability space is complete or not. Note that the converse is not true.
Given a r.v.~induces probability measure on its co-domain, we can define new r.v on the range. We have the
following relationship,
\begin{Proposition}
    Let $X$ be $\mbbY$ valued r.v.~from a probability space $\probS$ to a 
    measurable space $\metricS{\mbbY}{\famN}$. Let $Y$ be a $\measMap{\famN}{\famO}$ 
    function from $\metricS{\mbbY}{\famN}$ in to $\metricS{\mbbW}{\famO}$. Let $\indMeas{\probMeas}{X}$ be the
    distribution on $\metricS{\mbbY}{\famN}$. Then $Y$ is a $\mbbW$ valued r.v.~on
    $\measureS{\mbbY}{\famN}{\indMeas{\mbbP}{X}}$ with the same distribution as the $\mbbW$ valued
    r.v.~$\fog{Y}{X}$ on $\probS$.
\end{Proposition}
\begin{proof}
    It follows the same ideas as in~\ref{prop:composition_meas_func}.
\end{proof}
Since a r.v.~is a measurable function, all the
properties in~\ref{prop:prop0_mfunc_composition_cont}-\ref{prop:prop4_mfunc_ae_limit} apply to a r.v.
We now specialize to the case of random variables mapping into $\metricS{\R}{\borelS{\R}}$. Let us (re) define
a distribution function for the case of a probability space.
\begin{Definition}[name=Distribution function]
    A real valued function $F$ defined on $\R$ is a distribution function if it is monotone increasing, right
    continuous, and 
    \[\limit{F(x)}{x}{-\infty} = 0,\quad \limit{F(x)}{x}{\infty} = 1.\]
\end{Definition}
Given a probability measure $\probMeas$ on $\R$, we can always define a distribution function
$\map{F}{\R}{\interval{0}{1}}$ as $F(x) = \probMeasure{\hInt{-\infty}{x}}$ and vice versa
by~\ref{thm:borel_measure_R}. Thus if we have a probability space $\probS$ and a r.v.~on it, we get an induced
probability measure (distribution) $\indMeas{\probMeas}{X}$ on $\metricS{\R}{\borelS{\R}}$,
which then corresponds to a distribution function $\map{F_X}{\R}{\interval{0}{1}}$ and is 
given by $F_X(x) = \indMeasure{\probMeas}{X}{\interval{-\infty}{x}}$.
Thus,
\[X \leadsto F_X.\]
Is the converse true, i.e.
\[F \leadsto X_F?\]
The following theorem answers this question.
\begin{Theorem}\label{thm:distribution_func_to_rv}
    If a function $\map{F}{\R}{\interval{0}{1}}$ is a distribution function, then there exists a
    r.v.~$\map{X}{\interval{0}{1}}{\R}$ defined
    on the probability space $\measureS{\interval{0}{1}}{\famL_{\interval{0}{1}}}{\mu}$ such that $F = F_X$,
    where $\famL_{\interval{0}{1}}$ is the family of Lebesgue measurable sets on $\interval{0}{1}$ and $\mu$
    is the corresponding Lebesgue measure.
\end{Theorem}
\begin{proof}
    Let us define $\map{X}{\interval{0}{1}}{\R}$ by,
    \[X(\omega) = \inf\set{x\in\R}{F(x)\geq \omega},\quad 0\leq\omega\leq 1.\]
    First, we will show that $X$ is a $\measMap{\famL_{\interval{0}{1}}}{\borelS{\R}}$ function and hence a 
    r.v. Fix an $a \in \R$ and consider the set $\fCompA{X}{\leq}{a}$. Since $F$ is increasing this set is
    just an interval $\interval{0}{c}$ where $c$ is the $\sup\set{\omega}{X(\omega)\leq a}$. Hence $X$ is a r.v.

    To show that $F = F_X$ we need to show that for any $y\in \R$,
    \begin{align*}
	F(y) = F_X(y) &= \indMeasure{\mu}{X}{\hInt{-\infty}{y}},\\
	& = \measure{\invIm{X}{\hInt{-\infty}{y}}},\\
	& = \measure{\set{\omega\in\Omega}{X(\omega)\leq y}}.
    \end{align*}
    Let $A = \set{\omega}{X(\omega)\leq y}$. Then $A = \interval{0}{c}$ where 
    $c = \sup\set{\omega}{X(\omega)\leq y}$. Hence $\measure{A} = c$. Thus we need
    to show that $F(y) = c = \sup\set{\omega}{X(\omega)\leq y}$.
    First note that if $\omega = F(y)$ then $X(\omega) = y$ because $F$ is increasing. Thus $F(y) \in A$. If
    we show that $F(y)$ is an upper bound for $A$, then we are done. If $\omega \in A$, then $X(\omega) \leq
    y$. Thus $F(X(\omega)) \leq F(y)$. Since $F$ is right continuous at $X(\omega)$, for any $\epsilon$ there
    is a $\delta$ such that $F(X(\omega) + \delta) - F(X(\omega)) < \epsilon$. Now $X(\omega)$ is the infimum
    of all those $x \in \R$ such that $F(x) \geq \omega$. Hence by definition of infimum, 
    there is an $x_0 \in\R$ such that $x_0 <
    X(\omega) + \delta$ and $F(x_0) \geq \omega$. Thus, because $F$ is increasing,
    \[F(X(\omega)+\delta) \geq F(x_0) \geq \omega.\]
    Hence $\omega < \epsilon + F(X(\omega))$. Since $\epsilon$ was arbitrary we get $\omega \leq
    F(X(\omega))$. Hence $F(y)$ is an upper bound for $A$.
\end{proof}
\section{Littlewood's three principles}
