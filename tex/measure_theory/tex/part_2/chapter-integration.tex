\chapter{Integration theory}
\section{Class of Riemann integrable functions}
\section{Abstract integration in measure space}
We have seen that Riemann integration doesn't behave nicely when interchanging limit order. To remedy this, we
will define the (Lebesgue) integral in a general measure space. In order to achieve this nice property we will
expand the functions that can be integrated. As we saw in the previous section, the basic idea of Riemann
Integration of a bounded function is to partition the domain. In this section we will partition the range of
the function. The theory will proceed in $3$ stages:
\begin{enumerate}
    \item
	Define integration for simple functions.
    \item
	Define integration for non-negative measurable functions.
    \item
	Define integration for general measurable functions.
\end{enumerate}
In each stage we will show that our integration satisfies the $3$ most important property of an integral:
\begin{properties}
\item
    Linearity
\item
    Additivity
\item
    Monotonicity
\end{properties}

We let $\measureS{X}{\famM}{\mu}$ be a measure space.
\begin{Definition}
    If $s$ is a simple function $s = \finiteSum{a_k\charFunc{E_k}}{k}{N}$,
    we define the ($\mu$-) integeral of $s$ by,
    \begin{equation}\label{eq:int_simple_func}
	\lInt{s}{\mu}{X} = \finiteSum{a_k\measure{E_k}}{k}{N}.
    \end{equation}
    For any $A \in \famM$, integral over $A$ is given by,
    \begin{equation}
	\lInt{s}{\mu}{A} := \lInt{s\charFunc{A}}{\mu}{X} = \finiteSum{a_k\measure{E_k\cap A}}{k}{N}
    \end{equation}
\end{Definition}
We have seen that a simple functions don't have unique representation and thus we need to check 
that the above definition is well defined.
\begin{Proposition}
    Integration of simple function as described by~\ref{eq:int_simple_func} is well defined.
\end{Proposition}
\begin{proof}
    Suppose $s = \finiteSum{a_k\measure{E_k}}{k}{N}$ is a simple function where $a_k$'s are not distinct. Note
    that $E_k$'s are disjoint and their union is $X$. Let $i = 1$ and $j_i$ be the smallest index of the set 
    $J_1 = \set{k}{a_k = a_1}$ and let $E_{j_1} = \bigcup\limits_{k\in J_1}E_k$ and $a_{j_1} = a_1$. If $a_2$
    is in $J_1$ we leave it alone and proceed until we find an $i \leq N$ such that $a_i \not \in J_1$. Thus,
    the sets $E_{j_i}$ are disjoint and their union is $X$. Let $N_j$ be the number of such $j_i$.  
    Then,
    $s = \series{a_{j_i}\charFunc{E_{j_i}}}{j_i}{j_1}{N_j}$ and hence,
    \[\lInt{s}{\mu}{X} = \series{a_{j_i}\measure{E_{j_i}}}{j_i}{j_1}{N_j} = 
	\finiteSum{a_{k}\measure{E_k}}{k}{N}, \]
    where the last equality is due to the additivity property of measure.
\end{proof}
Thus we don't have to worry about the representation of the simple function.
\begin{Proposition}[name=Properites of integral of simple functions]\label{prop:prop_int_simp_func}
    Let $A \in \famM$. 
    The integral of simple functions defined in~\ref{eq:int_simple_func} satisfies the following properties:
    \begin{properties}
    \item
	(Linearity) If $s_1,s_2$ are simple functions and $a,b \in \R$ then
	\[\lInt{(as_1 + bs_2)}{\mu}{A} = a\lInt{s_1}{\mu}{A} + b\lInt{s_2}{\mu}{A}.\]
    \item
	(Additivity) Let $s_1$ be a simple function. If $A,B$ are disjoint subsets of $\famM$ then,
	\[\lInt{s_1}{\mu}{A\cup B} = \lInt{s_1}{\mu}{A} + \lInt{s_1}{\mu}{B}.\]
    \item
	(Monotonicity) Let $s_1,s_2$ be two simple functions such that $s_1\leq s_2$. Then,
	\[\lInt{s_1}{\mu}{A} \leq \lInt{s_2}{\mu}{A}.\]
    \item
	(Triangle Inequality) Let $s_1$ be a simple function, then,
	\[\abs{\lInt{s_1}{\mu}{A}} \leq \lInt{\abs{s_1}}{\mu}{A}.\]
    \end{properties}
\end{Proposition}
\begin{proof}
    We prove in order. 
    \begin{itemize}
	\item
	    Assume $a,b = 1$. Let $s_1 = \finiteSum{a_i\charFunc{E_i}}{i}{N_1}$ and 
	    $s_2 = \finiteSum{b_j\charFunc{F_j}}{j}{N_2}$
	    We know that by~\ref{prop:prop_simple_func},
	    \[s_1 + s_2 = \series{(a_i + b_j)\charFunc{E_i\cap F_j}}{i,j}{1}{}.\]
	    Hence,
	    \begin{align*}
		\lInt{(s_1 + s_2)}{\mu}{A} &= 
		\series{(a_i + b_j)\measure{{E_i\cap F_j\cap A}}}{i,j}{1}{}\\  
		& = \series{a_i\measure{{E_i\cap F_j\cap A}}}{i,j}{1}{} + 
		\series{b_j\measure{{E_i\cap F_j\cap A}}}{i,j}{1}{} \\
		& = \sum\limits_{i}a_{i}\sum\limits_{j}\meas(E_i\cap F_j\cap A)  +
		\sum\limits_{j}b_{j}\sum\limits_{i}\meas(E_i\cap F_j\cap A)\\
		& = \sum\limits_{i}a_{i}\measure{E_i\cap A} + 
		\sum\limits_{j}b_{j}\measure{F_j\cap A}\\
		& = \lInt{s_1}{\mu}{A} + \lInt{s_2}{\mu}{A}.
	    \end{align*}
	    The equality in the $4^{th}$ line is due to the fact that $E_i$'s and $F_j$'s \emph{partition}
	    $X$ and hence $\lbrace E_i\cap A \rbrace$ and $\lbrace F_j\cap A \rbrace$ are partitions of $A$,
	    so we use the additivity of measure. It is easy to observe for any $a,b$ different than $1$.
	\item
	    Noting that measure is additive for disjoint sets we can observe,
	    \begin{align*}
		\lInt{s_1}{\mu}{A\cup B} &= \finiteSum{a_i\measure{E_i\cap (A\cup B)}}{i}{N}\\
		&= \finiteSum{a_i\measure{(E_i\cap A)\cup (E_i\cup B)}}{i}{N}\\
		&= \finiteSum{(a_i\measure{E_i\cap A}) + (a_i\measure{E_i\cup B})}{i}{N}\\
		&= \finiteSum{(a_i\measure{E_i\cap A})}{i}{N} + \finiteSum{(a_i\measure{E_i\cup B})}{i}{N}\\
		&= \lInt{s_1}{\mu}{A} + \lInt{s_1}{\mu}{B}
	    \end{align*}
	\item
	    If $s \geq 0$ is a simple function, then all its coefficient are greater than equal to $0$ and so
	    $\lInt{s}{\mu}{X} \geq 0$. Thus if $s = s_2 - s_2$, then $s \geq 0$ and we get the result.
	\item
	    Let $s_1 = \finiteSum{a_i\charFunc{E_i}}{i}{N_1}$. We get the result by observing that,
	    \[\abs{\finiteSum{a_i\measure{E_i}}{i}{N}} \leq \finiteSum{\abs{a_i}\measure{E_i}}{i}{N}.\]
    \end{itemize}
\end{proof}
Monotonicity shows that the Integral is a \textbf{positive operator} on the space of simple functions i.e.~it
maps positive simple functions to non-negative real numbers. 
\begin{Remark}\label{rmk:simple_func_meas_0}
    Note that if $s_1,s_2$ are simple functions such that $s_1 = s_2$ a.e.~then $\lInt{s_1}{\mu}{X} =
    \lInt{s_2}{\mu}{X}$ since the difference is $0$ except on a set of measure zero, 
    the integral of the difference is
    $0$ and hence we can use linearity to reach to the conclusion.

    If $A$ is any set of measure $0$, then $\lInt{s}{\mu}{A}$ must be $0$ provided we have a complete measure
    space. To see this, note that $\lInt{s}{\mu}{A} = \finiteSum{a_k\measure{E_k\cap A}}{k}{N}$. Since $A$ is
    a measure $0$, and $E_k \cap A \subset A$, we can see that if we have a complete measure space
    $\measure{E_k\cap A} = 0$ and hence $\lInt{s}{\mu}{A} = 0$. Since any measure space can be completed we
    are not too concerned about completeness.
\end{Remark}
We now embark on stage $2$; defining integral of non-negative measurable functions.
\begin{Definition}
    Let $\map{f}{X}{\extRealsPos}$ be a non-negative measurable function. Then, we define,
    \begin{equation}\label{eq:int_non_neg_meas_func}
	\lInt{f}{\mu}{X} = \sup\set{\lInt{s}{\mu}{X}}{0\leq s \leq f,\,\text{$s$ is simple}}.
    \end{equation}
    Note that the integral can be infinite. 
    We say that $f$ is ($\mu$-) integrable if $\lInt{f}{\mu}{X} < \infty$.
\end{Definition}
Note that because of~\ref{thm:approx_by_simple_func} (1), the above definition makes sense. We can always
approximate any measurable function by a sequence of simple functions from below and hence $f$ will be given
by the limit of such a sequence (which will be the sup).
For any $A \subset \famM$, we define\break{} 
$\lInt{f}{\mu}{A} := \lInt{f\charFunc{A}}{\mu}{X}$. 
\begin{Proposition}[name=Properties of the integral of non-negative measurable function]
    Let $A \in \famM$. Let $f,g$ be a non-negative extended real valued function defined on $X$ 
    such that its integral is given
    by~\ref{eq:int_non_neg_meas_func}. Then, it satifies the following properties:
    \begin{properties}
    \item
	(Linearity) For any $a,b \in \R$, 
	\[\lInt{(af + bg)}{\mu}{A} = a\lInt{f}{\mu}{A} + b\lInt{g}{\mu}{A}.\]
    \item
	(Additivity) If $A,B$ are disjoint subsets of $\famM$ then,
	\[\lInt{f}{\mu}{A\cup B} = \lInt{f}{\mu}{A} + \lInt{f}{\mu}{B}.\]
    \item
	(Monotonicity) Let $f\leq g$. Then,
	\[\lInt{f}{\mu}{A} \leq \lInt{g}{\mu}{A}.\]
    \end{properties}
\end{Proposition}
At this point only monotonicity is immediately observable from the definition.
\begin{proof}
    \textbf{(Monotonicity)}
    First let us consider the case when $A = X$. If $f = g$ the result is obvious. Assume $f < g$. 
    Approximate $f,g$ from below by sequence of non-negative simple functions. Then there will be a simple
    function $s$ such that $f < s \leq g$. Thus the collection of simple functions approximating $g$ contains
    the collection of simple functions approximating $f$. Hence taking the supremum of the integrals we get,
    \[\lInt{f}{\mu}{X} \leq \lInt{g}{\mu}{X}.\]
    For any $A \in \famM$, $f\leq g$ implies $f\charFunc{A}\leq g\charFunc{A}$ and hence we get
    the result by using the result on $A = X$ above.
\end{proof}
\begin{Remark}
    Linearity and additivity are not so obvious. We will need some additional results to be able to prove it. One
    such result is very useful and is called the Monotone Convergence theorem (MCT). Note that one direction
    of linearity is obvious. If we have simple functions $s_f,s_g$ approximating $f,g$, then the sequence of 
    $s_f + s_g$ will approximate $f+g$. Since $f,g$ are non-negative, the collection of $s_f + s_g$ will
    contain $s_f$ and $s_g$ and so by monotonicity $\lInt{(f+g)}{\mu}{X} \geq \lInt{f}{\mu}{X} +
    \lInt{g}{\mu}{X}$. The other inequality isn't apparent. Note that for any $a \in \R$, $\lInt{af}{\mu}{X}$
    is equal to $a\lInt{f}{\mu}{X}$. This is obvious from the definition.
\end{Remark}
\begin{Theorem}[name=Monotone convergence theorem (MCT)]\label{thm:monotone_conv_thm}
    If $\seq{f}{n}$ is a sequence of monotone increasing non-negative extended real valued measurable 
    functions,
    \[0\leq f_1 \leq f_2 \ldots f_n \leq f_{n+1} \leq \ldots f, \]
    and 
    \[\limit{f_n}{n}{\infty} = f, \]
    then,
    \[\limit{\lInt{f_n}{\mu}{X}}{n}{\infty} = \lInt{f}{\mu}{X}.\]
\end{Theorem}
\begin{proof}
    Let $I_n = \lInt{f_n}{\mu}{X}$ and $I = \lInt{f}{\mu}{X}$. We will show that,
    \[\limit{I_n}{n}{\infty} \leq I,\]
    and
    \[\limit{I_n}{n}{\infty} \geq I.\]
    By monotonicity, for any $n$ we have,
    \[I_n \leq I_{n+1} \leq I.\]
    Thus the sequence $\seq{I}{n}$ is monotone increasing and is bounded by $I$. Hence,
    $\limit{I_n}{n}{\infty}$ exists and is equal to $\sup\set{I_n}{n\in\Zplus}$ which is less than or equal to
    $I$. Thus,
    \[\limit{I_n}{n}{\infty}\leq I.\]
    To prove the other inequality let $s$ be a simple function such that $0\leq s \leq f$. This is possible
    due to~\ref{thm:approx_by_simple_func}. Consider the set,
    \[A_n = \set{x\in X}{f_n(x) \geq ts(x)},\]
    where $t \in (0,1)$ is a real number.
    Note that $\seq{A}{n}$ is an \emph{increasing} sequence of sets that increase to $X$
    i.e.~\[\atobUp{A_n}{X}.\]
    To see why this is true, all we need to check that if $x \in X$ is an arbitrary element of $X$, then it
    must be in atleast one of the sets $A_n$. If $f(x) = 0$ then each $f_n = 0 $ and so $x$ is in all $A_n$.
    If $f(x) > 0$, then there must be $n$ far out such that $x \in A_n$. This is because
    \[0 < ts(x) < s(x) \leq f(x),\]
    hence there must be a $f_n$ such that 
    \[0 < ts(x) < f_n(x) \leq f(x), \]
    because $\atob{f_n}{f}$ for any $x \in X$. This is the reason for taking a $t \in (0,1)$.
    Now because $f \geq f\charFunc{A_n}$, by monotonicity, 
    \[\lInt{f_n}{\mu}{X} \geq \lInt{f_n}{\mu}{A_n} \geq \lInt{ts}{\mu}{A_n} = t\lInt{s}{\mu}{A_n}.\]
    Let us write $s$ in standard representation,
    \[s = \finiteSum{a_k\charFunc{E_k}}{k}{N},\]
    for $E_k \in \famM$.
    Then,
    \begin{align*}
	\limit{\lInt{s}{\mu}{A_n}}{n}{\infty} &= \limit{\finiteSum{a_k\measure{E_k\cap
		    A_n}}{k}{N}}{n}{\infty}\\
	& = \finiteSum{a_k\limit{\measure{E_k\cap A_n}}{n}{\infty}}{k}{N}\\
	& = \finiteSum{a_k\measure{E_k\cap X}}{k}{N}\\
	& = \lInt{s}{\mu}{X}
    \end{align*}
    We used the fact that $\atob{\measure{A_n}}{\measure{X}}$ because $\atobUp{A_n}{X}$.
    Hence,
    \[\limit{\lInt{f_n}{\mu}{X}}{n}{\infty} \geq t\lInt{s}{\mu}{X}.\]
    Since $t\in (0,1)$ was arbitrary we must have,
    \[\limit{\lInt{f_n}{\mu}{X}}{n}{\infty} \geq \lInt{s}{\mu}{X}.\]
    This means that $\limit{I_n}{n}{\infty}$ is an upper bound for the set,
    \[\set{\lInt{s}{\mu}{X}}{0\leq s \leq f,\,\text{$s$ is simple}},\]
    and hence is greater than equal to the supremum of the set which is by definition equal to 
    $\lInt{f}{\mu}{X} = I$. Hence,
    \[\limit{I_n}{n}{\infty}\geq I.\]

\end{proof}
The monotone convergence theorem (MCT) is a very useful theorem. Immediately we can see that in the case where
we have a monotone increasing non-negative real valued function, we can \emph{bring} the limit \textbf{inside}
the integral. We will explore some important convergence theorem later. Now, we will use MCT to show 
linearity.
\begin{proof}
    \textbf{(Linearity)}. Let $\seq{s}{n}$, $\seq{t}{n}$ be non-negative simple functions such that
    $\atobUp{s_n}{f}$ and $\atobUp{t_n}{g}$. Hence $\atobUp{as_n + bt_n}{af + bg}$. Thus,
    \begin{align*}
	\lInt{(af+bg)}{\mu}{X} &= \limit{\lInt{(as_n + bt_n)}{\mu}{X}}{n}{\infty}\\
	&= \limit{\left(\lInt{as_n}{\mu}{X} + \lInt{bt_n}{\mu}{X}\right)}{n}{\infty}\\
	&= \limit{\lInt{as_n}{\mu}{X}}{n}{\infty} + \limit{\lInt{bt_n}{\mu}{X}}{n}{\infty} \\
	&= \lInt{af}{\mu}{X} + \lInt{bg}{\mu}{X},
    \end{align*}
    where the first and last equality is from~\ref{thm:monotone_conv_thm} (MCT), and the second inequality is
    due to linearity of the integral w.r.t simple functions. If $A$ is any set in $\famM$, then
    $\atobUp{s_n\charFunc{A}}{f\charFunc{A}}$ and $\atobUp{t_n\charFunc{A}}{g\charFunc{A}}$. 
    Hence $\atobUp{(as_n + bt_n)\charFunc{A}}{(af + bg)\charFunc{A}}$. Thus we can use the above result.
\end{proof}
Although additivity can now follow from what we have proved, we will prove additivity by showing 
that a non-negative measurable function induces a measure. First, a useful
lemma
\begin{Lemma}\label{lemma:int_induces_meas}
    Assume $f_1,f_2,\ldots$ are non-negative measurable functions defined on $X$. Then,
    \[\lInt{\infiniteSum{f_k}{k}}{\mu}{X} = \infiniteSum{\lInt{f_k}{\mu}{X}}{k}.\]
\end{Lemma}
\begin{proof}
    Let $F_n = \finiteSum{f_k}{k}{n}$ and let $F = \infiniteSum{f_k}{k}$. Then, since $f_k$'s are non-negative
    we have
    \[0 \leq F_1 \leq F_2 \leq \ldots F.\]
    Each $F_n$ is measurable and $\atobUp{F_n}{F}$ and hence $F$ is measurable. Thus we can use MCT to see
    that
    \begin{align*}
	\lInt{F}{\mu}{X} &= \limit{\lInt{F_n}{\mu}{X}}{n}{\infty}\\
	&= \limit{\finiteSum{\lInt{f_k}{\mu}{X}}{k}{n}}{n}{\infty}\\
	&= \infiniteSum{\lInt{f_k}{\mu}{X}}{k},
    \end{align*}
    where the second equality is due to linearity of the integral.
\end{proof}
\begin{Lemma}\label{lemma:int_set_meas_0}
    If $f$ is a non-negative extended real valued measurable function and $E\in \famM$ be a set such that
    $\measure{E} = 0$, then
    \[\lInt{f}{\mu}{E}= 0.\]
\end{Lemma}
\begin{proof}
    Note that $\lInt{f}{\mu}{E} = \lInt{f\charFunc{E}}{\mu}{X}$. If $\seq{s}{n}$ is a sequence of non-negative
    simple functions such that $\atobUp{s_n}{f}$, then $\atobUp{s_n\charFunc{E}}{f\charFunc{E}}$. But we have
    seen that (see~\ref{rmk:simple_func_meas_0}) $\lInt{s_n\charFunc{E}}{\mu}{X} = \lInt{s_n}{\mu}{E} = 0$ 
    and hence we get the result. 
\end{proof}
Next we come to an important statement.
\begin{Theorem}\label{thm:int_induces_meas}
    Let $f$ be a non-negative extended real valued function. Then,
    \[\mu_f(E) := \lInt{f}{\mu}{E},\]
    for each $E \in \famM$ defines a measure on $\metricS{X}{\famM}$.
\end{Theorem}
\begin{proof}
    From the Lemma above it is evident that $\mu_f(\emptyset) = 0$. Now let $E = \countUnion{E_k}{k}$ where
    the $E_k$'s are pairwise disjoint sets in $\famM$. Note that $\charFunc{E} =
    \infiniteSum{\charFunc{E_k}}{k}$ since the $E_k$'s are disjoint.
    Then,
    \begin{align*}
	\mu_f(E) &= \lInt{f}{\mu}{E}\\
	&= \lInt{f\charFunc{E}}{\mu}{X}\\
	&=\lInt{\infiniteSum{f\charFunc{E_k}}{k}}{\mu}{X}\\
	&=\infiniteSum{\lInt{f\charFunc{E_k}}{\mu}{X}}{k}\\
	&=\infiniteSum{\lInt{f}{\mu}{E_k}}{k}\\
	&=\infiniteSum{\mu_f(E_k)}{k},\\
    \end{align*}
    where the $4^{th}$ inequality is from the Lemma following the monotone convergence theorem. Hence $\mu_f$
    is a measure on $\metricS{X}{\famM}$.
\end{proof}
\begin{Corollary}[name=Additivity]
    Additivity of the integral.
\end{Corollary}
\begin{proof}
    \textbf{(Additivity)} If $A,B$ are disjoint sets in $\famM$, then $\mu_f(A\cup B) = \mu_f(A)
    +\mu_f(B)$ i.e.~$\lInt{f}{\mu}{A\cup B} = \lInt{f}{\mu}{A} + \lInt{f}{\mu}{B}$.
\end{proof}
\begin{Corollary}
    If $A \subset B$, then $\lInt{f}{\mu}{A} \leq \lInt{f}{\mu}{B}$.
\end{Corollary}
\begin{proof}
    This follows from the monotononcity of $\mu_f$ that $\mu_f(A) \leq \mu_f(B)$. 
\end{proof}

Now we embark on the final stage ($3$) of our construction. If $f$ is an extended real valued measurable
function, we have seen that we can decompose $f$ in two non-negative parts,
\[f = \fPlus{f} - \fMinus{f}.\]
We have already seen how to define the integral for $\fPlus{f},\fMinus{f}$. However, our definition allows
these integrals to be infinite and hence we must be careful about things like $\infty - \infty$. Note that
$\abs{f} = \fPlus{f} +\fMinus{f}$. 
\begin{Definition}
    If $\map{f}{X}{\extReals}$ is a measurable function, then
    \begin{equation}\label{eq:int_meas_func}
	\lInt{f}{\mu}{X} = \lInt{\fPlus{f}}{\mu}{X} - \lInt{\fMinus{f}}{\mu}{X},
    \end{equation}
    provided atleast one of the integral on the right hand side is not infinite. We say that $f$ is integrable
    if both $\lInt{\fPlus{f}}{\mu}{X}$ and $\lInt{\fMinus{f}}{\mu}{X}$ are finite. 
    Note that, this is equivalent to saying $\lInt{\abs{f}}{\mu}{X} < \infty$.
\end{Definition}
If $A \in \famM$ and $\lInt{f}{\mu}{X}$ is defined then,
\[\lInt{f}{\mu}{A} := \lInt{f\charFunc{A}}{\mu}{X}.\]

\begin{Proposition}[name=Properties of the integral of non-negative measurable function]
    Let $A \in \famM$. Let $f,g$ be extended real valued functions defined on $X$ 
    such that their integral is given
    by~\ref{eq:int_meas_func}. Then, it satifies the following properties:
    \begin{properties}
    \item
	(Linearity) For any $a,b \in \R$, 
	\[\lInt{(af + bg)}{\mu}{A} = a\lInt{f}{\mu}{A} + b\lInt{g}{\mu}{A}.\]
    \item
	(Additivity) If $A,B$ are disjoint sets in $\famM$ then,
	\[\lInt{f}{\mu}{A\cup B} = \lInt{f}{\mu}{A} + \lInt{f}{\mu}{B}.\]
    \item
	(Monotonicity) Let $f\leq g$. Then,
	\[\lInt{f}{\mu}{A} \leq \lInt{g}{\mu}{A}.\]
    \item
	(Triangle Inequality)
	\[\abs{\lInt{f}{\mu}{A}} \leq \lInt{\abs{f}}{\mu}{A}\]
    \end{properties}
\end{Proposition}
\begin{proof}
    We prove in order. 
    \begin{itemize}
	\item
	    We prove for $A = X$. The result for any $A \in \famM$ can then be
	    proved by using $f\charFunc{A}$ and $g\charFunc{A}$ instead of $f,g$.
	    We will
	    prove the results for integrable function i.e.~we assume the integrals of both 
	    $\fPlus{f},\fMinus{f}$ and $\fPlus{g},\fMinus{g}$ are 
	    finite. The general proof can then be completed by taking all the different cases.

	    We will show that 
	    $\lInt{af}{\mu}{X} = a\lInt{f}{\mu}{X}$ and \\ $\lInt{(f+g)}{\mu}{X} =
	    \lInt{f}{\mu}{X} + \lInt{g}{\mu}{X}$.

	    If $a > 0$, then $\fPlus{(af)} = a\fPlus{f}$ and $\fMinus{(af)} = a\fMinus{f}$. 
	    $\lInt{af}{\mu}{X}$ is $\lInt{\fPlus{(af)}}{\mu}{X} - \lInt{\fMinus{(af)}}{\mu}{X}$ and thus,
	    \[\lInt{af}{\mu}{X} = a\lInt{\fPlus{f}}{\mu}{X} - a\lInt{\fMinus{f}}{\mu}{X} = a\lInt{f}{\mu}{X}.\]
	    If $a < 0$ then $\fPlus{(af)} = -a\fMinus{f}$ and $\fMinus{(af)} = -a\fPlus{f}$ and the result
	    follows analogously.

	    Let $h = f+g$. Then $\fPlus{h} - \fMinus{h} = \fPlus{f} - \fMinus{f} + \fPlus{g} - \fMinus{g}$.
	    Hence,
	    \[\fPlus{h} + \fMinus{f} + \fMinus{g} = \fMinus{h} + \fPlus{f} + \fPlus{g}. \]
	    Thus, using the linearity of non-negative functions we get,
	    \[\lInt{\fPlus{h}}{\mu}{X} + \lInt{\fMinus{f}}{\mu}{X} + \lInt{\fMinus{g}}{\mu}{X} = 
		\lInt{\fMinus{h}}{\mu}{X} + \lInt{\fPlus{f}}{\mu}{X} + \lInt{\fPlus{g}}{\mu}{X}.\]
	    Since we have assumed each of them are finite, we get
	    \[\lInt{\fPlus{h}}{\mu}{X} - \lInt{\fMinus{h}}{\mu}{X} =  
		\lInt{\fPlus{f}}{\mu}{X} - \lInt{\fMinus{f}}{\mu}{X} + 
		\lInt{\fPlus{g}}{\mu}{X}- \lInt{\fMinus{g}}{\mu}{X}.\]
	    Hence we get the result.
	\item
	    For any set $E \in \famM$, $\fPlus{(f\charFunc{E})} = \fPlus{f}\charFunc{E}$ and
	    $\fMinus{(f\charFunc{E})} = \fMinus{f}\charFunc{E}$. Thus using $E = A\cup B$,
	    \[\lInt{f}{\mu}{A\cup B} = \lInt{f\charFunc{A\cup B}}{\mu}{X} = 
		\lInt{\fPlus{f}\charFunc{A\cup B}}{\mu}{X} - \lInt{\fMinus{f}\charFunc{A\cup B}}{\mu}{X}.\]
	    Now from additvity of non-negative functions,
	    \[\lInt{\fPlus{f}\charFunc{A\cup B}}{\mu}{X} = \lInt{\fPlus{f}\charFunc{A}}{\mu}{X} +  
		\lInt{\fPlus{f}\charFunc{B}}{\mu}{X},\]
	    and
	    \[\lInt{\fMinus{f}\charFunc{A\cup B}}{\mu}{X} = \lInt{\fMinus{f}\charFunc{A}}{\mu}{X} +  
		\lInt{\fMinus{f}\charFunc{B}}{\mu}{X}.\]
	    Hence, we get the result.
	\item
	    Assume $f$ is integrable. Since $f \leq g$, $0 \leq g - f$ and thus using linearity of integrals of 
	    non-negative functions we get the result. To complete the proof we just need to check all the
	    cases on the decompositons of $f,g$.
	\item
	    This follows from the fact that $-\abs{f} \leq f \leq \abs{f}$ and monotonicity above.
    \end{itemize}
\end{proof}
An integral of a measurable function can tell us something about the function. More precisely if an integral
is finite it must be the case that the function is finite. As with any case in measure theory, we should make
a deduction about properties a.e.~rather than pointwise.
\begin{Proposition}\label{prop:integrable_func_finite_ae}
    If $\map{f}{X}{\extReals}$ is an integrable function then $f$ is finite a.e.~on $X$.
\end{Proposition}
\begin{proof}
    Let $E = \set{x\in X}{\abs{f(x)} = \infty}$. Take any $t > 0$ and consider the function $t\charFunc{E}$. By
    definition $\abs{f} > t\charFunc{E}$ and hence from monotonicity $\lInt{\abs{f}}{\mu}{X} \geq
    \lInt{t\charFunc{E}}{\mu}{X} = t\measure{E}$. If $\measure{E}$ is finite, this would mean the
    $\lInt{\abs{f}}{\mu}{X} > t$ for any $t$ which would contradict the hypothesis that $f$ is integrable.
    Hence $\measure{E} = 0$ which means that both $\fPlus{f},\fMinus{f}$ are finite a.e.~on $X$ and hence $f$
    is finite a.e.~on $X$
\end{proof}
\begin{Proposition}
    Suppose that $\map{f}{X}{\extReals}$ is a measurable function. Then,
    \[\lInt{\abs{f}}{\mu}{X} = 0 \,\iff\, f=0\quad\text{a.e.}\]
\end{Proposition}
\begin{proof}
    Note that if $f=0$ a.e.~then $\abs{f} = 0$ a.e.~and vice versa. Hence, we can just replace $f$ by
    $\abs{f}$ in the Proposition above without changing anything. This just amounts to assuming that $f$ is
    positive.

    If $f$ is a simple function $f = \finiteSum{a_k\charFunc{E_k}}{k}{N}$ such that $f = 0$ a.e.~then it 
    means that either all its coefficient $a_k$'s are
    zero or the measure of its component sets $E_k$'s is zero and hence $\lInt{f}{\mu}{X} = 0$. If $f$ is not
    a simple function, then for any simple function $0\leq s\leq f$ implies $s = 0$ a.e.~and hence
    $\lInt{f}{\mu}{X} = 0$ by definition of the integral of an non-negative measurable function. 

    Now let $\lInt{\abs{f}}{\mu}{X} = 0$. Let $E$ be the set where $f$ is not equal to $0$. Since $f$ is
    assumed positive, this means that $E$ is the set $\fCompA{f}{>}{0}$. Let us define the following set,
    \[E_n = \set{x\in X}{f(x) \geq \frac{1}{n}}.\]
    Fix any $\epsilon > 0$. There is an $n \in \Zplus$ such that $1/n < \epsilon$ and hence if $f(x) >
    \epsilon$ then $f(x) > 1/n$. This means that,
    \[E \subset \countUnion{E_n}{n},\]
   
    Note that $0\leq \frac{1}{n}\charFunc{E_n}\leq f$ for all $n$ by construction and hence,
    \[\frac{1}{n}\measure{E_n} = \lInt{\frac{1}{n}\charFunc{E_n}}{\mu}{X} \leq \lInt{f}{\mu}{X} = 0.\]
    Thus,
    \[\measure{E} \leq \infiniteSum{\measure{E_n}}{n} \leq 0.\]
    Thus $E$ has $0$ measure.
\end{proof}
Intuitively, integrable functions should in some sense vanish at infinity since their integrals are finite. We
make precise this notion in the following theorem.
\begin{Theorem}[name=Absolute continuity]\label{thm:abs_cont_integral}
    Suppose $f$ is integrable extended real valued function on $X$. Then for any $\epsilon > 0$
    \begin{enumerate}
	\item
	    There is a set of finite measure $E\in\famM$ such that $\lInt{\abs{f}}{\mu}{\comp{E}} < \epsilon$.
	\item
	    There is a $\delta > 0$ such that for any $E \in \famM$,
	    \[\lInt{\abs{f}}{\mu}{E} < \epsilon,\]
	    whenever $\measure{E} < \delta$.
    \end{enumerate}
\end{Theorem}
\begin{proof}
    We prove in order. By replacing $f$ with $\abs{f}$, we don't change anything. Hence we can assume $f$ to
    be non-negative.
    \begin{enumerate}
	\item
	    Let $E_n = \set{x\in X}{0\leq f(x) \leq n}$. Then $E_1 \subset E_2 \subset \ldots$. Let 
	    $f_n = f\charFunc{E_n}$. Then $\atobUp{f_n}{f}$. Hence we can use MCT to state that
	    \[\limit{\lInt{f_n}{\mu}{X}}{n}{\infty} = \lInt{f}{\mu}{X}.\]
	    Thus there is an $N$ such that,
	    \[\abs{\lInt{f_n}{\mu}{X} - \lInt{f}{\mu}{X}} < \epsilon,\]
	    whenever $n \geq N$. Since $f \geq f_n$ for all $n$, we have
	    \[\lInt{f}{\mu}{X} - \lInt{f_N}{\mu}{X} < \epsilon,\]
	    which from linearity can be written as,
	    \[\lInt{(f - f\charFunc{E_N})}{\mu}{X} < \epsilon.\]
	    Since $1-\charFunc{E_N} = \charFunc{\indxComp{E}{N}}$ we get,
	    \[\epsilon > \lInt{f\charFunc{\indxComp{E}{N}}}{\mu}{X} =
		\lInt{f}{\mu}{\indxComp{E}{N}}.\]
	    $E_N$ is of finite measure by~\ref{prop:integrable_func_finite_ae}.
	    Hence, we get the result.
	\item
	    With the same definition for $f_n$ as above we can observe that,
	    \begin{align*}
		\lInt{f}{\mu}{E} &= \lInt{(f-f_n)}{\mu}{E} + \lInt{f_n}{\mu}{E} \\
		& \leq \lInt{(f-f_n)}{\mu}{X} + n\measure{E},
	    \end{align*}
	    for any $n\in\Zplus$ and $E \in \famM$.
	    The first term in the second line can be made smaller than $\frac{\epsilon}{2}$ by MCT for some
	    $N\in\Zplus$. Pick $\delta = \frac{\epsilon}{2N}$. Hence when $\measure{E} < \delta$,
	    \[\lInt{f}{\mu}{E} \leq \frac{\epsilon}{2} + \frac{\epsilon}{2} = \epsilon.\]
	    Thus, we get the result.
    \end{enumerate}
\end{proof}
\section{Limit theorems}
One of the basic questions in analysis is when can we interchange limit operations? Note that integration is a
limit operation (the supremum of simple functions is the limit of the simple functions that approximate a
given non-negative measurable function). If $\atob{f_n}{f}$ pointwise, when can one say that
\[\atob{\lInt{f_n}{\mu}{X}}{\lInt{f}{\mu}{X}}?\]
The monotone convergence theorem says that the above is true if $f_n$ increases to $f$ for non-negative
measurable functions. In a general case we may not have monontone sequences. The simplest case in which we can
guarantee the convergence of the integral is when $\atob{f_n}{f}$ \textbf{uniformly}. To see this, note that
\[\lInt{f_n}{\mu}{X} - \lInt{f}{\mu}{X} \leq \lInt{\abs{f_n-f}}{\mu}{X}\leq \measure{X}\epsilon,\]
for $n$ greater than some $N$. An important caveat is that here we have adopted the convenction that
\[\infty\cdot 0 = 0\cdot\infty = 0.\]
This certainly works for Riemann integral, however uniform convergence is too strong a demand. One of the
advantages of the construction of lebesgue integral is that we can (almost) acheive the convergence of
integrals when we have pointwise convergence of functions. A few cautionary examples follow,
\begin{Example}
    Let $\measureS{\R}{\famL}{\mu}$ be the space of lebesgue measure. 
    Let \\$\map{f_n}{\R}{\R}$ be given by,
    \begin{equation*}
	f_n(x) = 
	\begin{cases}
	    n &\,\text{if $0 < x < \frac{1}{n}$} \\
	    0 &\, \text{otherwise.}
	\end{cases}
    \end{equation*}
    Then $\atob{f_n}{0}$ pointwise but $\lInt{f_n}{\mu}{\R} = 1$. (Note that $f_n$ is a simple function and
    hence we were able to calculate it easily.)
\end{Example}

\begin{Example}
    Let $\measureS{\R}{\famL}{\mu}$ be the space of lebesgue measure. 
    Let \\$\map{f_n}{\R}{\R}$ be given by,
    \begin{equation*}
	f_n(x) = 
	\begin{cases}
	    n^2 &\,\text{if $0 < x < \frac{1}{n}$} \\
	    0 &\, \text{otherwise.}
	\end{cases}
    \end{equation*}
    Then $\atob{f_n}{0}$ pointwise but $\lInt{f_n}{\mu}{\R} = n$ and hence
    $\atob{\lInt{f_n}{\mu}{\R}}{\infty}$ (Note that $f_n$ is a simple function and
    hence we were able to calculate it easily.)
\end{Example}
Thus we need some extra assumptions to deduce convergence of integrals. The first important theorem
generalizes the monotonce convergence theorem for non-negative functions.
\begin{Theorem}[name=Fatou's Lemma]\label{thm:Fatous_lemma}
    Suppose that $\seq{f}{n}$ is a sequence of positive measurable functions $\map{f_n}{X}{\extRealsPos}$.
    Then
    \begin{equation}
	\lInt{\liminf\limits_{n} f_n}{\mu}{X} \leq \liminf\limits_{n}\lInt{f_n}{\mu}{X}.
    \end{equation}
\end{Theorem}
\begin{proof}
    Let $g_n = \linf{f_k}{k}{n}$. Then $\atobUp{g_n}{\liminf\limits_n f_n}$.
    Hence we can use MCT to deduce that,
    \[\limit{\lInt{g_n}{\mu}{X}}{n}{\infty} = \lInt{\liminf\limits_n f_n}{\mu}{X}.\]
    Now each $g_n \leq f_k$ for $k \geq n$ and hence,
    \[\lInt{g_n}{\mu}{X} \leq \lInt{f_k}{\mu}{X},\]
    for $k \geq n$. Hence,
    \[\lInt{g_n}{\mu}{X} \leq \inf\limits_{k\geq n}\lInt{f_k}{\mu}{X},\]
    and taking $n\to \infty$ we see that,
    \[\limit{\lInt{g_n}{\mu}{X}}{n}{\infty} \leq \liminf\limits_{n}\lInt{f_n}{\mu}{X}.\]
    Hence, we get the result. For any set $E \in\famM$, multiplying $f_n,f$ by $\charFunc{E}$ we can get the
    result for the integral over $E$.
\end{proof}
Now we come to the most important result in this chapter which gives us the conditions when to expect the
convergence of the integral of the functions given pointwise convergence of the functions.
\begin{Theorem}[name=Dominated Convergence theorem]\label{thm:dominated_conv_thm}
    Let $E\in\famM$ be any measurable set and let $\seq{f}{n}$ be a sequence of measurable functions such that
    for each $n$, $\abs{f_n} \leq g$ a.e.~on $E$ where $g$ is an integrable function over $E$. If
    $\atob{f_n}{f}$ a.e.~on $E$ then $f$ is integrable over $E$ and,
    \[\limit{\lInt{f_n}{\mu}{E}}{n}{\infty} = \lInt{f}{\mu}{E}.\]
\end{Theorem}
\begin{proof}
    Since $-g < f_n < g$ we have $g + f_n > g - g = 0$ and hence $g + f_n$ is a non-negative function such
    that $\atob{g+f_n}{f}$. Note that $\liminf\limits_n f_n = f$. Using
    Fatou's lemma on $g + f_n$ we get
    \begin{align*}
	\lInt{g}{\mu}{E} + \lInt{f}{\mu}{E} &= \lInt{(g+f)}{\mu}{E} \\
	&= \lInt{\liminf\limits_{n}(g+f_n)}{\mu}{E} \\
	&\leq \liminf\limits_{n}\lInt{(g+f_n)}{\mu}{E},\\
	&\quad = \lInt{g}{\mu}{E} + \liminf\limits_n\lInt{f_n}{\mu}{E}.
    \end{align*}
    Subtracting $\lInt{g}{\mu}{E}$ since $g$ is integrable we get,
    \[\lInt{f}{\mu}{E} \leq \liminf\limits_n\lInt{f_n}{\mu}{E}.\]
    If we show 
    \[\lInt{f}{\mu}{E} \geq \limsup\limits_n\lInt{f_n}{\mu}{E},\]
    we are done. Note that $(g-f_n) > 0$ and thus using the same argument we get,
    \[\lInt{g}{\mu}{E} - \lInt{f}{\mu}{E} \leq \lInt{g}{\mu}{E} + \liminf\limits_{n}(-\lInt{f_n}{\mu}{E}).\] 
    Since $\liminf\limits_{n}(-a_n) = -\limsup\limits_{n}a_n$ we get
    \[\lInt{f}{\mu}{E} \geq \limsup\limits_{n}\lInt{f_n}{\mu}{E}.\]
    Hence, we get the result.
\end{proof}
\section{Relation with Riemann Integral}
\section{Product spaces and Fubini's theorem}
Let $\measureS{X}{\famM}{\mu}$ and $\measureS{Y}{\famN}{\nu}$ be two measure spaces. We have already seen that
the product sigma algebra is defined by,
\[\famM\otimes\famN = \sigmaGen{\set{A\times B}{A\in\famM,B\in\famN}}.\]
\begin{Definition}
    Let $\famE = \set{A\times B}{A\in\famM,B\in\famN}$. Then $\famE$ is called the collection of measurable
    rectangles. By definition, $\famM\otimes\famN = \sigmaGen{\famE}$.
\end{Definition}
Our goal in this section would be introduce a measure $\Pi$ on the sigma algebra $\famM\otimes\famN$, 
such that,
\[\Pi(A\times B) = \mu(A)\nu(B).\]
Note that the sigma algebra contains sets $E$ which cannot be written as a product of sets $A\times B$ for
some $A \in \famM$ and $B \in \famN$.
Let us introduce some terminology and notation. 
\begin{Definition}[name=sections]
    Suppose that $X,Y$ are sets and that $E \subset X\times Y$.
    Then for each $x \in X$ and each $y \in Y$ the sections $E_x$ and $E^{y}$ are the subsets defined as,
    \[E_x = \set{y\in Y}{(x,y)\in E},\]
    \[E^{y} = \set{x\in X}{(x,y)\in E}.\]
    If $f$ is a function defined on $X\times Y$, the sections $f_x,f^{y}$ are the functions on $Y$ and $X$
    respectively given by, $f_x(y) = f(x,y)$ and $f^{y}(x) = f(x,y)$.
\end{Definition}
\begin{Lemma}
    Let $\metricS{X}{\famM}$ and $\metricS{Y}{\famN}$ be measure spaces.
    \begin{enumerate}
	\item
	    If $E \subset X\times Y$ belongs to the product sigma algebra $\famM\otimes\famN$, then, 
	    $E_x$ belongs to $\famN$ and $E^{y}$ belongs to $\famM$.
	\item
	    If $f$ is an extended real valued $\famM\otimes\famN$-measurable function defined on $X\times Y$,
	    then each section $f_x$ is $\famN$-measurable and each section $f^{y}$ is $\famN$-measurable.
    \end{enumerate}
\end{Lemma}
\begin{proof}
    We prove in order.
    \begin{enumerate}
	\item
	    Fix an $x \in X$.
	    Let \[\famF = \set{E\in\famM\otimes\famN}{E_x \in \famN}.\]
	    Note that $\famF \subset \famM\otimes\famN$. If we show that $\famF \supset \famE$ and $\famF$ is
	    a sigma algebra, we are done.
	    If $E \in \famE$, then $E = A \times B$ for some $A,B$ in $\famM$ and $\famN$ respectively. It is
	    easy to see that $E_x$ is either $B$ or $\emptyset$; and in both cases $E_x \in \famN$. Thus,
	    $\famE \subset \famF$. To show that $\famF$ is a sigma algebra, we need to show that it is closed
	    under complements and countable unions. Note that,
	    $\comp{(E_x)} = {(\comp{E})}_{x}$. Thus, if $E \in \famF$, then since $E \in \famM\otimes\famN$
	    which is a sigma-algebra, $\comp{E} \in \famM\otimes\famN$. Moreover, since $E_x \in \famN$ by
	    our hypothesis, $\comp{(E_x)}$ is in $\famN$ since $\famN$ is a sigma-algebra. Similarly for
	    countable unions. Since this is true for any $x \in X$, we get the result. The same arguments can
	    be used to show that $E^{y}$ belongs to $\famM$.
	\item
	    Note that,
	    \[{(\invIm{f}{D})} = \set{(x,y)}{f(x,y)\in D}.\]
	    We show that $\invIm{f_x}{D} = {(\invIm{f}{D})}_{x}$ i.e.~fix an $x \in X$ and
	    observe that,
	    \begin{equation*}
		\begin{aligned}
		    {(\invIm{f}{D})}_{x} &= \set{y}{f(x,y)\in D},\\
		    &= \set{y}{f_x(y) \in D},\\
		    &= \invIm{f_x}{D}.
		\end{aligned}
	    \end{equation*}
	    For any $D \in \borelS{\R}$, $\invIm{f}{D} \in \famM\otimes\famN$ and by the previous result,
	    ${(\invIm{f}{D})}_{x} \in \famN$. Thus, $\invIm{f_x}{D} \in \famN$ and so is $\famN$-measurable.
    \end{enumerate}
\end{proof}
\begin{Proposition}
    Let $\measureS{X}{\famM}{\mu}$, $\measureS{X}{\famN}{\nu}$ be $\sigma$-finite measure spaces. If $E$
    belongs to the product sigma-algebra $\famM\otimes\famN$, then the function that maps $x$ to $\nu(E_x)$
    is $\famM$-measurable and the function that maps $y$ to $\mu(E^{y})$ is $\famB$-measurable.
\end{Proposition}
\begin{proof}
    We will prove the Proposition in steps for the case of finite measure.
    \begin{enumerate}
	\item 
	    We show that the family of measurable rectangles $\famE$ is a $\pi$-class.
	    If $E_1 = A_1\times B_1$ and $E_2 = A_2 \times B_2$, then 
	    \[E_1\cap E_2 = (A_1\times B_1)\cap (A_2\times B_2) = (A_1\cap A_2)\times (B_1\cap B_2).\]
	    Thus, $E_1\cap E_2 \in \famE$.
	\item 
	    We show that a sub-collection $\famF$ of $\famM\otimes\famN$ satisfying the hypothesis in the
	    Proposition is a $\lambda$-class.
	    Let $\map{h}{X}{\R}$ be given by $h(x) = \nu(E_x)$. By the above Lemma, such a function is
	    well-defined. Similarly, let $\map{l}{Y}{\R}$ be given by $l(y) = \mu(E^{y})$.
	    Define,
	    \[\famF = 
	\set{E\in\famM\otimes\famN}{x\mapsto\nu(E_x),y\mapsto\mu(E_y)\hspace{0.05in}\text{are measurable}}.\]
	    Let $E = X \times Y$. Then, $E_x = Y$ and $E^{y} = X$. Consider $h(x) = \nu(E_x)$. 
	    Then, $\invIm{h}{\hInt{a}{\infty}}$ is
	    either $\emptyset$ or $X$; in both the cases $\invIm{h}{\hInt{a}{\infty}} \in \famM$. Thus, $h$ is
	    $\famM$-measurable for $E = X \times Y$ and by similar reasonings $l$ is $\famN$-measurable.
	    Let $E \in \famF$. Let $h(x) := \nu(\comp{(E_x)})$. Let $g(x) = \nu(Y)$ and $h'(x) = \nu(E_x)$.
	    Then, by our hypothesis $h'(x)$ is measurable. 
	    For any $x \in X$, $\nu(\comp{(E_x)}) = \nu(Y) - \nu(E_x)$ and hence $h$ is 
	    a sum of two measurable functions and is thus measurable. Thus, $\comp{E}\in \famF$. 
	    If $\seq{E}{n}$ is a sequence of pairwise disjoint sets in $\famF$, then for any $x$
	    \[\nu(\countUnion{{(E_n)}_x}{n}) = \limit{\finiteSum{\nu({(E_i)}_x)}{i}{n}}{n}{\infty}.\]
	    Thus, $\countUnion{E_n}{n} \in \famF$, because the limit of measurable functions is also
	    measurable, i.e.~, let $h'_n = \finiteSum{h_i}{i}{n}$, where $h_i(x) = \nu({(E_i)}_x)$. Then, each
	    $h_i$ is measurable by our hypothesis, and so $h'_n$ is measurable for each $n$ and hence $h$ is
	    measurable (being the limit of measurable functions.)
	    Hence, $\famF$ is a $\lambda$-class.
	\item
	    We appeal to $\pi-\lambda$ Theorem to conclude that $\famF = \famM\otimes\famN$.
	    By our construction, $\famF \subset \famM\otimes\famN$. Also we showed that $\famF \supset \famE$
	    and $\famF$ is a $\lambda$-class. Since, $\famE$ is a $\pi$-class, by the $\pi-\lambda$ Theorem,
	    $\famF \supset \sigmaGen{\famE} = \famM\otimes\famN$. Hence, $\famF = \famM\otimes\famN$.

	    We can extend this for the case when $\mu,\nu$ are sigma finite. In other words, we only need to
	    show Step $(2)$ when $\mu,\nu$ are sigma-finite. But this is easily seen by applying to the case
	    of $\nu(E_x\cap Y_n)$, where $\atobUp{Y_n}{Y}$ and $\nu(Y_n) < \infty$ and analogously for $\mu$.
    \end{enumerate}
\end{proof}

\begin{Theorem}
    Let $\measureS{X}{\famM}{\mu}$, $\measureS{Y}{\famF}{\nu}$ be $\sigma$-finite measure spaces. Then, there
    is a unique measure $\Pi$ on the sigma-algebra $\famM\otimes\famN$, such that,
    \[\Pi(A\times B) = \mu(A)\nu(B),\]
    for every $A \in \famM$ and $B \in \famN$. Furthermore, for any arbitrary set $E \in \famM\otimes\famN$,
    \[\Pi(E) = \lInt{\nu(E_x)}{\mu}{X} = \lInt{\mu(E^{y})}{\nu}{Y}.\]
    The measure $\Pi$ is the called the product measure and is sometimes denoted by $(\mu\times \nu)$.
\end{Theorem}
\begin{proof}
    For any $E \in \famM\otimes\famN$, define, 
    \begin{align*}
	\Pi_{1}(E) &:= \lInt{\nu(E_x)}{\mu}{X},\\ 
	\Pi_{2}(E) &:= \lInt{\mu(E_y)}{\nu}{Y}
    \end{align*}
    By the previous Theorem, $x\mapsto\nu(E_x)$ is a $\famM$-measurable function and so $\Pi_{1}$ is
    well-defined. Similarly $\Pi_2$ is well-defined.
    We show that $\Pi_{1},\Pi_{2}$ are measures.
    Clearly $\Pi_{1}(\emptyset) = 0$. Let $\seq{E}{n}$ be a sequence of measurable sets in
    $\famM\otimes\famN$. Then, ${(E_x)}_n$ is a sequence of measurable sets in $\famN$. Let $E =
    \countUnion{E_n}{n}$.
    \begin{align*}
	\Pi_{1}(E) &= \lInt{\nu(\countUnion{{(E_x)}_n}{n})}{\mu}{X}\\ 
	&= \lInt{\series{\nu({(E_x)}_n)}{n}{1}{\infty}}{\mu}{X}\\
	&= \series{\lInt{\nu({(E_x)}_n)}{\mu}{X}}{n}{1}{\infty}\\
	&= \series{\Pi_{1}(E_n)}{n}{1}{\infty}.
    \end{align*}
    Similarly, $\Pi_{2}$ is measure on $\famM\otimes\famN$.
    For any $E \in \famE$, where $\famE$ is the family of measurable rectangles, we can write $E = A \times B$
    for some $A \in \famM$ and $B \in \famN$. Then,
    \[\lInt{\nu(B)\charFunc{A}}{\mu}{X} = \nu(B)\mu(A) = \lInt{\mu(A)\charFunc{B}}{\nu}{Y}.\]
    Hence,
    $\Pi_1 = \Pi_2 = \Pi$ on $\famE$ which is a $\pi$-class. Thus by~\ref{thm:uniq_measures_pilam}, these two
    measure agree on $\sigmaGen{\famE} = \famM\otimes\famN$.
\end{proof}
WE can extend this to any finite collection of measurable spaces.
The following two theorems enable one to evaluate integrals with respect to product measures by evaluating
iterated integrals.
\begin{Proposition}[name=Tonelli's Theorem]\label{prop:tonelli_thm}
    Let $\measureS{X}{\famM}{\mu}$,$\measureS{Y}{\famN}{\nu}$ be $\sigma$-finite measure spaces, and let
    $\map{f}{X\times Y}{\interval{0}{\infty}}$ be $\famM\otimes\famN$-measurable. Then,
    \begin{enumerate}
	\item
	    the functions $x\mapsto\lInt{f_x}{\nu}{Y}$ and $y\mapsto\lInt{f^{y}}{\mu}{X}$ are $\famM,\famN$
	    measurable respectively and
	\item
	    $f$ satisfies
    \end{enumerate}
\end{Proposition}
