\chapter{The problem of measure: prescribing volume}
One of the most fundamental concepts in geometry is that of the generalized \emph{volume} or \textbf{measure}
of a solid body $E$ in one or more dimensions. In one dimension, this refers to \emph{length} of the body $E$,
and in two and three dimension we refer to the measure of $E$ as the \emph{area} and \emph{volume}
respectively. We can think of the measure as a \textbf{set function}, i.e.~we input a solid body and it
outputs its volume. A solid body can be abstracted as a set in $\Rn$. The Question is, can we prescribe a
measure to \emph{any} set in $\Rn$?  Classically, to get the volume of any body $E$, one would find a sequence
of approximations from within $E$ and from outside and thus get some lower and upper bounds on the volume. In
trying to generalize this to arbitrary sets in $\Rn$ we will follow the same procedure. Let us say the set
function we are after is denoted by $\mu$. In order for it to give us a reasonable volume we must expect $\mu$
to behave reasonably. By this, we mean
\begin{enumerate}
    \item
	If $E$ is an interval $\interval{a}{b}$ in $\R$ or a rectangle $\interval{a_1}{b_1} \times
	\interval{a_2}{b_2}$, then $\measure{E}$ must equal its length $b-a$ in $\R$ or area
	$(b_1-a_1)\times(b_2-a_2)$ in $\R^2$.
    \item
	(Additivity) If $E_1$, $E_2$ are disjoint then $\measure{E_1\cup E_2} = \measure{E_1}+\measure{E_2}$.
    \item
	(Translation invariance) If we translate $E$, we must not change its measure, 
	that is $\measure{E+h} = \measure{E}$ for any $h$.
\end{enumerate}
We will see that even such a basic requirement on our set function $\mu$ proves to be troublesome. In other
words, we will show that there are sets in $\R^n$ that fail to satisfy the above requirements. We do not want to
give up on any of these requirements. However, we will give up the requirement that all sets be measurable.

We start with measuring sets whose \emph{volume} we already know from elementary geometry.
\begin{Definition}[name= Intervals and rectangles]
    An interval $I$ is a subset of $\R$ of the form $\interval{a}{b} = \set{x\in\R}{a\leq x \leq b}$,
    $\interval[open right]{a}{b} = 
    \set{x\in\R}{a\leq x < b}$, $\hInt{a}{b} = \set{x\in\R}{a < x \leq b}$ or $(a,b)
    = \set{x\in\R}{a < x < b}$. We define the volume (length) of an interval $I$ by $\volumeMeasure{I} = b-a$. 
    We also denote this by $\abs{I}$.
    A rectangle $R \in \Rn $ is a Cartesian product of $n$ intervals, $R = \finiteProduct{I_i}{i}{n}$. We
    define the volume of a rectangle $R$ as $\finiteProduct{\volumeMeasure{I_i}}{i}{n}$. 
\end{Definition}
For example a closed rectangle given by,
\[R = \rectRn{a}{b},\]
has a volume given by,
\[\volumeMeasure{R} = \volumeMeasure{\rectRn{a}{b}} = \finiteProduct{(b_i-a_i)}{i}{n}.\]


\begin{figure}
    \includestandalone[width=0.40\textwidth]{tex/tikz_figures/cube}
    \caption{Rectangles in $\R^3,\R^2,\R$.}\label{fig:tikz:rectangle_in_rn}
\end{figure}

\begin{Definition}[name=Elementary set]
    An elementary set is any subset of $\Rn$ which is the union of a finite number of disjoint rectangles.
\end{Definition}
Note that any rectangle is an elementary set. It is easy to see that if $E,F$ are elementary sets in $\Rn$, then $E\cup F$, $E\cap F$, $\setDiff{E}{F}$ are
also elementary sets.
\begin{figure}
    \includestandalone[width=0.40\textwidth]{tex/tikz_figures/measure_elem_set}
    \caption{Measure of an elementary set. We show two decomposition of 
	the elementary set.}\label{fig:tikz:meas_elem_set}
\end{figure}
How do we define the volume of an elementary set? Since an elementary set is a collection of finite disjoint
rectangles, it makes sense that the volume
of an elementary set is just the sum of the volumes of its component rectangles. Thus,
\begin{Theorem}[name=Measure of an elementary set]\label{thm:meas_elem_set}
    Let $E\subset\Rn$ be an elementary set. We define the (volume) measure of $E$ as,
    \[\measure{E} = \finiteSum{\volumeMeasure{R_i}}{i}{m},\]
    where $E = \finiteDisjUnion{R_i}{i}{m}$. Moreover, this measure is well defined, i.e.~if $E =
    \finiteDisjUnion{S_j}{j}{k}$ its measure remains the same.
\end{Theorem}

\begin{proof}
    First we will show that for any interval $I$, we can compute its length by the following formula
    \[\volumeMeasure{I} = \limit{\frac{1}{N}\#(I\cap \frac{1}{N}\Z)}{N}{\infty},\]
    Where $\frac{1}{N}\Z = \set{\frac{n}{N}}{n\in\Z}$ and $\#$ refers to the cardinality of the finite set
    $I\cap \frac{1}{N}\Z$. It suffices to prove this for the unit interval $I = \interval{0}{1}$. For any $N$,
    $I\cap \frac{1}{N}\Z$ is the set $\lbrace \frac{0}{N},\frac{1}{N},\dots,\frac{N}{N}\rbrace$ and thus 
    $\#I\cap \frac{1}{N}\Z = N+1$. Hence $\limit{\frac{1}{N}\#(I\cap \frac{1}{N}\Z)}{N}{\infty}$ is
    equal to $\limit{\frac{N+1}{N}}{N}{\infty} = 1 = \volumeMeasure{I}$.
    See~\ref{fig:tikz:interval_discretization}. Thus for any rectangle $R_i \subset
    \Rn$, we have
    \[\volumeMeasure{R_i} = \limit{\frac{1}{N^n}\#(R_i\cap \frac{1}{N}\Zn)}{N}{\infty}.\] 
    Thus,
    \[\volumeMeasure{E} = \finiteSum{\volumeMeasure{R_i}}{i}{m} = 
	\limit{\frac{1}{N^n}\#(E\cap \frac{1}{N}\Zn)}{N}{\infty},\]
    which remains same no matter what is the decomposition of $E$. 
\end{proof}
\begin{figure}
    \includestandalone[width=0.40\textwidth]{tex/tikz_figures/interval_discretization}
    \caption{Discretization of the unit interval.}\label{fig:tikz:interval_discretization}
\end{figure}
\begin{figure}
    \includestandalone[width=0.40\textwidth]{tex/tikz_figures/interval_discretization2}
    \caption{Discretization of the unit square.}\label{fig:tikz:interval_discretization2}
\end{figure}
We cannot extend this formula as the measure for any set in $\Rn$ since the limit may fail to exist. However,
for elementary sets we can observe some important properties.
\begin{properties}
\item
    If $E,F$ are disjoint elementary sets then $\measure{E\cup F} = \measure{E} + \measure{F}$.
\item
    If $E = \emptyset$, then $\measure{E} = 0$.
\item
    If $E = R$, where $R$ is a rectangle in $\Rn$, then $\measure{E} = \volumeMeasure{R}$.
\item
    If $E,F$ are elementary sets such that $E \subset F$ then $\measure{E} \leq \measure{F}$.
\end{properties}
\begin{figure}
    \includestandalone[width=0.40\textwidth]{tex/tikz_figures/in_out_approx}
    \caption{Approximating from within and without by an elementary set.}\label{fig:tikz:in_out_approx}
\end{figure}
How do we extend this notion of measure to arbitrary sets in $\Rn$? 
We can follow the ideas first expounded by the Greeks i.e by
approximating an arbitrary set by elementary sets from \textbf{inside} and \textbf{outside}. In other words we
can define an approximate \textbf{outer} measure as follows:
\[\outMeas(E) = \inf\limits_{B\supset E, B\, \text{elementary}} \measure{B}.\]
That is we \emph{cover} $E$ by an elementary set $B$ and then take the infimum over all the measure of such
$B$. Since $B = \finiteUnion{R_i}{i}{n}$, this is equivalent to saying,
\[\outMeas(E) = \inf\set{\finiteSum{\volumeMeasure{R_i}}{i}{n}}{\finiteUnion{R_i}{i}{n}\supset E}.\]
Similary we approximate $E$ from inside by
\[\mu_{\ast}(E) = \sup\set{\finiteSum{\volumeMeasure{S_j}}{j}{n}}{\finiteUnion{R_j}{j}{n}\subset E}.\]
Whenever these two coincide we say that $\measure{E} = \mu_{\ast}(E) = \outMeas(E)$.
See~\ref{fig:tikz:in_out_approx}. This is the construction
by Jordan and appeals to the classical approximation ideas. However, there is a problem with such a
construction. In analysis, sets occur as a limit of a sequence of sets and as such taking $\inf,\sup$ over
finite unions of rectangles might prove to be trouble some. Hence, we tinker with this approximation and allow
for a countable union of rectangles. To get a motiviation for using countable rectangles we
show the decomposition theorem for an open set in $\R$.
\begin{Proposition}[name=Decomposition of an open interval]\label{prop:open_set_countable_interval}
    Any open subset $G\subset\R$ is a countable union of disjoint open intervals.
\end{Proposition}
\begin{proof}
    Let $G$ be an open set in $\R$. Then for any $x \in G$, there is an interval around $x$ that is entirely
    contained in $G$. We wish to find the maximal interval containing $x$. To this end, let
    \[a_x = \inf\set{a < x}{(a,x)\subset G},\]
    \[b_x = \sup\set{x < b}{(x,b)\subset G},\]
    with possibly infinite values for $a_x,b_x$. We take $I_x = (a_x,b_x)$. Clearly $x \in I_x$ and for any
    set $J$ such that $x\in J$, we must have $J \subset I_x$. Then,
    \[G = \bigcup\limits_{x\in G}I_x.\]
    Now suppose $I_x \cap I_y \neq \emptyset$. Thus there is an $x$ such that $x \in I_x$ and $x \in I_y$ and
    thus $x \in I_x \cup I_y$. Since $I_x$ is the maximal interval containing $x$ we must have $I_x \cup I_y
    \subset I_x$. Similarly $I_x \cup I_y \subset I_y$. This is only possible when $I_x = I_y$. Thus we have
    a disjoint union. To show that the union is countable, we note that every interval $I_x$ contains a
    rational number $r_x$. Since $I_x,I_y$ are disjoint, $r_x \neq r_y$. Thus we can index each $I_x$ with the
    corresponding rational number $r_x$. Note, there are an infinite number of rational numbers in each $I_x$.
    To pick one, we invoke the Axiom of Choice. Hence, we have the result.
\end{proof}
It would make sense to define the measure of an open set $G \subset \R$ to be $\measure{G} =
\infiniteSum{\volumeMeasure{I_{x_i}}}{i}$. Thus, we need a countable collection of rectangles instead of using
just a finite collection. Unfortunately this decomposition doesn't quite extend to higher dimension.
Fortunately, there is a similar decompostion for open sets in $\Rn$ that works for almost disjoint rectangles.
Let us first define almost disjoint.

\begin{Definition}[name=Almost disjoint rectangles] 
    We will denote the \emph{interior} of a rectangle $R$ by $\interior{R}$. Two rectangles
    $R_i,R_j$ are almost disjoint if $\interior{R_i}\cap\interior{R_j} = \emptyset$.
\end{Definition}
A cube $Q \subset \Rn$ is a \textbf{special} rectangle whose sides are all equal. It turns out that any open
set in $\Rn$ can be written as a countable collection of \emph{almost} disjoint cubes. To prove this, we will
define \emph{dyadic} cubes in $\Rn$.

Note that for any $x \in \R$ and for any $M \in \Zplus$, by the archimedes principle there is
an integer $k$ such that, $\frac{k}{M} \leq x < \frac{(k+1)}{M}$ i.e $k$ is the greatest integer of $Mx$. If we
choose $M = 2^N$ for any $N \in \Zplus$ we get a dyadic partition of $\R$ for every $N$. We can
extend this idea to $\Rn$.
\begin{Definition}[name=Dyadic cube]\label{def:dyadic_cube}
    A (closed) dyadic cube $\dyadicCube{k}{N} \in \Rn$ is given by 
    \[\dyadicCube{k}{N} = \set{\vect{x} \in \R}{\frac{k_i}{2^N} \leq x_i \leq 
	    \frac{k_i+1}{2^N} \quad 1\leq i \leq n},\] 
    where $\vect{k} \in \Zn$.
\end{Definition}
Let $\family{C}_n$ be the collection of all dyadic cubes. Note that $\family{C}_n \subset
\family{J}_{n}$, where $\family{J}_n$ is the family of all closed rectangles. For a fixed $N$, each
side of the cube is of length $\frac{1}{2^N}$. Hence the volume of the cube is 
${(\frac{1}{2^N})}^n$. Note that for any $\vect{x},\vect{y} \in \dyadicCube{k}{N}$ the distance
$\distRn{\vect{x}}{\vect{y}} \leq \frac{\sqrt{n}}{2^N}$. For clarity, we will denote $\family{C}_n^N$
to be the collection of all dyadic cubes of level $N$ in $\Rn$. Two dyadic cubes are almost disjoint
if they intersect only in the boundary.

\begin{Proposition}\label{prop:count_col_cubes}
    Every open set in $\Rn$ is a countable union of almost disjoint cubles.
\end{Proposition}
\begin{proof}
    Let $G$ be an open set. First we will show that there is a countable collection of almost
    disjoint dyadic cubes $\family{Q}=\set{Q_i \in \family{C}_n}{i\in\Zplus}$ such that,
    \[\countUnion{Q_i}{i} \subset G. \]
    We start with a collection of dyadic cubes of level $0$. We pick $Q_i \in
    \family{C}_n^0$ such that $Q_i \subset G$. Next, we identify $Q_k$ such that $Q_k \cap G \neq
    \emptyset$. Then we find $Q_j \subset Q_k$ such that $Q_j \in \family{C}^1_{n}$. If $Q_j \subset G$
    we add it to the collection $\family{Q}$ otherwise we repeat by finding $Q_j \in
    \family{C}^2_{n}$ and so on. Since by construction each $Q_i
    \subset G$ we get $\countUnion{Q_i}{i} \subset G$. This is illustrated
    in~\ref{fig:tikz:countable_closed_cubes}.

    Now consider an $\vect{x} \in G$. Since $G$ is open there is an $\epsilon > 0$ such that the
    open ball $\ball{\epsilon}{\vect{x}} \subset G$. Now, there is a $\dyadicCube{k}{N}$ 
    such that $\vect{x} \in \dyadicCube{k}{N}$, 
    however $\dyadicCube{k}{N}$ may or may not be completely in $G$. 
    But we can successively increase i.e., we find an $M > N$ such that $\dyadicCube{k}{M} \subset
    \dyadicCube{k}{N}$ and $\dyadicCube{k}{M} \subset G$. Thus it must be in $\family{Q}$ and hence,
    \[G \subset \countUnion{Q_i}{i}. \] 
    This is illustrated in~\ref{fig:tikz:countable_closed_cubes2}.
\end{proof}
\begin{figure}
    \includestandalone[width=0.40\textwidth]{tex/tikz_figures/countable_closed_cubes}
    \caption{Illustration of \emph{paving} of $G\subset \Rn$ by dyadic cubes. The cubes $Q_k$ 
	filled in red are such that $Q_k \cap G \neq \emptyset$. Thus we increase the level.
	}\label{fig:tikz:countable_closed_cubes}
\end{figure}
\begin{figure}
    \includestandalone[width=0.40\textwidth]{tex/tikz_figures/countable_closed_cubes2}
    \caption{Illustration of finding a dyadic cube containing $\vect{x}$ and inside
	$\ball{\epsilon}{\vect{x}}$.}\label{fig:tikz:countable_closed_cubes2}
\end{figure}


If we want to include countable collection of rectangles, we want to make sure that we are not changing the
elementary measure of a rectangle i.e if a rectangle is a countable union of almost disjoint rectangles then
its volume measure must equal the infinite sum of its component rectangles.  We collect 
all these facts in the following proposition. Note that since a cube is also a rectangle, the following
results also hold for a cube. To make the ideas clear we will show these facts for
$\R^2$. The proposition is valid for all $\Rd,d\geq 1$.
\begin{Proposition}\label{prop:rect_in_rn}
    Let $R$ be a two dimensional rectangle with sides parallel to the co-ordinate axes. Then,
    \begin{enumerate}
	\item
	    Suppose that $R = I_1 \times I_2$ where each $I_i \subset \interval{a}{b} \subset \R$. 
	    If each $I_i$ is an almost disjoint union of closed, bounded intervals i.e
	    \begin{align*}
		&\set{I_{i,j_i} \subset \interval{a}{b}}{1\leq j_i \leq N_i}, \quad \text{and}\\
		& I_i = \finiteUnion{I_{i,j_i}}{j_i}{N_i}.
	    \end{align*}
	    Define the rectangles,
	    \[S_{j_1,j_2} = I_{1,j_1}\times I_{2,j_2},\]
	    then,
	    \[\volumeMeasure{R} = \sum\limits_{j_1 = 1}^{N_1}\sum\limits_{j_2 =
		    1}^{N_2}\volumeMeasure{S_{j_1,j_2}}.\]
	\item
	   If a rectangle $R$ is an almost disjoint, finite union of rectangles, $\set{R_i}{1\leq i
	       \leq N}$, then 
	   \[\volumeMeasure{R} = \finiteSum{\volumeMeasure{R_i}}{i}{N}.\] Note that this is another proof
	   of~\ref{thm:meas_elem_set}. 
       \item
	   If a rectangle $R$ is a subset of another rectangle $S$, then
	   \[\volumeMeasure{R} \leq \volumeMeasure{S}.\]
       \item
	   If a rectangle $R$ is \emph{covered} by finite union of rectangles, $\set{R_i}{1\leq i
	       \leq N}$, then 
	   \[\volumeMeasure{R} \leq \finiteSum{\volumeMeasure{R_i}}{i}{N}.\] 
       \item 
	   If a rectangle $R$ is the countable union of almost disjoint rectangles,
	   $\set{R_i}{i\in\Zplus}$, then 
	   \[\volumeMeasure{R} = \infiniteSum{\volumeMeasure{R_i}}{i}.\] 
    \end{enumerate}
\end{Proposition}
\begin{figure}
    \includestandalone[width=0.35\textwidth]{tex/tikz_figures/rectangles_rn}
    \caption{Illustration of proof~\ref{prop:rect_in_rn} (1).}\label{fig:tikz:rectangles_rn}
\end{figure}
\begin{figure}
    \includestandalone[width=0.35\textwidth]{tex/tikz_figures/rectangles_rn2}
    \caption{Illustration of proof~\ref{prop:rect_in_rn} (2).}\label{fig:tikz:rectangles_rn2}
\end{figure}
\begin{figure}
    \includestandalone[width=0.35\textwidth]{tex/tikz_figures/rectangles_rn3}
    \caption{Illustration of proof~\ref{prop:rect_in_rn} (3).}\label{fig:tikz:rectangles_rn3}
\end{figure}
\begin{figure}
    \includestandalone[width=0.35\textwidth]{tex/tikz_figures/rectangles_rn5}
    \caption{Illustration of proof~\ref{prop:rect_in_rn} (5).}\label{fig:tikz:rectangles_rn5}
\end{figure}
\begin{proof}
    We prove in order,
    \begin{enumerate}
	\item
	    Let us denote the length of an interval $I$ by $\lvert I \rvert$, then since $I$ is an elementary
	    set we can see that, (for example using the formula in~\ref{thm:meas_elem_set})
	    \[\lvert I_i \rvert = \finiteSum{\lvert I_{i,j_i}\rvert}{j_i}{N_i}.\]
	    By definition,
	    \[\volumeMeasure{R} = \lvert I_1\rvert \lvert I_2 \rvert.\]
	    Thus,
	    \begin{align*}
		\volumeMeasure{R} &= \lvert I_1\rvert \lvert I_2 \rvert \\
		&= \finiteSum{\lvert I_{1,j_1}\rvert}{j_1}{N_1}\finiteSum{\lvert I_{2,j_2}\rvert}{j_2}{N_2}\\
		&= \sum\limits_{j_1 = 1}^{N_1}\sum\limits_{j_2 = 1}^{N_2}\lvert
		I_{1,j_1}\rvert\lvert I_{2,j_2}\rvert\\
		&=\sum\limits_{j_1 = 1}^{N_1}\sum\limits_{j_2 =
		    1}^{N_2}\volumeMeasure{S_{j_1,j_2}}
	    \end{align*}
	    See~\ref{fig:tikz:rectangles_rn}.
	\item
	    If $R$ is the almost disjoint union of $R_i$ for $1\leq i \leq N$, then we can extend the
	    sides of each $R_i$. Each $\volumeMeasure{R_i}$ can then be expressed as a sum in $(1)$.
	    Also sides of $R$ have been partitioned and its volume measure can be expressed as a 
	    sum in $(1)$. Easy to
	    see that $\volumeMeasure{R} = \finiteSum{\volumeMeasure{R_i}}{i}{N}$. 
	    See~\ref{fig:tikz:rectangles_rn2}.
	\item
	    Note that both $R,S$ are elementary sets and we have see that volume measure preserves
	    monotonicity. Here we will give a different proof. 
	    If $R\subset S$, then extend the sides of $R$ to intersect the sides of $S$. Then using the result
	    above $\volumeMeasure{S}$ is the sum of almost disjoint rectangles $\lbrace R_i\rbrace$, 
	    where one of $R_i$ is equal to $R$. See~\ref{fig:tikz:rectangles_rn3}.
	    Hence we get the result.
	\item
	    If $R \subset \finiteUnion{R_i}{i}{N}$, then we get,
	    $R = \finiteUnion{(R\cap R_i)}{i}{N}$. However, each $R\cap R_i$ may not be disjoint. Let us
	    denote by $S_1 = R\cap R_1$ and subsequent $S_j = \setDiff{(R\cap R_j)}{(S_1\cup\dots S_{j-1})}$.
	    Then $R$ is the almost disjoint finite union of rectangles $S_j$ and by the result in (2)
	    $\volumeMeasure{R} = \finiteSum{\volumeMeasure{S_j}}{j}{N}$. But each $S_j \subset R_j$ and hence
	    by (3) above
	    $\volumeMeasure{S_j} \leq \volumeMeasure{R_j}$. Thus we get the result.
	\item
	    Note that $R$ \emph{covers} any finite almost disjoint union $\finiteUnion{R_i}{i}{N}$ i.e,
	    $\finiteUnion{R_i}{i}{N} \subset R$. Thus from above (roles reversed),
	    \[\volumeMeasure{\finiteUnion{R_i}{i}{N}} \leq \volumeMeasure{R}.\]
	    But since we have an almost finite disjoint union, using $(2)$,
	    \[\finiteSum{\volumeMeasure{R_i}}{i}{N} \leq \volumeMeasure{R}.\]
	    Taking the limit, we get
	    \[\volumeMeasure{R} \geq \infiniteSum{\volumeMeasure{R_i}}{i}.\]
	    To prove the other inequality we have to use a compactness argument.

	    Let us extend each ${R_i}$ to get an open covering of $R$. That is we get an
	    $S_i$ such that,
	    \[ \volumeMeasure{S_i} \leq \volumeMeasure{R_i} + \frac{\epsilon}{2^i}.\]
	    and $\interior{S_i} \supset R_i$.
	    (Note this means that we cover $R_i$ by $S_i$ which itself can be covered by something
	    larger than $R_i$ since each $R_i$ is bounded.) See~\ref{fig:tikz:rectangles_rn5}.
	    Thus we have an open cover of $R$ given by, 
	    $\family{G} = \set{\interior{S_i}}{i\in\Zplus}$. Since we are in a compact space there
	    is a finite subcover $\family{G}_{N}$ such that,
	    $R \subset \finiteUnion{S_i}{i}{N}$. Hence,
	    $\volumeMeasure{R} \leq \finiteSum{\volumeMeasure{S_i}}{i}{N}$. 
	    Taking the limit to $\infty$ we get the result.
    \end{enumerate}
\end{proof}

Thus we have seen, that to measure open sets we need a countable covering of elementary sets. But not all sets
are open sets. Since we are describing an abstract set function, we must be willing to accept strange sets.
The most famous strange set is the \textbf{Cantor} set.

To construct a Cantor set we begin with the unit interval $\interval{0}{1}$ and let $C_1$ denote the set
obtained from deleting the middle third open interval from $\interval{0}{1}$, that is
\[ C_1 = \interval{0}{\frac{1}{3}} \cup \interval{\frac{2}{3}}{1}.\]
To get to the next level, we delelte the middle third open interval from each of the above intervals to get,
\[ C_2 = \interval{0}{\frac{1}{9}}\cup \interval{\frac{2}{9}}{\frac{1}{3}} \cup
    \interval{\frac{2}{3}}{\frac{7}{9}} \cup \interval{\frac{8}{9}}{1}.\]
This procedure yields a sequence of closed sets $\seq{C}{k}$, where each $C_k$ is a finite union of closed
interval of length $\frac{1}{3^k}$, and $\decSetSeq{C}$.
\begin{Definition}
    The Cantor set $C$ is a compact set in $\R$ defined by,
    \[C = \countIntersection{C_k}{k},\]
    where $C_k$ is the set obtained by removing $2^k - 1$ middle third open intervals as described above.
\end{Definition}
\begin{figure}
    \includestandalone[width=0.55\textwidth]{tex/tikz_figures/cantor_set}
    \caption{First three levels of Cantor set in $\R$.}\label{fig:tikz:cantor_set}
\end{figure}
\begin{Proposition}\label{prop:cantor_set}
    Let $C$ be the Cantor set in $\R$. Then,
    \begin{enumerate}
	\item
	    $C$ does not contain any open intervals.
	\item
	    Every point of $C$ is a limit point of $C$.
	\item
	    $C$ is uncountable.
    \end{enumerate}
\end{Proposition}
\begin{proof}
    We prove $(1),(2)$.

    $(1)$ Let us assume that there is an interval $(a,b)$ such that $(a,b) \subset C$. Then for any $n \in
    \Zplus$, $(a,b) \subset C_n$. But since $C_n$ contains $2^n$ disjoint closed intervals of length
    $1/3^n$, $(a,b)$ must be a subset of one of these. But if we take $n$ large enough such that
    $1/3^n < (b-a)$ we get a contradiction. 

    $(2)$ Fix an $\epsilon > 0$.
    Since $x \in C$, $x \in C_n$ for every $n \in \Zplus$. But $C_n$ consists of disjoint closed
    intervals of length $1/3^n$ and hence $x$ belongs to one of these closed interval. If $x$ is the
    left end of the interval pick $y$ to be the right end of that interval, if $x$ is the right
    end of the interval pick $y$ to be the left end of that interval and in any other case pick
    $y$ to be the left end of the interval. Thus for any $n$ we have found a $y \in C$ such 
    that $\lvert y - x\rvert < 1/3^n$. Pick $n$ large enough such that $1/3^n < \epsilon$, then
    $\lvert y - x\rvert < \epsilon$. Since $\epsilon$ was arbitrary and $y \neq x$, we have shown
    that $x$ is a limit point of $C$ and thus there exists a sequence in $C$ that converges to $x$.
\end{proof}

What is the measure of the Cantor set? Approximating $C$ from \textbf{inside} seems tricky since $C$ doesn't
contain any open intervals. In what follows, we will approximate from outside by allowing a countable
collection of rectangles. As a convenience, we take the rectangles to be cubes. Nothing is lost in such a
construction.

Uptill now, we have a notion of measure for elementary sets in $\Rd$. We now construct a set function that
approximates \emph{any} set from \textbf{outside}. Such a construction will yield the familiar measure when
the set is an elementary set. Let us give a precise definition,
\begin{Definition}[name=Outer (volume) measure]
    If $E$ is any subset of $\Rd$, the outer measure of $E$ is,
    \[\outMeasure{E} = \inf\set{\infiniteSum{\volumeMeasure{Q_i}}{i}}{E\subset\countUnion{Q_i}{i}}.\]
    Hence,
    $\map{\outMeas}{\powSet{\Rd}}{\extRealsPos}$.
\end{Definition}
\begin{Example}
    What is the outer measure of a point $\vect{x}\in\Rd$? A point $\vect{x}$ is covered by $Q =
    \finiteProduct{I_i}{i}{d}$,
    where $I_i = \interval{x_i}{x_i}$, and its volume is equal to $0$.
\end{Example}
\begin{Example}
    What is the outer measure of a cube $Q$? Since $Q \subset Q$, from the definition of outer measure
    $\outMeasure{Q} \leq \volumeMeasure{Q}$. Now fix an $\epsilon$, by the definition of infimum and outer
    measure, there is a
    countable collection of cubes $\lbrace Q_i \rbrace$, such that $Q \subset \countUnion{Q_i}{i}$ and
    \[\infiniteSum{\volumeMeasure{Q_i}}{i} < \outMeasure{Q} + \epsilon.\]
    By extending the sides of $Q_j$ we can find a $S_j$ such that $\interior{S_j}\supset Q_j$ and
    $\volumeMeasure{S_j} \leq \volumeMeasure{Q_j} + \frac{\epsilon}{2^j}$. See~\ref{fig:tikz:rectangles_rn5}.
    Hence $Q \subset
    \countUnion{\interior{S_j}}{j}$ and since $Q$ is compact (closed and bounded), there exists a finite
    subcover such that (after renumbering)
    \[Q \subset \finiteUnion{\interior{S_j}}{j}{N} \subset \finiteUnion{S_j}{j}{N}.\]
    From~\ref{prop:rect_in_rn} (3), we have
    \[\volumeMeasure{Q} \leq \finiteSum{\volumeMeasure{S_j}}{j}{N}.\]
    Thus,
    \[\volumeMeasure{Q} \leq \finiteSum{\volumeMeasure{Q_j}+\frac{\epsilon}{2^j}}{j}{N} \leq
	\infiniteSum{\volumeMeasure{Q_j}}{j}.\]
    Hence $\volumeMeasure{Q} \leq \outMeasure{Q}$. Therefore we get that $\volumeMeasure{Q} =
    \outMeasure{Q}$. In fact, we can show that for any rectangle $R$ its volume measure equals the outer
    measure.
\end{Example}
Now we are ready to list some of the properties of outer measure.
\begin{Proposition}[name=Properties of (volume) outer measure]\label{prop:out_meas_rn_prop}
    Let $\outMeas$ be the outer measure defined for any subset in $\Rd$. Then,
    \begin{properties}
    \item
	(Monotonicity) If $E_1 \subset E_2$, then 
	\[\outMeasure{E_1} \leq \outMeasure{E_2}.\] 
    \item
	(Countable sub-additivity) If $E = \countUnion{E_j}{j}$, then
	\[\outMeasure{E} \leq \infiniteSum{\outMeasure{E_j}}{j}.\]
    \item
	(Approximation by open sets) If $E \subset \Rd$, then
	\[\outMeasure{E} = \inf\set{\outMeasure{G}}{E\subset G,\,\text{G open}}.\]
    \item
	If $E = E_1\cup E_2$ and $d(E_1,E_2) > 0$, then
	\[\outMeasure{E} = \outMeasure{E_1} + \outMeasure{E_2}.\]
    \item
	If $E$ is the countable union of almost disjoint cubes, i.e~$E = \countUnion{Q_i}{i}$, then
	\[\outMeasure{E} = \infiniteSum{\volumeMeasure{Q_j}}{j}.\] 
    \end{properties}
	
\end{Proposition}
\begin{proof}
    We prove in order.
    \begin{enumerate}
	\item
	    The collection of cubes that cover $E_2$ also cover $E_1$. Thus if $\famY_1$ is the collection of
	    cubes covering $E_1$ and $\famY_2$ is the collection of cubes covering $E_2$, then $\famY_1
	    \supset \famY_2$. Hence taking infimum of the sums of the volumes in the respective families, 
	    we get the result.
	\item
	    Fix an $\epsilon$. Then for each $j$ there is a collection of cubes $\lbrace Q^{j}_i \rbrace$ such
	    that $E_j \subset \countUnion{Q^j_i}{i}$ and,
	    \[\infiniteSum{\volumeMeasure{Q^j_i}}{i} < \outMeasure{E_j} + \frac{\epsilon}{2^j}.\]
	    We can observe that,
	    \[E \subset \bigcup\limits_{i,j}Q^{j}_i,\]
	    and hence,
	    \begin{align*}
		\outMeasure{E} &\leq \sum\limits_{i}\sum\limits_{j}\volumeMeasure{Q^j_i} \\
		&\quad = \sum\limits_{j}(\sum\limits_{i}\volumeMeasure{Q^j_i}) \\
		&\quad < \sum\limits_{j}(\outMeasure{E_j} + \frac{\epsilon}{2^j})\\
		&\quad \leq \infiniteSum{\outMeasure{E_j}}{j} + \epsilon
	    \end{align*}
	    Since, $\epsilon$ was arbitrary we get the result.
	\item
	    If there is a $G$ such that $G \supset E$, then by mononoticity $\outMeasure{G} \geq \outMeasure{E}$ 
	    and thus $\outMeasure{E}$ is less than or equal to the infimum of the outer measure of all such open 
	    sets $G$ such that $G \supset E$. For the other direction, fix an $\epsilon > 0$. Hence, there is
	    a collection of cubes $\lbrace Q_{\epsilon,i}\rbrace$ such that $E \subset
	    \countUnion{Q_{\epsilon},i}{i}$ and,
	    \[\infiniteSum{\volumeMeasure{Q_{\epsilon,i}}}{i} < \outMeasure{E} + \epsilon.\]
	    We can extend each $Q_{\epsilon,i}$ to get $\tilde{Q}_{\epsilon,i}$ such that,
	    \[\volumeMeasure{\interior{\tilde{Q}}_{\epsilon,i}} < \volumeMeasure{Q_{\epsilon,i}} +
		\frac{\epsilon}{2^i}.\]
	    Let $G_{\epsilon} = \countUnion{\interior{\tilde{Q}}_{\epsilon,i}}{i}$. 
	    Then $G_{\epsilon}$ is open and from
	    monotonicity,
	    \[\outMeasure{G_{\epsilon}}\leq \infiniteSum{\outMeasure{\interior{\tilde{Q}}_{\epsilon,i}}}{i} = 
		\infiniteSum{\volumeMeasure{\interior{\tilde{Q}}_{\epsilon,i}}}{i}.\]
	    Thus,
	    \begin{align*}
		\outMeasure{G_{\epsilon}}&\leq
		\infiniteSum{\volumeMeasure{\interior{\tilde{Q}}_{\epsilon,i}}}{i} \\
		&\leq\infiniteSum{(\volumeMeasure{Q_{\epsilon,i}}+\frac{\epsilon}{2^i})}{i} \\
		& < \outMeasure{E} + 2\epsilon,
	    \end{align*}
	    since, $\epsilon$ was arbitrary we get the result.
	\item
	    If $E = E_1 \cup E_2$, then from monotonicity $\outMeasure{E} \leq \outMeasure{E_1} +
	    \outMeasure{E_2}$. Thus we need to show the other inequality.

	    Fix an $\epsilon > 0$. There is a collection of cubes $\famQ = \lbrace Q_i \rbrace$ such that 
	    $E \subset
	    \countUnion{Q_i}{i}$ and
	    \[\infiniteSum{\volumeMeasure{Q_i}}{i} < \outMeasure{E} + \epsilon.\]
	    Let us construct a countable collection that covers $E$. 
	    Consider the set $E_1$. If a cube $Q_i$ intersects $E_1$ only
	    and not $E_2$ we add $i$ to an index set $J_1$. If a cube $Q_i$ intersects both $E_1$ and $E_2$, 
	    then we
	    refine the cube until it intersects only $E_1$. This can be done since if a cube intersects both 
	    $E_1$ and
	    $E_2$ its diameter must be larger than $d(E_1,E_2)$. Thus refining the cube (by a higher dyadic 
	    level)
	    we can get it to intersect only one. Thus we have an index set $J_1$ such that $E_1 \subset
	    \bigcup\limits_{i\in J_1}{Q_i}$, where each $Q_i, i\in J_1$ is a subset of cubes in $\famQ$. 
	    Similarly
	    we can find a $J_2$ such that $E_2 \subset \bigcup\limits_{i\in J_2}{Q_i}$. Thus,
	    \begin{align*}
		\outMeasure{E_1} + \outMeasure{E_2} &\leq \sum\limits_{i\in J_1}\outMeasure{Q_i} +
		\sum\limits_{i\in J_2}\outMeasure{Q_i} \\
		& \quad = \sum\limits_{i\in J_1}\volumeMeasure{Q_i} +
		\sum\limits_{i\in J_2}\volumeMeasure{Q_i} \\
		&\quad \leq \sum\limits_{Q_i\in\famQ}{\volumeMeasure{Q_i}} \\
		&\quad < \outMeasure{E} + \epsilon.
	    \end{align*}
	\item
	    From countable sub-additivity, $\outMeasure{E} \leq \infiniteSum{\volumeMeasure{Q_i}}{i}$. For the
	    other direction, we can \textbf{reduce} the sides of each cube $Q_i$ to get a cube $\tilde{Q}_i$
	    such that $\tilde{Q}_i$ is strictly a subset of $Q_i$ and,
	    \[\volumeMeasure{Q_i} \leq \volumeMeasure{\tilde{Q}_i} + \frac{\epsilon}{2^i}.\]
	    For any an finite number $N$ then we have
	    $d(\tilde{Q}_i,\tilde{Q}_j) > 0$ where $i,j$ range over $N$. Hence, we can use the result above to
	    deduce that for any $N$,
	    \[\outMeasure{\finiteUnion{\tilde{Q}_i}{i}{N}} = \finiteSum{\volumeMeasure{\tilde{Q}_i}}{i}{N} \geq
		\finiteSum{(\volumeMeasure{Q_i}- \frac{\epsilon}{2^j})}{i}{N}.\]
	    Note that $\epsilon(\frac{1}{2} + \cdots + \frac{1}{2^N}) < \epsilon$ for any $N$ and
	    $\finiteUnion{\tilde{Q}_i}{i}{N} \subset E$. Thus,
	    \[\outMeasure{E} \geq \finiteSum{\volumeMeasure{Q_i}}{i}{N} - \epsilon. \]
	    Since this is true for any $N$, we can take the limit $\atob{N}{\infty}$ to get,
	    \[\infiniteSum{\volumeMeasure{Q_i}}{i} \leq \outMeasure{E} + \epsilon.\]
	    Thus we get the result.
    \end{enumerate}
\end{proof}
It is crucial to observe that we do not have additivity for arbitrary disjoint sets. That is if $E_1,E_2$ are
disjoint then we cannot conclude that $\outMeasure{E_1\cup E_2} = \outMeasure{E_1} + \outMeasure{E_2}$. For
example take $E_k = \emptyset$ for $k > N$ in~\ref{prop:out_meas_rn_prop} (2) above 
and note that $\outMeasure{\emptyset} = 0$. This
is only true when $E_1$ and $E_2$ are \textbf{metrically} disjoint.  Also from the last property, 
using~\ref{prop:count_col_cubes} the outer measure of an open set written as the union of a countable
collection of almost disjoint cubes is the infinite sum of the volume of those cubes.

Since additivity is a crucial property we need identify those sets that preserve additivity, more importantly
countable additivity. The notion of measurability isolates a collection of subsets in $\Rd$ for which the
exterior measure statisfies all our desired properties including countable additivity. Since open sets satisfy
countable additivity, we make the following criteria for measurablity.
\begin{Definition}[name= (Volume) Lebesgue measure in $\Rn$]
    A subset $E \subset \Rd$ is Lebesgue measurable or simply measurable, if for any $\epsilon > 0$ there is
    an open set $G$ such that $E\subset G$ and,
    \[\outMeasure{\setDiff{G}{E}} \leq \epsilon.\]
    When this is true we denote the measure of $E$ by $\measure{E} = \outMeasure{E}$.
\end{Definition}
We will show that the collection of measurable sets behave nicely w.r.t.~set operations like unions,
intersection, complements. In particular, we will show that a countable union of measurable sets are
measurable, countable intersection of measurable sets are measurable and the complement of measurable set is
measurable. Moreover, the measure of elementary set is the usual volume and we preserve (countable) additivity
of disjoint sets and translation invariance. However we will also show that the class of measurable sets is
strictly smaller than $\powSet{\Rd}$.

Before we list the properties of measurable sets, we will list a useful lemma.
\begin{Lemma}
    If $F \subset \Rd$ is closed and $K \subset \Rd$ is compact and these two sets are disjoint, then
    $d(F,K) > 0$.
\end{Lemma}
\begin{proof}
    Let $x \in K$. Then $x \not\in F$ and hence there is a $\delta_x > 0$ such that $\ball{\delta_x}{x}\cap
    F = \emptyset$. Hence for any $y \in F$, $d(x,y) > \delta_x$. By choosing a smaller $\delta_x$ we can make
    $d(x,F) \geq 2\delta_x$. Now the collection $\set{\ball{\delta_x}{x}}{x\in K}$ is an open cover for $K$
    and since $K$ is compact there is a finite subcover $\set{\ball{\delta_{x_i}}{x_i}}{1\leq i\leq N}$. Let
    $\delta = \min\set{\delta_{x_i}}{1\leq i \leq N}$. Then for any $x \in K$, there is a $x_j$ such that
    $d(x,x_j) < \delta_j$. Pick any $y\in F$, then $d(y,x) \geq d(y,x_j) - d(x,x_j) > \delta$ and hence
    $d(F,K) > 0$.	
\end{proof}
\begin{Theorem}[name=Properties of Lebesgue (volume) measure]\label{thm:prop_leb_meas_rn}
    Let $\mu$ be the Lebesgue measure and let $\famL$ be the collection of all (Lebesgue) measurable sets. 
    We observe the following:
    \begin{properties}
    \item
	If $G$ is an open subset of $\Rd$, then $G \in \famL$.
    \item
	(Completeness) If $\outMeasure{E} = 0$, then $E \in \famL$. In particular if $F$ is a 
	subset of a set of exterior measure $0$, then $F \in \famL$.
    \item
	If $\lbrace E_i \rbrace$ is a countable collection of measurable sets, then
	\[\countUnion{E_i}{i} \in \famL.\]
    \item
	If $F$ is a closed subset of $\Rd$, then $F \in \famL$.
    \item
	If $E \in \famL$, then $\comp{E} \in \famL$. 
    \item
	If $\lbrace E_i \rbrace$ is a countable collection of measurable sets, then
	\[\countIntersection{E_i}{i} \in \famL.\]
    \item
	If $\lbrace E_i \rbrace$ is a countable collection of pairwise disjoint measurable sets, then
	\[\measure{\countUnion{E_i}{i}} = \infiniteSum{\measure{E_i}}{i}.\]
    \end{properties}
\end{Theorem}



















Note that a set $E$ has Lebesgue measure zero if $\measure{E} = 0$. 
We have a useful characterization of sets of Lebesgue measure zero.
\begin{Proposition}[name=Lebesgue measure $0$]\label{prop:lebesgue_meas_0}
    A set $E \subset \Rn$ has Lebesgue measure zero iff for any $\epsilon > 0$ there is a sequence of
    rectangles $\lbrace R_i \rbrace \subset \Rn$ with their sides parallel to the co-odrdinate axes
    such that,
    \[E \subset \countUnion{R_i}{i}\quad\text{and}\quad \infiniteSum{\volumeMeasure{R_i}}{i} <
	\epsilon\]
\end{Proposition}

