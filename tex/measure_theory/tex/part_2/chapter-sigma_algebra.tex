\chapter{Elements of measure theory: measurable sets}
In this chapter, we will define the abstract notion of measurability of sets.
%%%%%%%%%%%%%%%%%%%%%%%%%%%%%%%%%%%%%%%%%%%%%%%%%%%%%%%%%%%%%%%%%%%%%%%%%%%%%%%%%%%%%%%%%%%%%%%%%%%%%%%%%%%%%%
\section{Sigma Algebra of sets}
\begin{Definition}[name=Power set]
    Let $X$ be a non-empty set. The family of all subsets of $X$ is called the \emph{power set} of $X$
    and is denoted by $\powSet{X}$.
\end{Definition}

\begin{Definition}[name=Increasing sequence of sets]
    Let $\seq{A}{n}$ be a sequence of subsets of $X$. If $\incSetSeq{A}$ and $\countU{A} = A$, then
    we say that the $A_n$ form an increasing sequence of sets and increase to $A$. We denote this by
    $\atobUp{A}{A}$.
\end{Definition}

\begin{Definition}[name=Decreasing sequence of sets]
    Let $\seq{A}{n}$ be a sequence of subsets of $X$. If $\decSetSeq{A}$ and $\countI{A} = A$, then
    we say that the $A_n$ form an decreasing sequence of sets and decrease to $A$. We denote this by
    $\atobDown{A_n}{A}$.
\end{Definition}

Given two subsets $A,B \subset X$, we can write the union as a disjoint union as follows:
\begin{equation*}
    A \bigcup B = \left(A\right) \disjU \left(B \cap \comp{A}\right)
\end{equation*}
\begin{figure}
  \includestandalone[width=0.5\textwidth]{tex/tikz_figures/disjU}
  \caption{Disjoint union of two sets}\label{fig:tikz:disjU}
\end{figure}
Here we use $\disjU$ to denote that the union is between elements that are disjoint. 
The above observation along with the \emph{DeMorgan's} Law can be used to show the following
statements that equate an arbitrary union to disjoint union.

\begin{Remark}\label{rmk:disjU}
    Let $\seq{A}{n}$ be a sequence of subsets of $X$. Then,
    \begin{enumerate}
	\item $\finiteU{A}{n} = \left(A_1\right) \disjU \left(A_2 \cap \indxComp{A}{1}\right)
	    \dots \disjU \left(A_n\cap\indxComp{A}{n-1}\dots\cap\indxComp{A}{1}\right)$. 
	    Let us construct new sets $F_k = (A_k) \bigcap \comp{(\bigcup_{i = 1}^{k-1}
		A_i)}$ for $k > 1$ and $F_1 = A_1$. Thus $\countU{A} = \countDisjU{F}$. Note that
	    $F_k$ is just $\setDiff{A_k}{\bigcup_{i = 1}^{k-1}A_i}$. In this way,
	    we can get a sequence of pairwise disjoint sets $\seq{F}{n}$ whose countable 
	    union is the same as the countable union of the originial sequence.
	\item If $\atobUp{A_n}{A}$ then, 
	    \begin{enumerate}
		\item $ A_n = \finiteU{A}{n} = A_1 \disjU \left(\setDiff{A_2}{A_1}\right) \dots \disjU
		    \left(\setDiff{A_n}{A_{n-1}}\right)$,
		\item $A = \countU{A} = \dot\bigcup_{i \in \Zplus}\left({\setDiff{A_i}{A_{i-1}}}\right)$, 
		    where we take $A_0 = \emptyset$.
	    \end{enumerate}
	\item If $\atobUp{A_n}{A}$ then $\atobDown{\indxComp{A}{n}}{\comp{A}}$. If
	    $\atobDown{A_n}{A}$ then $\atobUp{\indxComp{A}{n}}{\comp{A}}$.
	    
    \end{enumerate}
\end{Remark}

\begin{Definition}[name=Algebra of sets]
    Let $X$ be a non-empty set. An \emph{algebra} on $X$ is a \emph{non-empty} collection of sets
    $\algebra{A} \subset \powSet{X}$ with the following properties,
    \begin{enumerate}
	\item If $E \in \algebra{A}$ then $\comp{E} \in \algebra{A}$.
	\item If $E_1,E_2,\dots,E_n \in \algebra{A}$ then $\bigcup_{i=1}^{n}E_i \in \algebra{A}$.
    \end{enumerate} 
\end{Definition}


\begin{Definition}[name=Sigma-Algebra of sets]
    Let $X$ be a non-empty set. A $\sigmaAlgebra$  on $X$ is a 
    \emph{non-empty} collection of sets $\algebra{A} \subset \powSet{X}$ with the following properties,
    \begin{enumerate}
	\item If $E \in \algebra{A}$ then $\comp{E} \in \algebra{A}$.
	\item If $\seq{E}{n} \in \algebra{A}$ then $\countU{E} \in \algebra{A}$.
    \end{enumerate} 
\end{Definition}
\begin{Theorem}[name=Properties of sigma algebras]\label{thm:prop_sigmaA}
    If $\family{A}$ is a $\sigma - \text{algebra}$ on $X$ then,
    \begin{properties}
	\item If $E_1,E_2,E_3,\dots, E_n \in \algebra{A}$ then $\finiteU{E}{n} \in \algebra{A}$.
	\item $X, \emptyset \in \family{A}$.
	\item If $\seq{E}{n} \in \family{A}$ then $\countI{E} \in \family{A}$.
	\item If $E_1,E_2,E_3,\dots, E_n \in \algebra{A}$ then $\finiteI{E}{n} \in \algebra{A}$.
	\item If $A,B \in \algebra{A}$ then $\setDiff{A}{B} \in \algebra{A}$.
    \end{properties}
\end{Theorem}
Note that for an algebra the closure under countable intersection is generally not true. 
Thus a $\sigma - \text{algebra}$ on a set $X$ is an algebra that is closed under countable union and
intersection.
\begin{proof}
    We prove in order,
    \begin{enumerate}
	\item Let $E_i = E_n$ for $i > n$. Then $\finiteU{E}{n} = \countU{E} \in \algebra{A}$.
	\item Since $\algebra{A}$ is non empty there is a $E \subset X$ such that $E \in
	    \algebra{A}$. Thus $\comp{E} \in \algebra{A}$. Thus $X = E \cup \comp{E} \in
	    \algebra{A}$ from the above property. Since $X \in \algebra{A}, \comp{X} = \emptyset \in
	    \algebra{A}$.
	\item From DeMorgan's Law.
	\item Since $E_1,E_2,\dots,E_n \in \algebra{A}, \indxComp{E}{1},\dots,\indxComp{E}{n} \in
	    \algebra{A}$. From $(1), \finiteUComp{E}{n} \in \algebra{A}$. Thus its complement
	    $\finiteI{E}{n}$ is in $\algebra{A}$.
	\item From above, $\setDiff{A}{B} = A \cap {\comp{B}}$ which is in $\algebra{A}$.
    \end{enumerate}
\end{proof}
\begin{Example}\label{ex:type_of_sigma_alg}
    The following are all $\sigmaAlgebra$s.
    \begin{enumerate}
	\item $\powSet{X}$ is a $\sigmaAlgebra$. It is called the maximal $\sigmaAlgebra$ on $X$.
	\item $\lbrace \emptyset, X \rbrace$ is called the minimal $\sigmaAlgebra$ on
	    $X$.
	\item For any $B \subset X$, the collection $\lbrace \emptyset, B, \comp{B}, X\rbrace$ is a
	    $\sigmaAlgebra$.
	\item The collection $\algebra{A} = \set{A \subset X}{A \, \text{is countable or} \,
	       	\comp{A}\,\text{is countable}}$ is a $\sigmaAlgebra$.
	    \begin{proof}
		$\comp{(\comp{A})} = A$. Thus if $A \in \algebra{A}$, it is either countable or
		$\comp{A}$ is countable. Thus $\comp{A} \in \algebra{A}$. Let $\seq{A}{n} \in
		\algebra{A}$. If all are countable then the countable union is countable and so is
		in $\algebra{A}$. If not there exist an index $j \in \Zplus$ such that
		$\indxComp{A}{j}$ is countable. Thus $\comp{(\countU{A})} = \countIComp{A} \subset
		\indxComp{A}{j} \in \algebra{A}$.
	    \end{proof}
	\item (\textbf{Restricted} $\sigmaAlgebra$)
	    Let $E \subset X$ be any set and let $\algebra{A}$ be a $\sigmaAlgebra$ on $X$. Then
	    the collection $\algebra{A}_{E} = \set{E \cap A}{A \in \algebra{A}}$ is $\sigmaAlgebra$
	    on $E$.
	    \begin{proof}
		Let $E_1 \in \algebra{A}_{E}$, then $E_1 = E \cap A$ for some $A \in \algebra{A}$.
		The complement of $E_1$ in $E$ is given by $E \cap \indxComp{E}{1}$. Thus we have $E
		\cap \indxComp{E}{1} = E \cap (\comp{E} \cup \comp{A})$ and so $E \cap \indxComp{E}{1} =
		(E \cap \comp{E}) \cup (E \cap \comp{A})$. Thus $E \cap \indxComp{E}{1} = E \cap
		\comp{A} \in \algebra{A}_{E}$. Let $\seq{E}{n}$ be a sequence in $\algebra{A}_{E}$.
		Each $E_i = E \cap A_i$ for some $A_i \in \algebra{A}$. Therefore $\countU{E} =
		E \cap \countU{A} \in \algebra{A}_{E}$.
	    \end{proof}
    \end{enumerate}
\end{Example}
\begin{Theorem}[name=Equivalent Characterization of $\sigmaAlgebra$]\label{thm:eq_ch_sigmaA}
    A collection of sets $\algebra{A}$ is a $\sigmaAlgebra$ on $X$ iff it is closed under
    complements and for any sequence of pairwise disjoint sets contained in the collection 
    their union is also contained in the collection. 
\end{Theorem}
\begin{proof}
    $\Rightarrow$ is immediately evident from the definition of $\sigmaAlgebra$. 

    $\Leftarrow$ Consider a sequence of sets $\seq{E}{n} \in \algebra{A}$. Using~\ref{rmk:disjU},
	we can construct a sequence of pairwise disjoint sets $\seq{F}{n}$, such that $\countU{A} =
	\countDisjU{F} \in \algebra{A}$. 
\end{proof}
\begin{Definition}[name=Generated $\sigmaAlgebra$]
    Let $X$ be any non-empty set and let $\family{E} \subset \powSet{X}$. The $\sigmaAlgebra$
    generated by $\family{E}$ is the unique smallest $\sigmaAlgebra$ containing $\family{E}$ and is
    denoted by $\sigmaGen{\family{E}}$.
\end{Definition}
In set theory a \emph{smallest} set in a collection is the set that is contained in every other set
of the collection. The next theorem guarantees the existence of a smallest $\sigmaAlgebra$.
\begin{Theorem}[name=Sigma Algebra generated by an arbitrary collection]\label{thm:sigma_al_gen}
    Let $X$ be a non-empty set,
    \begin{enumerate}
	\item
	    The intersection of any collection of $\sigmaAlgebra$ on $X$ is
	    itself a $\sigmaAlgebra$.
	\item 
	    Let $\family{E} \subset \powSet{X}$. There is a unique smallest $\sigmaAlgebra$
	    $\sigmaGen{\family{E}}$ containing $\family{E}$ in the sense that any $\sigmaAlgebra$ 
	    containing $\family{E}$ must contain $\sigmaGen{\family{E}}$.
    \end{enumerate}
\end{Theorem}
\begin{proof}
    Given a non-empty set $X$,
    \begin{enumerate}
	\item
	    Let $\mathcal{G}$ be a collection of $\sigmaAlgebra$ on $X$. Then,
	    \begin{equation*}
		\algebra{F} = \bigcap_{\algebra{A} \in \mathcal{G}}\algebra{A} =
		\set{E{\subset}X}{\forEv{\algebra{A}\in\mathcal{G}},E\in\algebra{A}}.
	    \end{equation*}
	    Let $E \in \algebra{F}$. Thus $\forEv{\algebra{A} \in \mathcal{G}}$, $E \in \algebra{A}$ and
	    hence $\comp{E} \in \algebra{A}$ for all $\algebra{A} \in \mathcal{G}$. Thus $E \in
	    \algebra{F}$. Similary for a sequence of sets $\seq{E}{n}$ in $\algebra{F}$, 
	    we have that $\countU{E} \in \algebra{F}$.
	\item
	    Let $\mathcal{G} = \set{\algebra{A}\,\text{on}\,X}{\family{E}\subset\algebra{A}}$. Then
	    $\mathcal{G}$ is a collection of $\sigmaAlgebra$ on $X$.
	    Define $\sigmaGen{\family{E}} = \bigcap_{\algebra{A} \in \mathcal{G}}\algebra{A}$. From above,
	    $\sigmaGen{\family{E}}$ is a $\sigmaAlgebra$. By definition for any $\sigmaAlgebra$ $\algebra{A}$
	    on $X$ that contains $\family{E}$, $\algebra{A} \in \mathcal{G}$ and so
	    $\sigmaGen{\family{E}}
	    \subset\algebra{A}$.
    \end{enumerate}
\end{proof}
\begin{Remark}\label{rmk:obs_sigma_gen}
    We make a few observations about the generated $\sigmaAlgebra$.
    \begin{enumerate}
	\item
	    If $\family{E}$ is a $\sigmaAlgebra$, then $\family{E} = \sigmaGen{E}$.
	    \begin{proof}
		Note that $\family{E} \subset \family{E}$ and so if $\family{E}$ is a
		$\sigmaAlgebra$, from the proof above $\sigmaGen{\family{E}} \subset \family{E}$.
		That $\family{E} \subset \sigmaGen{\family{E}}$ is evident from the definition of
		$\sigmaGen{\family{E}}$.
	    \end{proof}
	\item
	    For any $A \subset X$, we have $\sigmaGen{\lbrace A \rbrace} = \lbrace \emptyset, A,
	    \comp{A},X\rbrace$.
	\item If $\family{E} \subset \family{F}$, then $\sigmaGen{\family{E}} \subset
	    \sigmaGen{\family{F}}$.
	    \begin{proof}
		$\family{E} \subset \family{F} \subset \sigmaGen{\family{F}}$. Thus
		$\sigmaGen{\family{E}} \subset \sigmaGen{\family{F}}$.
	    \end{proof}
    \end{enumerate}
\end{Remark}
A very important principle to prove statements about generated sigma algebra is called the
\emph{principle of good sets} which is described in the following remark:
\begin{Remark}[name=Principle of good sets]\label{rmk:prin_good_sets}
    Suppose we want to show that for any collection $\family{E}$, the generated sigma algebra
    $\sigmaGen{\family{E}}$ has a certain property $P$. IF we can find a sub-collection $\family{F}
    \subset \sigmaGen{\family{E}}$ that has this property $P$, and show that $\family{F}$ is a
    $\sigmaAlgebra$, then $\sigmaGen{\family{E}}$ has property $P$, provided
    $\family{E}\subset\family{F}$.
\end{Remark}
We can generate new $\sigmaAlgebra$ from existing $\sigmaAlgebra$ using inverse maps.
\begin{Theorem}[name=Pre-Image $\sigmaAlgebra$]\label{thm:pre_img_sigma}
    Let $\map{f}{X}{Y}$ be a function.
    \begin{enumerate}
	\item
	    If $\algebra{A}$ is a $\sigmaAlgebra$ on $Y$, then
	    \begin{equation*}
		\invIm{f}{\algebra{A}} := \set{\invIm{f}{A}}{A \in \algebra{A}}
	    \end{equation*}
	    is a $\sigmaAlgebra$ on $X$.
	\item
	    If $\algebra{B}$ is a $\sigmaAlgebra$ on $X$, then
	    \begin{equation*}
		\set{A\subset Y}{\invIm{f}{A}\in\algebra{B}}
	    \end{equation*}
	    is a $\sigmaAlgebra$ on $Y$.
	\item
	    If $\family{E}$ is a collection of sets in $Y$, then
	    \begin{equation*}
		\sigmaGen{\invIm{f}{\family{E}}} = \invIm{f}{\sigmaGen{\family{E}}}
	    \end{equation*}
    \end{enumerate}
\end{Theorem}
\begin{proof}
    Given a function $\map{f}{X}{Y}$, 
    \begin{enumerate}
	\item Let $E \in \invIm{f}{\algebra{A}}$. Thus $E = \invIm{f}{A}$ for some $A \in
	    \algebra{A}$. $\comp{E} = \comp{(\invIm{f}{A})} = \invIm{f}{\comp{A}}$ and 
	    thus $\comp{E}\in\invIm{f}{\algebra{A}}$. Let $\seq{E}{n}$ be a sequence of sets in
	    $\invIm{f}{\algebra{A}}$. Then $E_i = \invIm{f}{A_i}$. Thus $\countU{E} =
	    \bigcup_{i\in\Zplus}\invIm{f}{A_i} = \invIm{f}{\countU{A}} \in \invIm{f}{\algebra{A}}$.
	\item Let $E$ be in the collecion, hence, $E \subset Y$ such that $\invIm{f}{E} \in 
	    \algebra{B}$, hence $\invIm{f}{\comp{E}} = \comp{(\invIm{f}{E})} \in \algebra{B}$. 
	    Thus $\comp{E}$ is in the collection. Similar argument for countable union.
	\item
	    We need to show $\sigmaGen{\invIm{f}{\family{E}}} \subset 
	    \invIm{f}{\sigmaGen{\family{E}}}$ and $ \invIm{f}{\sigmaGen{\family{E}}} \subset 
	    \sigmaGen{\invIm{f}{\family{E}}}$.
	    
	    For any $A \in \family{E}$, $A \in \sigmaGen{\family{E}}$ and hence
	    $\invIm{f}{\family{E}} \subset \invIm{f}{\sigmaGen{\family{E}}}$. But from
	    $(1)$, we know that $\invIm{f}{\sigmaGen{\family{E}}}$ is a $\sigmaAlgebra$, and
	    thus $\sigmaGen{\invIm{f}{\family{E}}} \subset
	    \invIm{f}{\sigmaGen{\family{E}}}$.

	    The other direction is tricky. We will use~\ref{rmk:prin_good_sets}. What we want to show
	    is that for any $A \in \sigmaGen{\family{E}}$ we should have,
	    \begin{equation*}
		\invIm{f}{A} \in \sigmaGen{\invIm{f}{\family{E}}}.
	    \end{equation*}
	    Thus consider the collection of good sets $\family{F}$, such that
	    \begin{equation*}
		\family{F} = \set{A \in \sigmaGen{\family{E}}}{\invIm{f}{A} \in 
		    \sigmaGen{\invIm{f}{\family{E}}}}
	    \end{equation*}
	    Clearly $\family{F}$ is not empty because $\family{E} \subset \family{F}$. This
	    is easily observed because $\invIm{f}{\family{E}} \subset
	    \sigmaGen{\invIm{f}{\family{E}}}$. If we show that $F$ is a $\sigmaAlgebra$ we
	    are done.
	    Let $E\in \family{F}$. Therefore $\invIm{f}{E} \in
	    \sigmaGen{\invIm{f}{\family{E}}}$ but
	    this means that $\comp{E} \in \family{F}$ since $\invIm{f}{\comp{E}} =
	    \comp{(\invIm{f}{E})} \in \sigmaGen{\invIm{f}{\family{E}}}$. Similary for 
	    countable unions. Thus $\family{F}$ is a $\sigmaAlgebra$ containing $\family{E}$ 
	    and so $\sigmaGen{\family{E}} \subset \family{F}$.
    \end{enumerate}
\end{proof}
Of particular interest is the situation involving a metric space, for example $\Rn$. In this case,
we have a built in natural concept of distance plus the associated topology of sets, for example 
open sets, closed sets etc.

\begin{Definition}[name=Borel Sets]
    Let $\metricS{X}{d}$ be a metric space. The $\sigmaAlgebra$ generated by the collection of open
    sets in $X$ is called the \emph{Borel} $\sigmaAlgebra$ on $X$ and is denoted by $\borelS{X}$.
    Its members are called the \emph{Borel sets}.
\end{Definition}
We can equivalently generate the Borel $\sigmaAlgebra$ using the family of closed sets. The Borel
sets include open sets, countable union an intersection of open sets, closed sets, countable union
and intersection of closed sets and so on. 
\begin{Definition}
    A countable intersection of open sets is called a $G_{\delta}$ set, a countable union of closed
    sets is called a $F_{\sigma}$ set, a countable union of $G_{\delta}$ set is called
    $G_{\delta\sigma}$ and so on.
\end{Definition}
The Borel $\sigmaAlgebra$ on $\Rn$, $\borelS{\Rn}$ is particularly important. When $n = 1$, we get
the Borel $\sigmaAlgebra$ on the real line, $\borelS{\R}$. The most interesting generators are the
family of open and half-open rectangles,

\begin{align*}
    &\family{J}_{n} = \set{\interval{a_1}{b_1} \times \dots \interval{a_n}{b_n}}{a_j,b_j \in \R}, \\	
    &\family{J}^{o}_{n} = \set{(a_1,b_1) \times \dots (a_n,b_n)}{a_j,b_j \in \R}, \\	
    &\family{J}^{cr}_{n} = \set{\interval[open left]{a_1}{b_1} \times \dots 
	\interval[open left]{a_n}{b_n}}{a_j,b_j \in \R}, \\	
    &\family{J}^{cl}_{n} = \set{\interval[open right]{a_1}{b_1} \times \dots 
	\interval[open right]{a_n}{b_n}}{a_j,b_j \in \R}. \\	
    &\family{J}^{o}_{\Q_{n}} = \set{(a_1,b_1) \times \dots 
	(a_n,b_n)}{a_j,b_j \in \Q}. \\	
\end{align*}
Let us denote by $\family{G},\family{F},\family{K}$, the collection of open, closed and compact sets
in $\Rn$. The following theorem is very useful in realizing the generating sets of $\borelS{\Rn}$.
\begin{Theorem}[name=Generating Borel Sets in $\Rn$]\label{thm:gen_borel_rn}
    We have,
    \begin{enumerate}
	\item 
	    $\borelS{\Rn} = \sigmaGen{\family{G}}$.
	\item
	    $\borelS{\Rn} = \sigmaGen{\family{F}}$.
	\item
	    $\borelS{\Rn} = \sigmaGen{\family{K}}$.
	\item
	    $\borelS{\Rn} = \sigmaGen{\family{J}^{o}_{n}}$.
	\item
	    $\borelS{\Rn} = \sigmaGen{\family{J}^{cr}_{n}}$.
	\item
	    $\borelS{\Rn} = \sigmaGen{\family{J}^{cl}_{n}}$.
	\item
	    $\borelS{\Rn} = \sigmaGen{\family{J}_{n}}$.
    \end{enumerate}
\end{Theorem}
\begin{proof}
    $(1)$ is just the definition of Borel sets. For any open set $U \in \family{G}$, $\comp{U} \in
    \family{F}$. Thus $U \in \sigmaGen{F}$ i.e $\family{G} \subset \sigmaGen{\family{F}}$ and thus
    $\sigmaGen{\family{G}} \subset \sigmaGen{\family{F}}$. A similar argument with roles reversed 
    leads to $\sigmaGen{\family{F}} \subset \sigmaGen{\family{G}}$. Any compact set $K$ is also a
    closed set so $\sigmaGen{\family{K}} \subset \sigmaGen{\family{F}}$. Let us write a closed set
    as a countable union of compact sets in $\Rn$. If $F \in \family{F}$, then let $F_i = F \cap
    \closure{\ball{i}{\vect{0}}}, i\in \Zplus$. Each $F_i$ is an intersection of closed sets and 
    is bounded and hence is compact in $\Rn$. Moreover,
    \begin{equation*}
	F = \countU{F}.
    \end{equation*}	
    Thus $F \in \sigmaGen{\family{K}}$ and so $\sigmaGen{\family{F}} \subset \sigmaGen{\family{K}}$.
    Thus we have shown the following,
    \begin{equation*}
	\borelS{\Rn} = \sigmaGen{\family{G}} = \sigmaGen{\family{K}} = \sigmaGen{\family{F}}.
    \end{equation*}
    Let $R \in \family{J}^{o}_{n}$ be a open rectangle (box) in $\Rn$. Any open rectangle is an open
    set, thus 
    $\sigmaGen{\family{J}^{o}_{\Q_{n}}} \subset \sigmaGen{\family{J}^{o}_{n}} 
    \subset \sigmaGen{\family{G}}$. Now for the other direction consider any open set $U \in
    \family{G}$. $U = \bigcup\limits_{p\in U}\ball{\epsilon_{p}}{p}$. But for any open ball we can
    inscribe a rectangle with rational endpoints in it and so,
    \begin{equation*}
	U = \bigcup\limits_{R \in \family{J}^{o}_{\Q_{n}};R \subset U}R
    \end{equation*}
    Thus $\family{G} \subset \sigmaGen{\family{J}^{o}_{\Q_{n}}}$ and so we get the following,
    \begin{equation*}
	\borelS{\Rn} = \sigmaGen{\family{J}^{o}_{n}}
    \end{equation*}
    Let $R^{cr} \in \family{J}^{cr}_{n}$ be a half-open rectangle in $\Rn$ closed at right.
    Thus, $R^{cr} = \interval[open left]{a_1}{b_1} \times \dots 
    \interval[open left]{a_n}{b_n}$ which can be written as,
    \begin{equation*}
	\interval[open left]{a_1}{b_1} \times \dots 
	\interval[open left]{a_n}{b_n} = \bigcap\limits_{j\in\Zplus}(a_1,b_1+1/j) \times \dots
	(a_n,b_1+1/j).
    \end{equation*}	    
    Thus, $\family{J}^{cr}_{n} \subset \sigmaGen{\family{J}^{o}_{n}}$. Let $R \in 
    \family{J}^{o}_{n}$ be a open rectangle (box) in $\Rn$. 
    Thus, $R = (a_1,b_1) \times \dots (a_n,b_n)$ which can be written as,
    \begin{equation*}
	(a_1,b_1)\times\dots(a_n,b_n) = \bigcup\limits_{j\in\Zplus}
	\interval[open left]{a_1}{b_1-1/j}\times\dots\interval[open left]{a_n}{b_1-1/j}.
    \end{equation*}
    Thus,$\family{J}^{o}_{n} \subset \sigmaGen{\family{J}^{cr}_{n}}$. Hence we have shown,
    \begin{equation*}
	\sigmaGen{\family{J}^{o}_{n}} = \sigmaGen{\family{J}^{cr}_{n}}
    \end{equation*}
    A similar argument can be done for the family $\family{J}^{cl}_{n}$.
\end{proof}
\begin{Corollary}\label{thm:gen_borel_R}
    $\borelS{\R}$ is generated by each of the following:
    \begin{enumerate}
	\item
	    (Open intervals): $\family{E}_1 = \set{(a,b)}{a < b}$.
	\item
	    (closed intervals): $\family{E}_2 = \set{\interval{a}{b}}{a < b}$.
	\item
	    (half open closed right intervals): $\family{E}_3 = \set{\hInt{a}{b}}{a < b}$
	\item
	    (half open closed left intervals): $\family{E}_4 = \set{\interval[open right]{a}{b}}{a < b}$
	\item
	    (open rays): $\family{E}_5 = \set{(a,\infty)}{a \in \R}$
	\item
	    (open rays): $\family{E}_6 = \set{(-\infty,a)}{a \in \R}$
	\item
	    (closed rays): $\family{E}_7 = \set{\interval[open right]{a}{\infty}}{a \in \R}$
	\item
	    (closed rays): $\family{E}_8 = \set{\hInt{a}{\infty}}{a \in \R}$
    \end{enumerate}
\end{Corollary}
Before moving onto the subject of measures, we'll give a technical result that will be useful in
constructing measure for the Borel sets. The following just abstracts the class of half open
rectangles.
\begin{Definition}[name=Elementary family]
    Let $X$ be a non-empty set. An elementary family is a collection $\family{E}$ of subsets of $X$
    such that,
    \begin{enumerate}
	\item
	    $\emptyset \in \family{E}$.
	\item
	    If $E,F \in \family{E}$ then $\intersection{E}{F} \in \family{E}$.
	\item
	    If $E \in \family{E}$, then $\comp{E}$ is a finite disjoint union of members of
	    $\family{E}$.
    \end{enumerate}
\end{Definition}
\begin{Example}\label{ex:hint_elem_fam}
    Consider the family $\family{J}^{cr}_{1}$ of half-open closed right interval in $\R$. If we
    include $-\infty,\infty$ then we refer to sets of this family as \textbf{h-intervals}. These are
    collection of subsets of $\R$ of the form $\interval[open left]{a}{b}, (a,\infty), \emptyset$. 
    $\family{J}^{cr}_{1}$ is an elementary family. To check the last condition of the definition,
    consider the set $\interval[open left]{a}{b}$. Its complement is set $\interval[open
    left]{-\infty}{a} \cup (b,\infty)$ which is the disjoint union of two \textbf{h-intervals}. 
\end{Example}
\begin{Theorem}[name=Constructing an algebra from elementary family]\label{thm:const_algebra_elem}
    If $\family{E}$ is an elementary family, the collection, $\algebra{A}$ of finite disjoint unions
    of members of $\family{E}$ is an algebra.
\end{Theorem}
\begin{proof}
    First note that $\algebra{A}$ is not empty since $\family{E} \subset \algebra{A}$. This is
    because any $E \in \family{E}$ can be written as the disjoint union of $E \disjU \emptyset$.

    Let $A \in \algebra{A}$. Then $A = E_1 \disjU E_2 \disjU \dots \disjU E_n$. We need to show that
    $\comp{A} \in \algebra{A}$. Note that $\comp{A} = \indxComp{E}{1} \cap \indxComp{E}{2} \cap
    \dots \cap \indxComp{E}{n}$. However each $\indxComp{E}{m} = E^{m}_{1} \disjU \dots \disjU
    E^{m}_{n_m}$. Therefore $\comp{A}$ is,
    \[\bigcap\limits_{m =1}^{n}\disjU\limits_{j=1}^{n_m}E^{m}_{j} = 
	\disjU\set{E_{j_1}^{1}\cap\dots\cap E_{j_n}^n}{1\leq j_m \leq n_m,\, 1\leq m \leq n},\]
    which is a disjoint union of finite sets in $\family{E}$ and hence in $\algebra{A}$.

    Let us show that $\algebra{A}$ is closed under union of two sets. By induction we will get
    closure of finite union.
    Let $A,B \in \algebra{A}$. Then $A = \finiteDisjU{C_i}{n_A}$ and $B = \finiteDisjU{D_i}{n_B}$. We
    need to show that $A\cup B \in \algebra{A}$ i.e the following set,
    \[A \cup B = \disjU\set{C_{j_A}\cup D_{j_B}}{1\leq j_A \leq n_A; 1\leq j_B \leq n_B}.\]

    Each $C_{j_A}\cup D_{j_B}$ belong to $\algebra{A}$. 
    To see this, let us show that if $C,D \in \family{E}$ then $C \cup D \in \algebra{A}$.
    Note that, $\comp{D} = \finiteDisjU{E_i}{J}$. 
    Thus $C \cap \comp{D} = \finiteDisjU{C\cap E_i}{J}$. Since
    $C,E_i \in \family{E}$ we get $C \cap \comp{D} \in \algebra{A}$. But $C \cup D = (C \cap
    \comp{D}) \disjU (D)$. Since $(C \cap \comp{D})$ is the disjoint union of sets in $\family{E}$,
    $C \cup D$ is then a disjoint union of sets in $\family{E}$ and hence $C\cup D \in \algebra{A}$.
    Thus from induction if $A_1, A_2, \dots, A_n \in \family{E}$ then $\finiteU{A}{n} \in
    \algebra{A}$. 
    Hence, each $C_{j_A}\cup D_{j_B}$ belong to $\algebra{A}$.
    Thus by induction on $n_A \times n_B$ we get
    the result. Now we can induct on a finite sequence of sets $A_i \in \algebra{A}$ to show that
    $\algebra{A}$ is closed under finite unions. 
\end{proof}
\begin{Remark}\label{rmk:hinterval}
    Note that if $\family{E}$ is the collection $\family{J}^{cr}_{1}$, then the collection
    $\algebra{A}$ of finite disjoint unions of \textbf{h-intervals} is an algebra
    by~\ref{thm:const_algebra_elem}. But note that $\family{E} \subset \family{A}$ and thus
    $\sigmaGen{\family{E}} \subset \sigmaGen{\family{A}}$. However, 
    $\sigmaGen{\family{A}} = \borelS{\R}$ which is also equal to $\sigmaGen{\family{E}}$
    from~\ref{thm:gen_borel_rn}.
\end{Remark}

\section{Measures}
We now consider how to measure the size of the sets in a given $\sigmaAlgebra$ on a set $X$.
\begin{Definition}[name=Measure]
    Let $X$ be a set on which there is a $\sigmaAlgebra$ $\algebra{M}$. A measure on the measurable
    space $\metricS{X}{\algebra{M}}$ is a function \[\map{\mu}{\algebra{M}}{\interval{0}{\infty}}\]
    satisfying,
    \begin{enumerate}
	\item
	    $\measure{\emptyset} = 0$,
	\item
	    ($\sigma$-additivity) If $\seq{E}{n}$ be a sequence of pairwise disjoint sets in 
	    $\algebra{M}$, then
	    \begin{equation*}
		\measure{\countDisjU{E}} = \sumInf{\measure{E_i}}.
	    \end{equation*}
    \end{enumerate}
\end{Definition}
A \emph{pre-measure} is function that satisfies the above on an \emph{algebra} and not necessarily
on a $\sigmaAlgebra$. Note that for a \emph{pre-measure}, additivity is statisfied if the countable
union of a sequence of sets is in the algebra. We will denote a pre-measure by $\preMeas$. 
The triple $\measureS{X}{\algebra{M}}{\mu}$ is called a \emph{measure space}.
A \emph{finite measure} is a measure with $\measure{X} < \infty$, and a \emph{probability measure}
is a measure with $\measure{X} = 1$. Note that for a finite measure, for any $E \in \algebra{M}$,
$\measure{E} < \infty$. A measure is said to be $\sigmaFinite$, if $\algebra{M}$
contains a sequence $\incSetSeq{A}$ such that $\countU{A} = X$ and $\measure{A_i} < \infty$ for
every $i \in \Zplus$. Most measures encountered in practice are at least $\sigmaFinite$, and
non-$\sigmaFinite$ measures have some strange behaviour. A measure $\mu$ is $\sigma$ semi-finite if
any set $F \in \algebra{M}$ such that $\measure{F} = \infty$ there is a set $E \in \algebra{M}$ and
$E \subset F$ such that $\measure{E} < \infty$.

\begin{Definition}[name=Finitely additive measure]
    If $\seq{E}{i}$ are disjoint sets in $\algebra{M}$, and $\measure{\finiteDisjU{E}{n}} =
    \sumFinite{\measure{E_i}}{n}$, then $\mu$ is finitely additive measure.
\end{Definition}
Note that a pre-measure will always satisfy finite additivity.
\begin{Example}
    Let $X$ be an uncountable set and let $\algebra{M}$ be the $\sigmaAlgebra$,
    \begin{equation*}
	\set{A \subset X}{A \, \text{is countable or} \,
	    \comp{A}\,\text{is countable}}
    \end{equation*}
    Then the following set function defined on $\algebra{M}$,
    \begin{equation*}
	\measure{E} =
	\begin{cases}
	    1 &\text{E is countable}\\
	    0 &\text{$\comp{E}$ is countable}
	\end{cases}
    \end{equation*}
    is a finitely additive measure but is not countably additive.
\end{Example}
We collect all the important properties of a measure in the following theorem. The following
properties will also be satisfied by a pre-measure provided we check that the countable union and
intersection of a sequence of sets belong to the algebra.
\begin{Theorem}[name=Properties of measure]\label{thm:prop_of_meas}
    Let $\measureS{X}{\algebra{M}}{\mu}$ be a measure space. Then,
    \begin{properties}
	\item 
	    (finite-additivity) $E,F \in \algebra{M}$ and $E \cap F = \emptyset$, then 
	    \[\measure{E\disjU F} = \measure{E} + \measure{F}.\]
	\item
	    (monotonicity) If $E,F \in \algebra{M}$ and $E \subset F$ then 
	    \[\measure{E} \leq \measure{F}.\]
	    Moreover if $\measure{F} < \infty$, then 
	    \[\measure{\setDiff{F}{E}} = \measure{F} - \measure{E}.\]
	\item
	    (sub-additivity) If $\seq{E}{i}$ be a sequence of sets in $\algebra{M}$, 
	    then \[\measure{\countU{E}} \leq \sumInf{\measure{E_i}}.\]
	\item
	    (continuity from below) If $\seq{E}{i} \in \algebra{M}$, and \[\incSetSeq{E},\text{then}\]
	    \[\measure{\countU{E}} = \lim\limits_{i\to\infty}\measure{E_i}.\]
	\item
	    (continuity from above) If $\seq{E}{i} \in \algebra{M}$, 
	    and \[\decSetSeq{E}\] and $\measure{E_n} < \infty$ for some $n \in \Zplus$, 
	    then \[\measure{\countI{E}} =
	    \lim\limits_{i\to\infty}\measure{E_i}.\]
    \end{properties}
\end{Theorem}
\begin{proof}
    We prove in order.
    \begin{enumerate}
	\item 
	    Let $E_1 = E$ and $E_2 = F$ and $E_i = \emptyset$ for all $i \geq 3, i \in \Zplus$.
	    Then, $\measure{E\disjU F} = \measure{\countDisjU{E}} = \measure{E} + \measure{F}$.
	\item
	    $F = E \disjU F\cap\comp{E}$, and so $\measure{F} = \measure{E} +
	    \measure{F\cap\comp{E}}$, and hence $\measure{E} \leq \measure{F}$. 
	    (Since $E,F \in \algebra{M}$, $F\cap\comp{E}$ is also in
	    $\algebra{M}$, the result follows from $(1)$ because $\mu$ is a non-negative set
	    function). Now if $\measure{F} < \infty$, then $\measure{E} < \infty$ and so we can
	    subtract the $\measure{E}$ to obtain $\measure{F} - \measure{E} =
	    \measure{F\cap\comp{E}} = \measure{\setDiff{F}{E}}$.
	\item 
	    From~\ref{rmk:disjU}, we can construct disjoint sets out of regular sets as follows,
	    let $F_1 = E_1$ and $F_k = E_k - \finiteU{E}{k-1}$ for $k \geq 2$. 
	    Note that each $\measure{F_k} \leq \measure{E_k}$ and $\countU{E} = \countDisjU{F}$. 
	    Hence, \[\measure{\countU{E}} = \measure{\countDisjU{F}} = \sumInf{\measure{F_i}} \leq
		\sumInf{\measure{E_i}}.\]
	\item
	   From~\ref{rmk:disjU}, we can construct disjoint sets as above, let $F_1 = E_1$ and 
	   $F_k = E_k - \finiteU{E}{k-1} = E_k - E_{k-1}$ for $k \geq 2$.
	   See~\ref{fig:tikz:measure_prop}. Also note that 
	   $E_n = \finiteDisjU{F_i}{n}$ and so $\measure{E_n} = \measure{\finiteDisjU{F}{n}} =
	   \sumFinite{\measure{F_i}}{n}$. Thus,
	   \begin{align*}
	       \measure{\countU{E}} &=\measure{\countDisjU{F}}  \\
	       &=\sumInf{\measure{F_i}} \\
	       &=\lim\limits_{n\to\infty}\sumFinite{\measure{F_i}}{n} \\
	       &=\lim\limits_{n\to\infty}\measure{E_n}. 
	   \end{align*}
	   


       \item
	   
	   Let $F_k = E_n - E_k$ for $k > n$, and $F_k = \emptyset$ for $k \leq n$. 
	   Note that $E_n \supset E_{n+1} \supset E_{n+2} \dots$, 
	   and so $F_{n+1} \subset F_{n+2} \dots$ i.e.~$F_{n+1} = E_n - E_{n+1} \subset E_n -
	   E_{n+2} = F_{n+2}$ and so on. See~\ref{fig:tikz:measure_prop2}. Hence,
	   \[\bigcup\limits_{i \geq n+1} F_i = \bigcup\limits_{i \geq n+1}(\setDiff{E_n}{E_i}) =
	       \setDiff{E_n}{\bigcap\limits_{i \geq n+1}E_i}.\]
	   Thus we get,
	   \[\measure{\countU{F}} = \measure{E_n} - \measure{\bigcap\limits_{i \geq n+1}E_i}.\]
	   because, $E_{n+k} \subset E_n$ and $\measure{E_n} < \infty$.
	   Also we get a increasing sequence of sets $F_{i} \subset
	   F_{i+1} \subset F_{i+2} \dots$, for $i > n$. Thus we know that,
	   \[\measure{\countU{F}} = \lim\limits_{i\to\infty}\measure{F_i} = \lim\limits_{k\to\infty}
	       \measure{\setDiff{E_n}{E_{n+k}}}.\]
	   Since $E_{n+k} \subset E_n$ and $\measure{E_n} < \infty$ we get,
	  \[\measure{\countU{F}} = \lim\limits_{i\to\infty}\measure{F_i} = 
	      \measure{E_n} - \lim\limits_{k\to\infty}\measure{E_{n+k}}.\]
	  Thus equating the two we get our result.
 
    \end{enumerate}
\end{proof}

\begin{figure}
    \includestandalone[width=0.45\textwidth]{tex/tikz_figures/measure_prop}
    \caption{Illustration of proof~\ref{thm:prop_of_meas} $(4)$}\label{fig:tikz:measure_prop}
\end{figure}
\begin{figure}
    \includestandalone[width=0.45\textwidth]{tex/tikz_figures/measure_prop2}
    \caption{Illustration of proof~\ref{thm:prop_of_meas} $(5)$}\label{fig:tikz:measure_prop2}
\end{figure}

\section{Sets of measure zero and completion of measure}
We know that we want to deal with sets of measure zero. There is a technical issue about such sets
that we settle now.
\begin{Definition}[name=Sets of measure zero]
    If $\measureS{X}{\algebra{M}}{\mu}$ is a measure space, a set $E \in \algebra{M}$ with
    $\measure{E} = 0$ is called a set of \emph{measure zero}. 
\end{Definition}
\begin{Definition}[name=Almost everywhere]
    If a statement about points $x \in X$ is true except for $x$ in a set of \emph{measure zero}, we
    say that the statement is true \emph{almost everywhere} and denote it by a.e.
\end{Definition}
\begin{Proposition}[name=Countable union of sets of measure $0$]\label{prop:count_union_meas_0}
    A countable union of sets of measure zero has measure zero.
\end{Proposition}
\begin{proof}
    From the sub-additivity property of measure, if $\seq{E}{i}$ is a sequence of measure zero then
    $\measure{\countU{E}} \leq \sumInf{\measure{E_i}} = 0$. Since measure is a non-negative set
    function we get $\measure{\countU{E}} = 0$.
\end{proof}
\begin{Remark}
    If $\measure{F} = 0$ for some $F \in \algebra{M}$, then $\measure{E} = 0$ for any $E \subset F$
    whenever $E \in \algebra{M}$. However, $E$ may not be in $\algebra{M}$. This is a technical
    issue that we need to resolve. We will do so my adding to the $\algebra{M}$ all those
    subsets of sets in $\algebra{M}$ that have measure $0$. This is called the completion of the
    measure $\mu$.
\end{Remark}
\begin{Definition}[name=Complete measure space]
    If $\measureS{X}{\algebra{M}}{\mu}$ is a measure space such that $\algebra{M}$ contains all
    subsets of sets in $\algebra{M}$ with measure $0$, then $\measureS{X}{\algebra{M}}{\mu}$ is
    \emph{complete}.
\end{Definition}
Completeness eliminates some annoying issues and it can always be obtained by enlarging the domain of
a given measure to obtain an equivalent measure in the following sense:
\begin{Theorem}[name=Completion of a measure]\label{thm:comp_of_meas}
    Let $\measureS{X}{\algebra{M}}{\mu}$ be a measure space. Let $\family{N} = \set{N \in
	\algebra{M}}{\measure{N} = 0}$. Define,
    \[\closure{\algebra{M}} = \set{E\cup F}{E \in \algebra{M},\text{$\thereIs{N\in\family{N}}$ such
	    that $F \subset N$}},\]
     then $\closure{\algebra{M}}$ is a $\sigmaAlgebra$ on $X$ that contains $\algebra{M}$. Moreover,
     the unique measure $\xoverline{\mu}$ on $\closure{\algebra{M}}$ defined by 
     $\compMeasure{E\cup F} = \measure{E}$ for all $E \in \algebra{M}$ makes
     $\measureS{X}{\closure{\algebra{M}}}{\xoverline{\mu}}$ complete.
\end{Theorem}
\begin{proof}
    Clearly $\algebra{M} \subset \closure{\algebra{M}}$. We need to show that
    $\closure{\algebra{M}}$ is a $\sigmaAlgebra$. Let $A \in \closure{\algebra{M}}$. Then $A = E
    \cup F$ such that there is a $N \in \family{N}$ with $F \subset N$. Note that $F \subset N
    \implies \comp{N} \subset \comp{F}$ and so $\comp{F} = \comp{N} \disjU \setDiff{N}{F}$. 
    Thus $\comp{A} = \comp{E}\cap\comp{F} = \comp{E}\cap(\comp{N}\cup(\setDiff{N}{F}))$. 
    Thus the distributive law yields,
    \[\comp{A} = (\comp{E}\cap\comp{N})\cup (\comp{E}\cap{N}\cap\comp{F}).\]
    Now $(\comp{E}\cap\comp{N}) \in \algebra{M}$ since both $E,N \in \algebra{M}$.
    Also $(\comp{E}\cap{N}\cap\comp{F}) \subset N$ and hence $\comp{A} \in \closure{\algebra{M}}$.
    Let $\seq{A}{i}$ be a sequence of sets in $\algebra{M}$. Then each $A_i = E_i \cup F_i$ such
    that there is a $N_i \in \family{N}$ such that $F_i \subset N_i$. Thus,
    \[\countU{A} = (\countU{E})\, \bigcup\, (\countU{F}).\]
    $\countU{E} \in \algebra{M}$ and $\countU{F} \subset \countU{N}$.
    From~\ref{prop:count_union_meas_0}, $\countU{N} \in \family{N}$ and hence $\countU{A} \in
    \closure{\algebra{M}}$. Thus $\closure{\algebra{M}}$ is a $\sigmaAlgebra$.

    We have to check if $\compMeasure{A}$ is well defined i.e if $A = E_1 \cup F_1 = E_2 \cup F_2$
    then $\measure{E_1} = \measure{E_2}$. To see this, $E_1 \subset E_1 \cup F_1 = E_2 \cup F_2
    \subset E_2 \cup N_2 \in \algebra{M}$. Thus $\measure{E_1} \leq \measure{E_2} + \measure{N_2} =
    \measure{E_2}$. Similarly, $\measure{E_2}  \leq \measure{E_1}$. 
    
    Thus to show that 
    $\measureS{X}{\closure{\algebra{M}}}{\xoverline{\mu}}$ is complete we need to show that for any
    $A \in \closure{\algebra{M}}$, if $\compMeasure{A} = 0$, then for any $B \subset A$ we must have
    $B \in \closure{\algebra{A}}$. Since $A = E \cup F$ where $F \subset N$ for some $N \in
    \family{N}$, we have $\compMeasure{A} = \measure{E}$. But this implies that $\measure{E} = 0$
    and so $E \in \family{N}$. Thus $A \in \family{N}$. But $B \subset A$ and so $B = \emptyset
    \cup B$ where $\emptyset \in \algebra{M}$ and $B \subset A \in \family{N}$. Thus $B \in
    \closure{\algebra{M}}$. 
\end{proof}
\begin{Example}
    Consider a set $X \neq \emptyset$ and the minimal sigma 
    algebra $\algebra{M} = \lbrace X,\emptyset\rbrace$ with a measure $\mu$ defined to be $0$ 
    for any set in the $\algebra{M}$. Clearly $\measureS{X}{\algebra{M}}{\mu}$ is not complete. To
    get a completion of $X$ we must add all subsets of $X$ to the $\sigmaAlgebra$. Thus
    $\closure{\algebra{M}} = \powSet{X}$.
\end{Example}
\section{Construction of measures}
A $\sigmaAlgebra$ is a huge set and specifying a set function that satisfies the requirements of a
measure is not a trivial pursuit. Instead we work with set functions that can be easily constructed 
and then derive a measure from it by restricting or extending its domain. One such way is the
\emph{outer measure}. As we will see an outer measure can be derived easily from certain set 
functions. Once an outer measure is constructed it can be restricted to generate a measure.
\begin{Definition}[name=Outer measure]
    If $X$ is a non-empty set, an outer measure on $X$ is a function
    $\map{\outMeas}{\powSet{X}}{\interval{0}{\infty}}$ such that,
    \begin{enumerate}
	\item
	    \[\outMeasure{\emptyset} = 0.\]
	\item
	    (Monotonicity) For any $A \subset B$,
	    \[\outMeasure{A} \leq \outMeasure{B}.\]
	\item
	    (sub-additivity) For any $\seq{A}{i}$ in $\powSet{X}$,
	    \[\outMeasure{\countU{A}} \leq \sumInf{\outMeasure{A_i}}.\]
    \end{enumerate}
\end{Definition}
It is easy to see that every measure whose domain is the $\powSet{X}$ is an outer measure. Now we
will show that an outer measure can be restricted to yield a measure. We will use Carath{\'e}dory's
theorem to establish this result.

\begin{Definition}[name=$\outMeasurable$ sets: Carath{\'e}dory's Criterion]\label{def:carath_crit}
    Given any outer measure $\outMeas$, we say that a set $E$ is $\outMeasurable$ if,
    \[\outMeasure{A} = \outMeasure{A\cap E} + \outMeasure{A\cap \comp{E}},\]
    for any subset $A \subset X$.
\end{Definition}
We want those sets $E$ such that no matter what $A \subset X$ we take, outer measure of $A$ is
additive on $A$ (w.r.t E). This is shown in~\ref{fig:tikz:caratheodary}. Note that $A$
itself is not required to be $\outMeasurable$. A vague motivation for this is, if $E$ is a
\emph{good} set and $A \supset E$, Carath{\'e}dory's criteria states that its outer measure
$\outMeasure{E} = \outMeasure{E\cap A}$ is equal to $\outMeasure{A} -
\outMeasure{\intersection{A}{\comp{E}}}$. While the later concerns measuring $E$ from
    \emph{inside} $A$, the term $\outMeasure{E} = \outMeasure{E\cap A}$ concerns measuring $E$ from
    \emph{outside} $E$. This is seen from the right side of~\ref{fig:tikz:caratheodary} and thus
    \emph{good} sets have the same \emph{measure}{\textemdash}inside or outside.

We can just say measurable instead of $\outMeasurable$ if $\outMeas$ is clear from the context.
\begin{figure}
  \includestandalone[width=0.75\textwidth]{tex/tikz_figures/caratheodary}
  \caption{$\carathCrit$: $\outMeasure{A}$ is the sum of the two colors.}\label{fig:tikz:caratheodary}
\end{figure}

\begin{Proposition}\label{prop:carath_crit}
    Given any outer measure $\outMeas$, we say that a set $E$ is $\outMeasurable$ if,
    \[\outMeasure{A} \geq \outMeasure{A\cap E} + \outMeasure{A\cap \comp{E}},\]
    for any subset $A \subset X$.
\end{Proposition}
\begin{proof}
    From the~\ref{fig:tikz:caratheodary} it is easy to see that $A = \intersection{A}{E} \disjU
    \intersection{A}{\comp{E}}$. Since $\outMeas$ is sub-additive we have $\outMeasure{A} \leq
    \outMeasure{\intersection{A}{E}} + \outMeasure{\intersection{A}{\comp{E}}}$. Thus, for $E$ to be
    $\outMeasurable$, we must have,
    \[\outMeasure{A} \geq \outMeasure{A\cap E} + \outMeasure{A\cap \comp{E}},\]
    for any subset $A \subset X$.
\end{proof}
\begin{Theorem}[name=Carath{\'e}odory Theorem]\label{thm:carath_restr_thm}
    Let $X$ be a non-empty set with an outer measure $\outMeas$ defined on $\powSet{X}$. Define,
    \[\algebra{M} = \set{E \subset X}{E\,\text{is $\outMeasurable$}}.\]
    Then, $\algebra{M}$ is a $\sigmaAlgebra$ and the restriction of $\outMeas$ on $\algebra{M}$
    is a measure. Thus, 
    $\measureS{X}{\algebra{M}}{\restrict{\outMeas}{\algebra{M}}}$ is a measurable space.
\end{Theorem}
\begin{proof}
    We first need to show that $\algebra{M}$ is a $\sigmaAlgebra$. A priori we don't know if
    $\algebra{M}$ is empty or not. To this end we will show that $\emptyset \in \algebra{M}$.
    If $\outMeasure{E} = 0$ then, 
    we need to show that for any $A \subset X$, 
    $\outMeasure{A} \geq \outMeasure{A\cap E} + \outMeasure{A\cap \comp{E}}$. Since,
    $A \supset \intersection{A}{\comp{E}}$ and hence $\outMeasure{A} \geq \intersection{A}{\comp{E}}$
    from monotonicity of $\outMeas$. Similarly, $ E\supset \intersection{A}{E}$ and 
    $\outMeasure{E} \geq \intersection{A}{E}$. Thus, adding, $\outMeasure{A} + \outMeasure{E} \geq 
    \outMeasure{A\cap E} + \outMeasure{A\cap \comp{E}}$. Since
    $\outMeasure{E} = 0$, the result follows. Thus $E$ is $\outMeasurable$. This means that
    $\emptyset \in \algebra{M}$ since $\outMeasure{\emptyset} = 0$. 

    If $E \in \algebra{M}$ then for any $A \subset X$, 
    $\outMeasure{A} = \outMeasure{\intersection{A}{E}} + \outMeasure{\intersection{A}{\comp{E}}}$. 
    Replacing $E$ by $\comp{E}$ we get the same expression and thus $\comp{E} \in \algebra{M}$.

    Let $E_1,E_2 \in \algebra{M}$. Then they satisfy $\carathCrit$~\ref{def:carath_crit}. Thus,
    \begin{align*}
	\carathEq{A_1}{E_1} \\
	\carathEq{A_2}{E_2} 
    \end{align*}
    for any $A_1,A_2 \subset X$. Pick $A_1 = A \subset X$ and $A_2 = A \cap E_1$ and so we get,
    \begin{align*}
	\carathEq{A}{E_1} \\
	\carathEq{A\cap\comp{E_1}}{E_2} 
    \end{align*}
    Substituting for $\outMeasure{\intersection{A}{\indxComp{E}{1}}}$ we get,
    \[\outMeasure{A} = \outMeasure{\intersection{A}{E_1}} +
	\outMeasure{\intersection{\intersection{A}{\comp{E_1}}}{E_2}} +
	\outMeasure{\intersection{A}{\comp{(\union{E_1}{E_2})}}}.\]
    Now, $\intersection{A}{(\union{E_1}{E_2})} =
    (\intersection{A}{E_1})\disjU(\intersection{\intersection{A}{\comp{E_1}}}{E_2})$ is easily seen
    as the disjoint union of two sets $X,Y$ where $X = A\cap E_1$, $Y = A \cap E_2$, and hence,
    \[\outMeasure{\intersection{A}{E_1}} +
    \outMeasure{\intersection{\intersection{A}{\comp{E_1}}}{E_2}} \geq 
    \outMeasure{\intersection{A}{(\union{E_1}{E_2})}}.\]
    Thus,
    \begin{align*} \carathGeq{A}{(\union{E_1}{E_2})}\end{align*}which from~\ref{prop:carath_crit} means that
    $\union{E_1}{E_2} \in \algebra{M}$. Thus we get finite additivity. Hence we have shown that
    $\algebra{M}$ is an algebra. 
    
    Showing that $\algebra{M}$ is closed under a countable union of an arbitrary sequence of sets is
    a little tricky. Instead we will show that $\algebra{M}$ is closed under countable union of a
    sequence of disjoint sets. Then $\algebra{M}$ will be a $\sigmaAlgebra$
    by~\ref{thm:eq_ch_sigmaA}. We will do this in 3 steps:
    Let $\seq{E}{i}$ be a sequence of pairwise \emph{disjoint} sets of $\algebra{M}$ and let
    $S_n = \finiteU{E}{n}$ and $S = \countU{E}$.
    
    \textbf{\large{Step 1}}:
    We will show that for any $n \geq 1$,
    \[\outMeasure{\intersection{A}{S_n}} = \sumFinite{\outMeasure{\intersection{A}{E_i}}}{n}.\]
    Note that since we proved $\algebra{M}$ is an algebra, $\algebra{M}$ is closed under finite
    unions and hence $S_n \in \algebra{M}$. We will use induction on $n$. The case $n=1$ is trivial.
    Suppose it is true for some $k > 1 \in \Zplus$, then since $S_{k} \in \algebra{M}$ we get, 
    \begin{align*}
	\carathEqIndx{A\cap S_{k+1}}{S}{k} \\
	&= \outMeasure{\intersection{A}{S_k}} + \outMeasure{\intersection{A}{E_{k+1}}} \\
	&= \sumFinite{\outMeasure{\intersection{A}{E_i}}}{k} + \outMeasure{\intersection{A}{E_{k+1}}} \\	
	&= \sumFinite{\outMeasure{\intersection{A}{E_i}}}{k+1}
    \end{align*}

    \textbf{\large{Step 2}}:
    We will show,
    \[\outMeasure{\intersection{A}{S}} = \sumInf{\outMeasure{\intersection{A}{E_i}}}.\]
    From sub-additivity of $\outMeas$,
    \[\outMeasure{\intersection{A}{S}} = \outMeasure{\countU{E}} \leq \sumInf{\outMeasure{E_i}}\] 
    To show the other direction,
    $S \supset S_n$ for any $n \in \Zplus$ and hence from monotonicity and \textbf{\large{Step 1}},
    \[\outMeasure{\intersection{A}{S}} \geq \outMeasure{\intersection{A}{S_n}} =
	\sumFinite{\measure{E_i}}{n}.\]
    Taking the limit as $\atob{n}{\infty}$ we get the desired result.

    \textbf{\large{Step 3}}:
    We will show,
    \begin{align*} \carathGeq{A}{S} \end{align*}
    Consider an $n \in \Zplus$ and since $S_n \in \algebra{M}$ we have,
    \begin{align*} \carathGeqIndx{A}{S}{n} \end{align*}
    But for any $n \in \Zplus$, $\indxComp{S}{n} \supset \comp{S}$ and thus using
    \textbf{\large{Step 1}} 
    \begin{align*}
	\outMeasure{A} \geq &\outMeasure{\intersection{A}{S_n}} + 
	\outMeasure{\intersection{A}{\comp{S}}} \\
	& = \sumFinite{\outMeasure{E_i}}{n} + \outMeasure{\intersection{A}{\comp{S}}}
    \end{align*}
    taking the limit as $\atob{n}{\infty}$ and using \textbf{\large{Step 2}} we get the result.
    

    Thus $\algebra{M}$ is a $\sigmaAlgebra$. It is easy to show that
    $\mu = \restrict{\outMeas}{\algebra{M}}$ is a measure. Indeed all we need to show is that $\mu$
    satisfies $\sigmaAdd$. 
    Let $\seq{E}{i}$ be a sequence of pairwise \emph{disjoint} sets of $\algebra{M}$ and let
    $S = \countU{E}$.
    From \textbf{\large{Step 2}}, 
    \[\outMeasure{\intersection{A}{S}} = \sumInf{\outMeasure{\intersection{A}{E_i}}}\] for any
    $A\subset X$. If we restrict $\outMeas$ to $\algebra{M}$, this means,
     \[\measure{\intersection{A}{S}} = \sumInf{\measure{\intersection{A}{E_i}}}\] for any
    $A\in\algebra{M}$. Since $S \in \algebra{M}$ take $A = S$ and thus
    \[\measure{\countDisjU{E}} = \sumInf{\measure{E_i}}.\]

    Also interersting to note is that $\measureS{X}{\algebra{M}}{\mu}$ is complete. To see this take
    any $E \in \algebra{M}$ such that $\measure{E} = 0$ and let $B \subset E$. Clearly $B \subset X$
    and thus, \[\outMeasure{B} \leq \outMeasure{E} = \measure{E} = 0.\]

    Hence, $\outMeasure{B} = 0$ which means $B \in \algebra{M}$ (as proved earlier).
\end{proof}
Thus we have seen that any outer measure can be restricted to a measure. Hence for this Theorem to be
useful we must have an outer measure. We noted that describing a measure for all members of a
$\sigmaAlgebra$ is no easy matter but now we have this outer measure on the maximal $\sigmaAlgebra$!
This is not a concern since the next theorem shows that we can easily construct an outer measure
from simple set functions defined on a simpler class. For example, lebesgue measure is defined on
intervals. At this point it will do us well to tabulate the properties of all the measures we
have seen. This is done in~\ref{tab:prop_measures}. 

\begin{table}
    \caption{Properties of measure, pre-measure and outer-measure}\label{tab:prop_measures}
    \begin{tabular}{llcccr}
	\toprule
	Properties  & Measure $\mu$ & Pre-Measure $\preMeas$ & Outer-Measure $\outMeas$ \\
	\midrule
	domain & $\sigmaAlgebra$ & algebra & $\powSet{X}$\\
	monotonicity & $\checkmark$ & $\checkmark$ & \checkmark\\
	finite-additvity & $\checkmark$ & $\checkmark$ & -\\
	finite-sub-additvity & $\checkmark$ & $\checkmark$ &$\checkmark$ \\
	countable-additvity & $\checkmark$ &-&-\\
	countable-sub-additvity & $\checkmark$ &-&$\checkmark$\\
	\bottomrule
    \end{tabular}
\end{table}

\begin{Example}\label{ex:out_meas}
    Let $X$ be an infinite set and define,
    \begin{equation*}
	\outMeasure{E} =
	\begin{cases}
	    \lvert E \rvert& \text{if $E$ is finite},\\
	    \infty& \text{if $E$ is infinite}.
	\end{cases}
    \end{equation*}
    Then $\outMeas$ is an outer measure.
\end{Example}
\begin{proof}
    Clearly $\map{\outMeas}{\powSet{X}}{\extRealsPos}$ and $\outMeasure{\emptyset} = 0$. Let us check
    monotonicity. If $A \subset B \subset X$, then if both are finite we have 
    $\lvert A \rvert \leq \lvert B \rvert$ and so $\outMeasure{A} \leq \outMeasure{B}$. If $B$ is
    infinite then its obvious.
    Consider a sequence $\seq{A}{i}$ of sets in $X$. If any of the $A_i$ is infinite then 
    $\outMeasure{\countU{A}} = \infty = \sumInf{\outMeasure{A_i}}$. Here we treat terms like $\infty +
    \infty$ to be equal to $\infty$. If all $A_i$ are finite then there are two cases. Either
    $\countU{A}$ is finite or it is infinite. (For expample, $\lvert A_i \rvert = i$, then
    $\countU{A}$ is infinite if $A_i$ are pairwise disjoint). 
    If $\countU{A}$ is finite then we know that $\lvert \countU{A} \rvert
    \leq \sumInf{\lvert A_i \rvert}$ and hence the result follows. If $\countU{A}$ is infinite then
    we can create a sequence of disjoint finte sets $F_k = \setDiff{A_k}{\finiteU{A}{k-1}}$. Then,
    $\countU{A} = \countDisjU{F}$. Thus 
    $\outMeasure{\countU{F}} = \sumInf{\outMeasure{F_i}} = \infty$ since we have finite disjoint
    sets, but $\sumInf{\outMeasure{F_i}}
    \leq \sumInf{\outMeasure{A_i}}$ and hence we get,
    $\outMeasure{\countU{A}} \leq \sumInf{\outMeasure{A_i}}$. All finite sets are $\outMeas$
    measurable.
\end{proof}

In~\ref{fig:tikz:construct_out_measure}, we illustrate a way of constructing outer measure. Say we
have a set $A_k$. If we want to define an outer measure we start with a sequence of 
simple sets $E^{k}_i$ whose union $\bigcup\limits_{i}E^{k}_{i} \supset A_k$. If we have a simple
set function (maybe a pre-measure) $\rho$ for these sets then we get an overestimated outer measure 
of $A$ given by 
$\sumInf{\rho(E^k_i)}$. The idea is now to take all those collections of sets whose union contain 
$A_k$ and then take the infimum of these overestimated outer measure. The infimum is then precisely
the outer measure of $A_k$.


\begin{Theorem}[name=Constructing outer measure]\label{thm:constr_out_measure}
    Let $X$ be a non-empty set, $\family{E}\subset\powSet{X}$ be non-empty family of subsets of $X$
    such that $\emptyset,X \in \family{E}$. Let there be a function
    $\map{\rho}{\family{E}}{\extRealsPos}$ such that $\rho(\emptyset) = 0$. For any $A \in
    \powSet{X}$, define
    \[\outMeasure{A} = \inf\set{\sumInf{\rho(E_i)}}{\lbrace E_i \rbrace \subset
	    \family{E}\,\text{and}\,A\subset\countU{E}},\]
    then, $\outMeasure{A}$ is an outer measure (induced by $\rho$).
\end{Theorem}

\begin{figure}
  \includestandalone[width=0.5\textwidth]{tex/tikz_figures/construct_out_measure}
  \caption{Construction of outer measure}\label{fig:tikz:construct_out_measure}
\end{figure}

\begin{proof}
    We will first check that the infimum exists. The set is certainly not empty since $X
    \in \family{E}$ and $A \subset X$; atleast one such $E_i = X$ exists. Morever we are taking
    infimum over positive reals that are bounded below by $0$ and hence the infimum exists.
    
    Lets check if $\outMeasure{\emptyset} = 0$. Since $\emptyset \in \family{E}$ it is trivally
    contained in a collection of empty sets. Since $\rho(\emptyset) = 0$ we get the result.

    Let us see if we get monotonicity. Let $A \subset B$. For any collection $\lbrace E_i \rbrace
    \subset \family{E}$ if $\countU{E} \supset {B}$ then $\countU{E} \supset {A}$. Thus the set,
    \begin{align*}
	&\set{\sumInf{\rho(E_i)}}{\lbrace E_i \rbrace \subset
	    \family{E}\,\text{and}\,A\subset\countU{E}}\\ 
	&\hspace{2cm}\supset \\
	&\set{\sumInf{\rho(E_i)}}{\lbrace E_i \rbrace \subset
	    \family{E}\,\text{and}\,B\subset\countU{E}},
    \end{align*}
    hence the result follows.

    Let $\seq{A}{k}$ be a sequence of sets in $X$ such that for each $k$ there is a sequence of
    sets $\left(E^k_i\right)$ such that,
    \begin{align*}
	&A_k \subset \bigcup\limits_{i}E^k_i \, \text{and,} \\
	&\sumInf{\rho(E^k_i)} \leq \outMeasure{A_k} + \epsilon,
    \end{align*}	
    for any $\epsilon > 0$. Pick $\epsilon = \frac{1}{2^k}$.
    Now,
    \[\bigcup\limits_{k}A_k \subset\bigcup\limits_{i,k}{E^k_i}.\]
    Let $A = \bigcup\limits_{k}A_k$, thus $A \subset \bigcup\limits_{i,k}E^i_k$, and so by definition
    \begin{align*}
	\outMeasure{A} &\leq \sum\limits_{i,k=1}^{\infty}\rho(E^i_k) \\
	&\leq \sum\limits_{k=1}^{\infty}(\outMeasure{A_k} + \frac{1}{2^k})\\
	&\leq \sum\limits_{k=1}^{\infty}\outMeasure{A_k}.
    \end{align*}
    Thus $\outMeas$ is an outer measure.
\end{proof}
Note, at this point there is no relation between $\rho$ and the outer measure induced by $\rho$.
Very little demands are made on $\rho$ and its domain $\family{E}$. An interesting question to ask
would be{\textemdash}what if we had some additional structure on $\family{E}$, particularly what if 
$\family{E}$
were an algebra? In that case, if additionaly $\rho = \preMeas$ is a pre-measure, we have a relation
between $\preMeas$ and the outer measure $\outMeas$ induced by $\preMeas$. The following Proposition
showcases this relation.
\begin{Proposition}\label{prop:out_meas_iduced_pre_measure}
    If $\preMeas$ is a pre measure on $\algebra{E}$ and the outer measure induced by $\preMeas$ is
    given as in~\ref{thm:constr_out_measure}, then
    \begin{enumerate}
	\item
	    $\restrict{\outMeas}{\algebra{E}} = \preMeas$.
	\item
	    Every set in $\algebra{E}$ is $\outMeasurable$.
    \end{enumerate}
\end{Proposition}
\begin{proof}
    We prove in order.
    \begin{enumerate}
	\item
	    We need to show that $\restrict{\outMeas}{\algebra{E}} \leq \preMeas$ and 
	    $\preMeas \leq \restrict{\outMeas}{\algebra{E}}$. Let $A \in \family{E}$.
	    Clearly, $A \subset A$, consider the collection $\lbrace E_i \rbrace$ where $E_1 = A$
	    and $E_i = \emptyset, i \geq 2$. Hence 
	    $\restrict{\outMeas}{\algebra{E}}(A) \leq \preMeas(A)$. 
	    To prove the other inequality, for any
	    $\epsilon > 0$ there is a collection $\lbrace E_i \rbrace \subset \family{E}$ 
	    such that $A \subset\countU{E}$ and, 
	    \[\sumInf{\preMeas(E_i)} < \restrict{\outMeas}{\algebra{E}}(A) + \epsilon.\]
	    Since $\preMeas$ is a pre-measure and $A \in \family{E}$, 
	    if $\countU{E} \in \algebra{E}$ we get from sub-additivity of
	    $\preMeas$, 
	    \[\preMeas(A) \leq \sumInf{\preMeas(E_i)}.\]
	    However, it may happen that $\countU{E} \not \in \algebra{E}$. 
	    In that case, observe that, 
	    \[A = \intersection{A}{\countU{E}} = \countU{A\cap E}.\]
	    We will write union of $A \cap E_i$ as a disjoint union using~\ref{rmk:disjU}.
	    Let $X_j = A\cap E_j $,
	    $F_j = X_j - \bigcup\limits_{i=1}^{j}X_i, j \geq 2, F_1 = X_1$.
	    Each $F_j \in \algebra{E}$, moreover $\disjU\limits_{j}F_j = A \in \algebra{E}$, thus
	    \[\preMeas(A) = \sumInf{\preMeas(F_i)} \leq \sumInf{\preMeas(E_i)}.\]
	    Thus,
	    \[\preMeas(A) \leq \sumInf{\preMeas(E_i)}.\]
	    Hence,
	    \[\preMeas(A) < \restrict{\outMeas}{\algebra{E}}(A) + \epsilon,\]
	    since $\epsilon$ is arbitrary we get the result.
	\item
	    Let $E \in \family{E}$. To show that $E$ is $\outMeasurable$ we have to show,
	    \begin{align*}
		\carathGeq{A}{E}
	    \end{align*}
	    for any $A \subset X$.
	    Fix an $\epsilon > 0$, then there is a collection $\lbrace B_i \rbrace \subset
	    \algebra{E}$ such
	    that $\countUnion{B_i}{i} \supset A$ and, 
	    \[\outMeasure{A} + \epsilon > \sumInf{\preMeas(B_i)}.\]
	    Since $E \in \algebra{E}$, we can write 
	    $B_i = (\intersection{B_i}{\comp{E}})\disjU (B_i \cap E)$ as a disjoint union of sets
	    in $\family{E}$. 
	    Using the fact that $\preMeas$ is a pre-measure, we get 
	    \[\preMeas(B_i) = \preMeas(\intersection{B_i}{\comp{E}}) + \preMeas(B_i \cap E).\]
	    Thus,
	    \begin{align*}
		\outMeasure{A} + \epsilon >& \sumInf{\preMeas(B_i \cap E)} + 
		\sumInf{\preMeas(\intersection{B_i}{\comp{E}})} \\
	       & \geq  \outMeasure{A\cap E} + \outMeasure{A\cap\comp{E}}.
	   \end{align*}
	   Since $\epsilon$ was arbitrary we get the result.
    \end{enumerate}
\end{proof}
Now we are ready to prove the extension theorem i.e given an algebra $\algebra{E}$ on a set $X$,
and a pre-measure $\preMeas$ we can extend
the pre-measure to a measure $\mu$ on a $\sigmaAlgebra$ containing $\algebra{E}$.
Such a construction can be achieved in two step:
\begin{enumerate}
    \item
	First extend the pre-measure $\preMeas$ on $\algebra{E}$ to an outer measure $\outMeas$ on
	$\powSet{X}$. See~\ref{thm:constr_out_measure}
	and~\ref{prop:out_meas_iduced_pre_measure}.
    \item
	Then restrict the outer measure $\outMeas$ to a $\sigmaAlgebra$ $\algebra{M}$.
	See~\ref{thm:carath_restr_thm}.
\end{enumerate}
This is an important result, since we start with simple set functions like measuring length of 
interval etc.\ and want to extend this notion to a larger class of sets. The following theorem
combines these observations and is called the Carath{\'e}odory Extension theorem.
\begin{Theorem}[name=Carath{\'e}odory Extension Theorem]\label{thm:carath_ext_thm}
    Let $X$ be a non-empty set and let $\algebra{E} \subset \powSet{X}$ be an algebra, $\preMeas$ a
    pre-measure on $\algebra{E}$ and $\sigmaGen{\algebra{E}}$ be the $\sigmaAlgebra$ generated by
    $\algebra{E}$. Then,
    \begin{enumerate}
	\item
	    There exists a measure $\mu$ on $\sigmaGen{\algebra{E}}$ whose restriction to
	    $\algebra{E}$ is $\preMeas$.
	\item
	    If $\nu$ is another measure on $\sigmaGen{\algebra{E}}$ that extends $\preMeas$, then
	    $\nu(E) \leq \measure{E}$ for any $E \in \sigmaGen{\algebra{E}}$, with equality when
	    $\measure{E} < \infty$. 
	\item
	    Additionally, if $\preMeas$ is $\sigmaFinite$, then $\mu$ is the unique extension of
	    $\preMeas$ to a measure on $\sigmaGen{\algebra{E}}$.
    \end{enumerate}
\end{Theorem}
\begin{proof}
    Given non-empty set $X$, an algebra $\algebra{E}$ with a pre-measure $\preMeas$,
    \begin{enumerate}
	\item
	    first extend $\preMeas$ to
	    $\outMeas$,~\ref{thm:constr_out_measure},~\ref{prop:out_meas_iduced_pre_measure}. Then
	    restrict $\outMeas$ to $\mu$ on $\sigmaAlgebra$ 
	    $\algebra{M}$,~\ref{thm:carath_restr_thm}. Since
	    $\algebra{M}$ contains $\algebra{E}$, it contains $\sigmaGen{\algebra{M}}$. Thus there is
	    a measure $\mu$ on $\sigmaGen{\algebra{E}}$ such that,
	    \[\mu = \restrict{\outMeas}{\sigmaGen{\algebra{E}}},\] and 
	    \[\restrict{\mu}{\algebra{E}} = \restrict{\outMeas}{\algebra{E}} = \preMeas.\]
	\item
	    Note that, \[\restrict{\nu}{\algebra{E}} = \preMeas = \restrict{\mu}{\algebra{E}}.\] 
	    Thus for any $A \in \algebra{E}$ we have $\nu(A) = \preMeas(A) = \measure{A}$. However,
	    if $\lbrace A_i \rbrace $ is a collection of sets $A_i \in \algebra{E}$, $\countU{A}$ may
	    or may not belong in $\algebra{E}$. In that case, it is not clear that if, 
	    $A = \countU{A}$, then $\nu(A) = \measure{A}$. But for any integer $n$,
	    \[\nu(\finiteU{A}{n}) = \mu(\finiteU{A}{n}).\]
	    To this end observe that,
	    $A_1 \subset (A_1\cup A_2) \subset \dots \subset \countU{A}$. 
	    Thus we have an increasing sequence of sets $\seq{(\finiteU{A}{n})}{n}$ and 
	    since $\nu,\mu$ are
	    measures we get,
	    \[\nu(A) = \lim\limits_{n\to\infty}\nu(\finiteU{A}{n}) =
		\lim\limits_{n\to\infty}\measure{\finiteU{A}{n}} = \measure{A}.\]
	    
	    Here we have used~\ref{thm:prop_of_meas} (4) with $E_i =
	    (\bigcup\limits_{j=1}^{i}A_j)$.

	    Now, consider any $E \in \sigmaGen{\algebra{E}}$ such that a collection $\lbrace A_i
	    \rbrace \subset \algebra{E}$ and $E \subset \countU{A}$. Then,
	    \[\nu(E) \leq \nu(\countU{A}) \leq \sumInf{\nu(A_i)} = \sumInf{\preMeas(A_i)}.\]
	    But this means that,
	    \[\nu(E) \leq \inf\set{\sumInf{\preMeas(A_i)}}{E\subset\countU{A}} = \measure{E},\]
	    since $E \in \sigmaGen{\algebra{E}}$.

	    For any $\epsilon > 0$, let $A = \countU{A}$ such that the collection $\lbrace A_i
	    \rbrace \subset \algebra{E}$ and $E \subset \countU{A}$ and,
	    $\sumInf{\preMeas(A_i)} < \measure{E} + \epsilon$. But $\preMeas(A_i) =
	    \measure{A_i}$ and $\measure{A} \leq \sumInf{\measure{A_i}}$ and hence,
	    $\measure{A} < \measure{E} + \epsilon$.
	    Now if $\measure{E} < \infty$, since $E \subset A$, $\measure{\setDiff{A}{E}} =
	    \measure{A} - \measure{E} < \epsilon$. Thus,
	    \begin{align*}
		\mu(E) \leq &\mu(A) &&\text{because $E \subset A$}\\
		& = \nu(A) &&\text{we proved this above}\\
	        & = \nu(E) + \nu(A\cap\comp{E})	&&\text{because $A = E \disjU A\cap\comp{E}$} \\
		& \leq \nu(E) + \mu(A\cap\comp{E}) &&\text{because $\nu(B) \leq \mu(B), \forall B
		    \in \sigmaGen{\algebra{E}}$} \\
		& \leq \nu(E) + \epsilon
	    \end{align*}	
	    Hence, $\measure{E} = \nu(E)$ whenever $\measure{E} < \infty$.
	\item
	    If $\preMeas$ is $\sigmaFinite$ then $X = \countU{A}$ with $\preMeas(A_i) < \infty$.
	    We can take $A_i$ to be
	    disjoint. (If not we can contruct sequence of disjoint set $F_i$ whose union is $X$).
	    Take any $E \in \sigmaGen{\algebra{E}}$ such that $E \subset \countDisjU{A}$ and hence $E =
	    \disjU\limits_{i}(E\cap A_i)$. Hence,
	    \[\measure{E} = \sumInf{\measure{E\cap A_i}} = \sumInf{\nu(E\cap A_i)} = \nu(E).\]
	    Note that the third equality is beacause $\measure{E\cap A_i} \leq \measure{A_i} =
	    \preMeas(A_i) < \infty$ and we can use the results from above. Hence $\nu = \mu$.
    \end{enumerate}
\end{proof}
\begin{Remark}[name=Relation between $\algebra{M}$ and $\sigmaGen{E}$]\label{rmk:rel_m_sigmaE}
    This is illustrated in~\ref{fig:tikz:carath_ext_thm}. 
\end{Remark}
\begin{figure}
  \includestandalone[width=0.5\textwidth]{tex/tikz_figures/carath_ext_thm}
  \caption{Embedding of the various sets. The measures must agree on the \emph{edges}.}
\label{fig:tikz:carath_ext_thm}
\end{figure}
\begin{Example}
    Show the usefulness of~\ref{prop:out_meas_iduced_pre_measure}.
\end{Example}
\section{Borel Measure on the real line}
We have now the machinary to measure all the Borel sets in $\R$. These are the sets that we care
about. However we only have a notion of measure for simple sets like intervals,
\textbf{h-intervals}, etc. Using such elementary notions we want to construct a measure on
$\borelS{\R}$. We will begin with a general construction that will yield us the Cumulative
Distribution Function (CDF).

To motive the idea, let us say we have a \emph{finite} measure $\mu$ on $\borelS{\R}$. Thus we can measure the
\text{h-intervals} like $\interval[open left]{-\infty}{x}$. Let $F(x) = 
\measure{\interval[open left]{-\infty}{x}}$. Then we can observe a few facts about $F$,
\begin{enumerate}
    \item
	(positive real valued) The function $F$ is such that $\map{F}{\R}{\extRealsPos}$.
    \item
	(increasing) If $x_1 \leq x_2$ then $\measure{\interval[open left]{-\infty}{x_1}} \leq 
	\measure{\interval[open left]{-\infty}{x_2}}$ from monotonicity of $\mu$. Thus $F(x_1) \leq
	F(x_2)$.
    \item
	(right continuous) If $\atobDown{x_n}{x}$ then $\interval[open left]{-\infty}{x} =
	\bigcap\limits_{n=1}^{\infty}\interval[open left]{-\infty}{x_n}$. Thus, 
	$\measure{\interval[open left]{-\infty}{x}} = 
	\lim\limits_{n\to\infty}\measure{\interval[open left]{-\infty}{x_n}}$, i.e.\, 
	$\atobDown{F(x_n)}{F(x)}$ as $\atobDown{x_n}{x}$.
    \item
	If $y > x$, then $\hIntInf{y} = \hIntInf{x} \disjU \hInt{x}{y}$ and hence,
	$\measure{\hInt{x}{y}} = F(y) - F(x)$.
\end{enumerate}
Now we will turn this around and create a measure on $\borelS{\R}$ using function $F$ with 
properties as seen above. Such functions are special and we define them below.
\begin{Definition}[name=Monotonic functions]
    A real valued function $F$ is \emph{increasing} if $F(x) \leq F(y)$ whenever $x < y$ for all $x,y \in
    \R$. A real valued function $F$ is \emph{decreasing} if $F(x) \geq F(y)$ whenever $x < y $ for
    all $x,y \in \R$. A real valued function $F$ is \emph{monotonic} if it is either increasing or
    decreasing.
\end{Definition}
\begin{Theorem}[name=Monotonic functions and one-sided limits]\label{thm:mono_func}
    If $\map{F}{\R}{\R}$ is a monotone function, then $F$ has right and left-hand limits at each
    point $x \in \R$,
    \begin{align*}
	&F(a^{+}) = \lim\limits_{\atobDown{x}{a}}F(x) =  
	\begin{cases}
	    \inf\limits_{x > a}F(x)& \text{If $F$ is increasing}\\
	    \sup\limits_{x < a}F(x)& \text{If $F$ is decreasing}
	\end{cases} \\
	&F(a^{-}) = \lim\limits_{\atobUp{x}{a}}F(x) = 
	\begin{cases}
	    \inf\limits_{x > a}F(x)& \text{If $F$ is increasing}\\
	    \sup\limits_{x < a}F(x)& \text{If $F$ is decreasing}
	\end{cases}
    \end{align*}
\end{Theorem}
\begin{Definition}[name=Right continuous function]
    A monotone function \[\map{f}{\R}{\R}\] is \emph{right continuous} if $F(a) = F(a^{+})$ for all
    $a\in\R$, thus,
    \[\lim\limits_{\atobDown{x}{a}}F(x) = F(a).\]
\end{Definition}
\begin{figure}
    \includestandalone[width=0.45\textwidth]{tex/tikz_figures/right_continuous}
    \caption{A right continuous increasing function}\label{fig:tikz:right_continuous}
\end{figure}
A right continuous function is show in~\ref{fig:tikz:right_continuous}.
Now we begin our construction of measure on $\borelS{\R}$. We have seen that the collection of
h-intervals is an elementary family (~\ref{ex:hint_elem_fam}). 
A collection of finite disjoint union of h-intervals is, then, an
algebra by~\ref{thm:const_algebra_elem}. If we can define a pre-measure on this algebra, then
by~\ref{thm:carath_ext_thm}, we will have a measure on $\borelS{\R}$ (~\ref{rmk:hinterval}). 
Thus our first step will be
constructing a pre-measure.

\begin{Proposition}[name=Pre-measure on collection of
    \textbf{h-intervals}]\label{prop:pre_meas_hint}
    Let $\family{J}^{cr}_{1}$ be the collection of \textbf{h-intervals} on $\R$ and let
    $\algebra{A}$ be the algebra of finite disjoint unions of h-intervals. Let
    $\map{F}{\R}{\R}$ be increasing and right continuous. If $\hInt{a_j}{b_j}$ ($j = 1\dots n$) 
    are disjoint h-intervals, define 
    \[\preMeasure{\finiteUnion{\hInt{a_j}{b_j}}{j}{n}} = \finiteSum{(F(b_j)-F(a_j))}{j}{n}.\] 
    Then, $\preMeas$ is a pre-measure.
\end{Proposition}
\begin{proof}
    First note that if $n=1$ and if $F(x) = x$, then the $\preMeas$ gives us the \emph{length} of
    an h-interval. To show that $\preMeas$ is a pre-measure on the algebra $\algebra{A}$, we need to
    show that $\preMeasure{\emptyset} = 0$ and $\preMeasure{\finiteDisjUnion{I}{i}{n}} =
    \finiteSum{\preMeasure{I_i}}{i}{n}$. Moreover, if $\countDisjUnion{I}{i} \in \algebra{A}$, then we need to
    show that $\preMeasure{\countDisjUnion{I_i}{i}} = \infiniteSum{\preMeasure{I_i}}{i}$. Note that
    $\emptyset = \hInt{b}{b}$ and thus $\preMeasure{\emptyset} = 0$. 


    We will complete the proof in following steps:
    \newline
    \textbf{Step 1}: Let us check finite additivity.
    Let $I \in \algebra{A}$. Then $I = \finiteDisjUnion{I_i}{i}{n}$ where $I_i = \hInt{a_i}{b_i}$. Note
    that $I_i \in \algebra{A}$ and hence, $\preMeasure{I_i} = F(b_i) - F(a_i)$. Thus,
    \[\preMeasure{I} = \finiteSum{(F(b_i)-F(a_i))}{i}{n} 
	= \finiteSum{\preMeasure{I_i}}{i}{n}.\]
    Hence, $\preMeas$ satisfies finite additivity.
    \\
    \textbf{Step 2}: We will show that $\preMeas$ is well defined.
    First let us look at a simple case. Let $I \in \algebra{A}$, 
    $I = \finiteDisjUnion{\hInt{a_i}{b_i}}{i}{n_I}$. Now assume $I = \hInt{a}{b}$. 
    If we re-label $a_i,b_i$, that is $a=a_1 < b_1 = a_2 < b_2 = a_3 \dots < b$,
    then $\preMeasure{I} = F(b) - F(a)$ from cancellation. 
    
    Note, however, such a representation is not unique. We could have a $J\in \algebra{A}$ such that
    $J = \finiteDisjUnion{\hInt{a_j}{b_j}}{j}{n_J} = \hInt{a}{b}$. But $\preMeasure{J}$
    would also yield $F(b) - F(a)$ by the same argument. See~\ref{fig:tikz:borel_pre_meas}. 

    Now, for a general case. If $I = \finiteDisjUnion{I_i}{i}{n_I} = \finiteDisjUnion{J_j}{j}{n_J}$, 
    then $I_i \subset \finiteDisjUnion{J_j}{j}{n_J}$ and thus 
    $I_i = \finiteDisjUnion{(I_i \cap J_j)}{j}{n_J}$. But $I_i \in \algebra{A}$, $I_i \cap J_j$ 
    is an h-interval and hence $\preMeasure{I_i} = \finiteSum{\preMeasure{I_i \cap J_j}}{j}{n_J}$.
    We have,
    $\preMeasure{I} = \finiteSum{\preMeasure{I_i}}{i}{n_I}$ and hence, 
    $\preMeasure{I} = \sum\limits_{i=1}^{n_I}\finiteSum{I_i \cap J_j}{j}{n_J}$. We can repeat this
    argument by noticing that $J_j \subset \finiteDisjUnion{I_i}{i}{n_I}$. Hence, we get
    \[\preMeasure{I} = \sum\limits_{i=1}^{n_I}\finiteSum{I_i \cap J_j}{j}{n_J} = \preMeasure{J}.\]
    Thus $\preMeas$ is well defined.
    \newline
    \textbf{Step 3}: We will show that if $I = \countDisjUnion{I_i}{i} \in \algebra{A}$, then
    $\preMeasure{I} = \preMeasure{\countDisjUnion{I_i}{i}} \geq \infiniteSum{\preMeasure{I_i}}{i}$.

    If $I = \countDisjUnion{I_i}{i} \in \algebra{A}$ then 
    $I = \finiteDisjUnion{\hInt{a_k}{b_k}}{k}{N}$. Since each $I_i$'s and each $\hInt{a_k}{b_k}$ 
    are disjoint we can partition the sequence finitely, i.e for each $1\leq k \leq N$ pick $n$ in 
    the sequence such that
    $I_{n}\subset \hInt{a_k}{b_k}$. The set $n_k = \set{n \in \Zplus}{I_n \subset \hInt{a_k}{b_k}}$
    indexs a subsequence of $I_i$. Hence there is atleast one such $n_k$, lets call it
    $n_0$ such that $n_0$ is infinite, otherwise we will have only a finite sequence of $I_i$.
    Without loss of generality we can disregard other $\hInt{a_k}{b_k}$ and just consider $I =
    \hInt{a_{n_0}}{b_{n_0}} = \hInt{a}{b}$ (after dropping the $n_0$).
    
    Note that for $n \in \Zplus$, we have from finite additivity,
    \[\preMeasure{I} = \finiteSum{\preMeasure{I_i}}{i}{n} + 
	\preMeasure{\disjU\limits_{i>n}I_i}.\]
    Since $F$ is increasing the last term is always greater than zero and thus,
    \[\preMeasure{I} \geq \finiteSum{\preMeasure{I_i}}{i}{n}.\]
    This is true for any $n \in \Zplus$ and thus taking $n \to \infty$, we get
    \[\preMeasure{I} \geq \infiniteSum{\preMeasure{I_i}}{i}.\]
    \newline
    \textbf{Step 4}: We will show that 
    $\preMeasure{I} \leq \infiniteSum{\preMeasure{I_i}}{i}$.
    Pick any $\epsilon > 0$. Note that $I = \hInt{a}{b}$. 
    Now assume $-\infty < a < b < \infty$. Since $F$ is right continuous at $a$,
    there is a $\delta > 0$ such that $F(a+\delta) - F(a) < \epsilon$, but this means that
    $\preMeasure{\hInt{a}{a+\delta}} < \epsilon$. Now, $I = \hInt{a}{a+\delta} \disjU 
    \hInt{a + \delta}{b}$ and since $I \in \algebra{A}$, we get
    \begin{align*}
	\preMeasure{I} =& \preMeasure{\hInt{a}{a+\delta}} + \preMeasure{\hInt{a+\delta}{b}} \\
	& < \epsilon + \preMeasure{\hInt{a+\delta}{b}} 
    \end{align*}
    Now that we have the right inequality we need to worry about $\preMeasure{\hInt{a+\delta}{b}}$.
    We will use a compactness argument.
    Note that $\interval{a+\delta}{b}$ is closed and bounded subset of $\R$ (from our assumption
    about a,b) and is thus compact. Let us use $I_i$'s to create an open cover. However each $I_i$ is
    an h-interval. We will use the right continuity of $F$ to create open intervals from each $I_i$.
    Since $F$ is right continuous at each $b_i$, there is a $\delta_{i} > 0$ such that $F(b_i +
    \delta_{i}) - F(b_i) < \frac{\epsilon}{2^i}$. Thus, 
    $F(b_i+\delta_{i}) < F(b_i) + \frac{\epsilon}{2^i}$.  Let 
    \[\family{G} = \set{(a_i,b_i + \delta_{i})}{i\in\Zplus}.\] be a family of open sets. Then,
    $\interval{a+\delta}{b} \subset \bigcup\limits_{i\in\Zplus}(a_i,b_i + \delta_{i})$.
    Thus $\family{G}$ is an open cover for the compact $\interval{a+\delta}{b}$, and thus there
    exists a finite subcover,
    \[\family{G}_{N} = \set{(a_i,b_i+\delta_{i})}{1\leq i \leq N}.\]
    (We discard any such open set that is contained in a larger open set). After relabeling, we can
    assume that,
    \[b_i + \delta_{i} \in (a_{i+1},b_{i+1}+\delta_{i+1}), 1\leq i\leq N. \] 
    See~\ref{fig:tikz:borel_pre_meas2}. Thus $F(b_i + \delta_i) \geq F(a_{i+1})$. 
    Note that $b < b_N + \delta_{N}$ and $a + \delta > a_1$, and thus since $F$
    is increasing $F(b) \leq F(b_N + \delta_{N})$ and $F(a+\delta) \geq F(a_1)$. 
    Thus,
    \begin{align*}
	\preMeasure{I} <& \epsilon + \preMeasure{\hInt{a+\delta}{b}} \\
	&= \epsilon + F(b) - F(a + \delta) \\
	&\leq \epsilon  + F(b_N + \delta_N) - F(a_1)  \\
	&= \epsilon  + F(b_N + \delta_N) - F(a_N) + \finiteSum{(F(a_{i+1}) - F(a_i))}{i}{N-1} \\
	&\leq \epsilon + F(b_N + \delta_N) - F(a_N) +
       	\finiteSum{(F(b_{i}+ \delta_{i}) - F(a_i))}{i}{N-1}\\
	&< \epsilon + \finiteSum{(F(b_i)+\frac{\epsilon}{2^i}-F(a_i))}{i}{N} \\
	&= \epsilon + \finiteSum{(\preMeasure{I_i} +\frac{\epsilon}{2^i})}{i}{N} \\
	&< \infiniteSum{(\preMeasure{I_i})}{i} + 2\epsilon.
    \end{align*}
    Since $\epsilon$ was arbitrary we get our result.
\end{proof}

\begin{figure}
    \includestandalone[width=0.45\textwidth]{tex/tikz_figures/borel_premeasure}
    \caption{Illustration of proof~\ref{prop:pre_meas_hint}-well defined}\label{fig:tikz:borel_pre_meas}
\end{figure}

\begin{figure}
    \includestandalone[width=0.65\textwidth]{tex/tikz_figures/borel_premeasure2}
    \caption{Illustration of proof~\ref{prop:pre_meas_hint}-finite sub-cover}\label{fig:tikz:borel_pre_meas2}
\end{figure}

Now we apply the framework for constructing measures for the Borel Sets.

\begin{Theorem}[name=Borel measure on the real line]\label{thm:borel_measure_R}
    If $\map{F}{\R}{\R}$ is any increasing, right continuous function, there is a unique Borel
    measure $\borelM{F}$ on $\R$ such that $\borelMeasure{F}{\hInt{a}{b}} = F(b) - F(a)$ for all
    $a,b \in \R$. If $G$ is another such function, we have $\borelM{F} = \borelM{G}$ iff $F-G$ is a
    constant. Conversely, if $\mu$ is a Borel measure on $\R$ that is finite on all bounded Borel
    sets and we define,
    \begin{equation*}
	F(x) =
	\begin{cases}
	    \measure{\hInt{0}{x}} &\text{if $x > 0$}\\
	    0 &\text{if $x = 0$} \\
	    -\measure{\hInt{-x}{0}} &\text{if $x < 0$},
	\end{cases}
    \end{equation*}	
    then $F$ is increasing and right continuous, and $\mu = \borelM{F}$.
\end{Theorem}
\begin{proof}
    The~\ref{prop:pre_meas_hint} implies that $F$ induces a pre measure $\mu_0$ 
    on the algebra $\algebra{A}$ of finite disjoint unions of h intervals. Thus, it can be 
    extended by~\ref{thm:carath_ext_thm} to a measure $\borelM{F}$ on $\sigmaAlgebra$ 
    generated by $\algebra{A}$ which is $\borelS{\R}$. 

    When $F,G$ differ by a constant they give rise to the same pre measure $\mu_0$ on $\algebra{A}$.
    Also $\mu_0$ is $\sigmaFinite$ since,
    \[\R = \bigcup\limits_{i = -\infty}^{\infty}\hInt{i}{i+1},\]
    hence by~\ref{thm:carath_ext_thm} they are equal on $\borelS{\R}$.

    As for the converse, we follow the same argument as the observations noted at the beginning of
    this section. The monotonicity of $\mu$ makes $F$ increasing. Since $\mu$ is continuous from
    above and below, we get the right continuity of $F$. Moreover $\mu\hInt{a}{b} = F(b) - F(a) =
    \borelM{F}$ on $\algebra{A}$ and thus $\mu = \borelM{F}$ by~\ref{thm:carath_ext_thm}.

\end{proof}
\begin{Remark}
    From~\ref{rmk:rel_m_sigmaE}, we know that $\borelS{R}$ is not necessarily complete. We can
    complete it to yield the set of all $\outMeasurable$ sets. This completion \emph{expands} the
    Borel sets and yields what is called the \emph{Lebesgue} measurable sets. We give a precise
    definition and few observations in the subsection below. 
\end{Remark}

\subsection{The Lebesgue-Stieltjes measure in $\R$}
\break{}

\begin{Definition}[name=Lebesgue-Stieltjes measure]
    If $\map{F}{\R}{\R}$ is increasing and right continuous, the \emph{complete} measure of
    $\borelM{F}$ denoted by $\closure{\borelM{F}}$ is called the Lebesgue-Stieltjes (L-S) measure. We
    usually drop the overbar and denote it by the same $\borelM{F}$. The L-S measure associated by
    the function $F(x) = x$ is called the \emph{Lebesgue} measure and is denoted by $\borelM{x} =
    \borelM{\algebra{L}} = \mu$. We denote its domain by $\algebra{L}$, and call it the class of
    \textbf{Lebesgue measurable sets}. 
\end{Definition}

We make some observations about the Lebesgue and Lebesgue-Stieltjes measures. We fix a monotone
increasing, right continuous function $\map{F}{\R}{\R}$, the associated \emph{complete} measure $\mu
= \borelM{F}$, and the domain $\algebra{M}$ of $\mu$. When $F(x) = x$, we denote $\algebra{M}$ as
$\algebra{L}$. The measure space is thus the triple $\measureS{\R}{\algebra{M}}{\mu}$. For any
$E\subset\algebra{M}$, $\mu$ is defined as,
\begin{equation}\label{eq:ls_measure}
    \measure{E} =
    \inf\set{\infiniteSum{\measure{\hInt{a_i}{b_i}}}{i}}{E\subset\countUnion{\hInt{a_i}{b_i}}{i}},
\end{equation}
where $\measure{\hInt{a}{b}} = F(b) - F(a)$.

\begin{Theorem}[name=Equivalent characterizations of L-S measures]\label{thm:equiv_lebesgue_meas}
    If $\measureS{\R}{\algebra{M}}{\mu}$ is a L-S measurable space then for any $E \in \algebra{M}$,
    \begin{itemize}
	\item (Measure through open intervals)
	    \begin{equation}\label{eq:ls_measure_open_interval}
		\measure{E} =
		\inf\set{\infiniteSum{\measure{(a_i,b_i)}}{i}}{E\subset\countUnion{(a_i,b_i)}{i}},
	    \end{equation}
	\item (Measure through open sets)
	    \begin{equation}\label{eq:ls_measure_open_set}
		\measure{E} =
		\inf\set{\measure{G}}{E\subset G,\,\text{G is open}},
	    \end{equation}
	\item (Measure through compact sets)
	    \begin{equation}\label{eq:ls_measure_compact_set}
		\measure{E} =
		\sup\set{\measure{K}}{E\supset K,\,\text{K is compact}}.
	    \end{equation}
    \end{itemize}
\end{Theorem}
\begin{proof}
    Let us first prove~\ref{eq:ls_measure_open_interval}. Given a collection of open intervals
    $(a_i,b_i)$, such that $E \subset \countUnion{(a_i,b_i)}{i}$, we construct a sequence of
    h-intervals,
    \[(a_i,b_i) = \countDisjUnion{\hInt{a_i}{b_i - 1/k}}{k} = \countDisjUnion{I_i^k}{k},\] 
    where, $I_i^k = \hInt{a_i}{b_i - 1/k}$. Therefore,
    \[E \subset \bigcup\limits_{i,k}^{\infty}I_i^k,\] and hence from~\ref{eq:ls_measure} we get,
    \[\measure{E} \leq \sum\limits_{i,k = 1}^{\infty}\measure{I_i^k}.\]
     From our construction we see, since each $\hInt{a_i}{b_i-1/k}$ are disjoint,
     (to be formal, note that $I_i^1 \subset I_i^1 \disjU I_i^2 \dots$, thus
     $\measure{\countUnion{I_i^k}{k}} =
     \lim\limits_{n\to\infty}\measure{\finiteDisjUnion{I_i^k}{k}{n}}$),
    \begin{align*}
	\measure{(a_i,b_i)} &= \lim\limits_{n\to\infty}\finiteSum{\measure{\hInt{a_i}{b_i -
		    1/k}}}{k}{n} \\
	&= \infiniteSum{\measure{\hInt{a_i}{b_i - 1/k}}}{k}
    \end{align*}
    And thus,
    \[\sum\limits_{i,k = 1}^{\infty}\measure{I_i^k} = \infiniteSum{\measure{(a_i,b_i)}}{i}\]
    Hence,
    \[ \measure{E} \leq \infiniteSum{\measure{(a_i,b_i)}}{i}.\]
    To get the other inequality, fix an $\epsilon > 0$. Hence there is a sequence of $I_i =
    \hInt{a_i}{b_i}$ such that $E \subset \countUnion{I_i}{i}$ and
    \[\infiniteSum{\measure{I_i}}{i} < \measure{E} + \epsilon.\] 
     Since, $F$ is right continuous at every $b_i$, there is $\delta_i > 0$ such that
     $F(b+\delta_i)-F(b) < \frac{1}{2^i}$. Now, 
     $(a_i,b_i + \delta_i) \subset \hInt{a_i}{b_i + \delta_i}$, thus 
     $\measure{(a_i,b_i+\delta_i)} \leq \measure{\hInt{a_i}{b_i+\delta_i}}$. We have,
     \begin{align*}
	 \measure{\hInt{a_i}{b_i+\delta_i}} &= F(b_i+\delta_i) - F(a_i) \\
	 &= F(b_i + \delta_i) - F(b_i) + F(b_i) - F(a_i) \\
	 &< \frac{1}{2^i} + \measure{\hInt{a_i}{b_i}}. \\
	 &\quad = \frac{1}{2^i} + \measure{I_i}
     \end{align*}
     Hence,
     \[\infiniteSum{\measure{(a_i,b_i)}}{i} \leq \infiniteSum{\measure{I_i}}{i} <
	 \measure{E} + \epsilon. \]
     Since $\epsilon$ was arbitrary we get the result.

     Now we will prove~\ref{eq:ls_measure_open_set}. Clearly, 
     if $E \subset G = \countUnion{(a_i,b_i)}{i}$ then, 
     $\measure{E} \leq \measure{G}$. Fix an $\epsilon > 0$, then from~\ref{eq:ls_measure_open_interval}, there
     is a collection of open intervals $(a_i,b_i)$ such that $E \subset \countUnion{(a_i,b_i)}{i}$
     and,
     \[\infiniteSum{\measure{(a_i,b_i)}}{i} < \measure{E} + \epsilon.\]
     Put $G = \countUnion{(a_i,b_i)}{i}$. Then $G$ is open and $E \subset {G}$. Also, $\measure{G}
     \leq \infiniteSum{\measure{(a_i,b_i)}}{i}$. Hence, the result follows.

     As for~\ref{eq:ls_measure_compact_set}, note that we are measuring from \emph{inside}. Compact
     sets in $\R$ are closed and bounded and thus are in $\borelS{\R}$ and thus in $\algebra{M}$. So
     for any $K \subset E$, where $K$ is compact, $\measure{K} \leq \measure{E}$ is evident. Thus,
     we need to show the other direction. We will do it by cases.\\ 
     \textbf{CASE I:} \\
     E is bounded. If E is also closed then the result is obvious ($K = E$). Thus assume $E$ is not
     closed. Hence, for any $\epsilon > 0$, we need to find a $K$ compact such that 
     $K \subset E$ and $\measure{K} > \measure{E} - \epsilon$. Since $E$ is not closed
     $\setDiff{\closure{E}}{E}$ is not empty. Using equation~\ref{eq:ls_measure_open_set} there is
     an open set $G$ such that $\setDiff{\closure{E}}{E} \subset G$ and 
     $\measure{G} < \measure{(\setDiff{\closure{E}}{E})}+ \epsilon$. Let $K =
     \setDiff{\closure{E}}{G}$, then $K$ is compact since it is the intersection of two closed sets and
     is bounded. See~\ref{fig:tikz:ls_compact}. Also $K \subset E$. From~\ref{fig:tikz:ls_compact},
     it is also evident that $E\cap\comp{K} = E\cap{G}$, i.e
     \begin{align*} 
	 E\cap\comp{K} =& E \cap ( \comp{\closure{E}} \cup G) \\
	 =& E \cap G
     \end{align*}
     Now, since $E = K \disjU (E\cap\comp{K}) = K \disjU (E \cap G)$, we have,
     \begin{align*}
	 \measure{K} &= \measure{E} - \measure{E\cap G} \\
	 &= \measure{E} - (\measure{G} - \measure{\setDiff{G}{E}}) \\ 
	 &\geq \measure{E} - \measure{G} + \measure{\setDiff{\closure{E}}{E}} \\
	 &> \measure{E} - \epsilon.
     \end{align*}
     \textbf{CASE II:}\\
     If $E$ is unbounded, then $E_j = E\cap\hInt{-j-1}{j}$ is bounded. Also $\atobUp{E_j}{E}$, hence
     $\measure{E} = \lim\limits_{j\to\infty}{\measure{E_j}}$. If $\measure{E} = \infty$, then the
     result is trivial. Hence, assume $\measure{E} < \infty$. Then for any $\epsilon$, there is a
     $N$ such that,
     \[\measure{E} < \measure{E_N} + \frac{\epsilon}{2}.\]
     Also since $E_N$ is bounded, by the preceding argument, there is a compact set $K_N$ such that
     $K_N \subset E_N$ and 
     \[\measure{K_N} > \measure{E_N} + \frac{3\epsilon}{2},\] and thus
     \[\measure{K_N} > \measure{E} + \epsilon.\]
     Thus we have shown that, for any $\epsilon > 0$, there is a compact set $K_N \subset E$, 
     such that 
     \[\measure{K_N} > \measure{E} + \epsilon.\]
\end{proof}

\begin{figure}
    \includestandalone[width=0.40\textwidth]{tex/tikz_figures/ls_compact}
    \caption{Illustration of proof of~\ref{eq:ls_measure_compact_set}-compact
	measure}\label{fig:tikz:ls_compact}
\end{figure}
Note that, the outer measure on $\powSet{\R}$ is defined as,
\[\outMeasure{A} = \inf\set{\sumInf{\preMeasure{I_i}}}{\lbrace I_i \rbrace \subset
	\family{E}\,\text{and}\,A\subset\countUnion{I}{i}},\]
where $\family{E}$ is the algebra of finite disjoint union of h-intervals, and $\preMeasure{I_i} =
F(b_i) - F(a_i)$, where $I_i = \hInt{a_i}{b_i}$. The set $A$ may or may 
not be in $\algebra{M}$. For it to be measurable, it must satisfy $\carathCrit$. 
Whenever $A \in \algebra{M}$, i.e A is L-S measurable, then we know $\outMeasure{A} = \measure{A}$.
In the theorem below, we give an equivalent criteria for a set $A\subset \R$ to be $L-S$ measurable.

\begin{Theorem}[name=Equivalent criteria for (L-S) measurablity]\label{thm:equiv_crit_LS_meas}
    A set $A \subset \R$ is (L-S) measurable iff for every $\epsilon > 0$ there is an open set
    $G_{\epsilon}\supset A$ such that,
    \[\outMeasure{\setDiff{G_{\epsilon}}{A}} < \epsilon.\]
\end{Theorem}
\begin{proof}
    Assume $A \subset \R$ is $(L-S)$ measurable. Then $\outMeasure{A} = \measure{A}$.
    Fix an $\epsilon > 0$. From~\ref{eq:ls_measure_open_set} there is an open set $G$ such that,
    \[\measure{G_{\epsilon}} < \measure{A} + \epsilon.\] Also $A$ satisfies $\carathCrit$,
    \begin{align*}
	\carathEq{G_{\epsilon}}{A}.
    \end{align*}
    Now, $A\cap G_{\epsilon} = A$. Assume $\outMeasure{A} < \infty$. Then,
    \[\outMeasure{G_{\epsilon}\cap\comp{A}} = \outMeasure{G_{\epsilon}} - \outMeasure{A} = 
	\measure{G_{\epsilon}} - \measure{A},\]
    since $G_{\epsilon}$ is an open set, hence $G_{\epsilon} \in \algebra{M}$ and thus 
    $\outMeasure{G_{\epsilon}} = \measure{G_{\epsilon}}$. But
    $\measure{G_{\epsilon}} - \measure{A} < \epsilon$. Hence, we get the result.
    If $\measure{A} = \infty$, then pick $A_j = A \cap \hInt{-j-1}{j}$. Since each $A_j$ is bounded,
    by the preceding argument there is an open set $G_j \supset A_j$ such that 
    $\measure{\setDiff{G_j}{A_j}} < \frac{\epsilon}{2^j}$. Let $G_{\epsilon} = \countUnion{G_j}{j}$.
    Now,
    \begin{align*}
	\measure{\setDiff{G_{\epsilon}}{A}} &= \measure{\setDiff{(\countUnion{G_j}{j})}{A}} \\
	& = \measure{\countUnion{(\setDiff{G_j}{A})}{j}} \\
	& \leq \measure{\infiniteSum{(\setDiff{G_j}{A})}{j}} \\
	& \leq \measure{\infiniteSum{(\setDiff{G_j}{A_j})}{j}} \\
	& \leq \epsilon.
    \end{align*}

    Assume that $A \subset \R$ and for any $\epsilon > 0$ there is an open set $G_{\epsilon}$ such
    that $\outMeasure{\setDiff{G_{\epsilon}}{A}} < \epsilon$. We need to show that $A$ is
    $\outMeasurable$. Let $E \subset \R$ be any arbitrary set. The main idea is to show that if $A$
    is measurable it will split \emph{E} w.r.t $\outMeas$.
    See~\ref{fig:tikz:equiv_ls_measurability}. This motivates the statemet,
    $\setDiff{E}{A} = (\setDiff{E}{G}) \disjU (\intersection{E}{(\setDiff{G}{A})})$. 
    Since $G$ is $\outMeasurable$ we have,
    \begin{align*}\carathGeq{E}{G}.\end{align*} Then from
    monotonicity of $\outMeas$ and noting that $\intersection{E}{(\setDiff{G}{A})} \subset 
    \setDiff{G}{A}$, $E\cap A \subset E\cap G$, we get
    \begin{align*}
	\outMeasure{E\cap A} + \outMeasure{E\cap\comp{A}} &= \outMeasure{E\cap A} 
	+ \outMeasure{(\setDiff{E}{G}) \disjU (\intersection{E}{(\setDiff{G}{A})})} \\
	&\leq \outMeasure{E\cap A} + \outMeasure{\setDiff{E}{G}} 
	+ \outMeasure{\intersection{E}{(\setDiff{G}{A})}} \\
	&\leq \outMeasure{E\cap G} + \outMeasure{\setDiff{E}{G}} 
	+ \outMeasure{\setDiff{G}{A}} \\
	&< \outMeasure{E} + \epsilon. \\
    \end{align*}
    Hence we get the result.
\end{proof}

\begin{figure}
    \includestandalone[width=0.40\textwidth]{tex/tikz_figures/equiv_ls_measurability}
    \caption{Illustration of proof of~\ref{thm:equiv_crit_LS_meas}- part of $E$ not in $A$ is the
	sum of two parts.}\label{fig:tikz:equiv_ls_measurability}
\end{figure}
The~\ref{thm:equiv_crit_LS_meas} states that a set is $(L-S)$ measurable if and only if it can be
approximated from \emph{outside} by an open set in such a way that the difference has arbitrarily
small outer measure. This condition can be adopted as the criteria for a $\outMeasurable$ set. 
However, the $\carathCrit$ is very general and is extremely useful for construction of other measures.

The following theorem gives another characterization of $L-S$ measurable sets, as ones that can
be \emph{squeezed} between open and closed sets.
\begin{Theorem}[name=Squeezing a measurable set by open and closed set]\label{thm:LS_meas_squeezed_closed_open}
    A subset $A \in \R$ is $(L-S)$ measurable if and only if for every $\epsilon > 0$, there is an
    open set $G_{\epsilon}$ and a closed set $F_{\epsilon}$ such that $G_{\epsilon} \supset A
    \supset F_{\epsilon}$ and,
    \[\measure{\setDiff{G_{\epsilon}}{F_{\epsilon}}} < \epsilon.\]
\end{Theorem}
\begin{proof}
    Assume that for every $\epsilon$ there is an open set and closed set $G_{\epsilon},F_{\epsilon}$
    respectively, such that, $G_{\epsilon} \supset A
    \supset F_{\epsilon}$ and
    \[\measure{\setDiff{G_{\epsilon}}{F_{\epsilon}}} < \epsilon.\]
    From monotonicity of $\outMeas$,
    \[\outMeasure{\setDiff{G_{\epsilon}}{A}} < \outMeasure{\setDiff{G_{\epsilon}}{F_{\epsilon}}}
	= \measure{\setDiff{G_{\epsilon}}{F_{\epsilon}}} < \epsilon, \]
    since $G_{\epsilon},F_{\epsilon} \in \algebra{M}$. Thus, from~\ref{thm:equiv_crit_LS_meas}, $A$
    is measurable.

    Now, assume $A\subset\R$ is $(L-S)$ measurable. Fix an $\epsilon > 0$. Then there is an open set 
    $G_{\epsilon}$ such that $\measure{\setDiff{G_{\epsilon}}{A}} < \epsilon/2$. Since $A$ is
    $(L-S)$ measurable, $\comp{A}$ is also $(L-S)$ measurable and hence there is an open set
    $H_{\epsilon}\supset \comp{A}$ such that 
    $\measure{\setDiff{H_{\epsilon}}{\comp{A}}} < \epsilon/2$. Let $F_{\epsilon} =
    \comp{H_{\epsilon}}$. Thus $F_{\epsilon} \subset A$. 
    Note that $A\cap\comp{F_{\epsilon}} = H_{\epsilon}\cap A = \setDiff{H_{\epsilon}}{\comp{A}}$. 
    Hence,
    \begin{align*}
	\measure{\setDiff{G_{\epsilon}}{F_{\epsilon}}} &= \measure{\setDiff{A}{F_{\epsilon}}} +
	\measure{\setDiff{G_{\epsilon}}{A}}\\
	& < \epsilon/2 + \epsilon/2 \\
	&\quad = \epsilon.
    \end{align*}
\end{proof}
\begin{Remark}\label{rmk:ls_meas_compact_open}
    In the theorem above, if $A \in \algebra{M}$ and $\measure{A} < \infty$, then we can replace the
    closed $F_{\epsilon}$ by a compact $K_{\epsilon}$ using~\ref{eq:ls_measure_compact_set}.
\end{Remark}
The next theorem states that any Borel set can be approximated up to a set of measure zero by a
Borel set.
\begin{Theorem}[name=Borel Approximation]\label{thm:borel_approximation}
    Suppose that $A\subset \R$ is $(L-S)$ measurable. Then there exists a $G_{\delta}$ and
    $F_{\sigma}$ set in $\borelS{\R}$ such that,
    \[G_{\delta} \supset A \supset F_{\sigma}, \quad \measure{\setDiff{G_{\delta}}{A}} =
	    \measure{\setDiff{F_{\sigma}}{A}} = 0.\]
\end{Theorem}
\begin{proof}
    Since, $A$ is $(L-S)$ measurable, for every $k \in\Zplus$, there is an open set $G_k$ and a
    closed set $F_k$ such that $G_k \supset A \supset F_k$ and  
    $\measure{\setDiff{G_k}{F_k}} \leq \frac{1}{k}$. 
    Put $G_{\delta} = \countIntersection{G_k}{k}$ and $F_{\sigma} = \countUnion{F_k}{k}$.
    Fix an $\epsilon > 0$. For every $k$,
    \begin{align*}
	G_{\delta}\cap\comp{A} &\subset G_k\cap\comp{A}\\
	& \subset G_k \cap \indxComp{F}{k}
    \end{align*}
    Thus picking a large $k$ such that $\frac{1}{k} < \epsilon$,
    $\measure{\setDiff{G_{\delta}}{A}} \leq \measure{\setDiff{G_k}{F_k}} < \epsilon$. Since $\epsilon$ was
    arbitrary, the result follows.
    Similarly,
    \begin{align*}
	A\cap\indxComp{F}{\sigma} &\subset G_k\cap\indxComp{F}{\sigma}\\
	& \subset G_k \cap \indxComp{F}{k},
    \end{align*}
    and the result follows.
\end{proof}

%\section{Borel and Lebesgue Measure in \texorpdfstring{$\Rn$}{}}
\section{Borel and Lebesgue Measure in {$\Rn$}}
Now we will extend the Borel measure on the real line to $\Rn$. If we view $\Rn$ as $\R \times \R
\dots \times \R$, then we can think of $\Rn$ as the n dimensional product of the Real line. However,
we do not have the machinary yet to deal with product measures. We will have to build the Borel
measure as we did in~\ref{prop:pre_meas_hint}. This will give rise to a distribution function in
$\Rn$ and the Lebesgue-Stieltjes measure will follow. 

The
key concept in that construction was that of the distribution function. Thus, given a finite measure
on $\R$ we were able to construct an increasing, right continuous function $F$ on $\R$ and vice
versa. See~\ref{thm:borel_measure_R}. Our goal would be to extend that idea to $\Rn$. Once we have a
borel measure on $\Rn$, the complete measure associated with it would give us the Lebesgue-Stieltjes
measure on $\Rn$. In $\R$ we looked at an elementary family of h-intervals. In $\Rn$, it will be the
family of right semi-closed intervals (rectangles) $\family{J}_n^{cr}$. 
However, as with $\R$ we need to include
$-\infty,\infty$ and these will give us the generalized \textbf{h-intervals}. 

If $\vect{a} = (a_1,\dots,a_n), \vect{b} = (b_1,\dots,b_n)$ are points in $\Rn$,
let us define $\hInt{\vect{a}}{\vect{b}}$ as the set $\set{\vect{x}\in\Rn}{a_i< x_i \leq b_i,\,1\leq
i\leq n}$. Similarly $(\vect{a},\infty) = \set{\vect{x}\in\Rn}{x_i > a_i,\,1\leq i \leq n}$ and 
$\hInt{-\infty}{\vect{b}} = \set{\vect{x}\in\Rn}{x_i\leq b_i,\,1\leq i\leq n}$.
Let $\family{E} \subset \Rn$ be the collection of all generalized h-intervals as described above.
Then $\family{E}$ is an elementary family and by~\ref{thm:const_algebra_elem} the family
$\algebra{A}$ of finite disjoint union of generalized h-intervals in $\family{E}$ is an algebra.
Moreover, $\sigmaGen{\family{A}} = \borelS{\Rn}$. Thus, we have to define a pre-measure $\preMeas$
on $\algebra{A}$ and by~\ref{thm:carath_ext_thm} we will have a unique measure on $\borelS{\Rn}$. We
proceed as in~\ref{prop:pre_meas_hint}, but we need to define what is means for a function $F$ in
$\Rn$ to be increasing and right continuous. The notion of a distribution function is more
complicated in $\Rn$. As with our motivation in the previous section, let us assume we have a finite
measure (for example the Lebesgue measure on a cube) in $\Rn$. In $\R$, we defined $F(x) =
\measure{\hInt{-\infty}{x}}$. 
Let us extend the same idea and define $\map{F}{\Rn}{\R}$ as,
\[F(x_1,x_2,\dots,x_n) = \measure{\set{\vect{\omega}\in\Rn}{\omega_{i}\leq x_i}}.\]
In $\R$, this lead to $\measure{\hInt{a}{b}} = F(b) - F(a)$. However, this wont be true in $\Rn$.
We need some more ideas to make a jump from the real line to $\Rn$. One is the notion of order. In
$\Rn$ there is no total order, however the following notion will suffice,
\begin{Definition}[name=Order in $\Rn$]
    Let $\vect{a},\vect{b}$ be points in $\Rn$. We say that $\vect{a}\leq\vect{b}$ iff $a_i\leq b_i$
    for all $1\leq i \leq n$.
\end{Definition}
To generalize $F(b) - F(a)$, we need to define the \emph{difference} operator of a function $F$ from
$\Rn$ to $\R$.
\begin{Definition}[name=Difference Operator]
    Let $\map{G}{\Rn}{\R}$. We define the difference of $G$ at the $ith$ cordinate evaluated at
    $a_i,b_i$ as,
    \[\diffOp{b_i}{a_i}G = G(x_1,x_2,\dots,x_{i-1},b_i,\dots,x_n) - 
	G(x_1,x_2,\dots,x_{i-1},a_i,\dots,x_n). \]
\end{Definition}
\begin{Definition}[name=Increasing Function in $\Rn$]
    Let $\map{F}{\Rn}{R}$ and for $\vect{a} \leq \vect{b}$, let us denote by
    $F(\hInt{\vect{a}}{\vect{b}})$ as,
    \[F(\hInt{\vect{a}}{\vect{b}}) = \diffOp{b_1}{a_1}\dots\diffOp{b_n}{a_n}F.\]
    The function $F$ is said to be increasing iff $F(\hInt{\vect{a}}{\vect{b}}) \geq 0$ whenever
    $\vect{a} \leq \vect{b}$.
\end{Definition}
The definition may seem peculiar but it is motivated by the following theorem,
\begin{Theorem}[name=Measure and Distribution function in $\Rn$]
    Let $\mu$ be a finite measure on $\borelS{\Rn}$ and define,
    \[F(\vect{x}) = \measure{\hInt{-\infty}{\vect{x}}},\]
    Then, if $\vect{a} \leq \vect{b}$, 
    \[\measure{\hInt{\vect{a}}{\vect{b}}} = F(\hInt{\vect{a}}{\vect{b}}).\]
\end{Theorem}
\begin{proof}
    We prove it for $n=3$ to make the notation simpler. 
    \begin{align*}
	\diffOp{b_3}{a_3}F &= F(x_1,x_2,b_3) - F(x_1,x_2,a_3) \\
	&= \measure{\set{\vect{\omega}}{\omega_{1}\leq x_1,\,\omega_2\leq x_2,\,\omega_3\leq b_3}}
	\\ 
	&\quad	- \measure{\set{\vect{\omega}}{\omega_{1}\leq x_1,\,\omega_2\leq x_2,\,\omega_3\leq
		a_3}} \\
	&= \measure{\set{\vect{\omega}}{\omega_{1}\leq x_1,\,\omega_2\leq x_2,\,a_3 < \omega_3\leq
		b_3}}
    \end{align*}
    Thus,
    \begin{align*}
	\diffOp{b_2}{a_2}\diffOp{b_3}{a_3}F &=\diffOp{b_2}{a_2}(F(x_1,x_2,b_3) - F(x_1,x_2,a_3)) \\
	&= F(x_1,b_2,b_3) - F(x_1,a_2,b_3) \\
	&\quad -F(x_1,b_2,a_3) + F(x_1,a_2,a_3)\\
	&= \measure{\set{\vect{\omega}}{\omega_1\leq x_1,\,a_2 < \omega_2 \leq b_2,\, a_3 < \omega_3
	    \leq b_3}}
    \end{align*}
    and thus we get the result from another application of the difference operator.
\end{proof}
For defining a right continuous function we need to define a \emph{right} limit. We say that a
sequence of points $(\vect{x}^{(n)})$ in $\Rn$ coverges to a point $\vect{x}$ from the \emph{right}
if for each co-ordinate $\atobDown{x_i^{(n)}}{x}$. We say that  $(\vect{x}^{(n)})$ \emph{decreases}
to $\vect{x}$ and denote it by $\atobDown{(\vect{x}^{(n)})}{\vect{x}}$.
\begin{Definition}[name=Right continuous function in $\Rn$]
    A function, $\map{F}{\Rn}{R}$ is right continuous if for a sequence of points 
    $\atobDown{\vect{x}^{(n)}}{\vect{x}}$; $\atobDown{F(\vect{x^{n}})}{F(\vect{x})}$.
\end{Definition}
\begin{Definition}[name=Distribution function in $\Rn$]
    A distribution function $F$ is an increasing, right continuous function $\map{F}{\Rn}{\R}$.
\end{Definition}
\begin{Theorem}[name=Borel measure in $\Rn$]\label{thm:borel_meas_rn}
    Let $F$ be a distribution function on $\Rn$ and set,
    \[\borelM{F} = F(\hInt{\vect{a}}{\vect{b}}),\quad \vect{a}\leq\vect{b}.\]
    Then $\borelM{F}$ is a measure on $\borelS{\Rn}$ and is called the borel measure on $\Rn$
    induced by $F$.
\end{Theorem}
\begin{proof}
    The proof follows the same idea as in~\ref{prop:pre_meas_hint}
    and~\ref{thm:equiv_lebesgue_meas}.
\end{proof}
\begin{Definition}[name=Lebesgue-Stieltjes (L-S) measure on $\Rn$]
    Let $F$ be a distribution function in $\Rn$ and $\borelM{F}$ the corresponding borel measure on
    $\Rn$. The completion of $\borelM{F}$ is called the Lebesgue-Stieltjes measure on $\Rn$ and we
    also denote is by $\borelM{F}$.
\end{Definition}
\begin{Example}
    Let $F_1,F_2,\dots,F_n$ be n distribution function on $\R$, and define,
    \[F(x_1,x_2,\dots,x_n) = F_1(x_1)\times F_2(x_2)\times\dots F_n(x_n).\]
    $F$ is a distribution function on $\Rn$ with,
    \[F(\hInt{\vect{a}}{\vect{b}}) = \finiteProduct{(F_i(b_i)-F_i(a_i))}{i}{n}.\]
    When $F_i = x$ for all $i$, $F(\hInt{\vect{a}}{\vect{b}}) = \finiteProduct{(b_i-a_i)}{i}{n}$ and
    the corresponding $L-S$ measure is the Lebesgue measure on $\Rn$.
\end{Example}
For a given distribution function $F$, we denote by $\measureS{\Rn}{\algebra{M}}{\mu}$ as
Lebesgue-Stieltjes $(L-S)$ measure space. In particular the space of Lebesgue measure is denoted by
$\measureS{\Rn}{\algebra{L}}{\mu}$. Thus by definition a set $E\subset \Rn$ is $L-S$ measurable if,
\begin{equation*}
    \measure{E} = \inf\set{\infiniteSum{F({\hInt{\vect{a}}{\vect{b}}}_{i})}{i}}{E \subset
	\countUnion{{\hInt{\vect{a}}{\vect{b}}}_i}{i}},
\end{equation*}
which is equivalent to
\begin{equation*}
    \measure{E} = \inf\set{\infiniteSum{\measure{R_i}}{i}}{E \subset
	\countUnion{R_i}{i}},
\end{equation*}
where $R_i$ are a sequence of closed rectangle with sides parallel to co-ordinate axes. 


It is clear that all equivalent criteria for $(L-S)$ measurability we addressed for $\R$ holds for
$\Rn$. Thus~\ref{thm:equiv_lebesgue_meas} holds in $\Rn$, in particular
\begin{equation}\label{eq:lebesgue_meas_rn_open_set}
    \measure{E} = \inf\set{\measure{G}}{E\subset G,\quad\text{G is open}},
\end{equation}
and 
\begin{equation}\label{eq:lebesgue_meas_rn_compact_set}
    \measure{E} = \sup\set{\measure{K}}{E\supset K,\quad\text{K is compact}}.
\end{equation}
To prove~\ref{eq:lebesgue_meas_rn_open_set}, note that if $E \subset G$ then $\outMeasure{E} \leq
\outMeasure{G}$. Since both $E,G$ are $(L-S)$ measurable their outer measure is just the $(L-S)$
measure $\mu$. Assume $\measure{E}$ is finite.  
To prove the other direction, for any $\epsilon$ there exist a sequence of rectangles
$\lbrace R_i \rbrace \subset \Rn$ such that $\infiniteSum{\measure{R_i}}{i} < \measure{E} +
\epsilon$. Since $\measure{E}$ is finite each rectangle $R_i$ is bounded and there is an open
rectangle $\interior{S_i}$ such that $R_i \subset \interior{S_i}$ and $\measure{{S_i}} <
\measure{R_i} + \frac{\epsilon}{2^i}$. Since $\interior{S_i}$ is an open set, let $G =
\countUnion{\interior{S_i}}{i}$. Thus we get our result.~\ref{eq:lebesgue_meas_rn_compact_set}
follows exactly as in~\ref{thm:equiv_crit_LS_meas} (3) with minor modification in the case that $\measure{E}$
is infinite.

Similarly as in~\ref{thm:equiv_crit_LS_meas},~\ref{thm:LS_meas_squeezed_closed_open}
and~\ref{thm:borel_approximation},
\begin{Theorem}[name=$(L-S)$ Measurability in $\Rn$]\label{thm:equiv_Lebesgue_meas_rn}
A set $E \subset \Rn \in \algebra{M}$ is $(L-S)$ measurable if and only if for every 
$\epsilon > 0$,
    \begin{enumerate}
	\item
	    there is an open set $G_{\epsilon} \subset \Rn$ such that
	    $G_{\epsilon} \supset E$ and,
	    \[\outMeasure{\setDiff{G_{\epsilon}}{E}} < \epsilon,\]
	\item
	    there is an open set $G_{\epsilon}$ and a closed set $F_{\epsilon}$, such that
	    $G_{\epsilon}\supset E \supset F_{\epsilon}$ and,
	    \[\measure{\setDiff{G_{\epsilon}}{F_{\epsilon}}} < \epsilon,\]
	\item
	    $E$ differs from a $G_{\delta}$ and a $F_{\sigma}$ set by a set of measure zero.
    \end{enumerate}
\end{Theorem}
%\section{\texorpdfstring{$\ast$}{} Probability Space}
\section{{$\ast$} Probability Space}
We now specialize the general measure theory to provide a description of probability.

\begin{Definition}[name=Probability space]
    A probability space  is a measure space $\measureS{X}{\algebra{M}}{\mu}$ with $\measure{X} = 1$. The
    measure $\mu$ is called a probability measure. The element of $\algebra{M}$ are called events.
    In classical notation, $X$ is denoted by $\Omega$ and is called the sample space, 
    $\algebra{M}$ is denoted by $\algebra{F}$ and
    $\mu$ by $\mathbb{P}$. Thus a probability space is denoted by $\probS$.
\end{Definition}
Note that if $E \in \algebra{F}$ then $\comp{E} \in \algebra{F}$. This means that
$\probMeasure{\comp{E}} = 1 - \probMeasure{E}$. This is a very useful result that is used very
frequently in probability.
\begin{Example}
    We consider a few probability spaces.
    \begin{enumerate}
	\item
	    Let $\Omega$ be countable set, $\Omega = \lbrace\omega_1,\omega_2,\dots\rbrace$. Suppose each
	    point $\omega_i$ has probability $p_i$ of occuring so that $\sum\limits_{i}p_i = 1$. Let
	    $\algebra{F} = \powSet{X}$. Define,
	    \[\probMeasure{A} = \sum\limits_{x_i\in A}p_i,\quad A\in\algebra{F}.\]
	    Then, $\probMeas$ is a probability measure. $\probS$ is called
	    the discrete probability space.
	\item
	    Let $\Omega = \interval{0}{1}$, $\algebra{F}$ be the lebesgue measurable sets in
	    $\interval{0}{1}$ and $\probMeas$ be the Lebesgue measure. This is called the uniform
	    distribution. 
	\item
	    Let $\Omega = \R$ and set $F(x) = \frac{1}{2} + \frac{1}{\pi}\arctan(x)$. The function
	    $F(x)$ is continuous, monotone increasing and non-negative with 
	    $\lim_{x\to-\infty}F(x) = 0$ and  $\lim_{x\to\infty}F(x) = 1$. The distribution function
	    $F$ defines a unique measure on $\borelS{\R}$. This is called the Cauchy distribution.
    \end{enumerate}
\end{Example}
Next, we describe a measure theoretic probability model.

\begin{Definition}[name=Measure theoretic probability model]
    Let $X$ be the space of a probabilistic process. A measure theoretic model of the
    process is an identification of $X$ with a set $\Omega$, a 
    $\sigmaAlgebra$ $\algebra{F}$ on $\Omega$ and a measure $\probMeas$ on $\algebra{F}$. 
    A subset $E\subset X$ is called a \emph{plausible} event if $\Omega_{E} \in \algebra{F}$, 
    where $\Omega_{E}$ is the set of 
    points in $\Omega$ which corresponds to points in $X$ for which $E$ occurs. 
    We denote the probability of $E$ by $P(E)$ and set it to $\probMeasure{\Omega_E}$.  
\end{Definition}
The notation $\Omega_{E}$ here signifies that $X$ could be any probabilistic set up where we
want to assign probabilities to subsets $E$ of $X$. In order to build a measure theoretic framework we
identify with $X$ a set which we call the sample space $\Omega$ and construct a $\sigmaAlgebra$ on $\Omega$. 
Such an identification and construction is not unique. To
assign probabilities to events $E$ in $X$, we see if the corresponding set is in $\algebra{F}$. If it is, we
assign it a probability; if not, we deem the set $E$ as not plausible within our measure theoretic framework.


Now we will build a measure theoretic probability model to analyze an experiment that involves a
sequence of infinite coin tosses. Let us assume we have a fair coin and we toss it infinitely many
times. Also it will be safe to assume that any outcome of a toss doesn't depend on what preceeded it
and will not affect any future outcomes. 
Ofcourse, such an experiment is just a thought experiment and cannot be practically carried
out. Our sample space will consist of \emph{points} like $(H,T,H,H,H,T,H,T,T,H,H,\dots,)$. We may
have events such as the event $E$ of all those infinite sequences such that the first toss is a $H$.
How do we assign a probability to such an event. First a couple of definitions.
\begin{Definition}[name=Bernoulli trial and sequences.]
    Suppose an experiment has two possible outcomes. A finite number of repetitions of the
    experiment is a called a Bernoulli trial. An infinite sequence of experiments is called a
    Bernoulli sequence.
\end{Definition}
Let $\mathcal{B}$ be the set of all Bernoulli sequences. Thus $\mathcal{B}$ is the space of
our experiment of infinite coin tosses. In order to prescribe probability to events in a measure
theoretic framework we need to identify $\mathcal{B}$ with the sample space of a probabilistic
space. The following proposition says that there is a one to one correspondence between the unit
interval and $\mathcal{B}$.

\begin{Proposition}
    There exists a $1-1$ correspondence from $\hInt{0}{1}$ into $\mathcal{B}$.
\end{Proposition}
\begin{proof}
    Let $\omega \in \interval{0}{1}$. Then $\omega$ can be written as a binary expansion given by,
    \[\omega = \infiniteSum{\frac{a_i}{2^i}}{i},\quad a_i= 0\,\text{or}\,1.\]
    Let $\map{f}{\interval{0}{1}}{\mathcal{B}}$ be the map given by,
    \begin{equation*}
	f(\omega) = 
	\begin{cases}
	    H\,&\text{if $a_i = 1$},\\
	    T\,&\text{if $a_i = 0$}.
	\end{cases}
    \end{equation*}
    The map defined above fails to be a function since there are real numbers that have two
    different binary expansion. For example $\frac{1}{2} = 0.10\dots$ or 
    $\frac{1}{2} = 0.0111\dots$. To avoid this trouble we chose the later expansion. What this means
    is that the function fails to be onto since we discard binary expansions terminating in an
    infinite sequences of $0$ which corresponds to Bernoulli sequence ending in infinite $T$'s. Thus
    there is a $1-1$ correspondence from $\hInt{0}{1}$ into $\mathcal{B}$.
\end{proof}
\begin{Remark}
    Since $\hInt{0}{1}$ is uncountable, this implies that $\mathcal{B}$ is uncountable. However,
    because the map isn't bijective we cannot identify every bernoulli sequence with a real number
    in the unit interval. However there is only a countable subset of $\mathcal{B}$ that is not
    mapped by $f$ above and since we know that a measure of a countable set is $0$ we can neglect
    this set to build our measure theoretic framework. To show this, let $\mathcal{B}_{\text{neg}}$ be the
    Bernoulli sequence that are not mapped by $f$. These are the sequences that end in infnite
    $T$'s. Let $\mathcal{B}_{\text{neg}}^{k}$ be the bernoulli sequences that end in inifinite $T$'s
    after the $k^{th}$ toss. Then $\mathcal{B}_{\text{neg}}^{k}$ is countable. But 
    $\mathcal{B}_{\text{neg}} = \bigcup\limits_{k}\mathcal{B}_{\text{neg}}^{k}$ and so the set that
    is not mapped by $f$ is countable. 
\end{Remark}
We can make $\mathcal{B}$ into a probability space by using the Lebesgue measure on the unit
interval.
\begin{Definition}[name=Borel Principle]
    Let $\Omega = \hInt{0}{1}$. If $E$ is a \textbf{plausible} event of Bernoulli sequences, we
    denote by $\Omega_{E}$ as the subset of real numbers in $\hInt{0}{1}$ that are Lebesgue
    measurable and are given by the corresponding binary expansion. 
    We set the probability of $E$, denoted by $P(E)$, as $\measure{\Omega_{E}}$.  
\end{Definition}
\begin{Example}
    Let $E = $ event where $H$ occurs on the first toss.
    $E = \lbrace H, X_1, X_2, X_3,\dots \rbrace$. The corresponding set in $\Omega$ is 
    \[\Omega_{E} = \set{\omega\in\Omega; x=0.1d_1d_2d_3\dots}{d_i = 0\,\text{or}\,1}.\]
    Is $E$ plausibe? That is is $\Omega_{E}$ Lebesgue measurable? The smallest number in
    $\Omega_{E}$ is $0.100000\dots$ while the largest number is $0.11111\dots$. Thus
    $\Omega_E = \hInt{1/2}{1}$ which is certainly Lebesgue measurable. Thus $P(E) =
    \measure{\hInt{1/2}{1}} = 1/2$.
\end{Example}
\begin{Example}
    Let $E$ be the event where the first $N$ tosses are prescribed. For example we could have for
    $N=3$, $HTT$ as the first $3$ tosses. Thus,
    \[\Omega_{E} = \set{\omega\in\Omega; x=0.a_1a_2\dots a_{N}d_1d_2d_3\dots}{d_i =
	    0\,\text{or}\,1}.\]
    Is $\Omega_{E}$ Lebesgue measurable? Again the smallest number is $a_1a_2\dots a_N
    00000\dots$ while the largest number is $a_1a_2\dots a_N 11111$. Let $s = a_1a_2\dots
    a_N000\dots$. Then $a_1a_2\dots a_N 111\dots = s + \sum\limits_{i \geq N+1}^{\infty}1/2^{i} = s
    = 1/2^N$. Thus $\Omega_{E} = \hInt{s}{s+\frac{1}{2^N}}$ which is certainly Lebesgue
    measurable. Thus $P(E) = \measure{\hInt{s}{s+\frac{1}{2^N}}} = \frac{1}{2^N}$.
\end{Example}

\begin{Example}
    Consider the event $E$ in which $H$ occurs in the $N^{th}$ toss. The corresponding set 
    \[\Omega_{E} = \set{\omega\in\Omega; x=0.d_1d_2\dots 1 d_{N+1}d_{N+2}d_{N+3}\dots}{d_i =
	    0\,\text{or}\,1}\]
    Here $\Omega_{E}$ can be thought of as a union of disjoint interval. As a concrete case,
    consider $N=3$. Then we have the following cases:
    $HTH,HHH,THH,HTH$. These correspond to 4 disjoint intervals of length $1/2^3$. Thus $P(E) = 4/8
    =1/2$. In general, $P(E) = 2^{N-1}/2^N = 1/2$.
\end{Example}
In all these examples the Events have corresponded to either intervals or union of intervals.
However, the power of a measure theoretic framework is not apparent since the above examples can be
easily seen even in a discrete setting. To show the power of the Borel principle, we will look at
the Weak Law of Large Numbers. See Appendix for a connection between weak law of large numbers and
the approximation of continuous functions by polynomials.

We believe that we should be able to detect the probabilities of Heads and Tails when flipping a
coin by examining the results of many experiments. For example suppose the probability that Head
occurs in a coin toss is $p$. If we don't know what $p$ is we can carry the experiment a large
number of times and record the number of Heads that occured. The fraction of Heads then must give us
some indication of $p$. However, a precise statement of this intuition is difficult to formulate.

Say we are tossing a fair coin whose results are independent of any other coin tosses. 
If we let $H_n$ to be the number of heads that occur in the first $n$ tosses, we \textbf{might} like
to show
\[\lim\limits_{n\to\infty}\frac{H_n}{n} = \frac{1}{2}.\]
However this is certainly not true. In other words, we could get a sequence that ends in $H$
infinitely or in $T$ infinitely. Intuitively, such sequences are not likely or typical. What is
clear is that we need to create a careful formulation.

\begin{Definition}[name=Number of Heads]
    For $\omega \in \Omega$, define
    \[S_N(\omega) = a_1 + \cdots + a_N,\quad \omega = 0.a_1a_2\dots\]
    $S_N$ gives the number of heads in the first $N$ tosses of the Bernoulli sequence corresponding
    to $\omega$. So $H_N = S_N(\omega)$.
\end{Definition}
Now, consider the event $E$ that the fraction of heads $H$ tends to $1/2$. The corresponding set 

\begin{equation}\label{eq:SLLN_lebesgue_set}
    \Omega_{E} = \set{\omega\in\Omega}{\atob{\frac{S_n(\omega)}{n}}{\frac{1}{2}}\quad\atob{n}{\infty}}
\end{equation}
Is $\Omega_{E}$ Lebesgue measurable? This set is highly complicated. What can we say about its
elements? If $\omega \in \Omega_{E}$ then no matter what $\epsilon$ we take, there is an
integer $k$ such that for all integers $n$ greater than $k$, the fraction of heads is within an
$\epsilon$ of $\frac{1}{2}$. That is,
\[\forEv{\epsilon}\thereIs{k}\forEv{n}\suchThat{n\geq
	k}{\implies}{\distR{\frac{S_n(\omega)}{n}}{\frac{1}{2}} < \epsilon}.\]
By the archimedes principle, for every $\epsilon > 0$, there is a positive integer $r$ such that
$\frac{1}{r} < \epsilon$. Thus the above condition can be written as,
\[\forEv{r}\thereIs{k}\forEv{n}\suchThat{n\geq
	k}{\implies}{\distR{\frac{S_n(\omega)}{n}}{\frac{1}{2}} < \frac{1}{r}}.\]
The advantage of such a formulation is that now we have positive integers as quantifiers. Let,
\[A_{n,r} = \set{\omega\in\Omega}{\distR{\frac{S_n(\omega)}{n}}{\frac{1}{2}} < \frac{1}{r}}.\]
This means that,
\[\frac{n}{2}(\frac{1}{2} - \frac{1}{r}) < a_1 + a_2 + \cdots + a_n < 
    \frac{n}{2}(\frac{1}{2} + \frac{1}{r}).\]
Since $a_i$ are either $0$ or $1$ we have a finite choice of these. Thus $A_{n,r}$ is a disjoint
union of intervals in $\Omega$ and is Lebesgue measurable. This means that,
\[\Omega_{E} = \bigcap\limits_{r = 1}^{\infty}\bigcup\limits_{k = 1}^{\infty}\bigcap\limits_{n\geq
    k}^{\infty}A_{n,r},\]
is Lebesgue measurable since it is just a countable union and intersection of a Lebesgue measurable
set. This is easy to see: if $\omega \in \Omega_{E}$ then for every $r$, 
$\omega \in \bigcup\limits_{k = 1}^{\infty}\bigcap\limits_{n\geq k}^{\infty}A_{n,r}$. This means that
there is a $k$ such that $\omega \in \bigcap\limits_{n\geq k}^{\infty}A_{n,r}$. Thus for all $n \geq
k$, $\omega \in A_{n,r}$, i.e $\distR{\frac{S_n(\omega)}{n}}{\frac{1}{2}} < \frac{1}{r}$.
Thus $\Omega_{E}$ is Lebesgue measurable in $\Omega$. 

What is $P(E)$ i.e.~what is $\measure{\Omega_{E}}$? Intiuitely, we would guess that $P(E) = 1$.
Indeed that is the case and the result is the famous \textbf{Strong Law of Large numbers}. To prove
this statement we will show $\measure{\comp{\Omega_{E}}} = 0$. There are two ways to show a set
has zero Lebesgue measure. Either the set is countable or it can be covered by intervals whose
countable sum of measures can be made arbitrarily small. See~\ref{prop:lebesgue_meas_0}. We can rule
out the first option which seems surprising at first but follows from a simple observation. 
Consider the map, $\map{\sigma}{\Omega}{\Omega}$ given by,
\[\sigma(\omega) = a_{1}11a_{2}11a_{3}11\cdots,\quad \omega = a_1a_2a_3\cdots.\]
Since $\sigma$ is injective, $\sigma(\Omega) \subset \Omega$ is uncountable. Fix an $\omega \in
\Omega$. Let $N = 3n$ for any positive integer $n$. Let $y \in \Omega$ be give by $\sigma(\omega)$.
Then $S_N(y)$ is the number of heads in the first $N$ tosses which is always greater than $2n$, thus
\[\frac{S_N(y)}{N} \geq \frac{2}{3}.\]
This means that $\sigma(\Omega) \subset \indxComp{\algebra{F}}{E}$ and hence
$\indxComp{\algebra{F}}{E}$ is uncountable.


But first we will show a \emph{weaker} 
statement. The notions weaker and stronger will be explained later when we describe measurable
functions. Given an $\epsilon > 0$ let 
\begin{equation}\label{eq:WLLN_lebesgue_set}
    \Omega_{E_N} =
\set{\omega\in\Omega}{\distR{\frac{S_n(\omega)}{N}}{\frac{1}{2}} \leq \epsilon}.
\end{equation} 
Note that,
$\Omega_{E_N}$ is Lebesgue measurable because we can re-formulate the definition by using
$\frac{1}{r}$, where $r$ is a positive integer such that $\frac{1}{r} < \epsilon$. 

We gather these two notions in the following theorem.
\begin{Theorem}[name=Law of large numbers for Bernoulli sequences]\label{thm:LLN_bernoulli_seq}
    Let $\Omega = \hInt{0}{1}$ be the sample space identified with the Bernoulli sequences. Let
    $\algebra{F}$ be the Lebesgue measurable sets in $\Omega$ and $\mu$ be the corresponding
    Lebesgue measure. Thus, $\measureS{\Omega}{\algebra{F}}{\mu}$ is a probability space. Let
    $\Omega_{E}$ be the set in $\algebra{F}$ as in~\ref{eq:SLLN_lebesgue_set} and
    $\Omega_{E_N}$ be the set in $\algebra{F}$ as in~\ref{eq:WLLN_lebesgue_set}. Then, 
    \begin{enumerate}
	\item
	    For every $\epsilon > 0$, 
	    \[\lim\limits_{N\to\infty}
		\measure{\set{\omega\in\Omega}{\distR{\frac{S_n(\omega)}{N}}{\frac{1}{2}} \leq \epsilon}} = 1\]
	    i.e.,
	    \[\atob{P({E_N})}{1}\,\text{as}\,\atob{N}{\infty}\]
	\item
	    \[\measure{\set{\omega\in\Omega}{\lim\limits_{N\to\infty}\frac{S_N(\omega)}{N}= \frac{1}{2}}}
	    = 1\]
	    i.e.,
	    \[P({E}) = 1.\]
    \end{enumerate}
\end{Theorem}
We will need a few more definitions and results from Riemann integration to prove this. The standard
approach will be to prove results about the complement of the set.

Each digit of the binary expansion of a real number in $\Omega$ (where we choose a unique binary
expansion) can be thought as a selection of a dyadic partition of $\Omega$ as in~\ref{def:dyadic_cube}.
At level $1$ we bisect the interval $\hInt{0}{1}$. The binary digit when in left is $0$ and $1$
when in right. At level $2$, we bisect each of these to get four intervals indexed as $00,01,10,12$.
Continuing, we achieve a dyadic partition of $\hInt{0}{1}$ at each level which corresponds to a
digit in the binary expansion. Since each digit corresponds to a Bernoulli trial, we get a
correspondence of the partition with the Bernoulli sequence. 

Each Level $k$ partitions $\Omega$  into $2^{k}$ 
intervals of $\interval[open left]{\frac{l}{2^{k}}}{\frac{l+1}{2^{k}}}$ 
for $0 \leq l < 2^{k}$.  
At each level of the partition we
define the Rademacher functions which take the value of $1$ when the binary digit is $1$ and
$-1$ when the binary digit is $0$. Thus $R_1$ has two values, $R_2$ has $4$ values, $R_n$ has $2^n$
values. This is made precise in the definition below. See~\ref{fig:tikz:rademacher}.

\begin{figure}
  \includestandalone[width=0.5\textwidth]{tex/tikz_figures/rademacher_1}
  \caption{We show the first three levels of dyadic partition with the corresponding Rademacher
      functions $R_1$, $R_2$ and $R_3$. Red shows that Rademacher functions takes a value $-1$,
      while blue shows that the function takes a value $1$.}\label{fig:tikz:rademacher}
\end{figure}

\begin{Definition}[name=Rademacher function]
    For $\omega \in \Omega$, we define the $k^{th}$ Rademacher function by,
    \begin{equation*}
	R_k(\omega) = 2a_k -1,\quad \omega = 0.a_1a_2\dots
    \end{equation*}
    This means that,
    \begin{equation*}
	R_k(\omega) = 
	\begin{cases}
	    1,\,&a_k = 1,\\
	    -1,\,&a_k = 0
	\end{cases}
    \end{equation*}
\end{Definition}
We can interpret $R_k$ like this. Suppose we bet on a sequence of coin tosses such that at each toss,
we win $\$1$ if it is heads and lose $\$1$ if it is tails. Then $R_k(\omega)$ is the amount won or
lost at the $k^{th}$ toss in the sequence of tosses represented by $\omega$.
\begin{Proposition}\label{prop:rademacher}
    Let $R_k$ denote the $k^{th}$ Rademacher function.
    \begin{enumerate}
	\item
	    $ \int\limits_{0}^{1}R_{k_1}(x)R_{k_2}(x)\dots R_{k_n}(x) = 0\,\text{or}\,1$
	    for any sequence $k_1 \leq k_2 \leq \cdots \leq k_n$. 
	\item
	   For $x \in \interval[open left]{0}{1}$, 
	    \[ \sumInf{R_i(x)2^{-i}} = 2x - 1. \]
    \end{enumerate}
\end{Proposition}
\begin{proof}
    We prove in order.
    \begin{enumerate}
	\item
	   Each $R_{k_i}$ is defined on partitions of $\interval[open left]{0}{1}$ into $2^{k_i}$ 
	   intervals of $\interval[open left]{\frac{l}{2^{k_i}}}{\frac{l+1}{2^{k_i}}}$ 
	   for $0 \leq l < 2^{k_i}$. Let the partition at level 
	   $k_n$ be given by 
	   \[\set{\Delta_{l}^{k_n} = \interval[open
	       left]{\frac{l}{2^{k_n}}}{\frac{l+1}{2^{k_n}}}}{0 \leq l < 2^{k_n}}.\] 
       
           If $k_1 < k_2 < \cdots < k_n$, $R_{k_i}$ for $k_i < k_n$ will be constant on each of these 
	   $\Delta_{l}^{k_n}$. Thus,
	   \begin{align*} 
	       \int\limits_{0}^{1}R_{k_1}(x)\dots R_{k_n}(x) & = \sum\limits_{l =
		   0}^{2^{k_n}-1}\int\limits_{\Delta_{l}^{k_n}}R_{k_1}(x)\dots R_{k_n}(x) \\
	       & = (R_{k_1}(x)\dots R_{k_{n-1}}(x))\int\limits_{\Delta_{0}^{k_n} + \Delta_{1}^{k_n}
	       }R_{k_n}(x) \\
	       & + \cdots + (R_{k_1}(x)\dots R_{k_{n-1}}(x))
	       \int\limits_{\Delta_{2^{k_n-1}}^{k_n} + \Delta_{2^{k_n}}^{k_n}}R_{k_n}(x)
	   \end{align*}  
	   Each of the integral is $0$ and hence the sum
	   is 0. If $k_1 = k_2 = k_3
	   \cdots = k_n$ and n is even then total integral is one otherwise integral is $0$.
       \item
	   Let us write $R_i(x) = 2*a_i - 1$ where $x = 0.a_1a_2\cdots$ is the binary
	   representation of $x$. Note that $x = \sum\limits_{i = 1}^{\infty}a_i/2^i$.
	   \begin{align*}
	       \sumInf{R_i(x)2^{-i}} &= \lim\limits_{n\to\infty}\sum\limits_{i=1}^{n}R_i(x)2^{-i} \\
	       &= \lim\limits_{n\to\infty}(2*(a_1/2 + a_2/4 + \cdots + a_n/2^n))\\ 
	       &\quad - \lim\limits_{n\to\infty}(1/2 + 1/4 + \cdots + 1/2^n) \\
	       & = 2*x - 1 
	   \end{align*}
	   We get the second term by using the definition of $R_i$, i.e $R_i(x) = 2*a_i - 1$. We can
	   split the limit because each converges. The first limit is just the binomial
	   representation of $x$ and the second is a geometric series.
    \end{enumerate}
\end{proof}
\begin{Definition}
    We define $W_N(\omega) = \finiteSum{R_k(\omega)}{k}{N}$. Then $W$ gives the total amount won or
    lost after the $N^{th}$ toss. Using the definition of $R_k$, we get
    \begin{align*}
	W_N(\omega) &= 2(a_1 + a_2 + \cdots + a_N) - N \\
	&= 2S_N(\omega) - N.
    \end{align*}
\end{Definition}
With the above definition, our condition in the set given by~\ref{eq:WLLN_lebesgue_set} becomes,
\[\lvert W_N(\omega) \rvert \leq 2\epsilon N.\]
We will need one more result before proving~\ref{thm:LLN_bernoulli_seq}. The special case of this
result will be proved later.
\begin{Proposition}\label{prop:chebychev_ineq_1}
    Let $f$ be a non-negative, piecewise constant function on $\Omega$ and $\alpha > 0 \in \R$ be a
    positive real number. Then,
    \[\measure{\set{\omega\in\Omega}{f(\omega) > \alpha}} <
	\frac{1}{\alpha}\int\limits_{0}^{1}f(\omega),\]
    where the integeral is the standard Riemann integral in $\R$.
\end{Proposition}
\begin{proof}
    Suppose $f$ is defined on the mesh $0 = \omega_{1} < \omega_{2} < \cdots < \omega_{k} = 1$ 
    such that $f(\omega) = c_i$
    whenever $\omega \in (\omega_{i},\omega_{i+1})$ for all $1 \leq i \leq k-1$.
    Then,
    \begin{align*}
	\int\limits_{0}^{1}f(\omega) &= \finiteSum{c_i({\omega}_{i+1} - {\omega}_i)}{i}{k} \\
	& \geq \finiteSumStack{c_i({\omega}_{i+1} - {\omega}_i)}{i=1}{c_i > \alpha}{k} \\
	& > \alpha \finiteSumStack{({\omega}_{i+1} - {\omega}_i)}{i=1}{c_i > \alpha}{k} \\
	& \quad = \alpha\measure{\set{\omega\in\Omega}{f(\omega) > \alpha}}.
    \end{align*}
\end{proof}
Now we are ready to prove~\ref{thm:LLN_bernoulli_seq} (1) which is the \textbf{WLLN} for Bernoulli
sequences.
\begin{proof}
    The complement of the set $\Omega_{E_N}$ is given by,
    \[\indxComp{\Omega}{E_N} = \set{\omega\in\Omega}{\lvert W_N(\omega) \rvert > 2\epsilon
	    N}.\]
    Since $\epsilon$ is arbitrary we can disregard the factor of $2$ and square both sides to
    remove the annoying absolute sign. Thus,
    \[\indxComp{\Omega}{E_N} = \set{\omega\in\Omega}{W^{2}_{N}(\omega)  >
	    N^2{\epsilon}^2}.\]
    Now we can use~\ref{prop:chebychev_ineq_1} in stating,
    \[\measure{\indxComp{\Omega}{E_N}} < 
	\frac{1}{N^2{\epsilon}^2}\int\limits_{0}^{1}W^2_N(\omega).\]
    To evaluate the integral we observe,
    \begin{align*}
	\int\limits_{0}^{1}W^2_N(\omega) &= \int\limits_{0}^{1}{(\finiteSum{R_k(\omega)}{k}{N})^2}\\
	&= \finiteSum{\int\limits_{0}^{1}R_k^2(\omega)}{k}{N} + 
	\finiteSumStack{\int\limits_{0}^{1}R_i(\omega)R_j(\omega)}{i,j=1}{i\neq j}{N}.
    \end{align*}
    The second summation is $0$ by~\ref{prop:rademacher} (1) while the first summation is $N$
    because each integral is $1$.
    Thus,
    \[\measure{\indxComp{\Omega}{E_N}} \leq \frac{1}{N^2{\epsilon}^2}N =
	\frac{1}{N{\epsilon}^2}.\]
    Thus,
    \[\atob{\measure{\indxComp{\Omega}{E_N}}}{0}\quad\text{as}\,\atob{N}{\infty}.\]

\end{proof}

We are now ready to prove~\ref{thm:LLN_bernoulli_seq} (2) which is the \textbf{SLLN} for the
Bernoulli sequence.
\begin{proof}
    We noted that to show $\measure{\indxComp{\Omega}{E}} = 0$, for any $\epsilon$ 
    we will have to find simple
    sets (closed intervals) $E_i$ such that $\indxComp{\Omega}{E} \subset \countUnion{E_i}{i}$
    and $\infiniteSum{\measure{E_i}}{i} < \epsilon$.
    There is $\delta > 0 \in \R$ such that the set,
    \[A_n = \set{\omega\in\Omega}{\lvert W_n(\omega) \rvert > \delta n}.\]
    is not empty for some $n$.
    How did we get this set? Consider an $\omega \in \indxComp{\Omega}{E}$. Then
    $\frac{S_n(\omega)}{n} \not \to \frac{1}{2}$. This is equivalent to saying,
    \[\thereIs{\delta}\forEv{N}\thereIs{n}\suchThat{n\geq
	    N}{\land}{\distR{\frac{S_n(\omega)}{n}}{\frac{1}{2}} > \delta}.\]
    Since $\distR{\frac{S_n(\omega)}{n}}{\frac{1}{2}} = \lvert 
    \frac{W_n(\omega)}{n}\rvert$ the set $A_n$ is not empty.

    Moreover, $A_n$ is just a finite union of disjoint h-intervals and is Lebesgue measurable.
    By~\ref{prop:chebychev_ineq_1} (removing the absolute sign by raising to $4^{th}$ power)
    \[\measure{A_n} < \frac{1}{{\delta}^4n^4}\int\limits_{0}^1(\finiteSum{R_k}{k}{n})^4.\]	
    The integrand yields $5$ kinds of terms,
    \begin{enumerate}
	\item
	    $R_j^4$ for $j = 1 \cdots n$.
	\item
	    $R_j^2 R_k^2$ for $j\neq k$.
	\item
	    $R_j^2 R_k R_l$ for $j\neq k \neq l$.
	\item
	    $R_j^3 R_k$ for $j \neq k$.
	\item
	    $R_j R_k R_l R_m$ for $j\neq k \neq l \neq m$.
    \end{enumerate}
    Using~\ref{prop:rademacher} it is easy to observe that only the first and second kind of terms integrate
    out to $1$ while others give zero. There  are $n$ terms of the first kind and $3n(n-1)$ terms
    involving the second kind.
    Thus,
    \[\measure{A_n} < \frac{3}{n^2{\delta}^{4}}.\]
    The idea is to cover $\indxComp{\Omega}{E}$ using the sets $A_n$. Since these sets are not
    necessarily closed we can take their closure to get closed intervals without changing the
    measure. Thus it is not necessary to find closed intervals. However, we need to make sure that the
    sequence of sets decrease in measure so that the countable sum can be made arbitrarily small.
    
    For a constant $C$, set $\delta_n = Cn^{\frac{-1}{2}}$. Then,
    \[\infiniteSum{\frac{3}{{\delta_n}^4n^2}}{n} =
	\frac{3}{C}\infiniteSum{\frac{1}{n^{\frac{3}{2}}}}{n}\] 
     converges and can be made smaller than any epsilon by choosing sufficiently large $C$.

     Let,\[E_n = \set{\omega\in\Omega}{\lvert W_n(\omega)\rvert > \delta_{n}n}.\]
     If $\indxComp{\Omega}{E} \subset \countUnion{E_n}{n}$, then
     \[\measure{\indxComp{\Omega}{E}} \leq \infiniteSum{\measure{A_n}}{n} =
	 \frac{3}{C}\infiniteSum{\frac{1}{n^{\frac{3}{2}}}}{n} < \epsilon.  \]
     Thus $\measure{\indxComp{\Omega}{E}} = 0$ provided 
     $\indxComp{\Omega}{E} \subset \countUnion{E_n}{n}$. However, this is easy to see. Taking
     complement what we need to show is 
     \[\countIntersection{\indxComp{E}{n}}{n} \subset \Omega_{E}.\] This is easily verified.
\end{proof}

We have seen that remarkable observations can be made by working in a measure theoretic framework.
Such arguments are standard in probability study. We isolate a few more concepts that are essential
in probability computation. One such technique we already saw when we showed that the set $\Omega_{E}$
is Lebesgue measurable by showing that is countable union and intersection of simple sets that are
Lebesgue measurable. In general, to prove that a set is measurable w.r.t a probability measure, we need
to show that it is an event in the underlying $\sigmaAlgebra$. This is usually done by showing that
that the set is a finite (or countable) combination of countable unions and intersections of elementary
sets that are events. We show a couple of examples to highlight this idea before giving a concrete 
definition.
\begin{Example}
    Let $E$ be the event where $\infiniteSum{\frac{R_n(\omega)}{n}}{n}$ converges. This is an interesting series.
    We know that $\infiniteSum{\frac{1}{n}}{n}$ diverges but $\infiniteSum{\frac{{(-1)}^{n}}{n}}{n}$
    converges. Is $E$ a plausible event? In other words is $\Omega_{E}$ Lebesgue measurable?
    Here $\Omega_{E}$ is given by,
    \[\set{\omega\in\Omega}{\infiniteSum{\frac{R_n(\omega)}{n}}{n} < \infty}.\]
    Let $T_n(\omega) = \finiteSum{\frac{R_k(\omega)}{k}}{k}{n}$. If $\omega\in\Omega_{E}$, then
    for any $\epsilon$, there is an integer $k$ such that
    $\distR{T_n(\omega)}{T_m(\omega)} < \epsilon$ whenever $n,m \geq \epsilon$. We can get rid of
    the $\epsilon$ by finding a positive integer $r$ such that $\frac{1}{r} < \epsilon$.
    Consider the set,
    \[A_{m,n,r} = \set{\omega\in\Omega}{\distR{T_n(\omega)}{T_m(\omega)} < \frac{1}{r}}.\]
    This set is Lebesgue measurable being a union of finite h-intervals in $\Omega$. We can write
    $\Omega_{E}$ as,
    \[\Omega_{E} =
	\bigcap\limits_{r=1}^{\infty}\bigcup\limits_{k=1}^{\infty}\bigcap\limits_{m,n \geq
	    k}^{\infty}A_{m,n,r}.\]
    This shows that $\Omega_{E}$ is Lebesgue measurable.
\end{Example}
\begin{Example}
    Let $E$ be the event where a prescribed sequence, say $HTTH$, occurs infinitely often. To
    describe $\Omega_{E}$, let $E_n$ be the event where the pattern occurs beginning at the
    $n^{th}$ step. $E_n$ is described by a finite number of conditions on the Rademacher functions,
    \[\Omega_{E_n} = \set{\omega\in\Omega}{R_n(\omega) =
	    1,R_{n+1}(\omega)=-1,R_{n+2}(\omega)=-1,R_{n+3}(\omega)=1}.\]
    How do we get $\Omega_{E}$ from $\Omega_{E_n}$? Since the pattern occurs infinitely
    often, no matter what integer $k$ we choose, there should be a $n$ far out such that
    $\omega\in \Omega_{E_n}$. Thus,
    \[\Omega_{E} = \bigcap\limits_{k=1}^{\infty}\bigcup\limits_{n\geq k}^{\infty}
	\Omega_{E_n}.\]
\end{Example}
What can we say about their probabilities? 
All these examples have a general pattern. Say we have a countable collection of events $\lbrace
E_1,E_2,\cdots\rbrace$. Let $E$ be the event that infinitely many of the $E_i$'s occur. Can we
determine the probability of $E$ from the probabilities of $E_i$'s? Two famous results{\textemdash} 
the Borel-Cantelli Lemmas{\textemdash}are relevant to the answer. First we will give a description of
the term infinitely often and almost always.

For a sequence of events $\seq{E}{n}$,we define the limit superior, and limit
inferior as.
\begin{align*}
    &\limSupSet{E}{n}{k} \\
    &\limInfSet{E}{n}{k}.
\end{align*}
\begin{Remark}\label{rmk:limsup_liminf_set}
    The terminology stems from the corresponding definitions for 
    a sequence of real numbers. For example if we let $\seq{s}{n}$ to be
    a sequence of real numbers then we can make observations of their limiting behavior by looking
    at the $\sup$ and $\inf$ of the tail sets. Let us define the following,
    \begin{align*}
	u_n &= \inf\set{s_k}{k\geq n} \\
	v_n &= \sup\set{s_k}{k\geq n} \\
    \end{align*}
    Then it is easy to see that $u_1 \leq u_2 \cdots$, while $v_1 \geq v_2 \geq \cdots$. Since this
    are monotonic sequence, if they converge they will converge to their $\sup$ and $\inf$
    respectively. Thus,
    \begin{align*}
	\liminf s_n & := \lim\limits_{n\to\infty}u_n = \sup\set{u_n}{n\in \Zplus} \\
	\limsup s_n & := \lim\limits_{n\to\infty}v_n = \inf\set{v_n}{n\in \Zplus} \\
    \end{align*}	
    
    Similarly, we can define the same for sequence of sets. However, to get supremum
    and infimum, we need an order relation. This is done through the relation $\subset$. Thus if we
    have a sequence of set $\seq{E}{n}$, the $\sup$ is defined as $\countUnion{E}{n}$. This is easy
    to see; as for any $n$, $E_n \subset \countUnion{E}{n}$. Similarly, the $\inf$ is defined
    as $\countIntersection{E}{n}$.   

    Now analogously we can define,
    \begin{align*}
	U_n &= \inf\set{E_k}{k\geq n} \\
	V_n &= \sup\set{E_k}{k\geq n} \\
    \end{align*}
    Thus, we have $U_1 \subset U_2 \subset U_3 \cdots$, while $V_1 \supset V_2 \supset V_3 \cdots$.
    Thus we see that $\atobUp{U_n}{\countUnion{U_n}{n}}$ while 
    $\atobDown{V_n}{\countIntersection{V_n}{n}}$. Hence,
    \begin{align*}
	\liminf E_n & := \lim\limits_{n\to\infty}U_n = \sup\set{U_n}{n\in \Zplus} \\
	\limsup E_n & := \lim\limits_{n\to\infty}V_n = \inf\set{V_n}{n\in \Zplus} \\
    \end{align*}	
    

    In probability, we denote the $\limsup E_n$ as the set $\lbrace E_n \quad \text{i.o}\rbrace$
    where the i.o stands for \textbf{infinitely often}. This is from the fact that if $\omega \in
    \limsup E_n$, then for any $n$ there is a $k$ such that $\omega \in E_k$. Thus no matter how far
    out we go, we can \emph{find} a $k$ farther such that $\omega \in E_k$. Hence, $\limsup E_n$ is
    the event that gives the point in $\Omega$ which are in infinitely many of the events.

    We denote the set $\liminf E_n$ as the set $\lbrace E_n \quad \text{a.a}\rbrace$ where the a.a
    stands for \textbf{almost always}. This is from the fact that if $\omega \in \liminf E_n$, then
    there is an $n$ such that for any $k$ greater than $n$, $\omega \in E_k$. Thus, $\liminf E_n$ is
    the event that all but a finite number of events occur. 

    If $\limsup E_n = \liminf E_n = E$, then we say that $\limit{E_n}{n}{\infty} = E$.
\end{Remark}
We will need the following result.
\begin{Proposition}\label{prop:prob_meas_limsup_liminf}
    Let $\probS$ be a probability space. If $\lbrace E_n \rbrace \subset \algebra{F}$ is a countable
    collection of events, then
    \[\probMeasure{\liminf E_n} \leq \liminf \probMeasure{E_n} \leq \limsup \probMeasure{E_n} \leq
	\probMeasure{\limsup E_n}.\]
\end{Proposition}
\begin{proof}
    Note that $(\probMeasure{E_n})$ is a sequence of positive numbers and so the middle inequality
    is trivial. First let us show that,
    \[\probMeasure{\liminf E_n} \leq \liminf \probMeasure{E_n}.\]
    Consider the set,
    \[U_n = \inf\set{E_k}{k\geq n} = \bigcap\limits_{k\geq n}E_k.\]
    Then, by~\ref{rmk:limsup_liminf_set}, $\seq{U}{n}$ is an increasing sequence, and by the properties
    of a measure (see~\ref{thm:prop_of_meas}),
    \[\probMeasure{\liminf E_n} = \probMeasure{\countUnion{U_n}{n}}
	=\lim\limits_{n\to\infty}\probMeasure{U_n}.\]
    Since $U_n \subset E_k$ for all $k \geq n$, $\probMeasure{U_n} \leq \probMeasure{E_k}$ for all
    $k\geq n$. Thus,
    \[\probMeasure{U_n} \leq \inf\set{\probMeasure{E_k}}{k\geq n}.\]
    Hence, taking limits
    \[\lim\limits_{n\to\infty} \probMeasure{U_n} \leq
	\lim\limits_{n\to\infty}\inf\set{\probMeasure{E_k}}{k\geq n} = \liminf\probMeasure{E_n}.\]
    We get the other inequality using an analogous argument.
\end{proof}

\begin{Theorem}[name=Continuity of Probability measure]\label{thm:continuity_prob_measure}
    Let $\set{E_n}{n\in\Zplus}$ be a countable collection of events in a probability space $\probS$.
    If $E = \lim\limits_{n\to\infty}E_n$, then $\limit{\probMeasure{E_n}}{n}{\infty} = \probMeasure{E}$. 
\end{Theorem}
\begin{proof}
    Since $\lim\limits_{n\to\infty}E_n = \limsup E_n = \liminf E_n$,
    using~\ref{prop:prob_meas_limsup_liminf} we get the desired result. This theorem highlights the
    continuity of the probability measure since,
    \[\limit{\probMeasure{E_n}}{n}{\infty} = \probMeasure{\lim\limits_{n\to\infty}E_n},\]
    interchanges the limit operation.
\end{proof}
\begin{Theorem}[name=First Borel-Cantelli Lemma]\label{thm:borel_cantelli_1}
    Given a probability space $\probS$ and a countable collection of events, $\lbrace E_n \rbrace
    \subset \algebra{F}$, let $E = \lbrace E_n \,\text{i.o}\rbrace$. Then,
    \[\text{If}\,\infiniteSum{\probMeasure{E_n}}{n} < \infty\quad\text{then}\, \probMeasure{E} = 0.\]
\end{Theorem}
\begin{proof}
    Let $V_n = \bigcup\limits_{k\geq n}E_k$. Then $E = \countIntersection{V_n}{n}$. Thus $E \subset
    V_n$ for every $n$. Now,
    \[\probMeasure{V_n} \leq \series{\probMeasure{E_k}}{k}{n}{\infty}.\]
    Since, $\infiniteSum{\probMeasure{E_n}}{n} < \infty$, for any $\epsilon$, there is an $N$ such
    that $\series{\probMeasure{E_k}}{k}{N}{\infty} < \epsilon$. Thus,
    \[\probMeasure{E} \leq \probMeasure{V_N} \leq 
	\series{\probMeasure{E_k}}{k}{N}{\infty} < \epsilon.\]
    Since $\epsilon$ was arbitrary we get the result.
\end{proof}
\begin{Example}[name=Run lengths]
    Consider the probability space $\probS$, where $\Omega = \hInt{0}{1}$, $\algebra{F}$ is the
    collection of Lebesgue measurable sets in $\Omega$ and $\probMeas = \mu$ is the Lebesgue
    measure. Thus we are in the Borel space of describing infinite coin tosses. For any $n$ define
    the run length function $l_n$ by,
    \begin{equation*}
	l_n(\omega) = \# \,\text{consecutive $1$'s in the binary expansion of $\omega$ starting at
	    the $n^{th}$ place}.
    \end{equation*}
    Thus $l_n(\omega) = k$ if,\[R_{n(\omega)} = 1, R_{n+1}(\omega) = 1,\ldots,R_{n+k-1} = 1,R_{n+k} =
	-1.\]
    A run length function gives us the number of consective heads in a sequence $\omega$ of
    infinite coin tosses. Take a sequence of non-negative integers $r_1,r_2,r_3,\ldots$ and let
    $E_n$ be the event that we have atleast $r_n$ consecutive heads starting at the $n^{th}$ toss. 
    Then, 
    \[\Omega_{E_n} = \set{\omega\in\Omega}{R_{n(\omega)} = 1, R_{n+1}(\omega) = 1,\ldots,R_{n+r_n-1} 
	    = 1}.\]
    Let $E = \lbrace E_n\,\text{i.o} \rbrace$. Then,
    \[P(E_n) = \measure{\Omega_{E_n}} = {(\frac{1}{2})}^{r_n}.\]
    By~\ref{thm:borel_cantelli_1}, if $\infiniteSum{{\left(\frac{1}{2}\right)}^{r_n}}{n} < \infty$, then
    $P(E) = 0$. As a concrete case take $r_n = n$. Then the probability of the event that consists
    of $n$ consecutive heads starting at the $n^{th}$ toss, infinitely often, is $0$.
\end{Example}
Next we define conditional probability and independence of events in a probability space.
\begin{Definition}[name=Conditional probability measure]
    Suppose $A,B$ are events in a probability space $\probS$ with $\probMeas(B) > 0$. The
    conditional probability (measure) of $A$ given $B$ is 
    \[\probMeasure{A|B} = \frac{\probMeasure{A\cap B}}{\probMeasure{B}}.\]
\end{Definition}
\begin{Remark}
    Fix an event $B \in \algebra{F}$. Then the the conditional probability measure is a 
    function $\map{\probMeasure{.|B}}{\algebra{F}}{\extRealsPos}$. Recall that $\algebra{F}_{B}$ is the
    restricted sigma algebra. See~\ref{ex:type_of_sigma_alg}. If we restrict the sample space to $B$, the measure space
    $\measureS{B}{\algebra{F}_{B}}{\probMeasure{.|B}}$ becomes a probability space. This is proved in the
    following theorem.
\end{Remark}
\begin{Theorem}[name=Conditioned probability space]
    Let $\probS$ be a probability space. For an given $B \in \algebra{F}$, 
    $\measureS{B}{\algebra{F}_{B}}{\probMeasure{.|B}}$ is also a probability space.
\end{Theorem}
\begin{proof}
    Note that $\algebra{F}_{B} = \set{E\cap B}{E \in \algebra{F}}$ is a $\sigmaAlgebra$.
    Clearly, $\cProbMeasure{\emptyset}{B} = 0$. Also,
    \[\cProbMeasure{B}{B} = \frac{\probMeasure{B\cap B}}{\probMeasure{B}} = 1.\]
    Now, let $\lbrace A_i \rbrace \subset \algebra{F}_{B}$ be a sequence of pairwise disjoint sets in the
    restricted sigma algebra $\algebra{F}_{B}$.
    \begin{align*}
	\cProbMeasure{(\countDisjUnion{A_i}{i})}{B} &= 
	\frac{\probMeasure{(\countDisjUnion{A_i}{i})\cap B}}{\probMeasure{B}} \\
	&= \frac{\probMeasure{\countDisjUnion{(A_i\cap B)}{i}}}{\probMeasure{B}} \\
	&= \frac{\infiniteSum{\probMeasure{(A_i\cap B)}}{i}}{\probMeasure{B}} \\
	&= \infiniteSum{\cProbMeasure{A_i}{B}}{i} \\
    \end{align*}
\end{proof}
The conditional probability measure of an event $A$ in a sample space, given the probability measure of an
event $B$ in a sample space corresponds to probability of event $A$ occuring knowing that $B$ has occured. We
give an example below to highlight this, i.e knowing that $B$ has occured, our sample space is now the
restricted probability space $\measureS{B}{\algebra{F}_{B}}{\probMeasure{.|B}}$. Hence calculating the
probability of $A$ is calculating the conditional probability measure of $A$ given $B$.
\begin{Example}\label{ex:independece_2_events}
    Let $\mathcal{B}$ be the probabilistic space of the Bernoulli sequence of infinite coin tosses. Let $A$ be
    the event that we tossed two consecutive Heads, and $B$ be the event that the first toss was head. What is
    the probability that $A$ occured given that $B$ occured? Recall that our  probability space $\probS$ 
    is given by $\Omega = \hInt{0}{1}$, $\algebra{F} = \text{Lebesgue measurable sets in $\Omega$}$, 
    and $\probMeas = \mu$, the Lebesgue measure.

    Since $B$ occured, we need to calulate $P(A|B)$. With $A$ we identify the corresponding set in the sigma
    algebra as \[\Omega_{A} = \set{\omega\in\Omega}{R_1(\omega) = 1,R_2(\omega) =1}.\]
    Hence $P(A) = \measure{\Omega_{A}} = \measure{\hInt{\frac{3}{4}}{1}} = \frac{1}{4}$.
    Similary $P(B) = \measure{\Omega_{B}} = \measure{\hInt{\frac{1}{2}}{1}} = \frac{1}{2}$.
    Thus \[P(A | B) = \measure{\Omega_A|\Omega_B} = 
	\frac{\measure{\Omega_{A} \cap \Omega_{B}}}{\measure{\Omega_{B}}} = \frac{1}{2}.\]
\end{Example}
The following theorem is very important whose proof is almost trivial.
\begin{Theorem}[name=Total probability theorem]\label{thm:total_prob_thm}
    Let $\probS$ be a probability space. If $\lbrace H_i \rbrace $ is a countable collection of pairwise
    disjoint events such that $\probMeasure{H_i} \neq 0$ for all $i$ and $\countUnion{H_i}{i} = \Omega$, 
    then for any event $A \in \algebra{F}$,
    \[\probMeasure{A} = \infiniteSum{\cProbMeasure{A}{H_i}\probMeasure{H_i}}{i}.\]
\end{Theorem}
\begin{proof}
    Since $A \subset \Omega$, we can observe that $ A \subset \countDisjUnion{H_i}{i}$. 
    Thus $A = \countDisjUnion{(A\cap H_i)}{i}$.
    Hence,
    \begin{align*}
	\probMeasure{A} &= \probMeasure{\countDisjUnion{(A\cap H_i)}{i}} \\
	& = \infiniteSum{\probMeasure{A\cap H_i}}{i} \\
	& = \infiniteSum{\cProbMeasure{A}{H_i}\probMeasure{H_i}}{i}.
    \end{align*}
\end{proof}
If knowing that $B$ occured doesn't change the probability of $A$, then we know that $A$ and $B$ are
independent. For example if $B$ is event that the first coin toss is a Head and $A$ is the event that the
second toss also a Head, then $A$ is independent of $B$. That is $P(A | B) = P(A)$.
We make this notion precise w.r.t a probability space.
\begin{Definition}[name=Independence of events]
    Given a probability space $\probS$, two events $A,B$ are independent if
    \[\probMeasure{A\cap B} = \probMeasure{A}\probMeasure{B}.\]
\end{Definition}
Note that in~\ref{ex:independece_2_events}, the events $A,B$ are NOT independent. Also, observe that the 
definition is symmetric.  
\begin{Proposition}
    If $A_1,A_2$ are independent events in a probability space $\probS$, then so are $\indxComp{A}{1}$ and
    $A_2$.
\end{Proposition}
\begin{proof}
    Note that, $A_2 = (A_2\cap A_1) \disjU (A_2 \cap \indxComp{A}{1})$. Thus,
    \begin{align*}
	&\probMeasure{A_2} = \probMeasure{A_2\cap A_1} + \probMeasure{A_2 \cap \indxComp{A}{1}} \\
	&\implies \probMeasure{A_2} - \probMeasure{A_2\cap A_1} = \probMeasure{A_2 \cap \indxComp{A}{1}} \\
	&\implies \probMeasure{A_2} - \probMeasure{A_2}\probMeasure{A_1} = 
	\probMeasure{A_2 \cap \indxComp{A}{1}} \\
	&\implies \probMeasure{A_2}(1 - \probMeasure{A_1}) = \probMeasure{A_2 \cap \indxComp{A}{1}} \\
	&\implies \probMeasure{A_2}\probMeasure{\indxComp{A}{1}} = \probMeasure{A_2 \cap \indxComp{A}{1}}
    \end{align*}
\end{proof}
We extend this idea for a collection of events.
\begin{Definition}[name=Independence of a collection of events and sigma algebras]
    A collection of events $\lbrace A_i \rbrace$ are independent if for each $n\in\Zplus$ and each selection
    of integers, $1\leq i_1 < i_2 < \cdots < i_k\leq n$,
    \[\probMeasure{A_{i_1}\cap\ldots\cap A_{i_k}} = \finiteProduct{\probMeasure{A_{i_j}}}{j}{k}.\]
    We say a collection of $\sigmaAlgebra$s $\lbrace \algebra{F}_i\rbrace$ are independent if each collection
    of events chosen individually from the $\sigmaAlgebra$s are independent.
\end{Definition}
\begin{Example}
    Three events $A,B,C$ in the probability space $\probS$ are independent if
    \begin{enumerate}
	\item
	    $\probMeasure{A\cap B} = \probMeasure{A}\probMeasure{B}$.
	\item
	    $\probMeasure{A\cap C} = \probMeasure{A}\probMeasure{C}$.
	\item
	    $\probMeasure{B\cap C} = \probMeasure{B}\probMeasure{C}$.
	\item
	    $\probMeasure{A\cap B\cap C} = \probMeasure{A}\probMeasure{B}\probMeasure{C}$.
    \end{enumerate}
\end{Example}
\begin{Example}
    Let $H_i$ be the event that the $i^{th}$ toss is a head in the space of Bernoulli sequences $\mathcal{B}$.
    Easy to check that the corresponding events $\Omega_{H_i} = \set{\omega\in\Omega}{R_i(\omega) = 1}$ 
    are independent.
\end{Example}
\begin{Proposition}
    If $\lbrace A_i\rbrace$ are independent then so are $\lbrace \indxComp{A}{i} \rbrace$.
\end{Proposition}
\begin{Theorem}[name=Second Borel-Cantelli Lemma]\label{thm:borel_cantelli_2}
    Let $\probS$ be a probability space and $\lbrace A_i \rbrace$ be a collection of independent events. Let
    $A =\lbrace A_n \,\text{i.o}\rbrace$. 
    \[\text{If}\,\infiniteSum{\probMeasure{A_i}}{i} = \infty\quad \text{then}\,\probMeasure{A} = 1.\]
\end{Theorem}
\begin{proof}
    $A = \limSupSet{A}{k}{n}$ and so $\comp{A} = \bigcup\limits_{k=1}\bigcap\limits_{n\geq k}\indxComp{A}{n}$.
    To show that $\probMeasure{A} = 1$, we will show that 
    $\probMeasure{\comp{A}} = 0$. It suffices to show that 
    $\probMeasure{\bigcap\limits_{n\geq k}\indxComp{A}{n}} = 0$ for all $k$. Due to independence,
    \[\probMeasure{\bigcap\limits_{n\geq k}^{l}\indxComp{A}{n}} =
	\prod\limits_{n=k}^{l}\probMeasure{\indxComp{A}{n}}.\]
    Now, $\probMeasure{\indxComp{A}{n}} = 1 - \probMeasure{A_n}$. Since $1-x\leq e^{-x}$,
    $\probMeasure{\indxComp{A}{n}} = 1 - \probMeasure{A_n} \leq e^{-\probMeasure{A_n}}$. Hence,
    \[\probMeasure{\bigcap\limits_{n\geq k}^{l}\indxComp{A}{n}} \leq
	\prod\limits_{n=k}^{l}e^{-\probMeasure{A_n}} = e^{-\sum\limits_{n=k}^{l}\probMeasure{A_n}}.\]
    Thus, as $\atob{l}{\infty}$, $\atob{\probMeasure{\indxComp{A}{n}}}{0}$.
\end{proof}
\begin{Example}\label{ex:borel_cantelli_2_ex1}
    Let $H_n$ be the event of a head at the $n^{th}$ toss. The corresponding event in the $\sigmaAlgebra$
    is\[\Omega_{H_n} = \set{\omega\in\Omega}{R_n(\omega) = 1}.\]
    Furthermore, $\probMeasure{\Omega_{H_n}} = \frac{1}{2}$. Since $\infiniteSum{\frac{1}{2}}{i} = \infty$,
    the probability that we see Heads infinitely often is $0$.
\end{Example}
\begin{Example}
    Any finite pattern in an infinite sequence of coin tosses occurs infinitely often with probability of $1$.
    To see a concrete case, consider the sequence $HTTH$. Let $E_n$ be the event where $HHTH$ occur starting
    at step $n$ and let $\Omega_{E_n}$ be the corresponding set in the $\sigmaAlgebra$. Then 
    \[\Omega_{E_n} = \Omega_{H_n}\cap\indxComp{\Omega}{H_{n+1}}
	\cap\indxComp{\Omega}{H_{n+2}}\cap\Omega_{H_{n+3}},\]
    where $\Omega_{H_n}$ is described in~\ref{ex:borel_cantelli_2_ex1}. From independence 
    $\probMeasure{\Omega_{E_n}} = \frac{1}{2^4}$.
    Since $\Omega_{E_n}$ and $\Omega_{E_{n+1}}$ are not independent we cannot
    apply~\ref{thm:borel_cantelli_2}. However, $\lbrace \Omega_{E_n},
    \Omega_{E_{n+4}},\Omega_{E_{n+8}},\cdots\rbrace$ are independent that satisy,
    \[\infiniteSum{\probMeasure{\Omega_{E_{4n+1}}}}{n} = \infty.\]
    So, $\probMeasure{\lbrace \Omega_{E_{4n+1}}\,\text{i.o} \rbrace } = 1$. But,
    \[\lbrace \Omega_{E_{4n+1}}\,\text{i.o} \rbrace \subset 
	\lbrace \Omega_{E_{n}}\,\text{i.o} \rbrace.\]
    Hence, $\probMeasure{\Omega_{E_{n}}\,\text{i.o} \rbrace} = 1$, since probability measure of any event
    is bounded by $1$.
\end{Example}
From now onwards, we will not be explicit in stating an underlying probabilistic process. We will denote the
probability of an event by its probability measure.
