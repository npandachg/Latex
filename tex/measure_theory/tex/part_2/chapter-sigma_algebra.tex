\chapter{Elements of measure theory: measurable sets}
In this chapter, we will define the abstract notion of measurability of sets.
%%%%%%%%%%%%%%%%%%%%%%%%%%%%%%%%%%%%%%%%%%%%%%%%%%%%%%%%%%%%%%%%%%%%%%%%%%%%%%%%%%%%%%%%%%%%%%%%%%%%%%%%%%%%%%
\section{Sigma Algebra of sets}
\begin{Definition}[name=Power set]
    Let $X$ be a non-empty set. The family of all subsets of $X$ is called the \emph{power set} of $X$
    and is denoted by $\powSet{X}$.
\end{Definition}

\begin{Definition}[name=Increasing sequence of sets]
    Let $\seq{A}{n}$ be a sequence of subsets of $X$. If $\incSetSeq{A}$ and $\countU{A} = A$, then
    we say that the $A_n$ form an increasing sequence of sets and increase to $A$. We denote this by
    $\atobUp{A}{A}$.
\end{Definition}

\begin{Definition}[name=Decreasing sequence of sets]
    Let $\seq{A}{n}$ be a sequence of subsets of $X$. If $\decSetSeq{A}$ and $\countI{A} = A$, then
    we say that the $A_n$ form an decreasing sequence of sets and decrease to $A$. We denote this by
    $\atobDown{A_n}{A}$.
\end{Definition}

Given two subsets $A,B \subset X$, we can write the union as a disjoint union as follows:
\begin{equation*}
    A \bigcup B = \left(A\right) \disjU \left(B \cap \comp{A}\right)
\end{equation*}
\begin{figure}
  \includestandalone[width=0.5\textwidth]{tex/tikz_figures/disjU}
  \caption{Disjoint union of two sets}\label{fig:tikz:disjU}
\end{figure}
Here we use $\disjU$ to denote that the union is between elements that are disjoint. 
The above observation along with the \emph{DeMorgan's} Law can be used to show the following
statements that equate an arbitrary union to disjoint union.

\begin{Remark}\label{rmk:disjU}
    Let $\seq{A}{n}$ be a sequence of subsets of $X$. Then,
    \begin{enumerate}
	\item $\finiteU{A}{n} = \left(A_1\right) \disjU \left(A_2 \cap \indxComp{A}{1}\right)
	    \dots \disjU \left(A_n\cap\indxComp{A}{n-1}\dots\cap\indxComp{A}{1}\right)$. 
	    Let us construct new sets $F_k = (A_k) \bigcap \comp{(\bigcup_{i = 1}^{k-1}
		A_i)}$ for $k > 1$ and $F_1 = A_1$. Thus $\countU{A} = \countDisjU{F}$. Note that
	    $F_k$ is just $\setDiff{A_k}{\bigcup_{i = 1}^{k-1}A_i}$. In this way,
	    we can get a sequence of pairwise disjoint sets $\seq{F}{n}$ whose countable 
	    union is the same as the countable union of the originial sequence.
	\item If $\atobUp{A_n}{A}$ then, 
	    \begin{enumerate}
		\item $ A_n = \finiteU{A}{n} = A_1 \disjU \left(\setDiff{A_2}{A_1}\right) \dots \disjU
		    \left(\setDiff{A_n}{A_{n-1}}\right)$,
		\item $A = \countU{A} = \dot\bigcup_{i \in \Zplus}\left({\setDiff{A_i}{A_{i-1}}}\right)$, 
		    where we take $A_0 = \emptyset$.
	    \end{enumerate}
	\item If $\atobUp{A_n}{A}$ then $\atobDown{\indxComp{A}{n}}{\comp{A}}$. If
	    $\atobDown{A_n}{A}$ then $\atobUp{\indxComp{A}{n}}{\comp{A}}$.
	\item We can construct a sequence $\seq{B}{n}$ such that $\atobUp{B_n}{A}$ where $A =
	    \countUnion{A_n}{n}$ as follows. $B_1 = A_1$, $B_2 = A_1 \cup A_2$ and so on, where $B_n =
	    \finiteUnion{A_k}{k}{n}$. This is an important trick to generate an increasing sequence of sets
	    whose countable unions are the same.
    \end{enumerate}
\end{Remark}

\begin{Definition}[name=Algebra of sets]
    Let $X$ be a non-empty set. An \emph{algebra} on $X$ is a \emph{non-empty} collection of sets
    $\algebra{A} \subset \powSet{X}$ with the following properties,
    \begin{enumerate}
	\item If $E \in \algebra{A}$ then $\comp{E} \in \algebra{A}$.
	\item If $E_1,E_2,\dots,E_n \in \algebra{A}$ then $\bigcup_{i=1}^{n}E_i \in \algebra{A}$.
    \end{enumerate} 
\end{Definition}


\begin{Definition}[name=Sigma-Algebra of sets]
    Let $X$ be a non-empty set. A $\sigmaAlgebra$  on $X$ is a 
    \emph{non-empty} collection of sets $\algebra{A} \subset \powSet{X}$ with the following properties,
    \begin{enumerate}
	\item If $E \in \algebra{A}$ then $\comp{E} \in \algebra{A}$.
	\item If $\seq{E}{n} \in \algebra{A}$ then $\countU{E} \in \algebra{A}$.
    \end{enumerate} 
\end{Definition}
\begin{Theorem}[name=Properties of sigma algebras]\label{thm:prop_sigmaA}
    If $\family{A}$ is a $\sigma - \text{algebra}$ on $X$ then,
    \begin{properties}
	\item If $E_1,E_2,E_3,\dots, E_n \in \algebra{A}$ then $\finiteU{E}{n} \in \algebra{A}$.
	\item $X, \emptyset \in \family{A}$.
	\item If $\seq{E}{n} \in \family{A}$ then $\countI{E} \in \family{A}$.
	\item If $E_1,E_2,E_3,\dots, E_n \in \algebra{A}$ then $\finiteI{E}{n} \in \algebra{A}$.
	\item If $A,B \in \algebra{A}$ then $\setDiff{A}{B} \in \algebra{A}$.
    \end{properties}
\end{Theorem}
Note that for an algebra the closure under countable intersection is generally not true. 
Thus a $\sigma - \text{algebra}$ on a set $X$ is an algebra that is closed under countable union and
intersection.
\begin{proof}
    We prove in order,
    \begin{enumerate}
	\item Let $E_i = E_n$ for $i > n$. Then $\finiteU{E}{n} = \countU{E} \in \algebra{A}$.
	\item Since $\algebra{A}$ is non empty there is a $E \subset X$ such that $E \in
	    \algebra{A}$. Thus $\comp{E} \in \algebra{A}$. Thus $X = E \cup \comp{E} \in
	    \algebra{A}$ from the above property. Since $X \in \algebra{A}, \comp{X} = \emptyset \in
	    \algebra{A}$.
	\item From DeMorgan's Law.
	\item Since $E_1,E_2,\dots,E_n \in \algebra{A}, \indxComp{E}{1},\dots,\indxComp{E}{n} \in
	    \algebra{A}$. From $(1), \finiteUComp{E}{n} \in \algebra{A}$. Thus its complement
	    $\finiteI{E}{n}$ is in $\algebra{A}$.
	\item From above, $\setDiff{A}{B} = A \cap {\comp{B}}$ which is in $\algebra{A}$.
    \end{enumerate}
\end{proof}
\begin{Example}\label{ex:type_of_sigma_alg}
    The following are all $\sigmaAlgebra$s.
    \begin{enumerate}
	\item $\powSet{X}$ is a $\sigmaAlgebra$. It is called the maximal $\sigmaAlgebra$ on $X$.
	\item $\lbrace \emptyset, X \rbrace$ is called the minimal $\sigmaAlgebra$ on
	    $X$.
	\item For any $B \subset X$, the collection $\lbrace \emptyset, B, \comp{B}, X\rbrace$ is a
	    $\sigmaAlgebra$.
	\item The collection $\algebra{A} = \set{A \subset X}{A \, \text{is countable or} \,
	       	\comp{A}\,\text{is countable}}$ is a $\sigmaAlgebra$.
	    \begin{proof}
		$\comp{(\comp{A})} = A$. Thus if $A \in \algebra{A}$, it is either countable or
		$\comp{A}$ is countable. Thus $\comp{A} \in \algebra{A}$. Let $\seq{A}{n} \in
		\algebra{A}$. If all are countable then the countable union is countable and so is
		in $\algebra{A}$. If not there exist an index $j \in \Zplus$ such that
		$\indxComp{A}{j}$ is countable. Thus $\comp{(\countU{A})} = \countIComp{A} \subset
		\indxComp{A}{j} \in \algebra{A}$.
	    \end{proof}
	\item (\textbf{Restricted} $\sigmaAlgebra$)
	    Let $E \subset X$ be any set and let $\algebra{A}$ be a $\sigmaAlgebra$ on $X$. Then
	    the collection $\algebra{A}_{E} = \set{E \cap A}{A \in \algebra{A}}$ is $\sigmaAlgebra$
	    on $E$.
	    \begin{proof}
		Let $E_1 \in \algebra{A}_{E}$, then $E_1 = E \cap A$ for some $A \in \algebra{A}$.
		The complement of $E_1$ in $E$ is given by $E \cap \indxComp{E}{1}$. Thus we have $E
		\cap \indxComp{E}{1} = E \cap (\comp{E} \cup \comp{A})$ and so $E \cap \indxComp{E}{1} =
		(E \cap \comp{E}) \cup (E \cap \comp{A})$. Thus $E \cap \indxComp{E}{1} = E \cap
		\comp{A} \in \algebra{A}_{E}$. Let $\seq{E}{n}$ be a sequence in $\algebra{A}_{E}$.
		Each $E_i = E \cap A_i$ for some $A_i \in \algebra{A}$. Therefore $\countU{E} =
		E \cap \countU{A} \in \algebra{A}_{E}$.
	    \end{proof}
    \end{enumerate}
\end{Example}
\begin{Theorem}[name=Equivalent Characterization of $\sigmaAlgebra$]\label{thm:eq_ch_sigmaA}
    An algebra of sets $\algebra{A}$ is a $\sigmaAlgebra$ on $X$ iff it is closed under
    complements and for any sequence of pairwise disjoint sets contained in the collection 
    their union is also contained in the collection. 
\end{Theorem}
\begin{proof}
    $\Rightarrow$ is immediately evident from the definition of $\sigmaAlgebra$. 

    $\Leftarrow$ Consider a sequence of sets $\seq{E}{n} \in \algebra{A}$. Using~\ref{rmk:disjU},
	we can construct a sequence of pairwise disjoint sets $\seq{F}{n}$, such that $\countU{A} =
	\countDisjU{F} \in \algebra{A}$. We need to $\algebra{A}$ to be an algebra so that $F_n = A_n\cap
	\indxComp{A}{1}\cap\dots\cap\indxComp{A}{n-1}$ is in $\algebra{A}$. 
\end{proof}
\begin{Definition}[name=Generated $\sigmaAlgebra$]
    Let $X$ be any non-empty set and let $\family{E} \subset \powSet{X}$. The $\sigmaAlgebra$
    generated by $\family{E}$ is the unique smallest $\sigmaAlgebra$ containing $\family{E}$ and is
    denoted by $\sigmaGen{\family{E}}$.
\end{Definition}
In set theory a \emph{smallest} set in a collection is the set that is contained in every other set
of the collection. The next theorem guarantees the existence of a smallest $\sigmaAlgebra$.
\begin{Theorem}[name=Sigma Algebra generated by an arbitrary collection]\label{thm:sigma_al_gen}
    Let $X$ be a non-empty set,
    \begin{enumerate}
	\item
	    The intersection of any collection of $\sigmaAlgebra$ on $X$ is
	    itself a $\sigmaAlgebra$.
	\item 
	    Let $\family{E} \subset \powSet{X}$. There is a unique smallest $\sigmaAlgebra$
	    $\sigmaGen{\family{E}}$ containing $\family{E}$ in the sense that any $\sigmaAlgebra$ 
	    containing $\family{E}$ must contain $\sigmaGen{\family{E}}$.
    \end{enumerate}
\end{Theorem}
\begin{proof}
    Given a non-empty set $X$,
    \begin{enumerate}
	\item
	    Let $\mathcal{G}$ be a collection of $\sigmaAlgebra$ on $X$. Then,
	    \begin{equation*}
		\algebra{F} = \bigcap_{\algebra{A} \in \mathcal{G}}\algebra{A} =
		\set{E{\subset}X}{\forEv{\algebra{A}\in\mathcal{G}},E\in\algebra{A}}.
	    \end{equation*}
	    Let $E \in \algebra{F}$. Thus $\forEv{\algebra{A} \in \mathcal{G}}$, $E \in \algebra{A}$ and
	    hence $\comp{E} \in \algebra{A}$ for all $\algebra{A} \in \mathcal{G}$. Thus $E \in
	    \algebra{F}$. Similary for a sequence of sets $\seq{E}{n}$ in $\algebra{F}$, 
	    we have that $\countU{E} \in \algebra{F}$.
	\item
	    Let $\mathcal{G} = \set{\algebra{A}\,\text{on}\,X}{\family{E}\subset\algebra{A}}$. Then
	    $\mathcal{G}$ is a collection of $\sigmaAlgebra$ on $X$.
	    Define $\sigmaGen{\family{E}} = \bigcap_{\algebra{A} \in \mathcal{G}}\algebra{A}$. From above,
	    $\sigmaGen{\family{E}}$ is a $\sigmaAlgebra$. By definition for any $\sigmaAlgebra$ $\algebra{A}$
	    on $X$ that contains $\family{E}$, $\algebra{A} \in \mathcal{G}$ and so
	    $\sigmaGen{\family{E}}
	    \subset\algebra{A}$.
    \end{enumerate}
\end{proof}
\begin{Remark}\label{rmk:obs_sigma_gen}
    We make a few observations about the generated $\sigmaAlgebra$.
    \begin{enumerate}
	\item
	    If $\family{E}$ is a $\sigmaAlgebra$, then $\family{E} = \sigmaGen{E}$.
	    \begin{proof}
		Note that $\family{E} \subset \family{E}$ and so if $\family{E}$ is a
		$\sigmaAlgebra$, from the proof above $\sigmaGen{\family{E}} \subset \family{E}$.
		That $\family{E} \subset \sigmaGen{\family{E}}$ is evident from the definition of
		$\sigmaGen{\family{E}}$.
	    \end{proof}
	\item
	    For any $A \subset X$, we have $\sigmaGen{\lbrace A \rbrace} = \lbrace \emptyset, A,
	    \comp{A},X\rbrace$.
	\item If $\family{E} \subset \family{F}$, then $\sigmaGen{\family{E}} \subset
	    \sigmaGen{\family{F}}$.
	    \begin{proof}
		$\family{E} \subset \family{F} \subset \sigmaGen{\family{F}}$. Thus
		$\sigmaGen{\family{E}} \subset \sigmaGen{\family{F}}$.
	    \end{proof}
    \end{enumerate}
\end{Remark}
A very important principle to prove statements about generated sigma algebra is called the
\emph{principle of good sets} which is described in the following remark:
\begin{Remark}[name=Principle of good sets]\label{rmk:prin_good_sets}
    Suppose we want to show that for any collection $\family{E}$, the generated sigma algebra
    $\sigmaGen{\family{E}}$ has a certain property $P$. IF we can find a sub-collection $\family{F}
    \subset \sigmaGen{\family{E}}$ that has this property $P$, and show that $\family{F}$ is a
    $\sigmaAlgebra$, then $\sigmaGen{\family{E}}$ has property $P$, provided
    $\family{E}\subset\family{F}$.
\end{Remark}
We can generate new $\sigmaAlgebra$ from existing $\sigmaAlgebra$ using inverse maps.
\begin{Theorem}[name=Pre-Image $\sigmaAlgebra$]\label{thm:pre_img_sigma}
    Let $\map{f}{X}{Y}$ be a function.
    \begin{enumerate}
	\item
	    If $\algebra{A}$ is a $\sigmaAlgebra$ on $Y$, then
	    \begin{equation*}
		\invIm{f}{\algebra{A}} := \set{\invIm{f}{A}}{A \in \algebra{A}}
	    \end{equation*}
	    is a $\sigmaAlgebra$ on $X$.
	\item
	    If $\algebra{B}$ is a $\sigmaAlgebra$ on $X$, then
	    \begin{equation*}
		\set{A\subset Y}{\invIm{f}{A}\in\algebra{B}}
	    \end{equation*}
	    is a $\sigmaAlgebra$ on $Y$.
	\item
	    If $\family{E}$ is a collection of sets in $Y$, then
	    \begin{equation*}
		\sigmaGen{\invIm{f}{\family{E}}} = \invIm{f}{\sigmaGen{\family{E}}}
	    \end{equation*}
    \end{enumerate}
\end{Theorem}
\begin{proof}
    Given a function $\map{f}{X}{Y}$, 
    \begin{enumerate}
	\item Let $E \in \invIm{f}{\algebra{A}}$. Thus $E = \invIm{f}{A}$ for some $A \in
	    \algebra{A}$. $\comp{E} = \comp{(\invIm{f}{A})} = \invIm{f}{\comp{A}}$ and 
	    thus $\comp{E}\in\invIm{f}{\algebra{A}}$. Let $\seq{E}{n}$ be a sequence of sets in
	    $\invIm{f}{\algebra{A}}$. Then $E_i = \invIm{f}{A_i}$. Thus $\countU{E} =
	    \bigcup_{i\in\Zplus}\invIm{f}{A_i} = \invIm{f}{\countU{A}} \in \invIm{f}{\algebra{A}}$.
	\item Let $E$ be in the collecion, hence, $E \subset Y$ such that $\invIm{f}{E} \in 
	    \algebra{B}$, hence $\invIm{f}{\comp{E}} = \comp{(\invIm{f}{E})} \in \algebra{B}$. 
	    Thus $\comp{E}$ is in the collection. Similar argument for countable union.
	\item
	    We need to show $\sigmaGen{\invIm{f}{\family{E}}} \subset 
	    \invIm{f}{\sigmaGen{\family{E}}}$ and $ \invIm{f}{\sigmaGen{\family{E}}} \subset 
	    \sigmaGen{\invIm{f}{\family{E}}}$.
	    
	    For any $A \in \family{E}$, $A \in \sigmaGen{\family{E}}$ and hence
	    $\invIm{f}{\family{E}} \subset \invIm{f}{\sigmaGen{\family{E}}}$. But from
	    $(1)$, we know that $\invIm{f}{\sigmaGen{\family{E}}}$ is a $\sigmaAlgebra$, and
	    thus $\sigmaGen{\invIm{f}{\family{E}}} \subset
	    \invIm{f}{\sigmaGen{\family{E}}}$.

	    The other direction is tricky i.e.
	    \[\invIm{f}{\sigmaGen{\family{E}}} \subset \sigmaGen{\invIm{f}{\family{E}}}.\] 
	    We will use~\ref{rmk:prin_good_sets}; that is, we will construct a sub-collection of 
	    $\invIm{f}{\sigmaGen{\family{E}}}$ which is a subset of $\sigmaGen{\invIm{f}{\family{E}}}$. What
	    is a sub-collection of $\invIm{f}{\sigmaGen{\family{E}}}$ that satisfies this property? 
	    It is precisely a collection of some sets $A$ in
	    $\sigmaGen{\family{E}}$ for which $\invIm{f}{A}$ is in $\sigmaGen{\invIm{f}{\family{E}}}$.
	    Thus consider the collection of good sets $\family{F}$, such that
	    \begin{equation*}
		\family{F} = \set{A \in \sigmaGen{\family{E}}}{\invIm{f}{A} \in 
		    \sigmaGen{\invIm{f}{\family{E}}}}
	    \end{equation*}
	    What we want to show
	    is that for any $A \in \sigmaGen{\family{E}}$ we should have,
	    \begin{equation*}
		\invIm{f}{A} \in \sigmaGen{\invIm{f}{\family{E}}}.
	    \end{equation*}
	    Clearly $\family{F}$ is not empty because $\family{E} \subset \family{F}$. This
	    is easily observed because $\invIm{f}{\family{E}} \subset
	    \sigmaGen{\invIm{f}{\family{E}}}$. If we show that $F$ is a $\sigmaAlgebra$ we
	    are done.
	    Let $E\in \family{F}$. Therefore $\invIm{f}{E} \in
	    \sigmaGen{\invIm{f}{\family{E}}}$ but
	    this means that $\comp{E} \in \family{F}$ since $\invIm{f}{\comp{E}} =
	    \comp{(\invIm{f}{E})} \in \sigmaGen{\invIm{f}{\family{E}}}$. Similary for 
	    countable unions. Thus $\family{F}$ is a $\sigmaAlgebra$ containing $\family{E}$ 
	    and so $\sigmaGen{\family{E}} \subset \family{F}$. By construction,
	    $\family{F}\subset\sigmaGen{\famE}$ and so $\sigmaGen{\famE} = \famF$ and hence we are done.
    \end{enumerate}
\end{proof}
Of particular interest is the situation involving a metric space, for example $\Rn$. In this case,
we have a built in natural concept of distance plus the associated topology of sets, for example 
open sets, closed sets etc.

\begin{Definition}[name=Borel Sets]
    Let $\metricS{X}{d}$ be a metric space. The $\sigmaAlgebra$ generated by the collection of open
    sets in $X$ is called the \emph{Borel} $\sigmaAlgebra$ on $X$ and is denoted by $\borelS{X}$.
    Its members are called the \emph{Borel sets}.
\end{Definition}
We can equivalently generate the Borel $\sigmaAlgebra$ using the family of closed sets. The Borel
sets include open sets, countable union an intersection of open sets, closed sets, countable union
and intersection of closed sets and so on. 
\begin{Definition}
    A countable intersection of open sets is called a $G_{\delta}$ set, a countable union of closed
    sets is called a $F_{\sigma}$ set, a countable union of $G_{\delta}$ set is called
    $G_{\delta\sigma}$ and so on.
\end{Definition}
The Borel $\sigmaAlgebra$ on $\Rn$, $\borelS{\Rn}$ is particularly important. When $n = 1$, we get
the Borel $\sigmaAlgebra$ on the real line, $\borelS{\R}$. The most interesting generators are the
family of open and half-open rectangles,

\begin{align*}
    &\family{J}_{n} = \set{\interval{a_1}{b_1} \times \dots \interval{a_n}{b_n}}{a_j,b_j \in \R}, \\	
    &\family{J}^{o}_{n} = \set{(a_1,b_1) \times \dots (a_n,b_n)}{a_j,b_j \in \R}, \\	
    &\family{J}^{cr}_{n} = \set{\interval[open left]{a_1}{b_1} \times \dots 
	\interval[open left]{a_n}{b_n}}{a_j,b_j \in \R}, \\	
    &\family{J}^{cl}_{n} = \set{\interval[open right]{a_1}{b_1} \times \dots 
	\interval[open right]{a_n}{b_n}}{a_j,b_j \in \R}. \\	
    &\family{J}^{o}_{\Q_{n}} = \set{(a_1,b_1) \times \dots 
	(a_n,b_n)}{a_j,b_j \in \Q}. \\	
\end{align*}
Let us denote by $\family{G},\family{F},\family{K}$, the collection of open, closed and compact sets
in $\Rn$. The following theorem is very useful in realizing the generating sets of $\borelS{\Rn}$.
\begin{Theorem}[name=Generating Borel Sets in $\Rn$]\label{thm:gen_borel_rn}
    We have,
    \begin{enumerate}
	\item 
	    $\borelS{\Rn} = \sigmaGen{\family{G}}$.
	\item
	    $\borelS{\Rn} = \sigmaGen{\family{F}}$.
	\item
	    $\borelS{\Rn} = \sigmaGen{\family{K}}$.
	\item
	    $\borelS{\Rn} = \sigmaGen{\family{J}^{o}_{n}}$.
	\item
	    $\borelS{\Rn} = \sigmaGen{\family{J}^{cr}_{n}}$.
	\item
	    $\borelS{\Rn} = \sigmaGen{\family{J}^{cl}_{n}}$.
	\item
	    $\borelS{\Rn} = \sigmaGen{\family{J}_{n}}$.
    \end{enumerate}
\end{Theorem}
\begin{proof}
    $(1)$ is just the definition of Borel sets. For any open set $U \in \family{G}$, $\comp{U} \in
    \family{F}$. Thus $U \in \sigmaGen{F}$ i.e $\family{G} \subset \sigmaGen{\family{F}}$ and thus
    $\sigmaGen{\family{G}} \subset \sigmaGen{\family{F}}$. A similar argument with roles reversed 
    leads to $\sigmaGen{\family{F}} \subset \sigmaGen{\family{G}}$. Any compact set $K$ is also a
    closed set so $\sigmaGen{\family{K}} \subset \sigmaGen{\family{F}}$. Let us write a closed set
    as a countable union of compact sets in $\Rn$. If $F \in \family{F}$, then let $F_i = F \cap
    \closure{\ball{i}{\vect{0}}}, i\in \Zplus$. Each $F_i$ is an intersection of closed sets and 
    is bounded and hence is compact in $\Rn$. Moreover,
    \begin{equation*}
	F = \countU{F}.
    \end{equation*}	
    Thus $F \in \sigmaGen{\family{K}}$ and so $\sigmaGen{\family{F}} \subset \sigmaGen{\family{K}}$.
    Thus we have shown the following,
    \begin{equation*}
	\borelS{\Rn} = \sigmaGen{\family{G}} = \sigmaGen{\family{K}} = \sigmaGen{\family{F}}.
    \end{equation*}
    Let $R \in \family{J}^{o}_{n}$ be a open rectangle (box) in $\Rn$. Any open rectangle is an open
    set, thus 
    $\sigmaGen{\family{J}^{o}_{\Q_{n}}} \subset \sigmaGen{\family{J}^{o}_{n}} 
    \subset \sigmaGen{\family{G}}$. Now for the other direction consider any open set $U \in
    \family{G}$. $U = \bigcup\limits_{p\in U}\ball{\epsilon_{p}}{p}$. But for any open ball we can
    inscribe a rectangle with rational endpoints in it and so,
    \begin{equation*}
	U = \bigcup\limits_{R \in \family{J}^{o}_{\Q_{n}};R \subset U}R
    \end{equation*}
    Thus $\family{G} \subset \sigmaGen{\family{J}^{o}_{\Q_{n}}}$ and so we get the following,
    \begin{equation*}
	\borelS{\Rn} = \sigmaGen{\family{J}^{o}_{n}}
    \end{equation*}
    Let $R^{cr} \in \family{J}^{cr}_{n}$ be a half-open rectangle in $\Rn$ closed at right.
    Thus, $R^{cr} = \interval[open left]{a_1}{b_1} \times \dots 
    \interval[open left]{a_n}{b_n}$ which can be written as,
    \begin{equation*}
	\interval[open left]{a_1}{b_1} \times \dots 
	\interval[open left]{a_n}{b_n} = \bigcap\limits_{j\in\Zplus}(a_1,b_1+1/j) \times \dots
	(a_n,b_1+1/j).
    \end{equation*}	    
    Thus, $\family{J}^{cr}_{n} \subset \sigmaGen{\family{J}^{o}_{n}}$. Let $R \in 
    \family{J}^{o}_{n}$ be a open rectangle (box) in $\Rn$. 
    Thus, $R = (a_1,b_1) \times \dots (a_n,b_n)$ which can be written as,
    \begin{equation*}
	(a_1,b_1)\times\dots(a_n,b_n) = \bigcup\limits_{j\in\Zplus}
	\interval[open left]{a_1}{b_1-1/j}\times\dots\interval[open left]{a_n}{b_1-1/j}.
    \end{equation*}
    Thus,$\family{J}^{o}_{n} \subset \sigmaGen{\family{J}^{cr}_{n}}$. Hence we have shown,
    \begin{equation*}
	\sigmaGen{\family{J}^{o}_{n}} = \sigmaGen{\family{J}^{cr}_{n}}
    \end{equation*}
    A similar argument can be done for the family $\family{J}^{cl}_{n}$.
\end{proof}
\begin{Corollary}\label{thm:gen_borel_R}
    $\borelS{\R}$ is generated by each of the following:
    \begin{enumerate}
	\item
	    (Open intervals): $\family{E}_1 = \set{(a,b)}{a < b}$.
	\item
	    (closed intervals): $\family{E}_2 = \set{\interval{a}{b}}{a < b}$.
	\item
	    (half open closed right intervals): $\family{E}_3 = \set{\hInt{a}{b}}{a < b}$
	\item
	    (half open closed left intervals): $\family{E}_4 = \set{\interval[open right]{a}{b}}{a < b}$
	\item
	    (open rays): $\family{E}_5 = \set{(a,\infty)}{a \in \R}$
	\item
	    (open rays): $\family{E}_6 = \set{(-\infty,a)}{a \in \R}$
	\item
	    (closed rays): $\family{E}_7 = \set{\interval[open right]{a}{\infty}}{a \in \R}$
	\item
	    (closed rays): $\family{E}_8 = \set{\hInt{a}{\infty}}{a \in \R}$
    \end{enumerate}
\end{Corollary}
Before moving onto the subject of measures, we'll give a technical result that will be useful in
constructing measure for the Borel sets. The following just abstracts the class of half open
rectangles.
\begin{Definition}[name=Elementary family]
    Let $X$ be a non-empty set. An elementary family is a collection $\family{E}$ of subsets of $X$
    such that,
    \begin{enumerate}
	\item
	    $\emptyset \in \family{E}$.
	\item
	    If $E,F \in \family{E}$ then $\intersection{E}{F} \in \family{E}$.
	\item
	    If $E \in \family{E}$, then $\comp{E}$ is a finite disjoint union of members of
	    $\family{E}$.
    \end{enumerate}
\end{Definition}
\begin{Example}\label{ex:hint_elem_fam}
    Consider the family $\family{J}^{cr}_{1}$ of half-open closed right interval in $\R$. If we
    include $-\infty,\infty$ then we refer to sets of this family as \textbf{h-intervals}. These are
    collection of subsets of $\R$ of the form $\interval[open left]{a}{b}, (a,\infty), \emptyset$. 
    $\family{J}^{cr}_{1}$ is an elementary family. To check the last condition of the definition,
    consider the set $\interval[open left]{a}{b}$. Its complement is set $\interval[open
    left]{-\infty}{a} \cup (b,\infty)$ which is the disjoint union of two \textbf{h-intervals}. 
\end{Example}
\begin{Theorem}[name=Constructing an algebra from elementary family]\label{thm:const_algebra_elem}
    If $\family{E}$ is an elementary family, the collection, $\algebra{A}$ of finite disjoint unions
    of members of $\family{E}$ is an algebra.
\end{Theorem}
\begin{proof}
    First note that $\algebra{A}$ is not empty since $\family{E} \subset \algebra{A}$. This is
    because any $E \in \family{E}$ can be written as the disjoint union of $E \disjU \emptyset$.

    Let $A \in \algebra{A}$. Then $A = E_1 \disjU E_2 \disjU \dots \disjU E_n$. We need to show that
    $\comp{A} \in \algebra{A}$. Note that $\comp{A} = \indxComp{E}{1} \cap \indxComp{E}{2} \cap
    \dots \cap \indxComp{E}{n}$. However each $\indxComp{E}{m} = E^{m}_{1} \disjU \dots \disjU
    E^{m}_{n_m}$. Therefore $\comp{A}$ is,
    \[\bigcap\limits_{m =1}^{n}\disjU\limits_{j=1}^{n_m}E^{m}_{j} = 
	\disjU\set{E_{j_1}^{1}\cap\dots\cap E_{j_n}^n}{1\leq j_m \leq n_m,\, 1\leq m \leq n},\]
    which is a disjoint union of finite sets in $\family{E}$ and hence in $\algebra{A}$.

    Let us show that $\algebra{A}$ is closed under union of two sets. By induction we will get
    closure of finite union.
    Let $A,B \in \algebra{A}$. Then $A = \finiteDisjU{C_i}{n_A}$ and $B = \finiteDisjU{D_i}{n_B}$. We
    need to show that $A\cup B \in \algebra{A}$ i.e the following set,
    \[A \cup B = \disjU\set{C_{j_A}\cup D_{j_B}}{1\leq j_A \leq n_A; 1\leq j_B \leq n_B}.\]

    Each $C_{j_A}\cup D_{j_B}$ belong to $\algebra{A}$. 
    To see this, let us show that if $C,D \in \family{E}$ then $C \cup D \in \algebra{A}$.
    Note that, $\comp{D} = \finiteDisjU{E_i}{J}$. 
    Thus $C \cap \comp{D} = \finiteDisjU{C\cap E_i}{J}$. Since
    $C,E_i \in \family{E}$ we get $C \cap \comp{D} \in \algebra{A}$. But $C \cup D = (C \cap
    \comp{D}) \disjU (D)$. Since $(C \cap \comp{D})$ is the disjoint union of sets in $\family{E}$,
    $C \cup D$ is then a disjoint union of sets in $\family{E}$ and hence $C\cup D \in \algebra{A}$.
    Thus from induction if $A_1, A_2, \dots, A_n \in \family{E}$ then $\finiteU{A}{n} \in
    \algebra{A}$. 
    Hence, each $C_{j_A}\cup D_{j_B}$ belong to $\algebra{A}$.
    Thus by induction on $n_A \times n_B$ we get
    the result. Now we can induct on a finite sequence of sets $A_i \in \algebra{A}$ to show that
    $\algebra{A}$ is closed under finite unions. 
\end{proof}
\begin{Remark}\label{rmk:hinterval}
    Note that if $\family{E}$ is the collection $\family{J}^{cr}_{1}$, then the collection
    $\algebra{A}$ of finite disjoint unions of \textbf{h-intervals} is an algebra
    by~\ref{thm:const_algebra_elem}. But note that $\family{E} \subset \family{A}$ and thus
    $\sigmaGen{\family{E}} \subset \sigmaGen{\family{A}}$. However, 
    $\sigmaGen{\family{A}} = \borelS{\R}$ which is also equal to $\sigmaGen{\family{E}}$
    from~\ref{thm:gen_borel_rn}.
\end{Remark}

We close this section by looking at a few more important family of sets.
\begin{Definition}[name=Monotone Class]
    Let $X$ be a non-empty set. A monotone class $\famE$ is a non-empty collection of subsets of $X$ which is
    closed under monotone sequence of sets, i.e.
    \begin{enumerate}
	\item If $\seq{A}{n}$ is a sequence of subsets of $X$ in $\famE$ such that $A_i \subset A_{i+1}$ for
	    each $i$, then $\countUnion{A_n}{n} \in \famE$.
	\item If $\seq{A}{n}$ is a sequence of subsets of $X$ in $\famE$ such that $A_i \supset A_{i+1}$ for
	    each $i$, then $\countIntersection{A_n}{n} \in \famE$.
    \end{enumerate}
\end{Definition}
\begin{Proposition}
    An algebra $\famA$ is a sigma algebra if and only if it is a monotone class.
\end{Proposition}
\begin{proof}
    Let $\famA$ be a sigma algebra. Then since it is closed under countable unions, $\famA$ is also a
    monotone class. 

    Let $\famA$ be a monotone class and consider a sequence $\seq{A}{n}$. From remark~\ref{rmk:disjU}, we can
    construct an increasing sequence $\seq{B}{n}$ such that $\countUnion{B_n}{n} = \countUnion{A_n}{n}$. Since
    $\famA$ is a monotone class, we get $\countUnion{A_n}{n} \in \famA$. Hence $\famA$ is a sigma-algebra.
\end{proof}

Given a non-empty collection of sets $\famE$, the minimal sigma algebra containing $\famE$, is definied to be
the sigma algebra generated by $\famE$. Similarly, we can talk about the minimal monotone class containing
$\famE$ as the intersection of all monotone classes containing $\famE$. We denote this by $\mcalM(\famE)$. The
next theorem is one of a type called monotone class theorems and its technique is quite illustrative of a
useful proof technique in measure theory using remark~\ref{rmk:prin_good_sets}.
\begin{Theorem}
    Let $\famE$ be an algebra. Then, $\mcalM(\famE) = \sigmaGen{\famE}$.
\end{Theorem}
\begin{proof}
    We need to show $\sigmaGen{\famE} \subset \mcalM(\famE)$ and $\mcalM(\famE) \subset \sigmaGen{\famE}$.

    Since $\famE$ is a field, $\sigmaGen{\famE}$ is a monotone class containing $\famE$, we get from minimality,
    $\mcalM(\famE) \subset \sigmaGen{\famE}$.

    For the other direction, if we show that $\mcalM(\famE)$ is a sigma algebra containg $\famE$, we are done,
    but this amounts to showing that $\mcalM(\famE)$ is an algebra.
    Thus, we need to show: $\mcalM(\famE)$ is an algebra, that is for any $A \in \mcalM(\famE)$, $\comp{A}\in
    \mcalM(\famE)$ and for any $A,B \in \mcalM(\famE)$, $A\cup B \in \mcalM(\famE)$. We will use the good sets
    principle.
    Let us define the following property that we want $\mcalM(\famE)$ to satisfy.
    \[P_1(A) := \comp{A}\in \mcalM(\famE) \text{and} \hspace{0.05in} \forEv{B\in\mcalM(\famE)}A\cup B \in
	\mcalM(\famE).\]
    Let,
    \[\famF_{1} = \set{A\in\mcalM(\famE)}{P_1(A)}.\]
    By definition, $\famF_{1}$ is an algebra and $\famF_{1}\subset\mcalM{(\famE)}$. Thus, we need to show that
    $\famF_{1} \supset \mcalM{(\famE)}$, which is equivalent to showing that 
    \begin{enumerate}
	\item
	    $\famF_{1} \supset \famE$ and,
	\item
	    $\famF_{1}$ is a monotone class.
    \end{enumerate}
    To show $(2)$, let $\seq{A}{n}$ be an increasing sequence of sets in $\famF_{1}$ and so
    $\countUnion{A_n}{n} \in \mcalM(\famE)$. Let $B$ be any set in $\mcalM(\famE)$, then $A_i \cup B$ is in
    $\famF_{1}$. Hence, $(A_i\cup B)$ is an increasing sequence in $\mcalM(\famE)$ and so 
    \[ \left(\countUnion{A_i}{i}\right) \cup B = \countUnion{(A_i\cup B)}{i} \in \mcalM(\famE).\]
    Thus, $\countUnion{A_n}{n}\in \famF_{1}$. Also $\indxComp{A}{1} \supset \indxComp{A}{2} \dots$, and so,
    \[\comp{(\countUnion{A_n}{n})} = \countIntersection{\indxComp{A}{n}}{n} \in \mcalM(\famE).\]
    Thus, $(2)$ is satisfied.

    Showing $(1)$ is tricky and hence we will show,
    \[\famF_1 \supset \famF_2 \supset \famE,\]
    for a suitable $\famF_2$ such that $\famF_2$ is monotone. 
    This would mean that $\famF_1 \supset \famF_2 \supset \mcalM(\famE)$.
    To construct $\famF_2$, we will define a new property:
    \[P_2(A) := \comp{A}\in\mcalM(\famE) \text{and} \hspace{0.05in} 
	\forEv{B\in\famE}A\cup B \in\mcalM(\famE).\]

    Let,
    \[\famF_{2} = \set{A\in\mcalM(\famE)}{P_2(A)}.\]
    Since, $\famE$ is an algebra, $\famE \subset \famF_{2} \subset \famF_{1}$. Moreover, by similar reasoning
    $\famF_{2}$ is a monotone class and so we get the result.
\end{proof}

\begin{Definition}[name=$\pi$ class]
    Let $X$ be a non-empty set and let $\famE$ be a collection of subsets of $X$ such that:
    \begin{enumerate}
	\item For any $A,B \in \famE$, $A\cap B \in \famE$,
    \end{enumerate}
    then $\famE$ is called a $\pi-$ class.
\end{Definition}
\begin{Definition}[name=$\lambda$ class]
    Let $X$ be a non-empty set. The collection of subsets of $X$, $\famE$ is called a $\lambda-$ class 
    (system) if,
    \begin{enumerate}
	\item
	    $X$ is in $\famE$.
	\item
	    $A \in \famE$ imples $\comp{A}\in \famE$.
	\item
	    If $(A_n)$ is a sequence of pairwise disjoint sets in $\famE$, then $\countUnion{A_n}{n}$ is in
	    $\famE$. 
    \end{enumerate}
\end{Definition}
It is easy to see that a $\lambda-$ class that is also a $\pi-$ class is a sigma algebra.
See~\ref{thm:eq_ch_sigmaA}. 
An equivalent definition of the $\lambda-$ class due to Dynkin is given as:
\begin{Definition}[name= $D$ class]
    Let $X$ be a non-empty set. The collection of subsets of $X$, $\famD$ is called a $D-$ class
    if,
    \begin{enumerate}
	\item
	    $X$ is in $\famE$.
	\item
	    $A,B \in \famE$ and $A \supset B$ imples ${A-B}\in \famE$.
	\item
	    If $(A_n)$ is a sequence of pairwise disjoint sets in $\famE$, then $\countUnion{A_n}{n}$ is in
	    $\famE$. 
    \end{enumerate}

\end{Definition}
It is easy to see that a $D-$ class is equivalent to a $\lambda-$ class. Suppose $\famA$ is a $D$ class and
consider $A \in \famA$. We know that $X$ is in $\famA$ and since $X \supset A$, $\comp{A} = X-A$ is in
$\famA$. Thus a $D-$ class is a $\lambda-$ class. Suppose $\famA$ is a $\lambda$ class, and consider $A,B \in
\famA$ such that $A \supset B$. Then, $A-B = A\cap\comp{B} = \comp{\comp{A}\cup B}$. Since $A \supset B$,
$\comp{A}$ and $B$ are disjoint and so $\comp{A}\cup B$ is in $\famA$ and hence its complement is in $\famA$.

If $\famE \subset \powSet{X}$ is non-empty, then the smallest $\lambda-$ class generated by $\famE$ is the
intersection of all the $\lambda-$ classes containing $\famE$ and is denoted by $\mcalD(\famE)$, where
$\mcalD$ stands for Dynkin.
\begin{Lemma}
    Let $X$ be a non-empty set and let $\famE \subset \powSet{X}$ be a non-empty collection of subsets of $X$.
    Then $\mcalD(\famE) \subset \sigmaGen{\famE}$.
\end{Lemma}
\begin{proof}
    $\sigmaGen{\famE}$ is a $\lambda-$ class containing $\famE$ and so by minimality contains $\mcalD(\famE)$.
\end{proof}
\begin{Theorem}
    Let $X$ be a non-empty set and let $\famE \subset \powSet{X}$ be a non-empty collection of subsets of $X$.
    If $\famE$ is a $\pi-$ class, then $\mcalD(\famE) = \sigmaGen{\famE}$.
\end{Theorem}
\begin{proof}
    We need to show $\mcalD(\famE) \subset \sigmaGen{\famE}$ and $\sigmaGen{\famE}\subset \mcalD(\famE)$.
    Since, a sigma-algebra is a $\lambda-$ class we immediately get $\sigmaGen{\famE} \supset \mcalD(\famE)$.
    For the other direction we will use~\ref{rmk:prin_good_sets} along with the ideas in the proof used for
    the monotone class.

    We need to show $\mcalD(\famE)$ is a sigma algebra containing $\famE$ and this amounts to showing that
    $\mcalD(\famE)$ is a $\pi-$ class. Thus, consider the sub-collection:
    \[\famF = \set{A \in \mcalD(\famE)}{\forEv{B\in\mcalD} A\cap B \in \mcalD(\famE)}.\]
    By construction $\famF \subset \mcalD(\famE)$ and is a $\pi-$ class. If we show that 
    $\famF \supset \famE$ and $\famE$ is a $\lambda-$ class we are done. 

    It is clear that $X$ is in $\famF$. Let $A$ be in $\famF$, and let $B$ be any set in $\mcalD(\famE)$. 
    Then,
    \[\comp{A}\cap B = \comp{(A\cup \comp{B})}.\]
    Since, $A \in \famF$, $A \in \mcalD(\famE)$ and since $B \in \mcalD(\famE), \comp{B}\in\mcalD(\famE)$. Thus,
    $\comp{(A\cup \comp{B})}\in\mcalD(\famE)$. Hence for any $B \in \mcalD(\famE)$, $\comp{A}\cap B \in \famF$
    whenever $A \in \famF$. Let $\seq{A}{n}$ be a sequence of sets in $\famE$ that are pairwise disjoint
    and let $B$ be any set in
    $\mcalD(\famE)$. Then, 
    \[(\countUnion{A_n}{n})\cap B = \countUnion{(A_n\cap B)}{n}.\]
    Since $\famF \subset \mcalD(\famE)$, $A_n\cap B \in \mcalD{\famE}$ for each $n$ and so the countable union
    is in $\mcalD(\famE)$.

    To show that $\famF \supset \famE$, we will construct a family $\famF_{2}$ such that $\famF \supset
    \famF_{2}\supset \famE$ and $\famF_{2}$ is a $\lambda-$ class.
    Let \[\famF_{2} = \set{A \in \mcalD(\famE)}{\forEv{B\in\famE} A\cap B \in \mcalD(\famE)}.\]
    Since $\famE$ is a $\pi-$ class $\famE \subset \famF_{2}$. If we show $\famF_{2}$ is a $\lambda-$ class,
    it would mean that $\famF_2 \supset \mcalD(\famE)$. But since by construction, $\famF_2 \subset \famF$ we
    get $\famF \supset \mcalD(\famE) \supset \famE$. Proving that $\famF_2$ is a $\lambda-$ class follows the
    same reasonings as we showed for $\famF$.

\end{proof}
\begin{Theorem}[name=$\pi-\lambda$ Theorem]
    If $\famE$ is a $\pi-$ class and $\famL$ is a $\lambda-$ class, then $\famE \subset \famL$ implies,
    $\sigmaGen{\famE}\subset \famL$.
\end{Theorem}
\begin{proof}
    Consider the minimal $\lambda-$ class containing $\famE$, $\mcalD(\famE)$. By definition, 
    $\mcalD(\famE)\subset \famL$. We showed that $\sigmaGen{\famE} = \mcalD(\famE)$ and thus, 
    $\sigmaGen{\famE} \subset\famL$.
\end{proof}
\section{Measures}
We now consider how to measure the size of the sets in a given $\sigmaAlgebra$ on a set $X$.
\begin{Definition}[name=Measure]
    Let $X$ be a set on which there is a $\sigmaAlgebra$ $\algebra{M}$. A measure on the measurable
    space $\metricS{X}{\algebra{M}}$ is a function \[\map{\mu}{\algebra{M}}{\interval{0}{\infty}}\]
    satisfying,
    \begin{enumerate}
	\item
	    $\measure{\emptyset} = 0$,
	\item
	    ($\sigma$-additivity) If $\seq{E}{n}$ be a sequence of pairwise disjoint sets in 
	    $\algebra{M}$, then
	    \begin{equation*}
		\measure{\countDisjU{E}} = \sumInf{\measure{E_i}}.
	    \end{equation*}
    \end{enumerate}
\end{Definition}
A \emph{pre-measure} is function that satisfies the above on an \emph{algebra} and not necessarily
on a $\sigmaAlgebra$. Note that for a \emph{pre-measure}, additivity is statisfied if the countable
union of a sequence of sets is in the algebra. We will denote a pre-measure by $\preMeas$. 
The triple $\measureS{X}{\algebra{M}}{\mu}$ is called a \emph{measure space}.
A \emph{finite measure} is a measure with $\measure{X} < \infty$, and a \emph{probability measure}
is a measure with $\measure{X} = 1$. Note that for a finite measure, for any $E \in \algebra{M}$,
$\measure{E} < \infty$. A measure is said to be $\sigmaFinite$, if $\algebra{M}$
contains a sequence $\incSetSeq{A}$ such that $\countU{A} = X$ and $\measure{A_i} < \infty$ for
every $i \in \Zplus$. Most measures encountered in practice are at least $\sigmaFinite$, and
non-$\sigmaFinite$ measures have some strange behaviour. A measure $\mu$ is $\sigma$ semi-finite if
any set $F \in \algebra{M}$ such that $\measure{F} = \infty$ there is a set $E \in \algebra{M}$ and
$E \subset F$ such that $\measure{E} < \infty$.

\begin{Definition}[name=Finitely additive measure]
    If $\seq{E}{i}$ are disjoint sets in $\algebra{M}$, and $\measure{\finiteDisjU{E}{n}} =
    \sumFinite{\measure{E_i}}{n}$, then $\mu$ is finitely additive measure.
\end{Definition}
Note that a pre-measure will always satisfy finite additivity.
\begin{Example}
    Let $X$ be an uncountable set and let $\algebra{M}$ be the $\sigmaAlgebra$,
    \begin{equation*}
	\set{A \subset X}{A \, \text{is countable or} \,
	    \comp{A}\,\text{is countable}}
    \end{equation*}
    Then the following set function defined on $\algebra{M}$,
    \begin{equation*}
	\measure{E} =
	\begin{cases}
	    1 &\text{E is countable}\\
	    0 &\text{$\comp{E}$ is countable}
	\end{cases}
    \end{equation*}
    is a finitely additive measure but is not countably additive.
\end{Example}
We collect all the important properties of a measure in the following theorem. The following
properties will also be satisfied by a pre-measure provided we check that the countable union and
intersection of a sequence of sets belong to the algebra.
\begin{Theorem}[name=Properties of measure]\label{thm:prop_of_meas}
    Let $\measureS{X}{\algebra{M}}{\mu}$ be a measure space. Then,
    \begin{properties}
	\item 
	    (finite-additivity) $E,F \in \algebra{M}$ and $E \cap F = \emptyset$, then 
	    \[\measure{E\disjU F} = \measure{E} + \measure{F}.\]
	\item
	    (monotonicity) If $E,F \in \algebra{M}$ and $E \subset F$ then 
	    \[\measure{E} \leq \measure{F}.\]
	    Moreover if $\measure{F} < \infty$, then 
	    \[\measure{\setDiff{F}{E}} = \measure{F} - \measure{E}.\]
	\item
	    (sub-additivity) If $\seq{E}{i}$ be a sequence of sets in $\algebra{M}$, 
	    then \[\measure{\countU{E}} \leq \sumInf{\measure{E_i}}.\]
	\item
	    (continuity from below) If $\seq{E}{i} \in \algebra{M}$, and \[\incSetSeq{E},\text{then}\]
	    \[\measure{\countU{E}} = \lim\limits_{i\to\infty}\measure{E_i}.\]
	\item
	    (continuity from above) If $\seq{E}{i} \in \algebra{M}$, 
	    and \[\decSetSeq{E}\] and $\measure{E_n} < \infty$ for some $n \in \Zplus$, 
	    then \[\measure{\countI{E}} =
	    \lim\limits_{i\to\infty}\measure{E_i}.\]
    \end{properties}
\end{Theorem}
\begin{proof}
    We prove in order.
    \begin{enumerate}
	\item 
	    Let $E_1 = E$ and $E_2 = F$ and $E_i = \emptyset$ for all $i \geq 3, i \in \Zplus$.
	    Then, $\measure{E\disjU F} = \measure{\countDisjU{E}} = \measure{E} + \measure{F}$.
	\item
	    $F = E \disjU F\cap\comp{E}$, and so $\measure{F} = \measure{E} +
	    \measure{F\cap\comp{E}}$, and hence $\measure{E} \leq \measure{F}$. 
	    (Since $E,F \in \algebra{M}$, $F\cap\comp{E}$ is also in
	    $\algebra{M}$, the result follows from $(1)$ because $\mu$ is a non-negative set
	    function). Now if $\measure{F} < \infty$, then $\measure{E} < \infty$ and so we can
	    subtract the $\measure{E}$ to obtain $\measure{F} - \measure{E} =
	    \measure{F\cap\comp{E}} = \measure{\setDiff{F}{E}}$. Note that, we actually just need the
	    finiteness of $\measure{E}$.
	\item 
	    From~\ref{rmk:disjU}, we can construct disjoint sets out of regular sets as follows,
	    let $F_1 = E_1$ and $F_k = E_k - \finiteU{E}{k-1}$ for $k \geq 2$. 
	    Note that each $\measure{F_k} \leq \measure{E_k}$ and $\countU{E} = \countDisjU{F}$. 
	    Hence, \[\measure{\countU{E}} = \measure{\countDisjU{F}} = \sumInf{\measure{F_i}} \leq
		\sumInf{\measure{E_i}}.\]
	\item
	   From~\ref{rmk:disjU}, we can construct disjoint sets as above, let $F_1 = E_1$ and 
	   $F_k = E_k - \finiteU{E}{k-1} = E_k - E_{k-1}$ for $k \geq 2$.
	   See~\ref{fig:tikz:measure_prop}. Also note that 
	   $E_n = \finiteDisjU{F_i}{n}$ and so $\measure{E_n} = \measure{\finiteDisjU{F}{n}} =
	   \sumFinite{\measure{F_i}}{n}$. Thus,
	   \begin{align*}
	       \measure{\countU{E}} &=\measure{\countDisjU{F}}  \\
	       &=\sumInf{\measure{F_i}} \\
	       &=\lim\limits_{n\to\infty}\sumFinite{\measure{F_i}}{n} \\
	       &=\lim\limits_{n\to\infty}\measure{E_n}. 
	   \end{align*}
	   


       \item
	   
	   Let $F_k = E_n - E_k$ for $k > n$, and $F_k = \emptyset$ for $k \leq n$. 
	   Note that $E_n \supset E_{n+1} \supset E_{n+2} \dots$, 
	   and so $F_{n+1} \subset F_{n+2} \dots$ i.e.~$F_{n+1} = E_n - E_{n+1} \subset E_n -
	   E_{n+2} = F_{n+2}$ and so on. See~\ref{fig:tikz:measure_prop2}. Hence,
	   \[\bigcup\limits_{i \geq n+1} F_i = \bigcup\limits_{i \geq n+1}(\setDiff{E_n}{E_i}) =
	       \setDiff{E_n}{\bigcap\limits_{i \geq n+1}E_i}.\]
	   Thus we get,
	   \[\measure{\countU{F}} = \measure{E_n} - \measure{\bigcap\limits_{i \geq n+1}E_i}.\]
	   because, $E_{n+k} \subset E_n$ and $\measure{E_n} < \infty$.
	   Also we get a increasing sequence of sets $F_{i} \subset
	   F_{i+1} \subset F_{i+2} \dots$, for $i > n$. Thus we know that,
	   \[\measure{\countU{F}} = \lim\limits_{i\to\infty}\measure{F_i} = \lim\limits_{k\to\infty}
	       \measure{\setDiff{E_n}{E_{n+k}}}.\]
	   Since $E_{n+k} \subset E_n$ and $\measure{E_n} < \infty$ we get,
	  \[\measure{\countU{F}} = \lim\limits_{i\to\infty}\measure{F_i} = 
	      \measure{E_n} - \lim\limits_{k\to\infty}\measure{E_{n+k}}.\]
	  Thus equating the two we get our result.
 
    \end{enumerate}
\end{proof}
The next theorem says that in a sigma-finite measurable space, there cannot be a countable collection of
disjoint sets with positive measure. The proof of the theorem is highly illustrative of a common technique
that pops up in measure theory. 

\begin{Theorem}
    Let $\measureS{X}{\famM}{\mu}$ be a sigma-finite measure space. Then, $\famM$ cannot contain an
    uncountable, disjoint collection of sets of positive measure.
\end{Theorem}
\begin{proof}
    Let $\famE = \set{E_{\alpha}}{\alpha \in A}$ be the collection of subsets of $X$ such that for each
    $\alpha \in A$, $\measure{E_{\alpha}} > 0$. We need to show that $\famE$ is a countable. Since, we have a
    sigma finite space, there is a sequence of sets $\seq{X}{n}$ such that $\atobUp{X_n}{X}$ and 
    $\measure{X_n} < \infty$ for each $n$. For any $E \in \famE$,
    \[E \subset X = \countUnion{X_n}{n},\]
    thus, $E = \countUnion{(E\cap X_n)}{n}$.
    For any $\epsilon > 0$, consider the collection,
    \[\famE_{n,\epsilon} = \set{E \in \famE}{\measure{(E\cap X_n)} > \epsilon}.\]
    If $E$ is any set in $\famE$, then there must be an $n,\epsilon$ such that $E \in \famE_{n,\epsilon}$.
    Using the archimedes principle then there must be a $k \in \Zplus$ such that $\frac{1}{k}< \epsilon$.
    Hence, let us consider the collection,
    \[\famE_{n,k} = \set{E \in \famE}{\measure{(E\cap X_n)} > \frac{1}{k}}.\]
    Clearly for each $n,k$, $\famE_{n,k} \subset \famE$. Also, by reasoning above, for any $E \in \famE$ there
    must be a $k,n \in \Zplus$, such that $E \in \famE_{n,k}$. Hence,
    \[\famE = \bigcup\limits_{n,k = 1}^{\infty}\famE_{n,k}.\]
    Hence, if we show that each $\famE_{n,k}$ is countable, then we are done.
    Consider a sequence of finite sets $E_1, E_2, \dots, E_m$ in one such $\famE_{n,k}$. Since, these sets are
    disjoint,
    \[\frac{m}{k} \leq \finiteSum{\measure{E_i\cap X_n}}{i}{m} = \measure{\finiteUnion{(E_i\cap X_n)}{i}{n}}
	\leq \measure{X_n}.\]
    Thus, $m \leq k\measure{X_n}$. Hence we only have a finite choice for the index $m$. Thus, $\famE_{n,k}$
    is finite.
\end{proof}

\begin{figure}
    \includestandalone[width=0.45\textwidth]{tex/tikz_figures/measure_prop}
    \caption{Illustration of proof~\ref{thm:prop_of_meas} $(4)$}\label{fig:tikz:measure_prop}
\end{figure}
\begin{figure}
    \includestandalone[width=0.45\textwidth]{tex/tikz_figures/measure_prop2}
    \caption{Illustration of proof~\ref{thm:prop_of_meas} $(5)$}\label{fig:tikz:measure_prop2}
\end{figure}

\section{Sets of measure zero and completion of measure}
We know that we want to deal with sets of measure zero. There is a technical issue about such sets
that we settle now.
\begin{Definition}[name=Sets of measure zero]
    If $\measureS{X}{\algebra{M}}{\mu}$ is a measure space, a set $E \in \algebra{M}$ with
    $\measure{E} = 0$ is called a set of \emph{measure zero}. 
\end{Definition}
\begin{Definition}[name=Almost everywhere]
    If a statement about points $x \in X$ is true except for $x$ in a set of \emph{measure zero}, we
    say that the statement is true \emph{almost everywhere} and denote it by a.e.
\end{Definition}
\begin{Proposition}[name=Countable union of sets of measure $0$]\label{prop:count_union_meas_0}
    A countable union of sets of measure zero has measure zero.
\end{Proposition}
\begin{proof}
    From the sub-additivity property of measure, if $\seq{E}{i}$ is a sequence of measure zero then
    $\measure{\countU{E}} \leq \sumInf{\measure{E_i}} = 0$. Since measure is a non-negative set
    function we get $\measure{\countU{E}} = 0$.
\end{proof}
\begin{Remark}
    If $\measure{F} = 0$ for some $F \in \algebra{M}$, then $\measure{E} = 0$ for any $E \subset F$
    whenever $E \in \algebra{M}$. However, $E$ may not be in $\algebra{M}$. This is a technical
    issue that we need to resolve. We will do so my adding to the $\algebra{M}$ all those
    subsets of sets in $\algebra{M}$ that have measure $0$. This is called the completion of the
    measure $\mu$.
\end{Remark}
\begin{Definition}[name=Complete measure space]
    If $\measureS{X}{\algebra{M}}{\mu}$ is a measure space such that $\algebra{M}$ contains all
    subsets of sets in $\algebra{M}$ with measure $0$, then $\measureS{X}{\algebra{M}}{\mu}$ is
    \emph{complete}.
\end{Definition}
Completeness eliminates some annoying issues and it can always be obtained by enlarging the domain of
a given measure to obtain an equivalent measure in the following sense:
\begin{Theorem}[name=Completion of a measure]\label{thm:comp_of_meas}
    Let $\measureS{X}{\algebra{M}}{\mu}$ be a measure space. Let $\family{N} = \set{N \in
	\algebra{M}}{\measure{N} = 0}$. Define,
    \[\closure{\algebra{M}} = \set{E\cup F}{E \in \algebra{M},\text{$\thereIs{N\in\family{N}}$ such
	    that $F \subset N$}},\]
     then $\closure{\algebra{M}}$ is a $\sigmaAlgebra$ on $X$ that contains $\algebra{M}$. Moreover,
     the unique measure $\xoverline{\mu}$ on $\closure{\algebra{M}}$ defined by 
     $\compMeasure{E\cup F} = \measure{E}$ for all $E \in \algebra{M}$ makes
     $\measureS{X}{\closure{\algebra{M}}}{\xoverline{\mu}}$ complete.
\end{Theorem}
\begin{proof}
    Clearly $\algebra{M} \subset \closure{\algebra{M}}$. We need to show that
    $\closure{\algebra{M}}$ is a $\sigmaAlgebra$. Let $A \in \closure{\algebra{M}}$. Then $A = E
    \cup F$ such that there is a $N \in \family{N}$ with $F \subset N$. Note that $F \subset N
    \implies \comp{N} \subset \comp{F}$ and so $\comp{F} = \comp{N} \disjU \setDiff{N}{F}$. 
    Thus $\comp{A} = \comp{E}\cap\comp{F} = \comp{E}\cap(\comp{N}\cup(\setDiff{N}{F}))$. 
    Thus the distributive law yields,
    \[\comp{A} = (\comp{E}\cap\comp{N})\cup (\comp{E}\cap{N}\cap\comp{F}).\]
    Now $(\comp{E}\cap\comp{N}) \in \algebra{M}$ since both $E,N \in \algebra{M}$.
    Also $(\comp{E}\cap{N}\cap\comp{F}) \subset N$ and hence $\comp{A} \in \closure{\algebra{M}}$.
    Let $\seq{A}{i}$ be a sequence of sets in $\algebra{M}$. Then each $A_i = E_i \cup F_i$ such
    that there is a $N_i \in \family{N}$ such that $F_i \subset N_i$. Thus,
    \[\countU{A} = (\countU{E})\, \bigcup\, (\countU{F}).\]
    $\countU{E} \in \algebra{M}$ and $\countU{F} \subset \countU{N}$.
    From~\ref{prop:count_union_meas_0}, $\countU{N} \in \family{N}$ and hence $\countU{A} \in
    \closure{\algebra{M}}$. Thus $\closure{\algebra{M}}$ is a $\sigmaAlgebra$.

    We have to check if $\compMeasure{A}$ is well defined i.e if $A = E_1 \cup F_1 = E_2 \cup F_2$
    then $\measure{E_1} = \measure{E_2}$. To see this, $E_1 \subset E_1 \cup F_1 = E_2 \cup F_2
    \subset E_2 \cup N_2 \in \algebra{M}$. Thus $\measure{E_1} \leq \measure{E_2} + \measure{N_2} =
    \measure{E_2}$. Similarly, $\measure{E_2}  \leq \measure{E_1}$. 
    
    Thus to show that 
    $\measureS{X}{\closure{\algebra{M}}}{\xoverline{\mu}}$ is complete we need to show that for any
    $A \in \closure{\algebra{M}}$, if $\compMeasure{A} = 0$, then for any $B \subset A$ we must have
    $B \in \closure{\algebra{A}}$. Since $A = E \cup F$ where $F \subset N$ for some $N \in
    \family{N}$, we have $\compMeasure{A} = \measure{E}$. But this implies that $\measure{E} = 0$
    and so $E \in \family{N}$. Thus $A \in \family{N}$. But $B \subset A$ and so $B = \emptyset
    \cup B$ where $\emptyset \in \algebra{M}$ and $B \subset A \in \family{N}$. Thus $B \in
    \closure{\algebra{M}}$. 
\end{proof}
\begin{Example}
    Consider a set $X \neq \emptyset$ and the minimal sigma 
    algebra $\algebra{M} = \lbrace X,\emptyset\rbrace$ with a measure $\mu$ defined to be $0$ 
    for any set in the $\algebra{M}$. Clearly $\measureS{X}{\algebra{M}}{\mu}$ is not complete. To
    get a completion of $X$ we must add all subsets of $X$ to the $\sigmaAlgebra$. Thus
    $\closure{\algebra{M}} = \powSet{X}$.
\end{Example}
\section{Construction of measures}
A $\sigmaAlgebra$ is a huge set and specifying a set function that satisfies the requirements of a
measure is not a trivial pursuit. Instead we work with set functions that can be easily constructed 
and then derive a measure from it by restricting or extending its domain. One such way is the
\emph{outer measure}. As we will see an outer measure can be derived easily from certain set 
functions. Once an outer measure is constructed it can be restricted to generate a measure.
\begin{Definition}[name=Outer measure]
    If $X$ is a non-empty set, an outer measure on $X$ is a function
    $\map{\outMeas}{\powSet{X}}{\interval{0}{\infty}}$ such that,
    \begin{enumerate}
	\item
	    \[\outMeasure{\emptyset} = 0.\]
	\item
	    (Monotonicity) For any $A \subset B$,
	    \[\outMeasure{A} \leq \outMeasure{B}.\]
	\item
	    (sub-additivity) For any $\seq{A}{i}$ in $\powSet{X}$,
	    \[\outMeasure{\countU{A}} \leq \sumInf{\outMeasure{A_i}}.\]
    \end{enumerate}
\end{Definition}
It is easy to see that every measure whose domain is the $\powSet{X}$ is an outer measure. Now we
will show that an outer measure can be restricted to yield a measure. We will use Carath{\'e}dory's
theorem to establish this result.

\begin{Definition}[name=$\outMeasurable$ sets: Carath{\'e}dory's Criterion]\label{def:carath_crit}
    Given any outer measure $\outMeas$, we say that a set $E$ is $\outMeasurable$ if,
    \[\outMeasure{A} = \outMeasure{A\cap E} + \outMeasure{A\cap \comp{E}},\]
    for any subset $A \subset X$.
\end{Definition}
We want those sets $E$ such that no matter what $A \subset X$ we take, outer measure of $A$ is
additive on $A$ (w.r.t E). This is shown in~\ref{fig:tikz:caratheodary}. Note that $A$
itself is not required to be $\outMeasurable$. A vague motivation for this is, if $E$ is a
\emph{good} set and $A \supset E$, Carath{\'e}dory's criteria states that its outer measure
$\outMeasure{E} = \outMeasure{E\cap A}$ is equal to $\outMeasure{A} -
\outMeasure{\intersection{A}{\comp{E}}}$. While the later concerns measuring $E$ from
    \emph{inside} $A$, the term $\outMeasure{E} = \outMeasure{E\cap A}$ concerns measuring $E$ from
    \emph{outside} $E$. This is seen from the right side of~\ref{fig:tikz:caratheodary} and thus
    \emph{good} sets have the same \emph{measure}{\textemdash}inside or outside.

We can just say measurable instead of $\outMeasurable$ if $\outMeas$ is clear from the context.
\begin{figure}
  \includestandalone[width=0.75\textwidth]{tex/tikz_figures/caratheodary}
  \caption{$\carathCrit$: $\outMeasure{A}$ is the sum of the two colors.}\label{fig:tikz:caratheodary}
\end{figure}

\begin{Proposition}\label{prop:carath_crit}
    Given any outer measure $\outMeas$, we say that a set $E$ is $\outMeasurable$ if,
    \[\outMeasure{A} \geq \outMeasure{A\cap E} + \outMeasure{A\cap \comp{E}},\]
    for any subset $A \subset X$.
\end{Proposition}
\begin{proof}
    From the~\ref{fig:tikz:caratheodary} it is easy to see that $A = \intersection{A}{E} \disjU
    \intersection{A}{\comp{E}}$. Since $\outMeas$ is sub-additive we have $\outMeasure{A} \leq
    \outMeasure{\intersection{A}{E}} + \outMeasure{\intersection{A}{\comp{E}}}$. Thus, for $E$ to be
    $\outMeasurable$, we must have,
    \[\outMeasure{A} \geq \outMeasure{A\cap E} + \outMeasure{A\cap \comp{E}},\]
    for any subset $A \subset X$.
\end{proof}
\begin{Theorem}[name=Carath{\'e}odory Theorem]\label{thm:carath_restr_thm}
    Let $X$ be a non-empty set with an outer measure $\outMeas$ defined on $\powSet{X}$. Define,
    \[\algebra{M} = \set{E \subset X}{E\,\text{is $\outMeasurable$}}.\]
    Then, $\algebra{M}$ is a $\sigmaAlgebra$ and the restriction of $\outMeas$ on $\algebra{M}$
    is a measure. Thus, 
    $\measureS{X}{\algebra{M}}{\restrict{\outMeas}{\algebra{M}}}$ is a measurable space.
\end{Theorem}
\begin{proof}
    We first need to show that $\algebra{M}$ is a $\sigmaAlgebra$. A priori we don't know if
    $\algebra{M}$ is empty or not. To this end we will show that $\emptyset \in \algebra{M}$.
    If $\outMeasure{E} = 0$ then, 
    we need to show that for any $A \subset X$, 
    $\outMeasure{A} \geq \outMeasure{A\cap E} + \outMeasure{A\cap \comp{E}}$. Since,
    $A \supset \intersection{A}{\comp{E}}$ and hence $\outMeasure{A} \geq \intersection{A}{\comp{E}}$
    from monotonicity of $\outMeas$. Similarly, $ E\supset \intersection{A}{E}$ and 
    $\outMeasure{E} \geq \intersection{A}{E}$. Thus, adding, $\outMeasure{A} + \outMeasure{E} \geq 
    \outMeasure{A\cap E} + \outMeasure{A\cap \comp{E}}$. Since
    $\outMeasure{E} = 0$, the result follows. Thus $E$ is $\outMeasurable$. This means that
    $\emptyset \in \algebra{M}$ since $\outMeasure{\emptyset} = 0$. 

    If $E \in \algebra{M}$ then for any $A \subset X$, 
    $\outMeasure{A} = \outMeasure{\intersection{A}{E}} + \outMeasure{\intersection{A}{\comp{E}}}$. 
    Replacing $E$ by $\comp{E}$ we get the same expression and thus $\comp{E} \in \algebra{M}$.

    Let $E_1,E_2 \in \algebra{M}$. Then they satisfy $\carathCrit$~\ref{def:carath_crit}. Thus,
    \begin{align*}
	\carathEq{A_1}{E_1} \\
	\carathEq{A_2}{E_2} 
    \end{align*}
    for any $A_1,A_2 \subset X$. Pick $A_1 = A \subset X$ and $A_2 = A \cap E_1$ and so we get,
    \begin{align*}
	\carathEq{A}{E_1} \\
	\carathEq{A\cap\comp{E_1}}{E_2} 
    \end{align*}
    Substituting for $\outMeasure{\intersection{A}{\indxComp{E}{1}}}$ we get,
    \[\outMeasure{A} = \outMeasure{\intersection{A}{E_1}} +
	\outMeasure{\intersection{\intersection{A}{\comp{E_1}}}{E_2}} +
	\outMeasure{\intersection{A}{\comp{(\union{E_1}{E_2})}}}.\]
    Now, $\intersection{A}{(\union{E_1}{E_2})} =
    (\intersection{A}{E_1})\disjU(\intersection{\intersection{A}{\comp{E_1}}}{E_2})$ is easily seen
    as the disjoint union of two sets $X,Y$ where $X = A\cap E_1$, $Y = A \cap E_2$, and hence,
    \[\outMeasure{\intersection{A}{E_1}} +
    \outMeasure{\intersection{\intersection{A}{\comp{E_1}}}{E_2}} \geq 
    \outMeasure{\intersection{A}{(\union{E_1}{E_2})}}.\]
    Thus,
    \begin{align*} \carathGeq{A}{(\union{E_1}{E_2})}\end{align*}which from~\ref{prop:carath_crit} means that
    $\union{E_1}{E_2} \in \algebra{M}$. Thus we get finite additivity. Hence we have shown that
    $\algebra{M}$ is an algebra. 
    
    Showing that $\algebra{M}$ is closed under a countable union of an arbitrary sequence of sets is
    a little tricky. Instead we will show that $\algebra{M}$ is closed under countable union of a
    sequence of disjoint sets. Then $\algebra{M}$ will be a $\sigmaAlgebra$
    by~\ref{thm:eq_ch_sigmaA}. We will do this in 3 steps:
    Let $\seq{E}{i}$ be a sequence of pairwise \emph{disjoint} sets of $\algebra{M}$ and let
    $S_n = \finiteU{E}{n}$ and $S = \countU{E}$.
    
    \textbf{\large{Step 1}}:
    We will show that for any $n \geq 1$,
    \[\outMeasure{\intersection{A}{S_n}} = \sumFinite{\outMeasure{\intersection{A}{E_i}}}{n}.\]
    Note that since we proved $\algebra{M}$ is an algebra, $\algebra{M}$ is closed under finite
    unions and hence $S_n \in \algebra{M}$. We will use induction on $n$. The case $n=1$ is trivial.
    Suppose it is true for some $k > 1 \in \Zplus$, then since $S_{k} \in \algebra{M}$ we get, 
    \begin{align*}
	\carathEqIndx{A\cap S_{k+1}}{S}{k} \\
	&= \outMeasure{\intersection{A}{S_k}} + \outMeasure{\intersection{A}{E_{k+1}}} \\
	&= \sumFinite{\outMeasure{\intersection{A}{E_i}}}{k} + \outMeasure{\intersection{A}{E_{k+1}}} \\	
	&= \sumFinite{\outMeasure{\intersection{A}{E_i}}}{k+1}
    \end{align*}

    \textbf{\large{Step 2}}:
    We will show,
    \[\outMeasure{\intersection{A}{S}} = \sumInf{\outMeasure{\intersection{A}{E_i}}}.\]
    From sub-additivity of $\outMeas$,
    \[\outMeasure{\intersection{A}{S}} = \outMeasure{\countU{E}} \leq \sumInf{\outMeasure{E_i}}\] 
    To show the other direction,
    $S \supset S_n$ for any $n \in \Zplus$ and hence from monotonicity and \textbf{\large{Step 1}},
    \[\outMeasure{\intersection{A}{S}} \geq \outMeasure{\intersection{A}{S_n}} =
	\sumFinite{\measure{E_i}}{n}.\]
    Taking the limit as $\atob{n}{\infty}$ we get the desired result.

    \textbf{\large{Step 3}}:
    We will show,
    \begin{align*} \carathGeq{A}{S} \end{align*}
    Consider an $n \in \Zplus$ and since $S_n \in \algebra{M}$ we have,
    \begin{align*} \carathGeqIndx{A}{S}{n} \end{align*}
    But for any $n \in \Zplus$, $\indxComp{S}{n} \supset \comp{S}$ and thus using
    \textbf{\large{Step 1}} 
    \begin{align*}
	\outMeasure{A} \geq &\outMeasure{\intersection{A}{S_n}} + 
	\outMeasure{\intersection{A}{\comp{S}}} \\
	& = \sumFinite{\outMeasure{E_i}}{n} + \outMeasure{\intersection{A}{\comp{S}}}
    \end{align*}
    taking the limit as $\atob{n}{\infty}$ and using \textbf{\large{Step 2}} we get the result.
    

    Thus $\algebra{M}$ is a $\sigmaAlgebra$. It is easy to show that
    $\mu = \restrict{\outMeas}{\algebra{M}}$ is a measure. Indeed all we need to show is that $\mu$
    satisfies $\sigmaAdd$. 
    Let $\seq{E}{i}$ be a sequence of pairwise \emph{disjoint} sets of $\algebra{M}$ and let
    $S = \countU{E}$.
    From \textbf{\large{Step 2}}, 
    \[\outMeasure{\intersection{A}{S}} = \sumInf{\outMeasure{\intersection{A}{E_i}}}\] for any
    $A\subset X$. If we restrict $\outMeas$ to $\algebra{M}$, this means,
     \[\measure{\intersection{A}{S}} = \sumInf{\measure{\intersection{A}{E_i}}}\] for any
    $A\in\algebra{M}$. Since $S \in \algebra{M}$ take $A = S$ and thus
    \[\measure{\countDisjU{E}} = \sumInf{\measure{E_i}}.\]

    Also interersting to note is that $\measureS{X}{\algebra{M}}{\mu}$ is complete. To see this take
    any $E \in \algebra{M}$ such that $\measure{E} = 0$ and let $B \subset E$. Clearly $B \subset X$
    and thus, \[\outMeasure{B} \leq \outMeasure{E} = \measure{E} = 0.\]

    Hence, $\outMeasure{B} = 0$ which means $B \in \algebra{M}$ (as proved earlier).
\end{proof}
Thus we have seen that any outer measure can be restricted to a measure. Hence for this Theorem to be
useful we must have an outer measure. We noted that describing a measure for all members of a
$\sigmaAlgebra$ is no easy matter but now we have this outer measure on the maximal $\sigmaAlgebra$!
This is not a concern since the next theorem shows that we can easily construct an outer measure
from simple set functions defined on a simpler class. For example, lebesgue measure is defined on
intervals. At this point it will do us well to tabulate the properties of all the measures we
have seen. This is done in~\ref{tab:prop_measures}. 

\begin{table}
    \caption{Properties of measure, pre-measure and outer-measure}\label{tab:prop_measures}
    \begin{tabular}{llcccr}
	\toprule
	Properties  & Measure $\mu$ & Pre-Measure $\preMeas$ & Outer-Measure $\outMeas$ \\
	\midrule
	domain & $\sigmaAlgebra$ & algebra & $\powSet{X}$\\
	monotonicity & $\checkmark$ & $\checkmark$ & \checkmark\\
	finite-additvity & $\checkmark$ & $\checkmark$ & -\\
	finite-sub-additvity & $\checkmark$ & $\checkmark$ &$\checkmark$ \\
	countable-additvity & $\checkmark$ &-&-\\
	countable-sub-additvity & $\checkmark$ &-&$\checkmark$\\
	\bottomrule
    \end{tabular}
\end{table}

\begin{Example}\label{ex:out_meas}
    Let $X$ be an infinite set and define,
    \begin{equation*}
	\outMeasure{E} =
	\begin{cases}
	    \lvert E \rvert& \text{if $E$ is finite},\\
	    \infty& \text{if $E$ is infinite}.
	\end{cases}
    \end{equation*}
    Then $\outMeas$ is an outer measure.
\end{Example}
\begin{proof}
    Clearly $\map{\outMeas}{\powSet{X}}{\extRealsPos}$ and $\outMeasure{\emptyset} = 0$. Let us check
    monotonicity. If $A \subset B \subset X$, then if both are finite we have 
    $\lvert A \rvert \leq \lvert B \rvert$ and so $\outMeasure{A} \leq \outMeasure{B}$. If $B$ is
    infinite then its obvious.
    Consider a sequence $\seq{A}{i}$ of sets in $X$. If any of the $A_i$ is infinite then 
    $\outMeasure{\countU{A}} = \infty = \sumInf{\outMeasure{A_i}}$. Here we treat terms like $\infty +
    \infty$ to be equal to $\infty$. If all $A_i$ are finite then there are two cases. Either
    $\countU{A}$ is finite or it is infinite. (For expample, $\lvert A_i \rvert = i$, then
    $\countU{A}$ is infinite if $A_i$ are pairwise disjoint). 
    If $\countU{A}$ is finite then we know that $\lvert \countU{A} \rvert
    \leq \sumInf{\lvert A_i \rvert}$ and hence the result follows. If $\countU{A}$ is infinite then
    we can create a sequence of disjoint finte sets $F_k = \setDiff{A_k}{\finiteU{A}{k-1}}$. Then,
    $\countU{A} = \countDisjU{F}$. Thus 
    $\outMeasure{\countU{F}} = \sumInf{\outMeasure{F_i}} = \infty$ since we have finite disjoint
    sets, but $\sumInf{\outMeasure{F_i}}
    \leq \sumInf{\outMeasure{A_i}}$ and hence we get,
    $\outMeasure{\countU{A}} \leq \sumInf{\outMeasure{A_i}}$. All finite sets are $\outMeas$
    measurable.
\end{proof}

In~\ref{fig:tikz:construct_out_measure}, we illustrate a way of constructing outer measure. Say we
have a set $A_k$. If we want to define an outer measure we start with a sequence of 
simple sets $E^{k}_i$ whose union $\bigcup\limits_{i}E^{k}_{i} \supset A_k$. If we have a simple
set function (maybe a pre-measure) $\rho$ for these sets then we get an overestimated outer measure 
of $A$ given by 
$\sumInf{\rho(E^k_i)}$. The idea is now to take all those collections of sets whose union contain 
$A_k$ and then take the infimum of these overestimated outer measure. The infimum is then precisely
the outer measure of $A_k$.


\begin{Theorem}[name=Constructing outer measure]\label{thm:constr_out_measure}
    Let $X$ be a non-empty set, $\family{E}\subset\powSet{X}$ be non-empty family of subsets of $X$
    such that $\emptyset,X \in \family{E}$. Let there be a function
    $\map{\rho}{\family{E}}{\extRealsPos}$ such that $\rho(\emptyset) = 0$. For any $A \in
    \powSet{X}$, define
    \[\outMeasure{A} = \inf\set{\sumInf{\rho(E_i)}}{\lbrace E_i \rbrace \subset
	    \family{E}\,\text{and}\,A\subset\countU{E}},\]
    then, $\outMeasure{A}$ is an outer measure (induced by $\rho$).
\end{Theorem}

\begin{figure}
  \includestandalone[width=0.5\textwidth]{tex/tikz_figures/construct_out_measure}
  \caption{Construction of outer measure}\label{fig:tikz:construct_out_measure}
\end{figure}

\begin{proof}
    We will first check that the infimum exists. The set is certainly not empty since $X
    \in \family{E}$ and $A \subset X$; atleast one such $E_i = X$ exists. Morever we are taking
    infimum over positive reals that are bounded below by $0$ and hence the infimum exists.
    
    Lets check if $\outMeasure{\emptyset} = 0$. Since $\emptyset \in \family{E}$ it is trivally
    contained in a collection of empty sets. Since $\rho(\emptyset) = 0$ we get the result.

    Let us see if we get monotonicity. Let $A \subset B$. For any collection $\lbrace E_i \rbrace
    \subset \family{E}$ if $\countU{E} \supset {B}$ then $\countU{E} \supset {A}$. Thus the set,
    \begin{align*}
	&\set{\sumInf{\rho(E_i)}}{\lbrace E_i \rbrace \subset
	    \family{E}\,\text{and}\,A\subset\countU{E}}\\ 
	&\hspace{2cm}\supset \\
	&\set{\sumInf{\rho(E_i)}}{\lbrace E_i \rbrace \subset
	    \family{E}\,\text{and}\,B\subset\countU{E}},
    \end{align*}
    hence the result follows.

    Let $\seq{A}{k}$ be a sequence of sets in $X$ such that for each $k$ there is a sequence of
    sets $\left(E^k_i\right)$ such that,
    \begin{align*}
	&A_k \subset \bigcup\limits_{i}E^k_i \, \text{and,} \\
	&\sumInf{\rho(E^k_i)} \leq \outMeasure{A_k} + \epsilon,
    \end{align*}	
    for any $\epsilon > 0$. Pick $\epsilon = \frac{1}{2^k}$.
    Now,
    \[\bigcup\limits_{k}A_k \subset\bigcup\limits_{i,k}{E^k_i}.\]
    Let $A = \bigcup\limits_{k}A_k$, thus $A \subset \bigcup\limits_{i,k}E^i_k$, and so by definition
    \begin{align*}
	\outMeasure{A} &\leq \sum\limits_{i,k=1}^{\infty}\rho(E^i_k) \\
	&\leq \sum\limits_{k=1}^{\infty}(\outMeasure{A_k} + \frac{1}{2^k})\\
	&\leq \sum\limits_{k=1}^{\infty}\outMeasure{A_k}.
    \end{align*}
    Thus $\outMeas$ is an outer measure.
\end{proof}
Note, at this point there is no relation between $\rho$ and the outer measure induced by $\rho$.
Very little demands are made on $\rho$ and its domain $\family{E}$. An interesting question to ask
would be{\textemdash}what if we had some additional structure on $\family{E}$, particularly what if 
$\family{E}$
were an algebra? In that case, if additionaly $\rho = \preMeas$ is a pre-measure, we have a relation
between $\preMeas$ and the outer measure $\outMeas$ induced by $\preMeas$. The following Proposition
showcases this relation.
\begin{Proposition}\label{prop:out_meas_iduced_pre_measure}
    If $\preMeas$ is a pre measure on $\algebra{E}$ and the outer measure induced by $\preMeas$ is
    given as in~\ref{thm:constr_out_measure}, then
    \begin{enumerate}
	\item
	    $\restrict{\outMeas}{\algebra{E}} = \preMeas$.
	\item
	    Every set in $\algebra{E}$ is $\outMeasurable$.
    \end{enumerate}
\end{Proposition}
\begin{proof}
    We prove in order.
    \begin{enumerate}
	\item
	    We need to show that $\restrict{\outMeas}{\algebra{E}} \leq \preMeas$ and 
	    $\preMeas \leq \restrict{\outMeas}{\algebra{E}}$. Let $A \in \family{E}$.
	    Clearly, $A \subset A$, consider the collection $\lbrace E_i \rbrace$ where $E_1 = A$
	    and $E_i = \emptyset, i \geq 2$. Hence 
	    $\restrict{\outMeas}{\algebra{E}}(A) \leq \preMeas(A)$. 
	    To prove the other inequality, for any
	    $\epsilon > 0$ there is a collection $\lbrace E_i \rbrace \subset \family{E}$ 
	    such that $A \subset\countU{E}$ and, 
	    \[\sumInf{\preMeas(E_i)} < \restrict{\outMeas}{\algebra{E}}(A) + \epsilon.\]
	    Since $\preMeas$ is a pre-measure and $A \in \family{E}$, 
	    if $\countU{E} \in \algebra{E}$ we get from sub-additivity of
	    $\preMeas$, 
	    \[\preMeas(A) \leq \sumInf{\preMeas(E_i)}.\]
	    However, it may happen that $\countU{E} \not \in \algebra{E}$. 
	    In that case, observe that, 
	    \[A = \intersection{A}{\countU{E}} = \countU{A\cap E}.\]
	    We will write union of $A \cap E_i$ as a disjoint union using~\ref{rmk:disjU}.
	    Let $X_j = A\cap E_j $,
	    $F_j = X_j - \bigcup\limits_{i=1}^{j}X_i, j \geq 2, F_1 = X_1$.
	    Each $F_j \in \algebra{E}$, moreover $\disjU\limits_{j}F_j = A \in \algebra{E}$, thus
	    \[\preMeas(A) = \sumInf{\preMeas(F_i)} \leq \sumInf{\preMeas(E_i)}.\]
	    Thus,
	    \[\preMeas(A) \leq \sumInf{\preMeas(E_i)}.\]
	    Hence,
	    \[\preMeas(A) < \restrict{\outMeas}{\algebra{E}}(A) + \epsilon,\]
	    since $\epsilon$ is arbitrary we get the result.
	\item
	    Let $E \in \family{E}$. To show that $E$ is $\outMeasurable$ we have to show,
	    \begin{align*}
		\carathGeq{A}{E}
	    \end{align*}
	    for any $A \subset X$.
	    Fix an $\epsilon > 0$, then there is a collection $\lbrace B_i \rbrace \subset
	    \algebra{E}$ such
	    that $\countUnion{B_i}{i} \supset A$ and, 
	    \[\outMeasure{A} + \epsilon > \sumInf{\preMeas(B_i)}.\]
	    Since $E \in \algebra{E}$, we can write 
	    $B_i = (\intersection{B_i}{\comp{E}})\disjU (B_i \cap E)$ as a disjoint union of sets
	    in $\family{E}$. 
	    Using the fact that $\preMeas$ is a pre-measure, we get 
	    \[\preMeas(B_i) = \preMeas(\intersection{B_i}{\comp{E}}) + \preMeas(B_i \cap E).\]
	    Thus,
	    \begin{align*}
		\outMeasure{A} + \epsilon >& \sumInf{\preMeas(B_i \cap E)} + 
		\sumInf{\preMeas(\intersection{B_i}{\comp{E}})} \\
	       & \geq  \outMeasure{A\cap E} + \outMeasure{A\cap\comp{E}}.
	   \end{align*}
	   Since $\epsilon$ was arbitrary we get the result.
    \end{enumerate}
\end{proof}
Now we are ready to prove the extension theorem i.e given an algebra $\algebra{E}$ on a set $X$,
and a pre-measure $\preMeas$ we can extend
the pre-measure to a measure $\mu$ on a $\sigmaAlgebra$ containing $\algebra{E}$.
Such a construction can be achieved in two step:
\begin{enumerate}
    \item
	First extend the pre-measure $\preMeas$ on $\algebra{E}$ to an outer measure $\outMeas$ on
	$\powSet{X}$. See~\ref{thm:constr_out_measure}
	and~\ref{prop:out_meas_iduced_pre_measure}.
    \item
	Then restrict the outer measure $\outMeas$ to a $\sigmaAlgebra$ $\algebra{M}$.
	See~\ref{thm:carath_restr_thm}.
\end{enumerate}
This is an important result, since we start with simple set functions like measuring length of 
interval etc.\ and want to extend this notion to a larger class of sets. The following theorem
combines these observations and is called the Carath{\'e}odory Extension theorem.
\begin{Theorem}[name=Carath{\'e}odory Extension Theorem]\label{thm:carath_ext_thm}
    Let $X$ be a non-empty set and let $\algebra{E} \subset \powSet{X}$ be an algebra, $\preMeas$ a
    pre-measure on $\algebra{E}$ and $\sigmaGen{\algebra{E}}$ be the $\sigmaAlgebra$ generated by
    $\algebra{E}$. Then,
    \begin{enumerate}
	\item
	    There exists a measure $\mu$ on $\sigmaGen{\algebra{E}}$ whose restriction to
	    $\algebra{E}$ is $\preMeas$.
	\item
	    If $\nu$ is another measure on $\sigmaGen{\algebra{E}}$ that extends $\preMeas$, then
	    $\nu(E) \leq \measure{E}$ for any $E \in \sigmaGen{\algebra{E}}$, with equality when
	    $\measure{E} < \infty$. 
	\item
	    Additionally, if $\preMeas$ is $\sigmaFinite$, then $\mu$ is the unique extension of
	    $\preMeas$ to a measure on $\sigmaGen{\algebra{E}}$.
    \end{enumerate}
\end{Theorem}
\begin{proof}
    Given non-empty set $X$, an algebra $\algebra{E}$ with a pre-measure $\preMeas$,
    \begin{enumerate}
	\item
	    first extend $\preMeas$ to
	    $\outMeas$,~\ref{thm:constr_out_measure},~\ref{prop:out_meas_iduced_pre_measure}. Then
	    restrict $\outMeas$ to $\mu$ on $\sigmaAlgebra$ 
	    $\algebra{M}$,~\ref{thm:carath_restr_thm}. Since
	    $\algebra{M}$ contains $\algebra{E}$, it contains $\sigmaGen{\algebra{M}}$. Thus there is
	    a measure $\mu$ on $\sigmaGen{\algebra{E}}$ such that,
	    \[\mu = \restrict{\outMeas}{\sigmaGen{\algebra{E}}},\] and 
	    \[\restrict{\mu}{\algebra{E}} = \restrict{\outMeas}{\algebra{E}} = \preMeas.\]
	\item
	    Note that, \[\restrict{\nu}{\algebra{E}} = \preMeas = \restrict{\mu}{\algebra{E}}.\] 
	    Thus for any $A \in \algebra{E}$ we have $\nu(A) = \preMeas(A) = \measure{A}$. However,
	    if $\lbrace A_i \rbrace $ is a collection of sets $A_i \in \algebra{E}$, $\countU{A}$ may
	    or may not belong in $\algebra{E}$. In that case, it is not clear that if, 
	    $A = \countU{A}$, then $\nu(A) = \measure{A}$. But for any integer $n$,
	    \[\nu(\finiteU{A}{n}) = \mu(\finiteU{A}{n}).\]
	    To this end observe that,
	    $A_1 \subset (A_1\cup A_2) \subset \dots \subset \countU{A}$. 
	    Thus we have an increasing sequence of sets $\seq{(\finiteU{A}{n})}{n}$ and 
	    since $\nu,\mu$ are
	    measures we get,
	    \[\nu(A) = \lim\limits_{n\to\infty}\nu(\finiteU{A}{n}) =
		\lim\limits_{n\to\infty}\measure{\finiteU{A}{n}} = \measure{A}.\]
	    
	    Here we have used~\ref{thm:prop_of_meas} (4) with $E_i =
	    (\bigcup\limits_{j=1}^{i}A_j)$.

	    Now, consider any $E \in \sigmaGen{\algebra{E}}$ such that a collection $\lbrace A_i
	    \rbrace \subset \algebra{E}$ and $E \subset \countU{A}$. Then,
	    \[\nu(E) \leq \nu(\countU{A}) \leq \sumInf{\nu(A_i)} = \sumInf{\preMeas(A_i)}.\]
	    But this means that,
	    \[\nu(E) \leq \inf\set{\sumInf{\preMeas(A_i)}}{E\subset\countU{A}} = \measure{E},\]
	    since $E \in \sigmaGen{\algebra{E}}$.

	    For any $\epsilon > 0$, let $A = \countU{A}$ such that the collection $\lbrace A_i
	    \rbrace \subset \algebra{E}$ and $E \subset \countU{A}$ and,
	    $\sumInf{\preMeas(A_i)} < \measure{E} + \epsilon$. But $\preMeas(A_i) =
	    \measure{A_i}$ and $\measure{A} \leq \sumInf{\measure{A_i}}$ and hence,
	    $\measure{A} < \measure{E} + \epsilon$.
	    Now if $\measure{E} < \infty$, since $E \subset A$, $\measure{\setDiff{A}{E}} =
	    \measure{A} - \measure{E} < \epsilon$. Thus,
	    \begin{align*}
		\mu(E) \leq &\mu(A) &&\text{because $E \subset A$}\\
		& = \nu(A) &&\text{we proved this above}\\
	        & = \nu(E) + \nu(A\cap\comp{E})	&&\text{because $A = E \disjU A\cap\comp{E}$} \\
		& \leq \nu(E) + \mu(A\cap\comp{E}) &&\text{because $\nu(B) \leq \mu(B), \forall B
		    \in \sigmaGen{\algebra{E}}$} \\
		& \leq \nu(E) + \epsilon
	    \end{align*}	
	    Hence, $\measure{E} = \nu(E)$ whenever $\measure{E} < \infty$.
	\item
	    If $\preMeas$ is $\sigmaFinite$ then $X = \countU{A}$ with $\preMeas(A_i) < \infty$.
	    We can take $A_i$ to be
	    disjoint. (If not we can contruct sequence of disjoint set $F_i$ whose union is $X$).
	    Take any $E \in \sigmaGen{\algebra{E}}$ such that $E \subset \countDisjU{A}$ and hence $E =
	    \disjU\limits_{i}(E\cap A_i)$. Hence,
	    \[\measure{E} = \sumInf{\measure{E\cap A_i}} = \sumInf{\nu(E\cap A_i)} = \nu(E).\]
	    Note that the third equality is beacause $\measure{E\cap A_i} \leq \measure{A_i} =
	    \preMeas(A_i) < \infty$ and we can use the results from above. Hence $\nu = \mu$.
    \end{enumerate}
\end{proof}
\begin{Remark}[name=Relation between $\algebra{M}$ and $\sigmaGen{E}$]\label{rmk:rel_m_sigmaE}
    This is illustrated in~\ref{fig:tikz:carath_ext_thm}. 
\end{Remark}
\begin{figure}
  \includestandalone[width=0.5\textwidth]{tex/tikz_figures/carath_ext_thm}
  \caption{Embedding of the various sets. The measures must agree on the \emph{edges}.}
\label{fig:tikz:carath_ext_thm}
\end{figure}
\begin{Example}
    Show the usefulness of~\ref{prop:out_meas_iduced_pre_measure}.
\end{Example}

\section{Borel Measure on the real line}
We have now the machinary to measure all the Borel sets in $\R$. These are the sets that we care
about. However we only have a notion of measure for simple sets like intervals,
\textbf{h-intervals}, etc. Using such elementary notions we want to construct a measure on
$\borelS{\R}$. We will begin with a general construction that will yield us the Cumulative
Distribution Function (CDF).

To motive the idea, let us say we have a \emph{finite} measure $\mu$ on $\borelS{\R}$. Thus we can measure the
\text{h-intervals} like $\interval[open left]{-\infty}{x}$. Let $F(x) = 
\measure{\interval[open left]{-\infty}{x}}$. Then we can observe a few facts about $F$,
\begin{enumerate}
    \item
	(positive real valued) The function $F$ is such that $\map{F}{\R}{\extRealsPos}$.
    \item
	(increasing) If $x_1 \leq x_2$ then $\measure{\interval[open left]{-\infty}{x_1}} \leq 
	\measure{\interval[open left]{-\infty}{x_2}}$ from monotonicity of $\mu$. Thus $F(x_1) \leq
	F(x_2)$.
    \item
	(right continuous) If $\atobDown{x_n}{x}$ then $\interval[open left]{-\infty}{x} =
	\bigcap\limits_{n=1}^{\infty}\interval[open left]{-\infty}{x_n}$. Thus, 
	$\measure{\interval[open left]{-\infty}{x}} = 
	\lim\limits_{n\to\infty}\measure{\interval[open left]{-\infty}{x_n}}$, i.e.\, 
	$\atobDown{F(x_n)}{F(x)}$ as $\atobDown{x_n}{x}$.
    \item
	If $y > x$, then $\hIntInf{y} = \hIntInf{x} \disjU \hInt{x}{y}$ and hence,
	$\measure{\hInt{x}{y}} = F(y) - F(x)$.
\end{enumerate}
Now we will turn this around and create a measure on $\borelS{\R}$ using function $F$ with 
properties as seen above. Such functions are special and we define them below.
\begin{Definition}[name=Monotonic functions]
    A real valued function $F$ is \emph{increasing} if $F(x) \leq F(y)$ whenever $x < y$ for all $x,y \in
    \R$. A real valued function $F$ is \emph{decreasing} if $F(x) \geq F(y)$ whenever $x < y $ for
    all $x,y \in \R$. A real valued function $F$ is \emph{monotonic} if it is either increasing or
    decreasing.
\end{Definition}
\begin{Theorem}[name=Monotonic functions and one-sided limits]\label{thm:mono_func}
    If $\map{F}{\R}{\R}$ is a monotone function, then $F$ has right and left-hand limits at each
    point $x \in \R$,
    \begin{align*}
	&F(a^{+}) = \lim\limits_{\atobDown{x}{a}}F(x) =  
	\begin{cases}
	    \inf\limits_{x > a}F(x)& \text{If $F$ is increasing}\\
	    \sup\limits_{x < a}F(x)& \text{If $F$ is decreasing}
	\end{cases} \\
	&F(a^{-}) = \lim\limits_{\atobUp{x}{a}}F(x) = 
	\begin{cases}
	    \inf\limits_{x > a}F(x)& \text{If $F$ is increasing}\\
	    \sup\limits_{x < a}F(x)& \text{If $F$ is decreasing}
	\end{cases}
    \end{align*}
\end{Theorem}
\begin{Definition}[name=Right continuous function]
    A monotone function \[\map{f}{\R}{\R}\] is \emph{right continuous} if $F(a) = F(a^{+})$ for all
    $a\in\R$, thus,
    \[\lim\limits_{\atobDown{x}{a}}F(x) = F(a).\]
\end{Definition}
\begin{figure}
    \includestandalone[width=0.45\textwidth]{tex/tikz_figures/right_continuous}
    \caption{A right continuous increasing function}\label{fig:tikz:right_continuous}
\end{figure}
A right continuous function is show in~\ref{fig:tikz:right_continuous}.
Now we begin our construction of measure on $\borelS{\R}$. We have seen that the collection of
h-intervals is an elementary family (~\ref{ex:hint_elem_fam}). 
A collection of finite disjoint union of h-intervals is, then, an
algebra by~\ref{thm:const_algebra_elem}. If we can define a pre-measure on this algebra, then
by~\ref{thm:carath_ext_thm}, we will have a measure on $\borelS{\R}$ (~\ref{rmk:hinterval}). 
Thus our first step will be
constructing a pre-measure.

\begin{Proposition}[name=Pre-measure on collection of
    \textbf{h-intervals}]\label{prop:pre_meas_hint}
    Let $\family{J}^{cr}_{1}$ be the collection of \textbf{h-intervals} on $\R$ and let
    $\algebra{A}$ be the algebra of finite disjoint unions of h-intervals. Let
    $\map{F}{\R}{\R}$ be increasing and right continuous. If $\hInt{a_j}{b_j}$ ($j = 1\dots n$) 
    are disjoint h-intervals, define 
    \[\preMeasure{\finiteUnion{\hInt{a_j}{b_j}}{j}{n}} = \finiteSum{(F(b_j)-F(a_j))}{j}{n}.\] 
    Then, $\preMeas$ is a pre-measure.
\end{Proposition}
\begin{proof}
    First note that if $n=1$ and if $F(x) = x$, then the $\preMeas$ gives us the \emph{length} of
    an h-interval. To show that $\preMeas$ is a pre-measure on the algebra $\algebra{A}$, we need to
    show that $\preMeasure{\emptyset} = 0$ and $\preMeasure{\finiteDisjUnion{I}{i}{n}} =
    \finiteSum{\preMeasure{I_i}}{i}{n}$. Moreover, if $\countDisjUnion{I}{i} \in \algebra{A}$, then we need to
    show that $\preMeasure{\countDisjUnion{I_i}{i}} = \infiniteSum{\preMeasure{I_i}}{i}$. Note that
    $\emptyset = \hInt{b}{b}$ and thus $\preMeasure{\emptyset} = 0$. 


    We will complete the proof in following steps:
    \newline
    \textbf{Step 1}: Let us check finite additivity.
    Let $I \in \algebra{A}$. Then $I = \finiteDisjUnion{I_i}{i}{n}$ where $I_i = \hInt{a_i}{b_i}$. Note
    that $I_i \in \algebra{A}$ and hence, $\preMeasure{I_i} = F(b_i) - F(a_i)$. Thus,
    \[\preMeasure{I} = \finiteSum{(F(b_i)-F(a_i))}{i}{n} 
	= \finiteSum{\preMeasure{I_i}}{i}{n}.\]
    Hence, $\preMeas$ satisfies finite additivity.
    \\
    \textbf{Step 2}: We will show that $\preMeas$ is well defined.
    First let us look at a simple case. Let $I \in \algebra{A}$, 
    $I = \finiteDisjUnion{\hInt{a_i}{b_i}}{i}{n_I}$. Now assume $I = \hInt{a}{b}$. 
    If we re-label $a_i,b_i$, that is $a=a_1 < b_1 = a_2 < b_2 = a_3 \dots < b$,
    then $\preMeasure{I} = F(b) - F(a)$ from cancellation. 
    
    Note, however, such a representation is not unique. We could have a $J\in \algebra{A}$ such that
    $J = \finiteDisjUnion{\hInt{a_j}{b_j}}{j}{n_J} = \hInt{a}{b}$. But $\preMeasure{J}$
    would also yield $F(b) - F(a)$ by the same argument. See~\ref{fig:tikz:borel_pre_meas}. 

    Now, for a general case. If $I = \finiteDisjUnion{I_i}{i}{n_I} = \finiteDisjUnion{J_j}{j}{n_J}$, 
    then $I_i \subset \finiteDisjUnion{J_j}{j}{n_J}$ and thus 
    $I_i = \finiteDisjUnion{(I_i \cap J_j)}{j}{n_J}$. But $I_i \in \algebra{A}$, $I_i \cap J_j$ 
    is an h-interval and hence $\preMeasure{I_i} = \finiteSum{\preMeasure{I_i \cap J_j}}{j}{n_J}$.
    We have,
    $\preMeasure{I} = \finiteSum{\preMeasure{I_i}}{i}{n_I}$ and hence, 
    $\preMeasure{I} = \sum\limits_{i=1}^{n_I}\finiteSum{I_i \cap J_j}{j}{n_J}$. We can repeat this
    argument by noticing that $J_j \subset \finiteDisjUnion{I_i}{i}{n_I}$. Hence, we get
    \[\preMeasure{I} = \sum\limits_{i=1}^{n_I}\finiteSum{I_i \cap J_j}{j}{n_J} = \preMeasure{J}.\]
    Thus $\preMeas$ is well defined.
    \newline
    \textbf{Step 3}: We will show that if $I = \countDisjUnion{I_i}{i} \in \algebra{A}$, then
    $\preMeasure{I} = \preMeasure{\countDisjUnion{I_i}{i}} \geq \infiniteSum{\preMeasure{I_i}}{i}$.

    If $I = \countDisjUnion{I_i}{i} \in \algebra{A}$ then 
    $I = \finiteDisjUnion{\hInt{a_k}{b_k}}{k}{N}$. Since each $I_i$'s and each $\hInt{a_k}{b_k}$ 
    are disjoint we can partition the sequence finitely, i.e for each $1\leq k \leq N$ pick $n$ in 
    the sequence such that
    $I_{n}\subset \hInt{a_k}{b_k}$. The set $n_k = \set{n \in \Zplus}{I_n \subset \hInt{a_k}{b_k}}$
    indexs a subsequence of $I_i$. Hence there is atleast one such $n_k$, lets call it
    $n_0$ such that $n_0$ is infinite, otherwise we will have only a finite sequence of $I_i$.
    Without loss of generality we can disregard other $\hInt{a_k}{b_k}$ and just consider $I =
    \hInt{a_{n_0}}{b_{n_0}} = \hInt{a}{b}$ (after dropping the $n_0$).
    
    Note that for $n \in \Zplus$, we have from finite additivity,
    \[\preMeasure{I} = \finiteSum{\preMeasure{I_i}}{i}{n} + 
	\preMeasure{\disjU\limits_{i>n}I_i}.\]
    Since $F$ is increasing the last term is always greater than zero and thus,
    \[\preMeasure{I} \geq \finiteSum{\preMeasure{I_i}}{i}{n}.\]
    This is true for any $n \in \Zplus$ and thus taking $n \to \infty$, we get
    \[\preMeasure{I} \geq \infiniteSum{\preMeasure{I_i}}{i}.\]
    \newline
    \textbf{Step 4}: We will show that 
    $\preMeasure{I} \leq \infiniteSum{\preMeasure{I_i}}{i}$.
    Pick any $\epsilon > 0$. Note that $I = \hInt{a}{b}$. 
    Now assume $-\infty < a < b < \infty$. Since $F$ is right continuous at $a$,
    there is a $\delta > 0$ such that $F(a+\delta) - F(a) < \epsilon$, but this means that
    $\preMeasure{\hInt{a}{a+\delta}} < \epsilon$. Now, $I = \hInt{a}{a+\delta} \disjU 
    \hInt{a + \delta}{b}$ and since $I \in \algebra{A}$, we get
    \begin{align*}
	\preMeasure{I} =& \preMeasure{\hInt{a}{a+\delta}} + \preMeasure{\hInt{a+\delta}{b}} \\
	& < \epsilon + \preMeasure{\hInt{a+\delta}{b}} 
    \end{align*}
    Now that we have the right inequality we need to worry about $\preMeasure{\hInt{a+\delta}{b}}$.
    We will use a compactness argument.
    Note that $\interval{a+\delta}{b}$ is closed and bounded subset of $\R$ (from our assumption
    about a,b) and is thus compact. Let us use $I_i$'s to create an open cover. However each $I_i$ is
    an h-interval. We will use the right continuity of $F$ to create open intervals from each $I_i$.
    Since $F$ is right continuous at each $b_i$, there is a $\delta_{i} > 0$ such that $F(b_i +
    \delta_{i}) - F(b_i) < \frac{\epsilon}{2^i}$. Thus, 
    $F(b_i+\delta_{i}) < F(b_i) + \frac{\epsilon}{2^i}$.  Let 
    \[\family{G} = \set{(a_i,b_i + \delta_{i})}{i\in\Zplus}.\] be a family of open sets. Then,
    $\interval{a+\delta}{b} \subset \bigcup\limits_{i\in\Zplus}(a_i,b_i + \delta_{i})$.
    Thus $\family{G}$ is an open cover for the compact $\interval{a+\delta}{b}$, and thus there
    exists a finite subcover,
    \[\family{G}_{N} = \set{(a_i,b_i+\delta_{i})}{1\leq i \leq N}.\]
    (We discard any such open set that is contained in a larger open set). After relabeling, we can
    assume that,
    \[b_i + \delta_{i} \in (a_{i+1},b_{i+1}+\delta_{i+1}), 1\leq i\leq N. \] 
    See~\ref{fig:tikz:borel_pre_meas2}. Thus $F(b_i + \delta_i) \geq F(a_{i+1})$. 
    Note that $b < b_N + \delta_{N}$ and $a + \delta > a_1$, and thus since $F$
    is increasing $F(b) \leq F(b_N + \delta_{N})$ and $F(a+\delta) \geq F(a_1)$. 
    Thus,
    \begin{align*}
	\preMeasure{I} <& \epsilon + \preMeasure{\hInt{a+\delta}{b}} \\
	&= \epsilon + F(b) - F(a + \delta) \\
	&\leq \epsilon  + F(b_N + \delta_N) - F(a_1)  \\
	&= \epsilon  + F(b_N + \delta_N) - F(a_N) + \finiteSum{(F(a_{i+1}) - F(a_i))}{i}{N-1} \\
	&\leq \epsilon + F(b_N + \delta_N) - F(a_N) +
       	\finiteSum{(F(b_{i}+ \delta_{i}) - F(a_i))}{i}{N-1}\\
	&< \epsilon + \finiteSum{(F(b_i)+\frac{\epsilon}{2^i}-F(a_i))}{i}{N} \\
	&= \epsilon + \finiteSum{(\preMeasure{I_i} +\frac{\epsilon}{2^i})}{i}{N} \\
	&< \infiniteSum{(\preMeasure{I_i})}{i} + 2\epsilon.
    \end{align*}
    Since $\epsilon$ was arbitrary we get our result.
\end{proof}

\begin{figure}
    \includestandalone[width=0.45\textwidth]{tex/tikz_figures/borel_premeasure}
    \caption{Illustration of proof~\ref{prop:pre_meas_hint}-well defined}\label{fig:tikz:borel_pre_meas}
\end{figure}

\begin{figure}
    \includestandalone[width=0.65\textwidth]{tex/tikz_figures/borel_premeasure2}
    \caption{Illustration of proof~\ref{prop:pre_meas_hint}-finite sub-cover}\label{fig:tikz:borel_pre_meas2}
\end{figure}

Now we apply the framework for constructing measures for the Borel Sets.

\begin{Theorem}[name=Borel measure on the real line]\label{thm:borel_measure_R}
    If $\map{F}{\R}{\R}$ is any increasing, right continuous function, there is a unique Borel
    measure $\borelM{F}$ on $\R$ such that $\borelMeasure{F}{\hInt{a}{b}} = F(b) - F(a)$ for all
    $a,b \in \R$. If $G$ is another such function, we have $\borelM{F} = \borelM{G}$ iff $F-G$ is a
    constant. Conversely, if $\mu$ is a Borel measure on $\R$ that is finite on all bounded Borel
    sets and we define,
    \begin{equation*}
	F(x) =
	\begin{cases}
	    \measure{\hInt{0}{x}} &\text{if $x > 0$}\\
	    0 &\text{if $x = 0$} \\
	    -\measure{\hInt{-x}{0}} &\text{if $x < 0$},
	\end{cases}
    \end{equation*}	
    then $F$ is increasing and right continuous, and $\mu = \borelM{F}$.
\end{Theorem}
\begin{proof}
    The~\ref{prop:pre_meas_hint} implies that $F$ induces a pre measure $\mu_0$ 
    on the algebra $\algebra{A}$ of finite disjoint unions of h intervals. Thus, it can be 
    extended by~\ref{thm:carath_ext_thm} to a measure $\borelM{F}$ on $\sigmaAlgebra$ 
    generated by $\algebra{A}$ which is $\borelS{\R}$. 

    When $F,G$ differ by a constant they give rise to the same pre measure $\mu_0$ on $\algebra{A}$.
    Also $\mu_0$ is $\sigmaFinite$ since,
    \[\R = \bigcup\limits_{i = -\infty}^{\infty}\hInt{i}{i+1},\]
    hence by~\ref{thm:carath_ext_thm} they are equal on $\borelS{\R}$.

    As for the converse, we follow the same argument as the observations noted at the beginning of
    this section. The monotonicity of $\mu$ makes $F$ increasing. Since $\mu$ is continuous from
    above and below, we get the right continuity of $F$. Moreover $\mu\hInt{a}{b} = F(b) - F(a) =
    \borelM{F}$ on $\algebra{A}$ and thus $\mu = \borelM{F}$ by~\ref{thm:carath_ext_thm}.

\end{proof}
\begin{Remark}
    From~\ref{rmk:rel_m_sigmaE}, we know that $\borelS{R}$ is not necessarily complete. We can
    complete it to yield the set of all $\outMeasurable$ sets. This completion \emph{expands} the
    Borel sets and yields what is called the \emph{Lebesgue} measurable sets. We give a precise
    definition and few observations in the subsection below. 
\end{Remark}

\subsection{The Lebesgue-Stieltjes measure in $\R$}
\break{}

\begin{Definition}[name=Lebesgue-Stieltjes measure]
    If $\map{F}{\R}{\R}$ is increasing and right continuous, the \emph{complete} measure of
    $\borelM{F}$ denoted by $\closure{\borelM{F}}$ is called the Lebesgue-Stieltjes (L-S) measure. We
    usually drop the overbar and denote it by the same $\borelM{F}$. The L-S measure associated by
    the function $F(x) = x$ is called the \emph{Lebesgue} measure and is denoted by $\borelM{x} =
    \borelM{\algebra{L}} = \mu$. We denote its domain by $\algebra{L}$, and call it the class of
    \textbf{Lebesgue measurable sets}. 
\end{Definition}

We make some observations about the Lebesgue and Lebesgue-Stieltjes measures. We fix a monotone
increasing, right continuous function $\map{F}{\R}{\R}$, the associated \emph{complete} measure $\mu
= \borelM{F}$, and the domain $\algebra{M}$ of $\mu$. When $F(x) = x$, we denote $\algebra{M}$ as
$\algebra{L}$. The measure space is thus the triple $\measureS{\R}{\algebra{M}}{\mu}$. For any
$E\subset\algebra{M}$, $\mu$ is defined as,
\begin{equation}\label{eq:ls_measure}
    \measure{E} =
    \inf\set{\infiniteSum{\measure{\hInt{a_i}{b_i}}}{i}}{E\subset\countUnion{\hInt{a_i}{b_i}}{i}},
\end{equation}
where $\measure{\hInt{a}{b}} = F(b) - F(a)$.

\begin{Theorem}[name=Equivalent characterizations of L-S measures]\label{thm:equiv_lebesgue_meas}
    If $\measureS{\R}{\algebra{M}}{\mu}$ is a L-S measurable space then for any $E \in \algebra{M}$,
    \begin{itemize}
	\item (Measure through open intervals)
	    \begin{equation}\label{eq:ls_measure_open_interval}
		\measure{E} =
		\inf\set{\infiniteSum{\measure{(a_i,b_i)}}{i}}{E\subset\countUnion{(a_i,b_i)}{i}},
	    \end{equation}
	\item (Measure through open sets)
	    \begin{equation}\label{eq:ls_measure_open_set}
		\measure{E} =
		\inf\set{\measure{G}}{E\subset G,\,\text{G is open}},
	    \end{equation}
	\item (Measure through compact sets)
	    \begin{equation}\label{eq:ls_measure_compact_set}
		\measure{E} =
		\sup\set{\measure{K}}{E\supset K,\,\text{K is compact}}.
	    \end{equation}
    \end{itemize}
\end{Theorem}
\begin{proof}
    Let us first prove~\ref{eq:ls_measure_open_interval}. Given a collection of open intervals
    $(a_i,b_i)$, such that $E \subset \countUnion{(a_i,b_i)}{i}$, we construct a sequence of
    h-intervals,
    \[(a_i,b_i) = \countDisjUnion{\hInt{a_i}{b_i - 1/k}}{k} = \countDisjUnion{I_i^k}{k},\] 
    where, $I_i^k = \hInt{a_i}{b_i - 1/k}$. Therefore,
    \[E \subset \bigcup\limits_{i,k}^{\infty}I_i^k,\] and hence from~\ref{eq:ls_measure} we get,
    \[\measure{E} \leq \sum\limits_{i,k = 1}^{\infty}\measure{I_i^k}.\]
     From our construction we see, since each $\hInt{a_i}{b_i-1/k}$ are disjoint,
     (to be formal, note that $I_i^1 \subset I_i^1 \disjU I_i^2 \dots$, thus
     $\measure{\countUnion{I_i^k}{k}} =
     \lim\limits_{n\to\infty}\measure{\finiteDisjUnion{I_i^k}{k}{n}}$),
    \begin{align*}
	\measure{(a_i,b_i)} &= \lim\limits_{n\to\infty}\finiteSum{\measure{\hInt{a_i}{b_i -
		    1/k}}}{k}{n} \\
	&= \infiniteSum{\measure{\hInt{a_i}{b_i - 1/k}}}{k}
    \end{align*}
    And thus,
    \[\sum\limits_{i,k = 1}^{\infty}\measure{I_i^k} = \infiniteSum{\measure{(a_i,b_i)}}{i}\]
    Hence,
    \[ \measure{E} \leq \infiniteSum{\measure{(a_i,b_i)}}{i}.\]
    To get the other inequality, fix an $\epsilon > 0$. Hence there is a sequence of $I_i =
    \hInt{a_i}{b_i}$ such that $E \subset \countUnion{I_i}{i}$ and
    \[\infiniteSum{\measure{I_i}}{i} < \measure{E} + \epsilon.\] 
     Since, $F$ is right continuous at every $b_i$, there is $\delta_i > 0$ such that
     $F(b+\delta_i)-F(b) < \frac{1}{2^i}$. Now, 
     $(a_i,b_i + \delta_i) \subset \hInt{a_i}{b_i + \delta_i}$, thus 
     $\measure{(a_i,b_i+\delta_i)} \leq \measure{\hInt{a_i}{b_i+\delta_i}}$. We have,
     \begin{align*}
	 \measure{\hInt{a_i}{b_i+\delta_i}} &= F(b_i+\delta_i) - F(a_i) \\
	 &= F(b_i + \delta_i) - F(b_i) + F(b_i) - F(a_i) \\
	 &< \frac{1}{2^i} + \measure{\hInt{a_i}{b_i}}. \\
	 &\quad = \frac{1}{2^i} + \measure{I_i}
     \end{align*}
     Hence,
     \[\infiniteSum{\measure{(a_i,b_i)}}{i} \leq \infiniteSum{\measure{I_i}}{i} <
	 \measure{E} + \epsilon. \]
     Since $\epsilon$ was arbitrary we get the result.

     Now we will prove~\ref{eq:ls_measure_open_set}. Clearly, 
     if $E \subset G = \countUnion{(a_i,b_i)}{i}$ then, 
     $\measure{E} \leq \measure{G}$. Fix an $\epsilon > 0$, then from~\ref{eq:ls_measure_open_interval}, there
     is a collection of open intervals $(a_i,b_i)$ such that $E \subset \countUnion{(a_i,b_i)}{i}$
     and,
     \[\infiniteSum{\measure{(a_i,b_i)}}{i} < \measure{E} + \epsilon.\]
     Put $G = \countUnion{(a_i,b_i)}{i}$. Then $G$ is open and $E \subset {G}$. Also, $\measure{G}
     \leq \infiniteSum{\measure{(a_i,b_i)}}{i}$. Hence, the result follows.

     As for~\ref{eq:ls_measure_compact_set}, note that we are measuring from \emph{inside}. Compact
     sets in $\R$ are closed and bounded and thus are in $\borelS{\R}$ and thus in $\algebra{M}$. So
     for any $K \subset E$, where $K$ is compact, $\measure{K} \leq \measure{E}$ is evident. Thus,
     we need to show the other direction. We will do it by cases.\\ 
     \textbf{CASE I:} \\
     E is bounded. If E is also closed then the result is obvious ($K = E$). Thus assume $E$ is not
     closed. Hence, for any $\epsilon > 0$, we need to find a $K$ compact such that 
     $K \subset E$ and $\measure{K} > \measure{E} - \epsilon$. Since $E$ is not closed
     $\setDiff{\closure{E}}{E}$ is not empty. Using equation~\ref{eq:ls_measure_open_set} there is
     an open set $G$ such that $\setDiff{\closure{E}}{E} \subset G$ and 
     $\measure{G} < \measure{(\setDiff{\closure{E}}{E})}+ \epsilon$. Let $K =
     \setDiff{\closure{E}}{G}$, then $K$ is compact since it is the intersection of two closed sets and
     is bounded. See~\ref{fig:tikz:ls_compact}. Also $K \subset E$. From~\ref{fig:tikz:ls_compact},
     it is also evident that $E\cap\comp{K} = E\cap{G}$, i.e
     \begin{align*} 
	 E\cap\comp{K} =& E \cap ( \comp{\closure{E}} \cup G) \\
	 =& E \cap G
     \end{align*}
     Now, since $E = K \disjU (E\cap\comp{K}) = K \disjU (E \cap G)$, we have,
     \begin{align*}
	 \measure{K} &= \measure{E} - \measure{E\cap G} \\
	 &= \measure{E} - (\measure{G} - \measure{\setDiff{G}{E}}) \\ 
	 &\geq \measure{E} - \measure{G} + \measure{\setDiff{\closure{E}}{E}} \\
	 &> \measure{E} - \epsilon.
     \end{align*}
     \textbf{CASE II:}\\
     If $E$ is unbounded, then $E_j = E\cap\hInt{-j-1}{j}$ is bounded. Also $\atobUp{E_j}{E}$, hence
     $\measure{E} = \lim\limits_{j\to\infty}{\measure{E_j}}$. If $\measure{E} = \infty$, then the
     result is trivial. Hence, assume $\measure{E} < \infty$. Then for any $\epsilon$, there is a
     $N$ such that,
     \[\measure{E} < \measure{E_N} + \frac{\epsilon}{2}.\]
     Also since $E_N$ is bounded, by the preceding argument, there is a compact set $K_N$ such that
     $K_N \subset E_N$ and 
     \[\measure{K_N} > \measure{E_N} + \frac{3\epsilon}{2},\] and thus
     \[\measure{K_N} > \measure{E} + \epsilon.\]
     Thus we have shown that, for any $\epsilon > 0$, there is a compact set $K_N \subset E$, 
     such that 
     \[\measure{K_N} > \measure{E} + \epsilon.\]
\end{proof}

\begin{figure}
    \includestandalone[width=0.40\textwidth]{tex/tikz_figures/ls_compact}
    \caption{Illustration of proof of~\ref{eq:ls_measure_compact_set}-compact
	measure}\label{fig:tikz:ls_compact}
\end{figure}
Note that, the outer measure on $\powSet{\R}$ is defined as,
\[\outMeasure{A} = \inf\set{\sumInf{\preMeasure{I_i}}}{\lbrace I_i \rbrace \subset
	\family{E}\,\text{and}\,A\subset\countUnion{I}{i}},\]
where $\family{E}$ is the algebra of finite disjoint union of h-intervals, and $\preMeasure{I_i} =
F(b_i) - F(a_i)$, where $I_i = \hInt{a_i}{b_i}$. The set $A$ may or may 
not be in $\algebra{M}$. For it to be measurable, it must satisfy $\carathCrit$. 
Whenever $A \in \algebra{M}$, i.e A is L-S measurable, then we know $\outMeasure{A} = \measure{A}$.
In the theorem below, we give an equivalent criteria for a set $A\subset \R$ to be $L-S$ measurable.

\begin{Theorem}[name=Equivalent criteria for (L-S) measurablity]\label{thm:equiv_crit_LS_meas}
    A set $A \subset \R$ is (L-S) measurable iff for every $\epsilon > 0$ there is an open set
    $G_{\epsilon}\supset A$ such that,
    \[\outMeasure{\setDiff{G_{\epsilon}}{A}} < \epsilon.\]
\end{Theorem}
\begin{proof}
    Assume $A \subset \R$ is $(L-S)$ measurable. Then $\outMeasure{A} = \measure{A}$.
    Fix an $\epsilon > 0$. From~\ref{eq:ls_measure_open_set} there is an open set $G$ such that,
    \[\measure{G_{\epsilon}} < \measure{A} + \epsilon.\] Also $A$ satisfies $\carathCrit$,
    \begin{align*}
	\carathEq{G_{\epsilon}}{A}.
    \end{align*}
    Now, $A\cap G_{\epsilon} = A$. Assume $\outMeasure{A} < \infty$. Then,
    \[\outMeasure{G_{\epsilon}\cap\comp{A}} = \outMeasure{G_{\epsilon}} - \outMeasure{A} = 
	\measure{G_{\epsilon}} - \measure{A},\]
    since $G_{\epsilon}$ is an open set, hence $G_{\epsilon} \in \algebra{M}$ and thus 
    $\outMeasure{G_{\epsilon}} = \measure{G_{\epsilon}}$. But
    $\measure{G_{\epsilon}} - \measure{A} < \epsilon$. Hence, we get the result.
    If $\measure{A} = \infty$, then pick $A_j = A \cap \hInt{-j-1}{j}$. Since each $A_j$ is bounded,
    by the preceding argument there is an open set $G_j \supset A_j$ such that 
    $\measure{\setDiff{G_j}{A_j}} < \frac{\epsilon}{2^j}$. Let $G_{\epsilon} = \countUnion{G_j}{j}$.
    Now,
    \begin{align*}
	\measure{\setDiff{G_{\epsilon}}{A}} &= \measure{\setDiff{(\countUnion{G_j}{j})}{A}} \\
	& = \measure{\countUnion{(\setDiff{G_j}{A})}{j}} \\
	& \leq \measure{\infiniteSum{(\setDiff{G_j}{A})}{j}} \\
	& \leq \measure{\infiniteSum{(\setDiff{G_j}{A_j})}{j}} \\
	& \leq \epsilon.
    \end{align*}

    Assume that $A \subset \R$ and for any $\epsilon > 0$ there is an open set $G_{\epsilon}$ such
    that $\outMeasure{\setDiff{G_{\epsilon}}{A}} < \epsilon$. We need to show that $A$ is
    $\outMeasurable$. Let $E \subset \R$ be any arbitrary set. The main idea is to show that if $A$
    is measurable it will split \emph{E} w.r.t $\outMeas$.
    See~\ref{fig:tikz:equiv_ls_measurability}. This motivates the statemet,
    $\setDiff{E}{A} = (\setDiff{E}{G}) \disjU (\intersection{E}{(\setDiff{G}{A})})$. 
    Since $G$ is $\outMeasurable$ we have,
    \begin{align*}\carathGeq{E}{G}.\end{align*} Then from
    monotonicity of $\outMeas$ and noting that $\intersection{E}{(\setDiff{G}{A})} \subset 
    \setDiff{G}{A}$, $E\cap A \subset E\cap G$, we get
    \begin{align*}
	\outMeasure{E\cap A} + \outMeasure{E\cap\comp{A}} &= \outMeasure{E\cap A} 
	+ \outMeasure{(\setDiff{E}{G}) \disjU (\intersection{E}{(\setDiff{G}{A})})} \\
	&\leq \outMeasure{E\cap A} + \outMeasure{\setDiff{E}{G}} 
	+ \outMeasure{\intersection{E}{(\setDiff{G}{A})}} \\
	&\leq \outMeasure{E\cap G} + \outMeasure{\setDiff{E}{G}} 
	+ \outMeasure{\setDiff{G}{A}} \\
	&< \outMeasure{E} + \epsilon. \\
    \end{align*}
    Hence we get the result.
\end{proof}

\begin{figure}
    \includestandalone[width=0.40\textwidth]{tex/tikz_figures/equiv_ls_measurability}
    \caption{Illustration of proof of~\ref{thm:equiv_crit_LS_meas}- part of $E$ not in $A$ is the
	sum of two parts.}\label{fig:tikz:equiv_ls_measurability}
\end{figure}
The~\ref{thm:equiv_crit_LS_meas} states that a set is $(L-S)$ measurable if and only if it can be
approximated from \emph{outside} by an open set in such a way that the difference has arbitrarily
small outer measure. This condition can be adopted as the criteria for a $\outMeasurable$ set. 
However, the $\carathCrit$ is very general and is extremely useful for construction of other measures.

The following theorem gives another characterization of $L-S$ measurable sets, as ones that can
be \emph{squeezed} between open and closed sets.
\begin{Theorem}[name=Squeezing a measurable set by open and closed set]\label{thm:LS_meas_squeezed_closed_open}
    A subset $A \in \R$ is $(L-S)$ measurable if and only if for every $\epsilon > 0$, there is an
    open set $G_{\epsilon}$ and a closed set $F_{\epsilon}$ such that $G_{\epsilon} \supset A
    \supset F_{\epsilon}$ and,
    \[\measure{\setDiff{G_{\epsilon}}{F_{\epsilon}}} < \epsilon.\]
\end{Theorem}
\begin{proof}
    Assume that for every $\epsilon$ there is an open set and closed set $G_{\epsilon},F_{\epsilon}$
    respectively, such that, $G_{\epsilon} \supset A
    \supset F_{\epsilon}$ and
    \[\measure{\setDiff{G_{\epsilon}}{F_{\epsilon}}} < \epsilon.\]
    From monotonicity of $\outMeas$,
    \[\outMeasure{\setDiff{G_{\epsilon}}{A}} < \outMeasure{\setDiff{G_{\epsilon}}{F_{\epsilon}}}
	= \measure{\setDiff{G_{\epsilon}}{F_{\epsilon}}} < \epsilon, \]
    since $G_{\epsilon},F_{\epsilon} \in \algebra{M}$. Thus, from~\ref{thm:equiv_crit_LS_meas}, $A$
    is measurable.

    Now, assume $A\subset\R$ is $(L-S)$ measurable. Fix an $\epsilon > 0$. Then there is an open set 
    $G_{\epsilon}$ such that $\measure{\setDiff{G_{\epsilon}}{A}} < \epsilon/2$. Since $A$ is
    $(L-S)$ measurable, $\comp{A}$ is also $(L-S)$ measurable and hence there is an open set
    $H_{\epsilon}\supset \comp{A}$ such that 
    $\measure{\setDiff{H_{\epsilon}}{\comp{A}}} < \epsilon/2$. Let $F_{\epsilon} =
    \comp{H_{\epsilon}}$. Thus $F_{\epsilon} \subset A$. 
    Note that $A\cap\comp{F_{\epsilon}} = H_{\epsilon}\cap A = \setDiff{H_{\epsilon}}{\comp{A}}$. 
    Hence,
    \begin{align*}
	\measure{\setDiff{G_{\epsilon}}{F_{\epsilon}}} &= \measure{\setDiff{A}{F_{\epsilon}}} +
	\measure{\setDiff{G_{\epsilon}}{A}}\\
	& < \epsilon/2 + \epsilon/2 \\
	&\quad = \epsilon.
    \end{align*}
\end{proof}
\begin{Remark}\label{rmk:ls_meas_compact_open}
    In the theorem above, if $A \in \algebra{M}$ and $\measure{A} < \infty$, then we can replace the
    closed $F_{\epsilon}$ by a compact $K_{\epsilon}$ using~\ref{eq:ls_measure_compact_set}.
\end{Remark}
The next theorem states that any Borel set can be approximated up to a set of measure zero by a
Borel set.
\begin{Theorem}[name=Borel Approximation]\label{thm:borel_approximation}
    Suppose that $A\subset \R$ is $(L-S)$ measurable. Then there exists a $G_{\delta}$ and
    $F_{\sigma}$ set in $\borelS{\R}$ such that,
    \[G_{\delta} \supset A \supset F_{\sigma}, \quad \measure{\setDiff{G_{\delta}}{A}} =
	    \measure{\setDiff{F_{\sigma}}{A}} = 0.\]
\end{Theorem}
\begin{proof}
    Since, $A$ is $(L-S)$ measurable, for every $k \in\Zplus$, there is an open set $G_k$ and a
    closed set $F_k$ such that $G_k \supset A \supset F_k$ and  
    $\measure{\setDiff{G_k}{F_k}} \leq \frac{1}{k}$. 
    Put $G_{\delta} = \countIntersection{G_k}{k}$ and $F_{\sigma} = \countUnion{F_k}{k}$.
    Fix an $\epsilon > 0$. For every $k$,
    \begin{align*}
	G_{\delta}\cap\comp{A} &\subset G_k\cap\comp{A}\\
	& \subset G_k \cap \indxComp{F}{k}
    \end{align*}
    Thus picking a large $k$ such that $\frac{1}{k} < \epsilon$,
    $\measure{\setDiff{G_{\delta}}{A}} \leq \measure{\setDiff{G_k}{F_k}} < \epsilon$. Since $\epsilon$ was
    arbitrary, the result follows.
    Similarly,
    \begin{align*}
	A\cap\indxComp{F}{\sigma} &\subset G_k\cap\indxComp{F}{\sigma}\\
	& \subset G_k \cap \indxComp{F}{k},
    \end{align*}
    and the result follows.
\end{proof}

%\section{Borel and Lebesgue Measure in \texorpdfstring{$\Rn$}{}}
\section{Borel and Lebesgue Measure in {$\Rn$}}
Now we will extend the Borel measure on the real line to $\Rn$. If we view $\Rn$ as $\R \times \R
\dots \times \R$, then we can think of $\Rn$ as the n dimensional product of the Real line. However,
we do not have the machinary yet to deal with product measures. We will have to build the Borel
measure as we did in~\ref{prop:pre_meas_hint}. This will give rise to a distribution function in
$\Rn$ and the Lebesgue-Stieltjes measure will follow. 

The
key concept in that construction was that of the distribution function. Thus, given a finite measure
on $\R$ we were able to construct an increasing, right continuous function $F$ on $\R$ and vice
versa. See~\ref{thm:borel_measure_R}. Our goal would be to extend that idea to $\Rn$. Once we have a
borel measure on $\Rn$, the complete measure associated with it would give us the Lebesgue-Stieltjes
measure on $\Rn$. In $\R$ we looked at an elementary family of h-intervals. In $\Rn$, it will be the
family of right semi-closed intervals (rectangles) $\family{J}_n^{cr}$. 
However, as with $\R$ we need to include
$-\infty,\infty$ and these will give us the generalized \textbf{h-intervals}. 

If $\vect{a} = (a_1,\dots,a_n), \vect{b} = (b_1,\dots,b_n)$ are points in $\Rn$,
let us define $\hInt{\vect{a}}{\vect{b}}$ as the set $\set{\vect{x}\in\Rn}{a_i< x_i \leq b_i,\,1\leq
i\leq n}$. Similarly $(\vect{a},\infty) = \set{\vect{x}\in\Rn}{x_i > a_i,\,1\leq i \leq n}$ and 
$\hInt{-\infty}{\vect{b}} = \set{\vect{x}\in\Rn}{x_i\leq b_i,\,1\leq i\leq n}$.
Let $\family{E} \subset \Rn$ be the collection of all generalized h-intervals as described above.
Then $\family{E}$ is an elementary family and by~\ref{thm:const_algebra_elem} the family
$\algebra{A}$ of finite disjoint union of generalized h-intervals in $\family{E}$ is an algebra.
Moreover, $\sigmaGen{\family{A}} = \borelS{\Rn}$. Thus, we have to define a pre-measure $\preMeas$
on $\algebra{A}$ and by~\ref{thm:carath_ext_thm} we will have a unique measure on $\borelS{\Rn}$. We
proceed as in~\ref{prop:pre_meas_hint}, but we need to define what is means for a function $F$ in
$\Rn$ to be increasing and right continuous. The notion of a distribution function is more
complicated in $\Rn$. As with our motivation in the previous section, let us assume we have a finite
measure (for example the Lebesgue measure on a cube) in $\Rn$. In $\R$, we defined $F(x) =
\measure{\hInt{-\infty}{x}}$. 
Let us extend the same idea and define $\map{F}{\Rn}{\R}$ as,
\[F(x_1,x_2,\dots,x_n) = \measure{\set{\vect{\omega}\in\Rn}{\omega_{i}\leq x_i}}.\]
In $\R$, this lead to $\measure{\hInt{a}{b}} = F(b) - F(a)$. However, this wont be true in $\Rn$.
We need some more ideas to make a jump from the real line to $\Rn$. One is the notion of order. In
$\Rn$ there is no total order, however the following notion will suffice,
\begin{Definition}[name=Order in $\Rn$]
    Let $\vect{a},\vect{b}$ be points in $\Rn$. We say that $\vect{a}\leq\vect{b}$ iff $a_i\leq b_i$
    for all $1\leq i \leq n$.
\end{Definition}
To generalize $F(b) - F(a)$, we need to define the \emph{difference} operator of a function $F$ from
$\Rn$ to $\R$.
\begin{Definition}[name=Difference Operator]
    Let $\map{G}{\Rn}{\R}$. We define the difference of $G$ at the $ith$ cordinate evaluated at
    $a_i,b_i$ as,
    \[\diffOp{b_i}{a_i}G = G(x_1,x_2,\dots,x_{i-1},b_i,\dots,x_n) - 
	G(x_1,x_2,\dots,x_{i-1},a_i,\dots,x_n). \]
\end{Definition}
\begin{Definition}[name=Increasing Function in $\Rn$]
    Let $\map{F}{\Rn}{R}$ and for $\vect{a} \leq \vect{b}$, let us denote by
    $F(\hInt{\vect{a}}{\vect{b}})$ as,
    \[F(\hInt{\vect{a}}{\vect{b}}) = \diffOp{b_1}{a_1}\dots\diffOp{b_n}{a_n}F.\]
    The function $F$ is said to be increasing iff $F(\hInt{\vect{a}}{\vect{b}}) \geq 0$ whenever
    $\vect{a} \leq \vect{b}$.
\end{Definition}
The definition may seem peculiar but it is motivated by the following theorem,
\begin{Theorem}[name=Measure and Distribution function in $\Rn$]
    Let $\mu$ be a finite measure on $\borelS{\Rn}$ and define,
    \[F(\vect{x}) = \measure{\hInt{-\infty}{\vect{x}}},\]
    Then, if $\vect{a} \leq \vect{b}$, 
    \[\measure{\hInt{\vect{a}}{\vect{b}}} = F(\hInt{\vect{a}}{\vect{b}}).\]
\end{Theorem}
\begin{proof}
    We prove it for $n=3$ to make the notation simpler. 
    \begin{align*}
	\diffOp{b_3}{a_3}F &= F(x_1,x_2,b_3) - F(x_1,x_2,a_3) \\
	&= \measure{\set{\vect{\omega}}{\omega_{1}\leq x_1,\,\omega_2\leq x_2,\,\omega_3\leq b_3}}
	\\ 
	&\quad	- \measure{\set{\vect{\omega}}{\omega_{1}\leq x_1,\,\omega_2\leq x_2,\,\omega_3\leq
		a_3}} \\
	&= \measure{\set{\vect{\omega}}{\omega_{1}\leq x_1,\,\omega_2\leq x_2,\,a_3 < \omega_3\leq
		b_3}}
    \end{align*}
    Thus,
    \begin{align*}
	\diffOp{b_2}{a_2}\diffOp{b_3}{a_3}F &=\diffOp{b_2}{a_2}(F(x_1,x_2,b_3) - F(x_1,x_2,a_3)) \\
	&= F(x_1,b_2,b_3) - F(x_1,a_2,b_3) \\
	&\quad -F(x_1,b_2,a_3) + F(x_1,a_2,a_3)\\
	&= \measure{\set{\vect{\omega}}{\omega_1\leq x_1,\,a_2 < \omega_2 \leq b_2,\, a_3 < \omega_3
	    \leq b_3}}
    \end{align*}
    and thus we get the result from another application of the difference operator.
\end{proof}
For defining a right continuous function we need to define a \emph{right} limit. We say that a
sequence of points $(\vect{x}^{(n)})$ in $\Rn$ coverges to a point $\vect{x}$ from the \emph{right}
if for each co-ordinate $\atobDown{x_i^{(n)}}{x}$. We say that  $(\vect{x}^{(n)})$ \emph{decreases}
to $\vect{x}$ and denote it by $\atobDown{(\vect{x}^{(n)})}{\vect{x}}$.
\begin{Definition}[name=Right continuous function in $\Rn$]
    A function, $\map{F}{\Rn}{R}$ is right continuous if for a sequence of points 
    $\atobDown{\vect{x}^{(n)}}{\vect{x}}$; $\atobDown{F(\vect{x^{n}})}{F(\vect{x})}$.
\end{Definition}
\begin{Definition}[name=Distribution function in $\Rn$]
    A distribution function $F$ is an increasing, right continuous function $\map{F}{\Rn}{\R}$.
\end{Definition}
\begin{Theorem}[name=Borel measure in $\Rn$]\label{thm:borel_meas_rn}
    Let $F$ be a distribution function on $\Rn$ and set,
    \[\borelM{F} = F(\hInt{\vect{a}}{\vect{b}}),\quad \vect{a}\leq\vect{b}.\]
    Then $\borelM{F}$ is a measure on $\borelS{\Rn}$ and is called the borel measure on $\Rn$
    induced by $F$.
\end{Theorem}
\begin{proof}
    The proof follows the same idea as in~\ref{prop:pre_meas_hint}
    and~\ref{thm:equiv_lebesgue_meas}.
\end{proof}
\begin{Definition}[name=Lebesgue-Stieltjes (L-S) measure on $\Rn$]
    Let $F$ be a distribution function in $\Rn$ and $\borelM{F}$ the corresponding borel measure on
    $\Rn$. The completion of $\borelM{F}$ is called the Lebesgue-Stieltjes measure on $\Rn$ and we
    also denote is by $\borelM{F}$.
\end{Definition}
\begin{Example}
    Let $F_1,F_2,\dots,F_n$ be n distribution function on $\R$, and define,
    \[F(x_1,x_2,\dots,x_n) = F_1(x_1)\times F_2(x_2)\times\dots F_n(x_n).\]
    $F$ is a distribution function on $\Rn$ with,
    \[F(\hInt{\vect{a}}{\vect{b}}) = \finiteProduct{(F_i(b_i)-F_i(a_i))}{i}{n}.\]
    When $F_i = x$ for all $i$, $F(\hInt{\vect{a}}{\vect{b}}) = \finiteProduct{(b_i-a_i)}{i}{n}$ and
    the corresponding $L-S$ measure is the Lebesgue measure on $\Rn$.
\end{Example}
For a given distribution function $F$, we denote by $\measureS{\Rn}{\algebra{M}}{\mu}$ as
Lebesgue-Stieltjes $(L-S)$ measure space. In particular the space of Lebesgue measure is denoted by
$\measureS{\Rn}{\algebra{L}}{\mu}$. Thus by definition a set $E\subset \Rn$ is $L-S$ measurable if,
\begin{equation*}
    \measure{E} = \inf\set{\infiniteSum{F({\hInt{\vect{a}}{\vect{b}}}_{i})}{i}}{E \subset
	\countUnion{{\hInt{\vect{a}}{\vect{b}}}_i}{i}},
\end{equation*}
which is equivalent to
\begin{equation*}
    \measure{E} = \inf\set{\infiniteSum{\measure{R_i}}{i}}{E \subset
	\countUnion{R_i}{i}},
\end{equation*}
where $R_i$ are a sequence of closed rectangle with sides parallel to co-ordinate axes. 


It is clear that all equivalent criteria for $(L-S)$ measurability we addressed for $\R$ holds for
$\Rn$. Thus~\ref{thm:equiv_lebesgue_meas} holds in $\Rn$, in particular
\begin{equation}\label{eq:lebesgue_meas_rn_open_set}
    \measure{E} = \inf\set{\measure{G}}{E\subset G,\quad\text{G is open}},
\end{equation}
and 
\begin{equation}\label{eq:lebesgue_meas_rn_compact_set}
    \measure{E} = \sup\set{\measure{K}}{E\supset K,\quad\text{K is compact}}.
\end{equation}
To prove~\ref{eq:lebesgue_meas_rn_open_set}, note that if $E \subset G$ then $\outMeasure{E} \leq
\outMeasure{G}$. Since both $E,G$ are $(L-S)$ measurable their outer measure is just the $(L-S)$
measure $\mu$. Assume $\measure{E}$ is finite.  
To prove the other direction, for any $\epsilon$ there exist a sequence of rectangles
$\lbrace R_i \rbrace \subset \Rn$ such that $\infiniteSum{\measure{R_i}}{i} < \measure{E} +
\epsilon$. Since $\measure{E}$ is finite each rectangle $R_i$ is bounded and there is an open
rectangle $\interior{S_i}$ such that $R_i \subset \interior{S_i}$ and $\measure{{S_i}} <
\measure{R_i} + \frac{\epsilon}{2^i}$. Since $\interior{S_i}$ is an open set, let $G =
\countUnion{\interior{S_i}}{i}$. Thus we get our result.~\ref{eq:lebesgue_meas_rn_compact_set}
follows exactly as in~\ref{thm:equiv_crit_LS_meas} (3) with minor modification in the case that $\measure{E}$
is infinite.

Similarly as in~\ref{thm:equiv_crit_LS_meas},~\ref{thm:LS_meas_squeezed_closed_open}
and~\ref{thm:borel_approximation},
\begin{Theorem}[name=$(L-S)$ Measurability in $\Rn$]\label{thm:equiv_Lebesgue_meas_rn}
A set $E \subset \Rn \in \algebra{M}$ is $(L-S)$ measurable if and only if for every 
$\epsilon > 0$,
    \begin{enumerate}
	\item
	    there is an open set $G_{\epsilon} \subset \Rn$ such that
	    $G_{\epsilon} \supset E$ and,
	    \[\outMeasure{\setDiff{G_{\epsilon}}{E}} < \epsilon,\]
	\item
	    there is an open set $G_{\epsilon}$ and a closed set $F_{\epsilon}$, such that
	    $G_{\epsilon}\supset E \supset F_{\epsilon}$ and,
	    \[\measure{\setDiff{G_{\epsilon}}{F_{\epsilon}}} < \epsilon,\]
	\item
	    $E$ differs from a $G_{\delta}$ and a $F_{\sigma}$ set by a set of measure zero.
    \end{enumerate}
\end{Theorem}

\subsection{Uniqueness of measures and properties of Lebesgue measures}
We will use the $\pi-\lambda$ theorem to discuss the uniqueness of measures and use that to show some
important properties of Lebesgue measure.
\begin{Theorem}\label{thm:uniq_measures_pilam}
    Assume that $\metricS{X}{\famM}$ is a measurable space and that $\famM = \sigmaGen{\famE}$, where $\famE$
    is a $\pi-$ class. Suppose that $\mu_{1},\mu_{2}$ are two measures on $\sigmaGen{\famE}$ such that they
    are sigma-finite on $\famE$. If $\mu_{1},\mu_{2}$ agree on $\famE$, then they agree on $\sigmaGen{\famE}$.
\end{Theorem}
\begin{proof}
    Since measures are sigma-finite on $\famE$, there is a sequence of subsets $\seq{X}{n}$ such that
    $\atobUp{X_n}{X}$ and $\mu_{1}(X_n) = \mu_{2}(X_n) < \infty$ for each $n$. For any $A \in
    \sigmaGen{\famE}$, we observe that,
    \[A \subset X = \countUnion{X_n}{n},\]
    and hence $A= \countUnion{(A\cap X_n)}{n}$. Thus, to show that $\mu_{1}(A) = \mu_{2}(A)$ for any $A \in
    \sigmaGen{\famE}$, we need to show that $\mu_{1}(A\cap X_n) = \mu_{2}(A\cap X_n)$ for each $n$; because if
    that holds, we get
    \[\mu_{1}(A) = \mu_{1}(\countUnion{(A\cap X_n)}{n}) = \lim\limits_{n\to\infty} \mu_1(A\cap X_n) =
	\lim\limits_{n\to\infty} \mu_2(A\cap X_n) = \mu_2(A),\]
    since $\atobUp{(A\cap X_n)}{(A\cap X)}$. 
    Thus, let $\famF_{j}$ be defined as the class of all those sets $A$ in $\sigmaGen{\famE}$ such that:
    \[\famF_{j} = \set{A\in\sigmaGen{\famE}}{\mu_1(A\cap X_n) = \mu_2(A\cap X_n)}.\]
    Since $\famE$ is a $\pi-$ class we see that for each $j$, $\famE \subset \famF_{j}$. By construction,
    $\famF_{j} \subset \sigmaGen{\famE}$. If we show that $\famF_{j}$ is a $\lambda-$ class, we are done;
    since then $\famF_{j} \supset \sigmaGen{\famE}$ by the $\pi-\lambda$ theorem and hence we will get
    $\famF_{j} = \sigmaGen{\famE}$ for each $j$.

    It is easy to show that $\famF_{j}$ is a $\lambda-$ class.
    \begin{enumerate}
	\item Clearly $X \in \famF_{j}$.
	\item Let $A \in \famF_{j}$. Note that
	    \[\mu_1(X_j) = \mu_1(X_j \cap A) + \mu_1(X_j \cap \comp{A})\]
	    Since $(X_j\cap A) \subset X_j$, $\mu_1(X_j\cap A)$ is finite.
	    Now,
	    \begin{align*}
		\mu_1(\comp{A}\cap X_j) &= \mu_1(X_j) - \mu_1(X_j \cap A) \\
		& = \mu_2(X_j) - \mu_2(X_j \cap A) \\
		& = \mu_2(\comp{A}\cap X_j).
	    \end{align*}
	\item Let $(A_n)$ be a sequence of pairwise disjoint sets in $\famF_{j}$
	    \begin{align*}
		\mu_1(X_j \cap \left(\countUnion{A_n}{n}\right)) &= \mu_1(\countUnion{(X_j\cap A_n)}{n}) \\
		& = \infiniteSum{\mu_1(X_j\cap A_n)}{n} \\
		& = \infiniteSum{\mu_2(X_j\cap A_n)}{n} \\
		& = \mu_2(\countUnion{(X_j\cap A_n)}{n}) \\
		& = \mu_2(X_j \cap \left(\countUnion{A_n}{n}\right)).
	    \end{align*}
	    Hence, $\countUnion{A_n}{n}$ is in $\famF_j$.
    \end{enumerate}
    Thus, $\famF_j$ is a $\lambda-$ class for each $j$. Hence, we get the result.
\end{proof}
