\chapter{Metric spaces}
In this chapter we will start with the description of metric spaces and some topological concepts involved in
such spaces. Metrics are abstraction of the concept of distance that is familiar to us.
\section{Basic definitions}
\begin{Definition}
    A metric space is a pair $\metricS{X}{d}$ where $X$ is a non-empty set and $d$ is a function
    \[\map{d}{X\times X}{\interval[open right]{0}{\infty}},\]
    called a metric, that satisfies the following properties for all $x,y,z$ in $X$:
    \begin{itemize}
	\item (positive reflexive)
	    $d(x,y) = 0$ if and only if $x = y$.
	\item (symmetric)
	    $d(x,y) = d(y,x)$.
	\item (triangle inequality)
	    $d(x,y) \leq d(x,z) + d(z,y)$.
    \end{itemize}
\end{Definition}
These properties are minimal in order to capture the idea of distance between two points.
\begin{Example}\label{ex:metric_space}
    We will show some basic examples of a metric space.
    \begin{enumerate}
	\item
	    Let $X = \R$ and define $d(x,y) = \distR{x}{y}$. This is the standard metric in $\R$.
	\item
	    Let $X = \Rn$ and for $\vx,\vy$ in $\Rn$ define 
	    \[d(\vx,\vy) = (\finiteSum{{(x_j - y_j)}^2}{j}{n})^{\frac{1}{2}}.\] 
	    The proof that $d$
	    is a metric relies on the famous \textbf{Cauchy-Schwarz} inequality. This metric is called the
	    \textbf{Euclidean} metric and in $\R^2$ gives us the standard Pythagorean distance.
	\item
	    Let $X = \Rn$ and for $\vx,\vy$ in $\Rn$ define
	    \[d(\vx,\vy) = \finiteSum{\distR{x_j}{y_j}}{j}{n}.\]
	    To verify that $\metricS{X}{d}$ is indeed a metric space we need to check if $d$ is a metric. The
	    hardest one to check is triangle inequality. But this follwows from the triangle inequality of
	    absolute values in $\R$ i.e.~for any $z_j$ in $\R$, $\distR{x_j}{z_j} \leq \distR{x_j}{y_j} +
	    \distR{y_j}{z_j}$. 
	\item
	    Let $X = \Rn$ and for $\vx,\vy$ in $\Rn$ define,
	    \[d(\vx,\vy) = \max\set{\distR{x_j}{y_j}}{1\leq j \leq n}.\]
	\item
	    Any non-empty set can be made into a metric space. Let $X$ be a non-empty set and for any $x,y \in
	    X$ define
	    \begin{equation*}
		d(x,y) = 
		\begin{cases}
		    0 &\,\text{if $x = y$},\\
		    0 &\, \text{if $x \neq y$}.
		\end{cases}
	    \end{equation*}
	    This is easily verified to be a metric space and $d$ is called the \textbf{discrete} metric.
	\item
	    Suppose $\metricS{X}{d}$ is given. If $Y$ is a nonempty subset of $X$, then $\metricS{Y}{d}$ is a
	    metric space and is referred to as a subspace. As a specific example consider the set $X=\R$ and
	    the metric space $\metricS{\R}{d}$ where $d$ is any metric. Then $Y = \interval{0}{1}$ defines a 
	    metric space $\metricS{Y}{d}$.
    \end{enumerate} 
\end{Example}
Now that we have a notion of distance we can define a notion of \textbf{closeness}. This is done through the
concept of \emph{balls} in a metric space.
\begin{Definition}[name=Open Ball]
    Let $\metricS{X}{d}$ be a metric space and let $x$ be an element of $X$ and $r$ be a positive real number.
    The set
    \[\ball{r}{x} = \set{y\in X}{d(y,x) < r},\]
    is called the open ball of radius $r$ around $x$.
\end{Definition}
\begin{Definition}[name=Closed Ball]
    Let $\metricS{X}{d}$ be a metric space and let $x$ be an element of $X$ and $r$ be a positive real number.
    The set
    \[\closure{\ball{r}{x}} = \set{y\in X}{d(y,x) \leq r},\]
    is called the closed ball of radius $r$ around $x$.
\end{Definition}
Notice that when $s < r$, $\closure{\ball{s}{x}} \subset \ball{r}{x}$ for any metric.
It is highly illustrative to figure out unit balls in a metric space. For example, let us take $X = \R^2$ and
consider two metric spaces $\metricS{X}{d_1}$ and $\metricS{X}{d_2}$ where $d_1$ is the Euclidean metric
defined as in~\ref{ex:metric_space} (2) and $d_2$ be the metric defined as in~\ref{ex:metric_space} (3).
How do the open balls of radius $1$ \emph{look} around the point $\vx = \mathbf{0}$ 
in these two metric spaces?

%%%%% Figure here on open balls %%%%%%%%%%
The notion of closeness can be made precise with the concept of \emph{open} sets in metric spaces.
\begin{Definition}[name=Open set]
    If $\metricS{X}{d}$ is a metric space, then a subset $G$ of $X$ is open if for each $x$ in $G$ there is a
    radius $r > 0$ such that the ball of radius $r$ around $x$ is contained entirely in $G$. This is denoted
    logically as,
    \[\forEv{x\in G}\thereIs{r > 0 \in \R}\suchThat{\ball{r}{x}}{\subset}{G}.\]
\end{Definition}
When is a set $G \subset X$ \textbf{NOT} open? If we can find an $x \in G$ such that no matter what $r > 0$ we
choose, the open ball of radius $r$ around $x$ is not contained in $G$. This is immediately evident by
negating the logical statement in the definition.
\begin{Definition}[name=Closed set]
    If $\metricS{X}{d}$ is a metric space, then a subset $F$ of $X$ is closed if the complement of $F$ in $X$
    is open i.e.~$\setDiff{X}{F}$ is open. This is denoted logically as,
    \[\forEv{x\in X}\suchThat{x\not\in F}{\implies}{\thereIs{r > 0\in\R}\ball{r}{x}\cap F = \emptyset}.\]
\end{Definition}
The above definition can be made entirely in terms of the set $F$ by taking the contrapositive of the
implication in the logical statement above i.e.~a set $F$ is closed if for any point $x \in X$ it is
the case that no matter what radius $r$ we take the ball of radius $r$ around $x$ always intersects $F$, then
$x$ must be in $F$. Logically,
\[\forEv{x\in X}\suchThat{\forEv{r > 0 \in \R}\ball{r}{x}\cap F \neq \emptyset}{\implies}{x\in F},\]
is equivalent to saying $F$ is closed in $X$. This notion gives us a nice way to judge when a set $F$ is
\textbf{NOT}
closed in $X$. For that to happen there must be an $x \in X$ such that no matter any radius $r$ we take, the
ball of radius $r$ about $x$ always intersects $F$ but $x$ does not belong to $F$.
\begin{Example}\label{ex:open_closed_set}
    We show a few examples of open and closed sets in a metric space $\metricS{X}{d}$.
    \begin{enumerate}
	\item
	    We observe that $X,\emptyset$ are both open and closed sets. In fact it is clear that $X$ is open
	    and that $\emptyset$ is open because it has no points. Then taking complements we see that they
	    both are closed.
	\item
	    The open ball is an open set. To see this let $G = \ball{r}{x}$ be an open ball in $X$. To show that
	    it is an open set we need, for any point $y$ in $G$, to find a radius $\delta > 0$ such that the
	    ball $\ball{\delta}{y} \subset G$. Let $\delta = r - d(y,x)$. This is greater than $0$ since $y$
	    belongs to $G$. Now to show that $\ball{\delta}{y} \subset G$ we need to show that for any $z \in
	    \ball{\delta}{y}$, $z$ must be in $G$ i.e.~$d(z,x) < r$. But this follows from the following
	    argument,
	    \begin{align*}
		d(z,x) &\leq d(z,y) + d(y,x),\\
		& \leq \delta + r - \delta, \\
		&\quad = r.
	    \end{align*}
	\item
	    The closed ball is a closed set. To see this let $F = \closure{\ball{r}{x}}$ be a closed ball in
	    $X$. To show that $F$ is closed we need to show that for any $y\in X$ if $y \not \in F$ then we
	    must be able to find a radius $\delta > 0$ such that $\ball{\delta}{y}$ doesn't intersect $F$. Let
	    $y$ be a point in $\setDiff{X}{F}$. Hence $d(y,x) > r$. Let $\delta = d(y,x) - r$. Hence $\delta >
	    0$. For any $z \in \ball{\delta}{y}$,
	    \begin{align*}
		d(x,z) + d(z,y) &\geq d(x,y)\\
		&\implies d(x,z) \geq d(x,y) - d(z,y), \\
		&\implies d(x,z) \geq r.
	    \end{align*}
	    Hence $\ball{\delta}{y}\cap F = \emptyset$.
	\item
	    Any finite subset of $X$ is closed. In fact if $F =\setX{x_1,x_2,x_3,\ldots,x_n}$, then for any
	    $y\in \setDiff{X}{F}$, let $\delta = \min\set{d(y,x_i)}{1\leq i \leq n}$. Then
	    $\ball{\frac{\delta}{2}}{y} \cap F = \emptyset$.
    \end{enumerate}
\end{Example}

Open sets are closed under arbitrary unions and closed sets are closed under arbitrary intersections. We make
this precise in the next proposition.
\begin{Proposition}\label{prop:open_sets_prop}
    Let $\metricS{X}{d}$ be a metric space and let $\famG$ be the collection of all open sets in $X$. Then,
    \begin{enumerate}
	\item $X,\emptyset$ are in $\famG$.
	\item If $G_1,G_2,\ldots,G_n$ are $n$ open sets in $\famG$, then $\finiteIntersection{G_i}{i}{n}$ is
	    also in $\famG$.
	\item If $\set{G_{\alpha}\in \famG}{\alpha \in I}$ is an arbitrary sub collection of open sets in
	    $\famG$ then $\indexUnion{G_{\alpha}}{\alpha}{I}$ is also in $\famG$.
    \end{enumerate}
\end{Proposition}
\begin{proof}
    We prove in order.
    \begin{enumerate}
	\item We showed this in the example.
	\item Let $x$ be in $\finiteIntersection{G_i}{i}{n}$. Then $x$ is in each $G_i$ and since $G_i$ is open
	    there is an $r_i > 0$ such that $\ball{r_i}{x} \subset G_i$. Take 
	    $r = \min\set{r_i}{1\leq i \leq n}$.
	    Then $r > 0$ and $\ball{\frac{r}{2}}{x} \subset G_i$ for all $i$'s. Hence
	    \[\ball{\frac{r}{2}}{x} \subset \finiteIntersection{G_i}{i}{n}.\]
	\item Let $x$ be in $\indexUnion{G_{\alpha}}{\alpha}{I}$. Then there is an $\alpha$ in $I$ such that
	    $x$ is in $G_{\alpha}$. Since $G_{\alpha}$ is open there is an $r > 0$ such that $\ball{r}{x}
	    \subset G_{\alpha}$. But $G_{\alpha} \subset \indexUnion{G_{\alpha}}{\alpha}{I}$ and hence 
	    \[\ball{r}{x} \subset \indexUnion{G_{\alpha}}{\alpha}{I}.\]
    \end{enumerate}
\end{proof}
\begin{Proposition}\label{prop:closed_sets_prop}
    Let $\metricS{X}{d}$ be a metric space and let $\famF$ be the collection of all closed sets in $X$. Then,
    \begin{enumerate}
	\item $X,\emptyset$ are in $\famF$.
	\item If $F_1,F_2,\ldots,F_n$ are $n$ closed sets in $\famF$, then $\finiteUnion{F_i}{i}{n}$ is
	    also in $\famF$.
	\item If $\set{F_{\alpha}\in \famF}{\alpha \in I}$ is an arbitrary sub collection of closed sets in
	    $\famF$ then $\indexIntersection{F_{\alpha}}{\alpha}{I}$ is also in $\famF$.
    \end{enumerate}
\end{Proposition}
\begin{proof}
    De Morgan's law of set complements.
\end{proof}

One must be very careful about the universe when talking about open and closed sets in a metric space. For
example if we consider $X = \R^2$ with the euclidean metric and $\R = Y \subset \R^2$ with the same Euclidean
metric, then an interval $(a,b)$ is open in $Y$ but is \textbf{NOT} open in $X$. To make this precise we will
define the notion of relative openess (or closedness).

\begin{Definition}[name=Open balls in subspace]
    Let $\metricS{X}{d}$ be a metric space and consider $Y \subset X$. We define an open ball in
    $\metricS{Y}{d}$ of radius $r$
    about a point $y \in Y$ as the set,
    \[\relBall{r}{y}{Y} = \set{z\in Y}{d(z,y) < r} = \ball{r}{y}\cap Y.\]
\end{Definition}
\begin{Example}
    Let $X = \R$ and let $Y = \interval{0}{1}$ with the metric given by the absolute value. 
    Then an open ball in $Y$ of radius $\frac{1}{2}$ about $0$ is
    $\relBall{\frac{1}{2}}{0}{Y} = \interval[open right]{0}{\frac{1}{2}}$ where as the open ball in $X$ is 
    $(-\frac{1}{2},\frac{1}{2})$.
\end{Example}

\begin{Proposition}\label{prop:open_relative}
    Let $\metricS{X}{d}$ be a metric space and let $Y$ be a subset of $X$.
    \begin{enumerate}
	\item
	    A subset $G$ of $Y$ is relatively open in $Y$ if and only if there is an open subset $U$ in $X$
	    with $G = U \cap Y$.
	\item
	    A subset $F$ of $Y$ is relatively closed in $Y$ if and only if there is a closed subset $U$ in $X$
	    with $F = U \cap Y$.
    \end{enumerate}
\end{Proposition}
\begin{proof}
    We will show the proof for only $(1)$.
   
    Let $G\subset Y$ be relatively open in $Y$. This means that for any
    $x \in G$ there is radius $r_x$ such that an open ball in $Y$, $\relBall{r_x}{x}{Y} \subset G$.
    Let $U =
    \bigcup\limits_{x\in G}\ball{r_x}{x}$. Then by~\ref{prop:open_sets_prop}, $U$ is open in $X$. We will show
    that $G \subset U \cap Y$ and $U\cap Y \subset G$.

    If $z \in G$ then $z \in \relBall{r_z}{z}{Y} = \ball{r_z}{z}\cap Y
    \subset U\cap Y$. Thus $G \subset U \cap Y$. If $z \in U\cap Y$. Then there is an
    $x\in G$ such that $z \in \ball{r_x}{x}$ and $z \in Y$ that is $z \in \relBall{r_x}{x}{Y}$ which is a
    subset of $G$. Thus $z \in G$ i.e.~$U\cap Y \subset G$. 

    Let $U$ be an open set in $X$ such that $G = U\cap Y$, that is to say,
    \[G \subset U \cap Y \subset G.\]
    If $x \in G$ then $x \in U \cap Y$ and since $U$ is open there is a $r > 0$ such that $\ball{r}{x} \subset
    U$. That means that \[\relBall{r}{x}{Y} = \ball{r}{x}\cap Y \subset U \cap Y \subset G.\] 
\end{proof}

Open and closed sets enable us to look at points in a metric space with a geometric lens. Given a set, a point
may either be \emph{inside} it, \emph{outside} it  or on the \emph{edge}. These notions can be made precise.
\begin{Definition}[name=Interior of a set]
    Let $\metricS{X}{d}$ be a metric space and let $A$ be a subset of $X$. The interior of a set $A$, denoted
    by $\interior{A}$ is the
    set defined by,
    \[\interior{A} = \bigcup\set{G\subset X}{G\,\text{is open and}\, G \subset A}.\]
    Thus if $x \in \interior{A}$ then there is an $r > 0$ such that the ball $\ball{r}{x}$ is entirely
    contained in $A$.
\end{Definition}
\begin{Definition}[name=Closure of a set]
    Let $\metricS{X}{d}$ be a metric space and let $A$ be a subset of $X$. The closure of $A$, denoted by
    $\cl{A}$, is the set defined by,
    \[\cl{A} = \bigcap\set{F\subset X}{F\,\text{is closed and}\, F\supset A}.\]
    Thus if $x\in \cl{A}$ then any closed set $F$ containing $A$ must include the point $x$.
\end{Definition}
\begin{Definition}[name=Boundary of a set]
    Let $\metricS{X}{d}$ be a metric space and let $A$ be a subset of $X$. The boundary of $A$, denoted by
    $\boundary{A}$, is the set defined by
    \[\boundary{A} = \cl{A}\cap\cl{(\setDiff{X}{A})}.\]
\end{Definition}
\begin{Remark}
    Since $\emptyset$ is open and is contained in every set, the interior of any set always contains the
    emptyset. This however, may be all the interior of a set, in other words there could be a $A \subset X$
    such that $\interior{A} = \emptyset$. Since $X$ is closed and contains any set, it may
    happen that it is the only closed set containing the set in other words there could be a $A \subset X$
    such that $\cl{A} = X$.

    It follows from~\ref{prop:open_sets_prop} that $\interior{A}$ is open (being the union of open sets) and
    from~\ref{prop:closed_sets_prop} that the
    $\cl{A}$ is closed (being the intersection of closed sets).
\end{Remark}
The following Proposition gives a useful characterization of the interior and closure of sets.
\begin{Proposition}\label{prop:characterization_int_closure}
    Let $\metricS{X}{d}$ be a metric space and let $A \subset X$. Then,
    \begin{enumerate}
	\item
	    $x \in \interior{A}$ if and only if there is an $r > 0$ such that $\ball{r}{x} \subset A$.
	\item
	    $x \in \cl{A}$ if and only if for every $r > 0$ the ball $\ball{r}{x}\cap A \neq \emptyset$.
    \end{enumerate}
\end{Proposition}
\begin{proof}
    We prove in order.
    \begin{itemize}
	\item
	    If $x \in \interior{A}$ then it follows from the definition that it belongs to an open set $G
	    \subset X$ and hence there is an $r > 0$ such that $\ball{r}{x} \subset G \subset A$.

	    Assume there is an $r > 0$ such that the $\ball{r}{x} \subset A$. Since an open ball is an open
	    set, let $G = \ball{r}{x}$. Then $x \in G \subset A$ and hence $x \in \interior{A}$. 
	\item
	    Take an $x \in \cl{A}$. Fix an $r > 0$. Assume $\ball{r}{x}\cap A = \emptyset$
	    i.e.~$\ball{r}{x}\subset(\setDiff{X}{A})$. This means that $(\setDiff{X}{\ball{r}{x}}) \supset A$.
	    Since $\ball{r}{x}$ is open, $(\setDiff{X}{\ball{r}{x}})$ is a closed set containing $A$ and 
	    hence must
	    contain $\cl{A}$. Thus $\ball{r}{x}\cap \cl{A} = \emptyset$. Hence we reach a
	    contradiction (in assuming $\ball{r}{x}\cap A = \emptyset$) 
	    since $x \in \cl{A}$ and $\cl{A}$ is closed.

	    To show the other implication we prove the contrapositive. Assume $x\not\in\cl{A}$
	    i.e.~$x\in\setDiff{X}{\cl{A}}$. Since
	    $\cl{A}$ is closed, this means that there is an $r > 0$ such that 
	    $\ball{r}{x}\subset\setDiff{X}{\cl{A}}$. But
	    $(\setDiff{X}{\cl{A}}) \subset (\setDiff{X}{A})$ since $A \subset \cl{A}$. Hence
	    $\ball{r}{x}\cap A = \emptyset$.
    \end{itemize}
\end{proof}
\begin{Example}
    Let $X$ be $\R$ and $A = \Q$. We know from elementary analysis that any interval contains a rational
    number and any interval with rational endpoints contain a real number. Thus $\cl{\Q} = \R$. Similarly any
    interval with rational endpoints contain irrational number and so $\interior{\Q} = \emptyset$. Using the
    same reasoning we see that $\cl{(\setDiff{\R}{\Q})} = \R$ and $\interior{(\setDiff{\R}{\Q})} = \emptyset$.
    Thus $\boundary{\Q} = \R$.
\end{Example}
\begin{Example}
    We will show that the closed ball is not generally the closure of the open ball. Let $\metricS{X}{d}$ be
    the discrete metric. Then for any $x \in X$, 
    $\ball{r}{x} = \setX{x} = \cl{\ball{r}{x}}$. But $\closure{\ball{r}{x}} = X$.
\end{Example}
\begin{Proposition}\label{prop:prop_closure_interior}
    Let $\metricS{X}{d}$ be a metric space and let $A$ be a subset of $X$. Then,
    \begin{enumerate}
	\item
	    $A$ is closed if and only if $A = \cl{A}$.
	\item
	    $A$ is open if an only if $A = \interior{A}$.
	\item
	    $\cl{A} = \setDiff{X}{\interior{(\setDiff{X}{A})}}$.
	\item
	    $\interior{A} = \setDiff{X}{\cl{(\setDiff{X}{A})}}$.
	\item
	    $\boundary{A} = \setDiff{\cl{A}}{\interior{A}}$.
	\item
	    If $A_1,\ldots,A_n$ are subsets of $X$, then $\cl{(\finiteUnion{A_k}{k}{n})} =
	    \finiteUnion{\cl{A_k}}{k}{n}$.
    \end{enumerate}
\end{Proposition}
\begin{proof}
    We prove in order.
    \begin{itemize}
	\item
	    First note that $A \subset \cl{A}$ for any $A \subset X$. If $A$ is closed then since $A \subset
	    A$ and $\cl{A} \subset F$ for all $F$ closed and containing $A$, $\cl{A} \subset A$. Thus if $A$
	    is closed then $A = \cl{A}$.

	    Since $\cl{A}$ is closed and $A = \cl{A}$, $A$ is closed.
	\item
	    Note that $\interior{A}$ is open and $\interior{A} \subset A$. Let $A$ be open and let $x$ be in
	    $A$. Hence there is an $r > 0$ such that $\ball{r}{x} \subset A$. But
	    by~\ref{prop:characterization_int_closure}, $x$ is in $\interior{A}$ and hence $A \subset
	    \interior{A}$.

	\item
	    For any set $B \subset X$, let us denote by $\comp{B}$ the set $\setDiff{X}{B}$. With this
	    terminology, first observe that $\interior{(\comp{A})} \subset \comp{A}$ and hence,
	    $A \subset \comp{(\interior{(\comp{A})})}$. Since $\comp{(\interior{(\comp{A})})}$ is closed and 
	    contains
	    $A$, we must have $\closure{A} \subset \comp{(\interior{(\comp{A})})}$.

	    Let $x$ be an element of $\comp{(\interior{(\comp{A})})}$ and hence 
	    $x \not\in \interior{(\comp{A})}$, which means that for any $r > 0$
	    $\ball{r}{x}\not\subset\comp{A}$; which is equivalent to saying that $\ball{r}{x}\cap A \neq
	    \emptyset$ for any $r > 0$ and hence $x \in \closure{A}$. Thus, we have shown that
	    $\comp{(\interior{(\comp{A})})} \subset A$.

	\item
	    Again, using the notation from above, note that $\cl{(\comp{A})} \supset \comp{A}$ and hence
	    $\comp{(\cl{(\comp{A})})} \subset A$. Since $\comp{(\cl{(\comp{A})})}$ is an open set contained in
	    $A$, we must have $\comp{(\cl{(\comp{A})})} \subset \interior{A}$.

	    Let $x$ be an element of $\interior{A}$. Then there is an $r > 0$ such that $\ball{r}{x} \subset
	    A$, which means that $\ball{r}{x} \cap \comp{A} = \emptyset$. But this means that $x \not\in
	    \cl{(\comp{A})}$, which is equivalent to saying that $x \in \comp{(\cl{(\comp{A})})}$. Hence, we
	    have shown that $\interior{A} \subset \comp{(\cl{(\comp{A})})}$.
	\item
	    Using the notation from above, note that $\boundary{A} = \cl{A}\cap\cl{\comp{A}}$. From $(4)$, we
	    see that $\cl{\comp{A}} = \comp{(\interior{A})}$ and hence $\boundary{A} =
	    \cl{A}\cap\comp{(\interior{A})} = \setDiff{\cl{A}}{\interior{A}}$.
	\item
	    Let $A = \finiteUnion{A_k}{k}{n}$. Note that $A \supset A_k$ and hence $\cl{A} \supset \cl{A_k}$ for
	    each $k$. Thus $\cl{A} \supset \finiteUnion{\cl{A_k}}{k}{n}$.
	    Each $\cl{A_k}$ is closed and since \textbf{finite} union of closed sets are close
	    $\finiteUnion{\cl{A_k}}{k}{n}$ is a closed set. Moreover since $A_k \subset \cl{A_k}$, we have $A
	    \subset \finiteUnion{\cl{A_k}}{k}{n}$. Thus we have a closed set containing $A$ and hence $\cl{A}
	    \subset \finiteUnion{\cl{A_k}}{k}{n}$.
    \end{itemize}
\end{proof}
\begin{Remark}
    Sometimes when it is obvious from the context that the set difference are being taken w.r.t.~$X$, we
    usually denote $\setDiff{X}{A}$ as $\comp{A}$. With such a notation $(3),(4)$ in the theorem above can be
    written as \break{}$\interior{(\comp{A})} = \comp{(\cl{A})}$ and $\cl{\comp{A}} = \comp{(\interior{A})}$.
\end{Remark}
A very useful concept involving the closure is the notion of density.
\begin{Definition}[name=Dense subset]
    A subset $E$ of a metric space $\metricS{X}{d}$ is dense if $\cl{E} = X$. This means that for any 
    $x \in X$, $\ball{r}{x}\cap E \neq \emptyset$ for any radius $r$.
\end{Definition}
\begin{Definition}[name=Separable]
    A metric space $\metricS{X}{d}$ is separable if it has a countable dense subset.
\end{Definition}
\begin{Example}
    We show some examples of dense subset.
    \begin{itemize}
	\item
	    The rational numbers are dense in $\R$. This is because any interval around a real number must
	    contain rational numbers. Thus $\Q^n$ is dense in $\Rn$. This means that $\Rn$ is separable.
	    This follows from a general fact that if
	    $A_1,A_2$ are dense subsets of $X_1,X_2$, then $A_1\times A_2$ is dense in $X_1\times X_2$. 
	    We will show this fact when we define a product metric space. 
	    
	\item
	    If $X$ is an set and $d$ is the discrete metric, then the only dense subset of $X$ is $X$ itself.
    \end{itemize}
\end{Example}
We now look into the important topic of sequences and its convergence in a metric space.
\section{Sequences and completeness}
Let $\metricS{X}{d}$ be a metric space.
\begin{Definition}[name=Sequence]
    A sequence $\seq{x}{n}$ in $X$ is a function that maps the positive integers to points in the metrics
    space $X$ i.e.~$n\mapsto x_n$. A subsequence of a sequence is defined by $\seq{x}{n_k}$, is a composition
    of the sequence function with a subsequence index non-decreasing 
    function that maps positive integers to positive integers
    i.e.~$k\mapsto n_k$ and $k_1 \leq k_2$ implies $n_{k_1} \leq n_{k_2}$. 
    Hence a subsequence is the function that maps positive integers $k$ to  points in $X$
    i.e.~$k\mapsto x_{n_k}$.
\end{Definition}
\begin{Example}
    Let $\metricS{R}{d}$ be the standard metric space in $\R$. Then $\seq{x}{n}$ given by $1,-1,1,-1,\cdots$
    is a sequence, while $\seq{x}{n_k}$ given by $1,-1,1,1,1\cdots$ is a subsequence. We can explicitly write
    the subsequence index function as $1\mapsto 1$, $2\mapsto 2$, $3\mapsto 3$, $4\mapsto 5$, $5\mapsto 7$ and
    so on.
\end{Example}
\begin{Definition}[name=Convergence]
    A sequence $\seq{x}{n}$ is said to converge in $X$ if for every $\epsilon > 0$ there is an integer $N$
    such that $d(x,x_n) < \epsilon$, whenever $n \geq N$. We denote it by $\atob{x_n}{x}$ or $x =
    \limit{x_n}{n}{\infty}$. Logically this means,
    \[\forEv{\epsilon > 0}\thereIs{N\in\Zplus}\forEv{n\in\Zplus}\suchThat{n\geq N}{\implies}{d(x_n,x) <
	    \epsilon}.\]
\end{Definition}
What does it mean for a sequence to not converge to $x$? From our logical implication, this means that there
is an $\epsilon > 0$ such that no matter what $N$ we take, we can find a $n$ in the sequence with $n \geq N$
but $d(x_n,x) \geq \epsilon$.
\begin{Proposition}
    If $\atob{x_n}{x}$ then any subsequence $x_{n_k}$ also coverges to $x$.
\end{Proposition}
\begin{proof}
    Let $\epsilon > 0$ be given. Since $\seq{x}{n}$ is convergent, there is an $N$ such that for all $n\geq
    N$, $d(x_n,x) < \epsilon$. However, when $n \geq N$, then $n_k \geq N$ since for all $k$, $n_k \geq n$
    (can be shown by induction). Thus $d(x_{n_k},x) < \epsilon$.
\end{proof}

There is a deep connection between closed sets in metric spaces and convergence of sequences.
\begin{Proposition}
    A subset $F\subset X$ is closed if and only if whenever $\seq{x}{n}$ is a sequence in $F$ and
    $\atob{x_n}{x}$, it follows that $x \in F$.
\end{Proposition}
\begin{proof}
    Let $F$ be closed. Assume there is a sequence $\seq{x}{n}$ in $F$ such that $\atob{x_n}{x}$. If $x \not
    \in F$, then by the definiton of closed set there is a radius $r > 0$ such that $\ball{r}{x}\cap
    =\emptyset$. But since $\atob{x_n}{x}$ there is an $N$ such that $d(x_n,x) < r$ for all $n \geq N$. Hence
    $x$ must belong to $F$.

    Let $x$ be an element of $X$. Assume $\ball{r}{x}\cap F \neq \emptyset$ for any $r > 0$. In particular, 
    for $ r = 1$, there is an $x_1$
    such that $d(x_{1},x) < 1$. Going on inductively we have for each $n$ a $x_n$ such that $d(x_{n},x) < 1/n$.
    This means that we constructed a sequence $\seq{x}{n}$ such that $\atob{x_n}{x}$. By our hypothesis $x$
    belongs to $F$ and hence $F$ is closed. (See the second logical implication following the definition of
    closed set.)
\end{proof}
\begin{Remark}
    In the proof above, we uncovered a nice observation about the closure of a set $A$. If $x$ is in the
    closure of $A$ then for all $r > 0$, $\ball{r}{x}\cap A \neq \emptyset$. We saw, how this leads to a
    sequence $\seq{x}{n}$ in $A$ such that $\atob{x_n}{x}$. However, this sequence may have only a finite
    number of distinct terms. In other words we could get a \textbf{constant} sequence $x_1 = x_2 = \cdots =
    x$.
\end{Remark}
\begin{Definition}[name=Limit point]
    If $A \subset X$, then a point $x \in X$ is called a limit point of $A$ if for every $\epsilon > 0$ there
    is a point $a$ in $\ball{\epsilon}{x}\cap A$ with $a \neq x$. Logically this means,
    \[\forEv{\epsilon > 0}\thereIs{a \in A}\suchThat{a\neq x}{\text{and}}{d(a,x) < \epsilon }.\]
\end{Definition}
What does it mean for a point $x$ to be \textbf{NOT} a limit point of $A$. If that is the case then there must
be an $\epsilon$ such that no matter what point $a$ in $A$ we take, if $a \neq x$ then the distance between
$a$ and $x$ is larger than $\epsilon$.
\begin{Definition}[name=Isolated point]
    A point $x \in A$ that is not a limit point of $A$ is called an isolated point. Thus if $x$ is an isloated
    point of $A$, then there must be an
    $\epsilon > 0$ such that $\ball{r}{x}\cap (\setDiff{A}{\setX{x}}) = \emptyset$.
\end{Definition}
\begin{Example}
    The following are some examples of limit points/isolated points.
    \begin{itemize}
	\item
	    Let $X = \R$ and let $A = (0,1) \cup {2}$. Note that $2$ is an isolated point of $A$ because the
	    ball $\ball{\frac{1}{3}}{2}$ doesn't intersect $(0,1)$. Each point in $\interval{0}{1}$ is a limit
	    point of $A$.
	\item
	    If $X = \R$ and $A = \Q$, then every point of $X$ is a limit point of $A$ and $A$ has no isolated
	    points.
	\item
	    If $X = \R$ and $A = \set{\frac{1}{n}}{n \in \Zplus}$, then $0$ is the limit point of $A$, while
	    all points of $A$ are isolated points of $A$.
    \end{itemize}
\end{Example}
\begin{Proposition}
    Let $A$ be a subset of $X$. Then,
    \begin{enumerate}
	\item
	    A point $x$ is a limit point of $A$ if and only if there is a sequence of distinct points in $A$
	    that converges to $x$.
	\item
	    $\cl{A} = A \cup \set{x}{x\,\text{is a limit point of $A$}}$.
	\item
	    $A$ is closed in $X$ if and only if it contains all its limit points.
    \end{enumerate}
\end{Proposition}
\begin{proof}
    We prove in order.
    \begin{enumerate}
	\item
	    Let $x$ be a limit point of $A$. Then for every $\epsilon > 0$ there is an $a \in A$ such that
	    $a \neq x$ and $d(a,x) < \epsilon$. We will define a sequence inductively.
	    First pick $\epsilon_1 = 1$. Then there is an $a_1 \neq x$ such that $d(a_1,x) < \epsilon_1$. Now
	    take radius to be $\frac{1}{2}$. Is there an $a_2 \neq a_1 \neq x$ such that $d(a_2,x) <
	    \frac{1}{2}$? It could so happen that the $a_1$ we found actually satisfies $d(a_1,x) <
	    \frac{1}{2}$. Thus to remedy this let $\epsilon_2 = \min\setX{\frac{1}{2},d(a_1,x)}$. Thus there
	    exists an $a_2 \neq a_1$ such that $d(a_2,x) < \epsilon_2$. Inductively, we have found for each
	    $n$ there is an $a_n$ different from $a_1,a_2,\ldots,a_{n-1}$ and $d(a_n,x) < \epsilon_n \leq
	    \frac{1}{n}$. Taking $n$ large enough we get a sequence of distinct terms $\seq{a}{n}$ such that
	    $\atob{a_n}{x}$.

	    Let $\seq{a}{n}$ be a sequence of distinct points in $A$ such that $\atob{a_n}{x}$. Note that $x
	    \neq a_n$ for any $n$. For any
	    $\epsilon > 0$ there is an $N$ such that $d(a_n,x) < \epsilon$. Thus for any $\epsilon$ there is
	    an $a_n$ such that $a_n \neq x$ and $d(a_n,x) < \epsilon$. Hence $x$ is a limit point of $A$.

	\item
	    If $x$ is a point of $A$ then $x$ certainly belongs to $\cl{A}$. Let $x$ be a limit point of $A$.
	    Then there is a sequence of distinct points $\seq{a}{n}$ in $A$ and hence in $\cl{A}$ such that
	    $\atob{a_n}{x}$. But since $\cl{A}$ is closed, $x$ must be in $\cl{A}$. Thus,
	    $A \cup \set{x}{x\,\text{is a limit point of $A$}} \subset \cl{A}$.

	    Let $x$ be a point in $\cl{A}$ and assume $x \not \in A$. Then for any $\epsilon > 0$, the
	    $\ball{\epsilon}{x}\cap A \neq \emptyset$. Since $x \not\in A$, there must be an $a \neq x$ such
	    that $a$ is in $A$ and $d(x,a) < \epsilon$. Hence $x$ is a limit point of $A$. Thus, $\cl{A}
	    \subset A \cup \set{x}{x\,\text{is a limit point of $A$}}$.

	\item
	    If $A$ is closed then $A = \cl{A}$ and hence from the result above it must contain all its limit
	    points.
    \end{enumerate}
\end{proof}
The closure of a set can also be characterized in a different way that has a geometric motivation. For this we
define the distance between a point and a set in a metric space.
\begin{Definition}[name=Distance]
    If $A \subset X$ and $x \in X$, then the distance from $x$ to $A$ is the number, denoted as $\dist(x,A)$,
    given by,
    \[\dist(x,A) = \inf\set{d(a,x)}{a\in A}.\]
\end{Definition}
When $x$ is in $A$, then clearly $\dist(x,A) = 0$. Is it possible for $\dist(x,A) = 0$ when $x\not\in A$?
\begin{Proposition}
    If $A \subset X$, then $\cl{A} = \set{x \in X}{\dist(x,A) = 0}$.
\end{Proposition}
\begin{proof}
    Let $x$ be in $\cl{A}$. Then there is a sequence $\seq{x}{n}$ in $A$ such that $\atob{x_n}{x}$. This means
    that for any $\epsilon$ there is an $N$ such that $d(x_n,x) < \epsilon$ whenever $n \geq N$. Hence
    $\inf\set{d(a,x)}{a\in A} < \epsilon$. Since $\epsilon$ was arbitrary, $\dist(x,A) = 0$.

    Let $x$ be a point in $X$ such that $\dist(x,A) = 0$. Then, for any $\epsilon > 0$ there is an $a \in A$
    such that $d(a,x) < \epsilon$. This means that for any $\epsilon > 0$ $\ball{\epsilon}{x} \cap A \neq
    \emptyset$. Hence $x \in \cl{A}$.
\end{proof}
We need to know apriori the limit of a sequence to check if we have convergence. This notion is not very
helpful as we would like to know if a given sequence convergence without any knowledge of its limit. We can do
this for certain sequences called the Cauchy sequence.
\begin{Definition}[name=Cauchy sequence]
    A sequence $\seq{x}{n}$ in $X$ is a Cauchy sequence if,
    \[\limit{d(x_n,x_m)}{n,m}{\infty} = 0.\]
    This means that for any $\epsilon > 0$ there is an $N\in\Zplus$ such that $d(x_n,x_m) < \epsilon$ whenever
    $n,m \geq N$.
\end{Definition}
\begin{Definition}[name=Complete metric space]
    A metric space $X$ is complete if every Cauchy sequence converges.
\end{Definition}
\begin{Remark}
    A few observations can be made about Cauchy sequences.
    \begin{itemize}
	\item
	    A convergent sequence is Cauchy.
	\item
	    A Cauchy sequence with a convergent subsequence is convergent.
    \end{itemize}
\end{Remark}
A useful geometric notion can be given to any set in a metric space.
\begin{Definition}[name=Diameter]
    For any set $E \subset X$, we define its diameter as
    \[\diam{E} = \sup\set{d(x,y)}{x,y\in E}.\]
\end{Definition}
\begin{Proposition}
    For any set $E \subset X$, $\diam{E} = \diam{\cl{E}}$.
\end{Proposition}
\begin{proof}
    For any $p,q$ in $E$, $p,q$ are in $\cl{E}$ and hence $\diam{E} \leq \diam{\cl{E}}$. Let $x,y$ be points
    in $\cl{E}$ such that $x \neq y$. Then ther are point $p,q$ in $E$ such that $d(p,x) < \frac{\epsilon}{2}$
    and $d(q,y) \leq \frac{\epsilon}{2}$. By (repeated use of) triangle inequlality,
    \[d(x,y) < d(x,p) + d(p,q) + d(q,y) < d(p,q) + \epsilon.\]
    This means that
    $\diam{\cl{E}} \leq \diam{E} + \epsilon$. Since $\epsilon$ was arbitray we get $\diam{\cl{E}} \leq \diam{E}$.
\end{proof}
\begin{Theorem}[name=Cantor Intersection theorem]
    A metric space is complete if and only if $\seq{F}{n}$ is a sequence of non-empty subsets satisfying,
    \begin{enumerate}
	\item
	    Each $F_n$ is closed,
	\item
	    $\decSetSeq{F}$,
	\item
	    $\diam{F_n}\to 0$,
    \end{enumerate}
    then $\countIntersection{F_n}{n}$ is a single point.
\end{Theorem}
\begin{proof}
    Assume we have a complete metric space $\metricS{X}{d}$ and let $\seq{F}{n}$ be a sequence of arbitrary 
    closed sets in $X$ such that,
    \begin{enumerate}
	\item
	    Each $F_n$ is closed,
	\item
	    $\decSetSeq{F}$,
	\item
	    $\diam{F_n}\to 0$,
    \end{enumerate}
    Let $\seq{x}{n}$ be a sequence of points in $X$ such that $x_i \in F_i$. 

    \textbf{Claim}: $\seq{x}{n}$ is Cauchy. Fix an $\epsilon > 0$. Since $\diam{F_n}\to 0$, there is an $N$
    such that $\diam{F_n} < \epsilon$ whenever $n \geq N$. For any $n,m \geq N$, $x_n,x_m \in F_N$ since
    $F_{N} \supset F_{N+1} \ldots$ and hence $d(x_n,x_m) < \epsilon$. 
    
    Since $\metricS{X}{d}$ is complete,
    there is an $x \in X$ such that $\atob{x_n}{x}$. Note that $\seq{x}{n}$ is in $F_1$ and since $F_1$ is
    closed $x$ is in $F_1$. But the sequence $\seq{x}{n > 1}$ is in $F_2$ and since $F_2$ is closed $x$ is in
    $F_2$. Thus continuing this way, since each $F_n$ is closed, $x \in \countIntersection{F_n}{n}$. Hence
    $\countIntersection{F_n}{n}$ is not empty. If $y$ is in $\countIntersection{F_n}{n}$, then $d(x,y) <
    \diam{F_n}$ for each $N$ and hence $d(x,y) = 0$ i.e $x = y$.  

    Now assume we have a sequence  $\seq{F}{n}$ of arbitrary closed sets in $X$ such that IF 
    \begin{enumerate}
	\item
	    Each $F_n$ is closed,
	\item
	    $\decSetSeq{F}$,
	\item
	    $\diam{F_n}\to 0$,
    \end{enumerate}
    is true then $\countIntersection{F_n}{n}$ is a single point. Let $\seq{x}{n}$ be a Cauchy sequence in $X$.
    Let $F_k = \cl{\set{x_n}{n\geq k}}$. Note that $\decSetSeq{F}$ and each $F_n$ is closed.

    \textbf{Claim}: $\diam{F_n}\to 0$. Fix an $\epsilon > 0$. Then there is an $N$ such that $d(x_n,x_m) <
    \epsilon$ whenever $n \geq N$. This means that $\diam{F_N} \epsilon$. But $F_N \supset F_{N+1} \ldots$ and
    hence $\diam{F_n} \leq \diam{F_N}$ for every $n \geq N$ i.e $\diam{F_n} < \epsilon$ whenever $n \geq N$.
    Thus $\diam{F_n}\to 0$.

    Hence $\countIntersection{F_n}{n}$ contains only one point, lets call it $x$. For any $x_n$, $d(x,x_n) <
    \diam{F_n}$ which goes to $0$ as $n\to \infty$.
\end{proof}
\begin{Example}
    $\R,\Rn$ are complete spaces. The proof is part of an elementary real analysis course.
\end{Example}
\begin{Proposition}
    If $\metricS{X}{d}$ is a complete metric space and $Y \subset X$, then $\metricS{Y}{d}$ is complete if and
    only if $Y$ is closed in $X$.
\end{Proposition}
\begin{proof}
    Assume $\metricS{Y}{d}$ is complete. Let $\seq{x}{n}$ be a sequence in $Y$ such that $\atob{x_n}{x}$ for
    some $x$ in $X$. Since a convergent sequence is Cauchy, $\seq{x}{n}$ is Cauchy and since $Y$ is complete
    there is a $y \in Y$ such that $\atob{x_n}{y}$. But since limits are unique $x = y$ and hence $Y$ is
    closed.

    Assume $Y$ be closed in $X$. Let $\seq{x}{n}$ be a Cauchy sequence. Since $Y \subset X$, $\seq{x}{n}$ is
    cauchy in $X$. Since $X$ is complete, there is a $x\in X$ such that $\atob{x_n}{x}$. But since $Y$ is
    closed $x$ is in $Y$ and hence $\metricS{Y}{d}$ is complete.
\end{proof}
