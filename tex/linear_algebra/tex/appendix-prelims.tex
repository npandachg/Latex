\chapter{Preliminary Concepts: Set theory}

\section{Some basic properties of integers}\label{App:integers}
In this section we will list some of the basic properties of Integers without including the proofs
of most of them. We denote the integers by $\Z$ and the positive integers by $\Z^{+}$. 
\begin{Definition}
    For any two integers $a,b$, we say that $a$ divides $b$ iff $b$ is a multiple of $a$ i.e 
    $ a | b \equiv b = ak$ for some $k \in \Z$.
\end{Definition}

There are times when, for two integers $a,b$, \emph{exact} division is not possible. In that case we
will appeal to Euclid's division theorem.

\begin{Theorem}[name= Division Theorem]
    Let $a,b \in \Z$ with $a \in \Z^{+}$. Then there are unique integers $q,r$ such that
    \begin{equation*}
	b = a*q + r \quad \text{and } 0\leq r < a.
    \end{equation*}
\label{thm:div}
\end{Theorem}

For two integers $a,b$ we can find all the integers that divide $a$ and all the integers that divide
$b$. We'll be interested in those integers that divide both $a$ and $b$, particularly the greatest
integer that divides them both. This integer is called the greatest common divisor and is denoted by
$\gcd(a,b)$ or $(a,b)$.
\begin{Definition}{Greatest common divisor}
    Let $a,b$ be two integers, at least one of which is non-zero. Then the greatest common divisor
    of $a,b$ is the unique positive integer $d$ such that
    \begin{itemize}
	\item $d$ is a common divisor, i.e $ d | a$ and $d | b$.
	\item $d$ is greater than every other common divisor, i.e if $ c | a$ and $c | b$ then $c
	    \leq d$.
    \end{itemize}
\end{Definition}

To compute the $\gcd(a,b)$ we will use Euclid's recursive algorithm. Two important properties of
$\gcd$ are:
\begin{itemize}
    \item If $ a | b$ then $\gcd(a,b) = a$.
    \item For two integers $a,b$ if $b = a*q + r$ then $\gcd(a,b) = \gcd(r,a)$.
\end{itemize}
As an example of the second property consider the $\gcd(30,72)$. We can write $ 72 = 30*2 + 12 $. So
$\gcd(30,72) = \gcd(12,30)$. We can continue further by writing $30 = 12*2 + 6$, so $\gcd(12,30) =
\gcd(6,12) = 6$. The last result came from the first property. 

\begin{Theorem}[name=Euclidean Algorithm]
    Suppose that $a,b \in \Z^{+}$. The following procedure defines a finite sequence of positive
    integers $a_0,a_1,\ldots,a_n$ such that $a_n = \gcd(a,b)$. 
    \begin{enumerate}
	\item Put $a_0 = b$ and $a_1 = a$.
	\item At stage $k \geq 1$ suppose $a_0,a_1,\ldots,a_k$ have been defined. Then using the
	    division theorem we can write
	    \begin{equation*}
		a_{k-1} = a_k*q_k + r_k.~\eqref{thm:div}
	    \end{equation*}
	\item If $r_k = 0$ then stop, $\gcd(a,b) = a_k$ else continue.
    \end{enumerate}
\label{thm:euclid_al}
\end{Theorem}

The $\gcd(a,b)$ can also be written as an integral combination of $a,b$. Let us define a linear
combination of two integers $a,b$.
\begin{Definition}
    For any two integers $a,b$, any integer of the form $m*a + n*b$ where $m,n \in \Z$ is called a
    linear combination of $a,b$.
\end{Definition}
For example given $-5,2$ then $30 = 2*(-5) + 20*2$ is a linear combination of $-5,2$. Such a linear
combination is by no means unique. A very important fact (theorem) is that the $\gcd(a,b)$ can be
written as a linear combination of $a,b$.

\begin{Theorem}[name=gcd as linear combination]
    For any two integers $a,b$, the $\gcd(a,b)$ is a linear combination of $a$ and $b$.
\end{Theorem}

Let us see how to get the linear combination. Let $a = 136$ and $b = 232$. From Euclid's
algorithm~\eqref{thm:euclid_al} we get the following:
\begin{align*}
    232 = 136*1 + 96 \\
    136 = 96*1 + 40 \\
    96 = 40*2 + 16 \\
    40 = 16*2 + 8 \\
    16 = 8*2 + 0
\end{align*}
This tells us that $\gcd(136,232) = 8$. To get $8$ as a linear combination of $136,232$ we will
write each of the numbers towards the left of the equations in the Euclidean Algorithm above as
linear combination of $136,232$. 
\begin{align*}
    232 &= (232)*1 + (136)*0 \\
    136 &= (232)*0 + (136)*1 
\end{align*}
\begin{align*}
    96 &= 232 - 136*1 \\ 
       &= (232)*1 + (136)*-1 
\end{align*}
Now $40 = 136 - 96*1$. And thus,
\begin{align*}
    40 &= 136 - \left[ (232)*1 + (136)*-1  \right] \\
       &= (232)*-1 + (136)*2 
\end{align*}
Similarly $16 = 96 - 40*2$ and so 
$16 = (232)*3 + (136)*-5 $. Thus we get 
\begin{equation*}
    8 = (232)*-7 + (136)*12
\end{equation*}

\begin{Definition}
    Two integers $a,b$ are co-prime iff $1$ can be written as linear combination of $a$ and $b$ i.e
    iff $\gcd(a,b) = 1$.
\end{Definition}

A very interesting problem known due to \emph{Diophantus} is that given three integers $a,b$
and $c$, find all possible  linear combinations of $a$ and $b$ that yield $c$ i~.~e~.~find all the
integers $m,n$ such that $c = m*a + n*b$.

\begin{Theorem}
    For positive integers $a,b,c$ there exists integers $m,n$ such that $m*a + n*b = c$ iff
    $\gcd(a,b) | c $
\label{thm:dioph}
\end{Theorem}

As an example consider the puzzle made famous in the movie Die Hard 3. Can $4$ gallon be poured
using on $3,5$ gallon jugs? Note that $\gcd(3,5) = 1$ which divides $4$. Thus we can find $m,n$ such
that $m*3 + n*5 = 4$. In this case $m = 3, n = -1$.

As another example consider the equation $m*140 + n*63 = 35$. Does this equation have any solution?
The $\gcd(63,140) = 7$ and we know that $7 | 35$ so the equation does have a solution. How do we
find it. Note that we can write $7$ as a linear combination of $63,140$ as $ 7 = (63)*9 +
(140)*-4$ and thus since $35 = 5*7$ we get
\begin{equation*}
    35 = 140*(-20) + (63)*45
\end{equation*}

How do we get all the solutions? Consider the \emph{homogenous} equation as above i.e 
$m*140 + n*63 = 0$. To find its solution we can observe
\begin{align*}
    m*140 + 63*n = 0 \, \Leftrightarrow & m*20 + n*9 = 0 \\
    \, \Leftrightarrow & 20m = -9n \\
    \, \Leftrightarrow & 9 \, \text{divides} \, 20m.
\end{align*}
Since $\gcd(9,20) = 1$, this means that $9$ divides $m$ and so $m = 9*q$ for some $q \in \Z$. Thus
plugging it in the equation we get $n = -20*q$. And so we get the solution pair $(m,n) =
(9*q,-20*q)$. The following proposition sums up what we did.

\begin{Proposition}
    Suppose $a,b$ are co-prime non-zero integers, then for $(m,n) \in \Z^2$,
    \begin{equation*}
	a*m + b*n = 0 \Leftrightarrow (m,n) = (b*q,-a*q) \quad \text{for some $q \in \Z$}.
    \end{equation*}
\label{prop:ch1DioH}
\end{Proposition}

To find all the solutions for a linear Diophantine equation we will appeal to the following
propostion,
\begin{Proposition}
    Suppose that $(m_0,n_0) \in \Z^2$ is a solution to the Diophantine equation $a*m + b*n = c$.
    Then, for $(m,n) \in \Z^2$, 
    \begin{equation*}
	a*m + b*n = c \, \Leftrightarrow \, a*(m - m_0) + b*(n - n_0) = 0.
    \end{equation*}
\label{prop:ch1Dio}
\end{Proposition}
It can be easily proven by using $c = a*m_0 + b*n_0$. And using~\ref{prop:ch1DioH} we can find the
solution of $a*(m - m_0) + b*(n - n_0) = 0$ given by $(m-m_0,n-n_0) = (b*q,-a*q)$ and so we can find
$(m,n) = (m_0 + b*q, n_0 - a*q)$.

As an example consider again the equation $140*m + 63*n = 35$. We found a particular solution given
by $(m_0,n_0) = (-20,45)$ and so the solution set by~\ref{prop:ch1Dio} is given by
$(m,n) = (-20 + 9*q, 45 - 20*q)$. Note that we divide by $7$ to get the homogenous case to be
co-prime numbers. Thus we get $9*q$ and $20*q$ instead of $63*q$ and $140*q$.



\subsection{Congruence class}
Another very useful characterization of integers is done by the congruence or modulo construction.
\begin{Definition}
    Let $m \in \Z^{+}$. Two integers $a,b$ are congruent modulo $m$ whenever $a-b$ is divisible by
    $m$. Equivalently $a$ and $b$ are congruent modulo $m$ whenver the remainders when $a,b$ is
    divided by $m$ are equal. We denote this by
    \begin{equation*}
	a \equiv b \pmod m.
    \end{equation*}
\end{Definition}
For example $20 = 34 \pmod 7$.

The congruence modulo exhibit the following important properties.
\begin{Proposition}
    Let $m \in \Z^{+}$. Then
    \begin{itemize}
	\item For all $a \in \Z \, a \equiv a \pmod m$.
	\item If $a,b \in \Z$ such that $a \equiv b \pmod m$ then $b \equiv a \pmod m$.
	\item If $a,b,c \in \Z$ such that $a \equiv b \pmod m$ and $ b \equiv c \pmod m$ then $a
	    \equiv c \pmod m$.
    \end{itemize}
\end{Proposition}

\begin{Proposition}
    Cancellation. Let $m \in \Z^{+}$ and let $a*b_1 \equiv a*b_2 \pmod m$. Then,
    \begin{itemize}
	\item $b_1 \equiv b_2 \pmod m \text{ whenever } \gcd(a,m) = 1$. 
	\item $b_1 \equiv b_2 \pmod {m/a} \text{ whenever } a | m $. 
    \end{itemize}
\end{Proposition}

\begin{Proposition}
    Arithmetic. Let $m \in \Z^{+}$ and let $a_1,a_2,b_1,b_2 \in \Z$ such that $a_1 \equiv a_2 \pmod
    m$ and $b_1 \equiv b_2 \pmod m$. Then
    \begin{itemize}
	\item $a_1 \pm b_1 \equiv a_2 \pm b_2 \pmod m$. 
	\item $a_1*b_1 \equiv a_2*b_2 \pmod m$.
    \end{itemize}
\end{Proposition}

An important question is to determine if $a*x \equiv b \pmod m$ exists for certain values of $x$? 
The following theorem tells us that there is a bijective correspondence between solving the 
linear Diophantine equation and the congruence equation i.e,
\begin{Theorem}
    For integers $a,b$ and positive integer $m \in \Z^{+}$, there is a bijection
    \begin{equation*}
	f: \lbrace (x,y) \in \Z^2 \lvert a*x + m*y = b \rbrace \to \lbrace x \in \Z \lvert a*x \equiv b
	\pmod m \rbrace 
    \end{equation*}
\end{Theorem}

Thus to get the solution for $a*x \equiv b \pmod m$ we must have
\begin{itemize}
    \item $\gcd(a,m) | b$. 
    \item $x$ is such that $(x,y)$ must be solutions to the linear Diophantine equation $a*x + m*y =
	b$.
\end{itemize}

Since arithmetic is allowed in the modulo framework we can define two inverse operations of addition
and multiplication. 
We say that $-a$ is the additive inverse of $a$ modulo $m$ whenever $a + (-a) = 0 \pmod m$.
Similarly $a^{-1}$ is the multiplicative inverse of $a$ modulo $m$ whenever $a*a^{-1} = 1 \pmod m$.
Note that for a multiplicative inverse to exist we must have that $\gcd(a,m) | 1$. This is only
possible when $a$ and $m$ are co-prime and $a$ is non-zero.

\begin{Definition}
    Given $m \geq 2 \in \Z^{+}$ and an integer $a$, the congruence class of $a$ modulo $m$ is the
    set of all integers which are congruent to $a$ modulo $m$. We denote this congruence class by
    ${\left[a\right]}_{m}$. Thus,
    \begin{equation*}
	{\left[a\right]}_{m} = \lbrace x \in \Z \, \lvert \, x \equiv a \pmod m \rbrace
    \end{equation*}
\end{Definition}
Since $x \equiv a \pmod m$ is the same as $a \equiv x \pmod m$, the equivalence class
${\left[a\right]}_m$ can be thought of as the set of all integers that give a remainder $a$ when
divided by $m$.

As an example consider $m = 2$. There are only two remainders possible $0$ and $1$ and hence there
are two equivalent classes ${\left[0\right]}_2$ and ${\left[1\right]}_2$. Similarly when $m = 6$ we
get ${\left[0\right]}_6$,${\left[1\right]}_6$,${\left[2\right]}_6$,${\left[3\right]}_6$,
${\left[4\right]}_6$ and ${\left[5\right]}_6$.

\begin{Definition}
    Given $m \geq 2 \in \Z^{+}$. The set of congruence class ${\left[0\right]}_m$,
    ${\left[1\right]}_m$, $\dots {\left[m-1\right]}_{m}$ is denoted by $\Z_{m}$ i.e 
    \begin{equation*}
	\Z_{m} = \lbrace {\left[0\right]}_m,{\left[1\right]}_m, \ldots, {\left[m-1\right]}_m \rbrace
    \end{equation*}
\end{Definition}

Note that $\Z_{m}$ is a finite set. We can define arithmetic in $\Z_{m}$ as follows:
$ {\left[a\right]}_m + {\left[b\right]}_m = {\left[a + b\right]}_m $ and 
$ {\left[a\right]}_m * {\left[b\right]}_m = {\left[a * b\right]}_m $.

For example consider $\Z_{3}$. Then 
\begin{itemize}
    \item $ {\left[1\right]}_3 +  {\left[2 \right]}_3  =  {\left[3\right]}_3 = 
	{\left[0\right]}_3 $.
    \item $ {\left[2 \right]}_3 * {\left[2 \right]}_3  = {\left[4 \right]}_3 = 
	{\left[1 \right]}_3 $.
\end{itemize}

Since arithmetic has been defined for $\Z_{m}$ we can consider the additive inverse and
multiplicative inverse of ${\left[a\right]}_m \in \Z_{m}$. Thus,
\begin{itemize}
    \item ${\left[b\right]}_m \in \Z_{m}$ is the additive inverse of ${\left[a\right]}_m$ whenever
	${\left[a\right]}_m + {\left[b\right]}_m = {\left[0\right]}_m$.
    \item ${\left[b\right]}_m \in \Z_{m}$ is the multiplicative inverse of ${\left[a\right]}_m$ whenever
	${\left[a\right]}_m * {\left[b\right]}_m = {\left[1\right]}_m$.
\end{itemize}

Again note that for multiplicative inverse of ${\left[a\right]}_m$ to exist it would mean that $a$
and $m$ are co-prime and $a$ is non-zero. As an example in $\Z_{10}$ only ${\left[1\right]}_m$, 
${\left[3\right]}_m$, ${\left[7\right]}_m$ and ${\left[9\right]}_m$ will have multiplicative
inverses.

\subsection{Prime factorization}
\begin{Definition}
    A positive integer $p$ is said to be prime iff the only positive divisors of $p$ are $p$ and
    $1$. If a number is not prime then it is composite.
\end{Definition}
Thus if a number $q > 1$ is composite then $q = a*b$ for $ 1 < a < q $ and $ 1 < b < q$.  

We state two important propositions without proof. 
\begin{Proposition}
    \begin{itemize}
	\item Every integer greater than $1$ can be written as a product of primes.
	\item Suppose that $p$ is prime and $p | a*b$, then $p | a$ or $p | b$.
    \end{itemize}
\end{Proposition}

A very important theorem about prime numbers is that every positive integer can be uniquely factored
as a product of primes.
\begin{Theorem}[name=Fundamental theorem of Arithmetic]
    Every positive integer greater than $1$ can be uniquely written as a product of prime numbers
    with the prime factors in the product written in non-decreasing order.
\end{Theorem}

For example $72 = 2^3 * 3^2$.

Note that if $m = p \geq 2 \in \Z^{+}$ is prime then every non-zero element of $\Z_{p}$ will have an
multiplicative inverse.

This completes an introduction to the integers. We will revisit them in the context of Group theory.
\section{Foundations of set theory}\label{App:set_theory}
In this section, we will cover the basics of axiomatic set theory and will conclude with the 
Axiom of Choice. Most proofs will be omitted.
The axiomatic theory of set is just based two (undefined) notions \textit{class} and the 
\textit{membership relation} denoted by $\in$. All objects are classes. However there are two kinds
of classes:
\begin{itemize}
    \item Sets
    \item Proper Classes
\end{itemize}
If $x,A$ are classes then the expression $x \in A$ means that $x$ is an element of $A$. This leads
to our first definition
\begin{Definition}
    Let $x$ be a class. If $x$ is belongs to some class $A$ then $x$ is called an element.
\end{Definition}
All elements are denoted by lower case letters. Hence whenever we write $x,y,z$ we mean classes that
belong to some class. Whenever we denote classes by capital letters $A,B,C$ then such a class may be
an element of some other class or may not be an element at all.

\begin{Definition}
    Let $A,B$ be classes. We define $A = B$ to mean that every class that has $A$ as its element
    must have $B$ as an element and vice versa. Logically,
    \begin{equation*}
	A = B \, \text{iff} \, (\forall X)\left[ A \in X \implies B \in X \land B \in X \implies A \in X
	\right].
    \end{equation*}
\end{Definition}

The first axiom is called the \textbf{axiom of Extent} and is an equivalent statement of the above
definition.
\begin{enumerate}[label=\bfseries Axiom 1:]
    \item $A = B$ iff $x \in A \iff x \in B$.  
\end{enumerate}

\begin{Definition}
    Let $A,B$ be classes; we define $A \subseteq B$ to mean that every element of $A$ is an element of
    $B$. $A$ is called a subclass of $B$.
\end{Definition}

The second axiom is called the \textbf{axiom of class construction} and defines way to construct 
sets from elements. For this we define a \textit{property} as,
\begin{Definition}
    A property $P(x)$ is a mathematical statement involving an element $x$ such that it can be
    expressed entirely in terms of the logical symbols $\in, \land, \lor, \lnot, \exists, \forall$
    and variables $x,y,z,A,B \dots$. 
\end{Definition}

\begin{enumerate}[label=\bfseries Axiom 2:]
    \item If $P(x)$ is a property then there exists a class $C$ whose elements are precisely those
	that satisfy $P(x)$. Logicall we denote $C$ as,
	\begin{equation*}
	    C = \left.\lbrace x : P(x) \rbrace\right..
	\end{equation*}
\end{enumerate}

If $A$ and $B$ are classes then the following properties gives very important classes,
\begin{enumerate}
    \item $P(x)$ is $x \in A \lor x \in B$.
    \item $P(x)$ is $x \in A \land x \in B$.
\end{enumerate}

The class satisfying the first property is called the union of $A,B$ and is denoted by $A \cup B$.
The class satisfying the second property is called the intersection of $A,B$ and is denoted by 
$A \cap B$.

\begin{Definition}
    The universal class $\mathcal{U}$ is the class of all the elements. Thus it contains classes
    that belong to some class. Thus, 
    \begin{equation*}
	\mathcal{U} = \left.\lbrace x : x = x \rbrace\right.
    \end{equation*}
\end{Definition}

\begin{Definition}
    The empty class $\emptyset$ is the class that has no elements. Thus,
    \begin{equation*}
	\emptyset = \left.\lbrace x : x \neq x \rbrace\right.
    \end{equation*}
\end{Definition}

\begin{Definition}
    If two classes $A,B$ have no elements in common, they are said to be disjoint. Thus, $A,B$ are
    disjoint if,
    \begin{equation*}
	A \cap B = \emptyset.
    \end{equation*}
\end{Definition}

\begin{Definition}
    The complement of a class $A$, $\comp{A}$ is the class of all elements that do not belong to $A$.
    Thus,
    \begin{equation*}
	\comp{A} =\left.\lbrace x : x \not\in A \rbrace\right.
    \end{equation*}
\end{Definition}

Note that for any class $A$, $\emptyset \subseteq A$, $A \subseteq \mathcal{U}$ and $A \cup
\comp{A} = \mathcal{U}$ and $A \cap \comp{A} = \emptyset$.
The Demorgan Laws provide a duality about union and intersection,
\begin{itemize}
    \item $\comp{(A \cup B)} = \comp{A} \cap \comp{B}$.
    \item $\comp{(A \cap B)} = \comp{A} \cup \comp{B}$.
\end{itemize}

It is very important fact that union, intersection and complement describe an algebra of classes
which is summarized below (easily proven),

\begin{itemize}
    \item Identity Laws:
	\begin{itemize}
	    \item $A \cup A = A$.
	    \item $A \cap A = A$.
	\end{itemize}
    \item Associative Laws:
	\begin{itemize}
	    \item $A \cup ( B \cup C) = (A \cup B)\cup C $.
	    \item $A \cap ( B \cap C) = (A \cap B)\cap C $.
	\end{itemize}
    \item Commutative Laws:
	\begin{itemize}
	    \item $A \cup B = B \cup A$.
	    \item $A \cap B = B \cap A$.
	\end{itemize}
    \item Distributive Laws:
	\begin{itemize}
	    \item $A \cup (B \cap C) = (A \cup B) \cap (A \cup C)$.
	    \item $A \cap (B \cup C) = (A \cap B) \cup (A \cap C)$.
	\end{itemize}
\end{itemize}

\begin{Definition}
    The difference of two classes $A,B$, $A - B$ is the class of those elements that are elements of $A$ but
    not of $B$. Thus,
    \begin{equation*}
	A - B = \left.\lbrace x : x \in A \land x \not\in B \rbrace\right.
    \end{equation*}
\end{Definition}
Note that $A - B = A \cap \comp{B}$.

If $a$ is an element then from Axiom $2$ we can construct the class that contains only $a$. Such a
class is called a \textit{singleton} and is given by:
\begin{equation*}
    \lbrace a \rbrace =\lbrace x : x = a \rbrace.
\end{equation*}

Similarly we can create the \textit{un-ordered} pair $\lbrace a,b \rbrace$ which is,
\begin{equation*}
    \lbrace a,b \rbrace = \lbrace x : x = a \lor x = b \rbrace.
\end{equation*}

It is easy to see that $\lbrace a,b \rbrace = \lbrace c,d \rbrace$ 
when $(a = c) \lor (b = d)$ or $ (a = d) \lor (b = c)$. Something that is very important is the notion of
ordered pair which we denote by $(a,b)$. What is important about ordered pairs is that if 
$(a,b) = (c,d)$ the $a = c$ and $b = d$. Thus the \textit{order} in which they appear in the set is
important.

\begin{Definition}
    If $a,b$ are elements, then the ordered pair is the class given by,
    \begin{equation*}
	(a,b) = \left.\lbrace \lbrace a \rbrace, \lbrace a,b \rbrace \rbrace\right.
    \end{equation*}
\end{Definition}

\begin{Definition}
    The Cartesian product of two classes $A$ and $B$ denoted by $A \times B$ 
    is the class of all ordered pairs $(x,y)$ where $x \in A$ and $y \in B$. Thus,
    \begin{equation*}
	A \times B = \left.\lbrace (x,y) : x \in A \land y \in B \right.\rbrace.
    \end{equation*}
\end{Definition}

A class of ordered pairs is called a \textit{Graph}. Thus any subclass of $\mathcal{U} \times
\mathcal{U}$ is a graph. If $G$ is a graph, we denote $G^{-1}$ to be the inverse graph given by,
\begin{equation*}
    G^{-1} = \left.\lbrace (y,x) : (x,y) \in G \right.\rbrace.
\end{equation*}

\begin{Definition}
    If $G,H$ are graphs, then $G \circ H$ is the graph defined as follows:
    \begin{equation*}
	G \circ H = \set{(x,y)}{\exists z \ni (x,z) \in H \, \land \, (z,y) \in G}
    \end{equation*}
\end{Definition}

\begin{Theorem}
    If $G,H,J$ are graphs, then the following hold:
    \begin{enumerate}
	\item $(G \circ H)\circ J = G \circ (H \circ J)$.
	\item ${(G^{-1})}^{-1} = G $.
	\item ${(G \circ H)}^{-1} = H^{-1} \circ G^{-1}$.
    \end{enumerate}
\end{Theorem}
\begin{proof}
    We prove in order,
    \begin{enumerate}
	\item let $(x,y) \in (G \circ H)\circ J$. Then there is a $ z \ni (x,z) \in J$ and 
	    $(z,y) \in (G \circ H) $. Thus there is a $u \ni (z,u) \in H$ and $(u,y) \in G$. Thus $(x,u) \in 
	    H \circ J$. And so $(x,y) \in G \circ (H \circ J)$. The argument can be reversed and 
	    so we get the equality of classes. 
	\item Let $(x,y) \in {(G^{-1})}^{-1}$. Hence $(y,x) \in G^{-1}$. Thus $(x,y) \in G$. The
	    other direction is similar.
	\item Let $(x,y) \in {(G \circ H)}^{-1}$. Hence $(y,x) \in (G \circ H)$. Thus there is a 
	    $z \ni (y,z) \in H$ and $(z,x) \in G$. Thus $(z,y) \in H^{-1}$ and $(x,z) \in G^{-1}$, 
	    which is just $(x,z) \in G^{-1}$ and $(z,y) \in H^{-1}$. Hence 
	    $(x,y) \in H^{-1} \circ G^{-1}$.
    \end{enumerate}
\end{proof}

\begin{Definition}
    Let $G$ be a graph. By the domain of $G$ we mean the class
    \begin{equation*}
	dom\,G = \set{x}{\exists y \ni (x,y) \in G}.
    \end{equation*}
\end{Definition}

\begin{Definition}
    Let $G$ be a graph. By the range of $G$ we mean the class
    \begin{equation*}
	range\,G = \set{y}{\exists x \ni (x,y) \in G}.
    \end{equation*}
\end{Definition}

\begin{Theorem}
    IF $G,H$ are graphs then
    \begin{enumerate}
	\item $dom\,G = range\,G^{-1}$.
	\item $dom\,G^{-1} = range\,G$.
	\item $dom\,(G \circ H) \subseteq dom\,H$.
	\item $range\,(G \circ H) \subseteq range\,G$.
    \end{enumerate}
\end{Theorem}

The proof of the third statement is as follows,
\begin{proof}
    Let $x \in dom\,(G \circ H)$. Thus there is a $y \ni (x,y) \in G \circ H$. Thus there
    is a $z \ni (x,z) \in H$ and $(z,y) \in G$. Thus the existence of $z \ni (x,z) \in H$
    means that $x \in dom\,H$.
\end{proof}

An important corollary of the above theorem is that if $range\,H = dom\,G$ then $dom\,G\circ H =
dom\,H$. To prove the equality we just need to show that $dom\, H \subseteq dom\,G\circ H$. Consider
an element $x \in dom\,H$. Thus there is a $z \ni (x,z) \in H$. Since range of $H$ equal to
$dom\,G$, this means that $z \in dom\,G$. Hence there is a $y \ni (z,y) \in G$ and so 
$(x,y) \in G\circ H$. Thus $x \in dom\,G\circ H$.

\begin{Definition}
    An indexed class is the class denoted by $\lbrace A_i : i \in I \rbrace$ where $I$ is the class
    whose elements are called indices.
\end{Definition}
Formally an indexed class is a graph $G$ and each $A_i = \set{x}{(i,x) \in G}$. Thus if $I =
\lbrace 1,2\rbrace$ and $A_1 = \lbrace a,b \rbrace $ and $A_2 = \lbrace e , f \rbrace$ then the
indexed class $\lbrace A_i : i \in I \rbrace$ is the graph $G = \lbrace (1,a) , (1,b) , (2,e), (2,f)
\rbrace$.

\begin{Definition}
    Let $\lbrace A_i : i \in I \rbrace$ be an indexed family. Then,
    \begin{enumerate}
	\item The union of the classes $A_i$ consists of all those elements $x$ that are contained
	    in atleast one $A_i$. 
	    \begin{equation*}
		\bigcup_{i \in I} A_i = \set{x}{\exists j \ni x \in A_j}.
	    \end{equation*}
	\item The intersection of the classes $A_i$ consists of all those elements $x$ that are 
	    contained in each $A_i$. 
	    \begin{equation*}
		\bigcap_{i \in I} A_i = \set{x}{\forall j, x \in A_j}.
	    \end{equation*}
    \end{enumerate}
\end{Definition}

\begin{Theorem}
    Let $\lbrace A_i : i \in I \rbrace$ be an indexed class and $B$ be any class. Then,
    \begin{enumerate}
	\item If $B \subseteq A_i$ for every $i \in I$ then $B \subseteq \bigcap_{i \in I} A_i$.
	\item If $A_i \subseteq B$ for every $i \in I$ then $\bigcup_{i \in I} A_i \subseteq B$.
    \end{enumerate}
\end{Theorem}

The Generalized DeMorgan's Laws and Distributive laws can be restated as:
\begin{Theorem}[name=DeMorgan's Laws]
    Let $\lbrace A_i : i \in I \rbrace$ be an indexed class. Then,
    \begin{enumerate}
	\item ${\comp{(\bigcup_{i \in I} A_i)}} = \bigcap_{i \in I} \comp{A_{i}}$
	\item ${\comp{(\bigcap_{i \in I} A_i)}} = \bigcup_{i \in I} \comp{A_{i}}$
    \end{enumerate}
\end{Theorem}

\begin{Theorem}[name=Distributive Laws]
    Let $\lbrace A_i : i \in I \rbrace$ and $\lbrace B_j : j \in J \rbrace$  be indexed classes. 
    Then,
    \begin{enumerate}
	\item ${(\bigcup_{i \in I} A_i) \bigcap (\bigcup_{j \in J} B_j) } = \bigcup_{(i,j) \in I
	    \times J} A_{i} \cap B_{j}$.
	\item ${(\bigcap_{i \in I} A_i) \bigcup (\bigcap_{j \in J} B_j) } = \bigcap_{(i,j) \in I
	    \times J} A_{i} \cup B_{j}$.
    \end{enumerate}
\end{Theorem}

\begin{Theorem}
    Let $\lbrace G_i : i \in I \rbrace$ be a family of graphs. Then,
    \begin{enumerate}
	\item $dom\,(\bigcup_{i \in I} G_i) = \bigcup_{i \in I} (dom\,G_i)$.
	\item $range\,(\bigcup_{i \in I} G_i) = \bigcup_{i \in I} (range\,G_i)$.
    \end{enumerate}
\end{Theorem}

In the begining we noted that there were two kinds of classes. We now define the most important kind
of class called \textit{Set}.
\begin{Definition}
    A class $X$ is called a set if there is a class $Y$ such that $X \in Y$.
\end{Definition}
If for all class $Y$, $X \not\in Y$ then $X$ is called a proper class. The remaining axioms all
concern sets. 

\begin{enumerate}[label=\bfseries Axiom 3:]
    \item Every subclass of a set is itself a set.  
\end{enumerate}
Such a subclass is called a \textit{subset}. Note that for any class $B$, if $A$ is a set then
$A\cap B \subseteq A$. Thus \textit{intersections} are sets. 
The next axiom gives the existence of sets.

\begin{enumerate}[label=\bfseries Axiom 4:]
    \item The empty class $\emptyset$ is a set.
\end{enumerate}

\begin{enumerate}[label=\bfseries Axiom 5:]
    \item If $a,b$ are sets then the un-ordered pair $\lbrace a , b \rbrace$ is a set. 
\end{enumerate}

Note that $\emptyset$ is a set and the set containing the emptyset $\lbrace{\emptyset}\rbrace$ is
also a set i.e we used $b = a$ in the above axiom to construct this set. 
Thus we can form a new set $\lbrace \emptyset,\lbrace \emptyset \rbrace \rbrace$. We can
continue this forever.

\begin{Definition}
    Let $A$ be a set. By the power set of $A$ we mean the class which contains all subsets of $A$.
    Thus,
    \begin{equation*}
	\powSet{A} = \set{X}{X \subseteq A}.
    \end{equation*}
\end{Definition}

The next two axioms concerns sets of sets. 

\begin{enumerate}[label=\bfseries Axiom 6:]
    \item If $\mathcal{A}$ is a set of sets then $\bigcup \mathcal{A}$ is also a set.
\end{enumerate}
Note that $\bigcup{\mathcal{A}}$ is the set $\set{x}{\exists A \in \mathcal{A} \ni x \in A}$.

\begin{enumerate}[label=\bfseries Axiom 7:]
    \item If $A$ is a set then $\powSet{A}$ is also a set.
\end{enumerate}

The following theorem shows that the cartesian product of two sets is also a set.
\begin{Theorem}
    If $A,B$ are sets then $A \times B$ is also a set.
\end{Theorem}
\begin{proof}
    We will show that $A \times B$ is an subset of $\powSet{\powSet{A\cup B}}$
    and thus by Axiom $3$ is a set.
    Let $(x,y) \in A \times B$. Note that $(x,y) = \lbrace \lbrace x \rbrace, \lbrace x,y \rbrace
    \rbrace$. But $\lbrace x \rbrace \in \powSet{A \cup B}$ and $\lbrace y \rbrace \in
    \powSet{A \cup B}$.
    Thus $\lbrace \lbrace x \rbrace, \lbrace x,y \rbrace \rbrace$ is a subset of $\powSet{A\cup B}$. 
    That is $(x,y) \in \powSet{\powSet{A\cup B}}$. Hence $A \times B$ is a subset of 
    $\powSet{\powSet{A\cup B}}$.
\end{proof}

Now we will look into the set-theoretic definition of functions.
A function $f$ is a \emph{triple} $\left(A,B,f\right)$ where $A,B$ are sets and $f \subseteq A
\times B$ is a graph satisfying the following conditions:
\begin{enumerate}[label=\bfseries F \arabic*:]
    \item For every $x \in A$ there is a $y \in B$ such that $(x,y) \in f$.
    \item For every $x \in A$, if $y_1,y_2 \in B$ such that $(x,y_1) \in f$ and $(x,y_2) \in f$ then
	$y_1 = y_2$.
\end{enumerate}

We usually denote $\left(A,B,f\right)$ as $ f : A \to B$ and write $(x,y) \in f$ as $f(x)= y$.
Thus the above conditions become:
\begin{enumerate}[label=\bfseries F \arabic*:]
    \item For every $x \in A$ there is a $y \in B$ such that $f(x) = y$.
    \item For every $x \in A$, $f(x) = y_1 $ and $f(x) = y_2$ implies $y_1 = y_2$. 
\end{enumerate}

We note that if $A,B$ are sets then any graph $f \subseteq A \times B$ is a function iff
\begin{enumerate}
    \item $dom\,f = A$.
    \item $range\,f \subseteq B$.
    \item $F2$ is satisfied.
\end{enumerate}

Some important functions are:
\begin{enumerate}
    \item INJECTIVE, A function $f : A \to B$ is said to be injective iff for $x_1,x_2 \in A$,
	$f(x_1) = f(x_2)$ implies that $x_1 = x_2$.
    \item SURJECTIVE, A function $f : A \to B$ is surjective iff for every $y \in B$ there is a $x
	\in A$ such that $y = f(x)$, i.e range $f$ $ = B$.
    \item BIJECTIVE, A function $f : A \to B$ is bijective iff it is both surjective and injective.
\end{enumerate}

Some examples are as follows:
\begin{enumerate}
    \item Identity function. The function $i_A : A \to A$ given by $i_A(x) = x$ 
	for every $ x\in A$ is called the identity function.  
    \item Inclusion function. Let $A,B$ be sets such that $B \subseteq A$. The function $i_B : B \to
	A$ is called the inclusion function. Note that $i_B(x) = x$ for every $x \in B$.
    \item Characteristic function. Let $2$ designate the class of all functions with two elements, say the
	class $\lbrace 0,1\rbrace$. If $B \subset A$, then the characteristic function of $B$ is
	given by $\chi_B : B \to 2$ such that whenever $x \in B$ then $\chi_B(x) = 1$, otherwise
	$\chi_B(x) = 0$. 
    \item Restriction function. Let $C \subseteq A$ and $f : A \to B$. Then the restriction of $f$
	to $C$ is the function given by $f_{|_C} : C \to B$ such that $f_{|_C}(x) = f(x)$ for every
	$x \in C$.
\end{enumerate}

The next theorem concerns composition.
\begin{Theorem}
    If $f : A \to B$ and $g : B \to C$ are functions then $g \circ f : A \to C $ is a function and
    $(g\circ f)(x) = g(f(x))$ for every $x \in A$.
\end{Theorem}
The proof is easy once we notice that since $g$ is a fuction $dom\,g = B$ and so range of $f$ is
subset of domain of $g$ thus $dom\,g\circ f = A$. Also we observe that $range\, g\circ f \subseteq
range\,g$ for any graph. That $F2$ is satisfied is easy.

\begin{Definition}
    A function $f : A \to B$ is invertible if the inverse graph is a function $f^{-1} : B \to A$.
\end{Definition}

\begin{Theorem}
    A function is invertible iff it is bijective. Furthermore, if a function is invertible then
    the inverse is a bijective function.
\end{Theorem}

A useful characterization of invertible function is the following:
\begin{Theorem}
    A function $f : A \to B$ is invertible iff there is a function $g : B \to A$ such that $g \circ
    f = i_A$ and $f\circ g = i_B$. If such a function $g$ exists then $g = f^{-1}$.
\end{Theorem}

\begin{Theorem}
    A function $f : A \to B$ is injective iff there is a function $g : B \to A$ such that $g \circ
    f = i_A$.
\end{Theorem}
\begin{proof}
    Let a function $g : B \to A$ be such that $g\circ f (x) = x $ for every $x \in A$. Consider
    $x_1,x_2$ such that $f(x_1) = f(x_2)$. Thus $x_1 = (g \circ f)(x_1) = g(f(x_1)) = g(f(x_2)) = 
    x_2$. Hence $f$ is injective. Consider an injective function $f$ and fix an element $a \in A$.
    Construct $g : B \to A$ as follows. If $y \in range\,f$ let $g(y) = f(x)$. If $y \not \in
    range\,f$ then $g(y) = a$. Easy to see that $g\circ f = i_A$. 
\end{proof}

\begin{Theorem}
    Let $f : A \to B$ and $g : B \to C$ be functions. Then we can state the following about the
    composition $g \circ f$,
    \begin{enumerate}
	\item If $f,g$ are injective then so is $g \circ f$.
	\item If $f,g$ are surjective then so is $g \circ f$.
	\item If $f,g$ are bijective then so is $g \circ f$.
    \end{enumerate}
\end{Theorem}

Now we will define Direct and Inverse images of sets under a function.
\begin{Definition}
    Let $f : A \to B$ be a function and consider a set $C \subseteq A$. Then the direct image of $C$
    under $f$ is the set of all images of elements of $C$,
    \begin{equation*}
	f(C) = \set{y \in B}{\exists x \in C \ni f(x) = y}.
    \end{equation*}
\end{Definition}

\begin{Definition}
    Let $f : A \to B$ be a function and consider a set $D \subseteq B$. Then the inverse image of $D$
    under $f$ is the set of all elements in $A$ whose images are elements of $D$,
    \begin{equation*}
	\invIm{f}{D} = \set{x \in A}{\exists y \in D \ni f(x) = y}.
    \end{equation*}
\end{Definition}
We can alternatively write the inverse image of $D$ as the set 
\[\invIm{f}{D} = \set{x \in A}{f(x) \in D}.\]
It is important to see how direct images and inverse images act on generalized unions and
intersections. The next theorem summarizes there actions.
\begin{Theorem}
    Let $f : A \to B$ and let ${\lbrace C_i \rbrace}_{i \in I}$ and 
    ${\lbrace D_i \rbrace}_{i \in I}$ be sub-families in $A$ and $B$ respectively. Then,
    \begin{enumerate}
	\item $\dirIm{f}{\indexUnion{C_i}{i}{I}} = \indexUnion{\dirIm{f}{C_i}}{i}{I}$,
	\item $\dirIm{f}{\indexIntersection{C_i}{i}{I}} \subseteq \indexIntersection{\dirIm{f}{C_i}}{i}{I}$,
	\item $\invIm{f}{\indexUnion{D_i}{i}{I}} = \indexUnion{\invIm{f}{D_i}}{i}{I}$,
	\item $\invIm{f}{\indexIntersection{D_i}{i}{I}} = \indexIntersection{\invIm{f}{D_i}}{i}{I}$,
    \end{enumerate}
\end{Theorem}
Thus, we see that the inverse image is well behaved w.r.t unions and intersections (and
complements). Note that the inverse and direct image are functions that maps powersets, i.e
$C(\in \powSet{A}) \mapsto \dirIm{f}{C}(\in \powSet{B})$ and $D (\in \powSet{B}) 
\mapsto \invIm{f}{D} (\in \powSet{A})$. This is easy to see if we observe that if $C_1 = C_2$ then
$\dirIm{f}{C_1} = \dirIm{f}{C_2}$. Similarly for inverse images. However, the converse 
is not true in general.

We have defined the \emph{product} of two classes as $A \times B$ as the class of all the ordered
pairs in $A$ and $B$. We can extend this idea to the product of finite classes $A_1,A_2,\dots,A_n$
as the class of all \emph{n-tuple} $(a_1,a_2,\dots,a_n)$ such that $a_i \in A_i$ for all $1\leq i
\leq n$. However, we have a potential problem if we an arbitrary indexed family, 
$\set{A_i}{i\in I}$. In such a case we have to redefine what a product of classes mean.

\begin{Definition}
    Let $\set{A_i}{i\in I}$ be an indexed family of classes; let
    \[A = \indexUnion{A_i}{i}{I}.\]
    The product of the classes $A_i$ is defined to be the class
    \[\indexProduct{A_i}{i}{I} = \set{f}{\map{f}{I}{A}\,\text{and $f(i)\in A_i$}\,\forEv{i\in I}}.\]
\end{Definition}
We will adopt the following notational convention: we designate elements of a product
$\indexProduct{A_i}{i}{I}$ with boldface letters $\vect{a},\vect{b}$ etc. If $\vect{a}$ is an
element of $\indexProduct{A_i}{i}{I}$, we will denote by $a_j$ as $\vect{a}(j)$. We call $a_j$ as
the $j^{th}$ co-ordinate of $\vect{a}$.
Let $A = \indexProduct{A_i}{i}{I}$, corresponding to each index we define a function
$\map{\Pi_{i}}{A}{A_i}$ by
$\Pi_i(\vect{a}) = a_i$. We call $\Pi_i$ as the $i^{th}-projection$ of $A$ to $A_i$.

\begin{Definition}
    If $A,B$ are classes, we denote by $B^A$ as the class of all functions whose domain is $A$ and
    whose co-domain is $B$.
\end{Definition}
In particular if $2 = \lbrace 0,1\rbrace $ 
denotes the class of two elements, then $2^A$ is the class of all functions 
from $A$ to $\lbrace 0,1\rbrace$.
\begin{Theorem}
    If $A$ is a set, then  $\powSet{A}$ and $2^A$ are in $1-1$ correspondence.
\end{Theorem}
\begin{proof}
We will show that there is a function $\map{f}{\powSet{A}}{2^A}$, such that $f$ is injective.
For any $B \in \powSet{A}$ define $f(B) = \charFunc{B}$. Easy to see that $f$ is injective. Infact
there is a bijection. Let $g \in 2^A$. Define $B = \invIm{g}{\lbrace 1 \rbrace }$. 
Then $g = \charFunc{B}$.
\end{proof}

Let us list a couple more axioms that will \emph{almost} complete the set construction axiom. 

\begin{enumerate}[label=\bfseries Axiom 9:]
    \item If $A$ is a non-empty set, there is an element $a \in A$ such that $a\cap {A} =
	\emptyset$.
\end{enumerate}
The above axiom states that a set is disjoint from its elements. Hence if $A$ is a set, the
singleton $\lbrace A \rbrace \neq A$.
\begin{enumerate}[label=\bfseries Axiom 10:]
    \item If $A$ is a set and $\map{f}{A}{B}$ is a surjective function, then $B$ is a set.
\end{enumerate}

Next we define relations on sets.
\begin{Definition}
    Let $A$ be a class, by a relation $R$ in $A$ we mean an arbitrary subclass of $A \times A$.
\end{Definition}
Let $R$ be a relation in $A$, then
\begin{enumerate}
    \item
	(Reflexive) $R$ is reflexive if for every $a \in A$, $(a,a) \in R$. 
    \item
	(Irreflexive) $R$ is irreflexive if for every $a \in A$, $(a,a) \not \in R$. 
    \item
	(Symmetric) $R$ is symmetric if $(a,b) \in R \implies (b,a) \in R$. 
    \item
	(Asymmetric) $R$ is asymmetric if $(a,b) \in R \implies (b,a) \not \in R$. 
    \item
	(Anti-symmetric) $R$ is anti-symmetric if $(a,b),(b,a) \in R \implies a=b $. 
    \item
	(Transitive) $R$ is transitive if $(a,b) ,(b,c) \in R \implies (a,c) \in R$. 
\end{enumerate}
\begin{Definition}
    A relation $R$ in $A$ is called an equivalence relation if it is Reflexive, Transitive and
    Symmetric.
\end{Definition}
\begin{Definition}
    A relation $R$ in $A$ is called a partial order relation if it is Reflexive, Transitive and
    Anti-symmetric.
\end{Definition}
\begin{Definition}
    A relation $R$ in $A$ is called a strict order relation if it is Irreflexive, Transitive and
    Asymmetric.
\end{Definition}

\section{Countability}
