\chapter{Systems of Linear equations}

In this chapter we will study certain rank-preserving operations on matrices and view them in the theory of
solving systems of linear equations. Before we proceed further, we will look at different ways of looking at
matrix-matrix multiplication.

Let $A = {\left[a_{ij}\right]}_{m\times p}$, let $B = {\left[b_{ij}\right]}_{p \times n}$ and 
$C = {\left[c_{ij}\right]}_{m \times n}$ such that $C = AB$. Then we can make certain observations about the rows
and column of $C$.
\begin{Remark}
    The $i^{th}$ row of $C$ can be viewed as the linear combination of the matrix $B$ with the $i^{th}$ row of
    $A$. Thus,
    \[C[i,:] = \finiteSum{a_{ik}B}{k}{p}.\]
    The $j^{th}$ column of $C$ can be viewed as the linear combination of the matrix $A$ with the $j^{th}$
    column of $B$. Thus,
    \[C[:,j] = \finiteSum{b_{kj}A}{k}{p}.\]
\end{Remark}
\begin{Definition}[name = Elementary operations]
    Let $A = {\left[a_{ij}\right]}_{m\times n}$ be a matrix of size $m\times n$. Any one of the following
    operations on the rows (columns) of $A$ is called an elementary row (column) operation:
    \begin{enumerate}
	\item (TYPE 1) interchanging any two rows (columns) of $A$,
	\item (TYPE 2) multiplying any row (column) of $A$ by a nonzero constant.
	\item (TYPE 3) adding any constant multiple of row (column) of $A$ to another row (column) of $A$.
    \end{enumerate}
\end{Definition}
\begin{Definition}[name = Elementary matrix]
    An $n\times n$ elementary matrix is a matrix obtained by performing an elementary operation on $I_n$. To
    A TYPE 1 elementary matrix is given by $E_1(i,j)$, a TYPE 2 elementary matrix is given by $E_2(i,c)$ and a
    TYPE 3 elementary matrix is given by $E_3(i,j,c)$, where:
    \begin{enumerate}
	\item
	    $E_1(i,j)$ is $I_n$ with rows $i,j$ swapped.
	\item
	    $E_2(i,c)$ is $I_n$ with the row $i$ multiplied by $c$.
	\item
	    $E_3(i,j,c)$ is $I_n$ with row $i$ equal to row ($i$) $+$ $c*$row ($j$). 
    \end{enumerate}
\end{Definition}
\begin{Example}
    Consider $I_3$. We show a few examples of elementary matrices.
    $E_1(1,2)$ is given by
    \begin{equation*}
	E_1(1,2) = 
	\begin{pmatrix} 
	    0 & 1 & 0 \\
	    1 & 0 & 0 \\
	    0 & 0 & 1
	\end{pmatrix}
    \end{equation*}
    $E_2(2,-5)$ is given by
    \begin{equation*}
	E_2(2,-5) = 
	\begin{pmatrix} 
	    1 & 0 & 0 \\
	    0 & -5 & 0 \\
	    0 & 0 & 1
	\end{pmatrix}
    \end{equation*}
    $E_3(1,3,2)$ is given by
    \begin{equation*}
	E_3(1,3,2) = 
	\begin{pmatrix} 
	    1 & 0 & 2 \\
	    0 & 1 & 0 \\
	    0 & 0 & 1
	\end{pmatrix}
    \end{equation*}
\end{Example}

%%%%%%%%% merge from previous

%%%%%%%% done merge
\begin{Corollary}
    Elementary row and column operations are rank preserving.
\end{Corollary}
\begin{Theorem}
    The rank of a matrix equals the number of linearly independent columns.
\end{Theorem}
\begin{proof}
\end{proof}
\begin{Theorem}
    Let $A$ be a matrix of size $m\times n$ of rank $n$. Then, $r \leq m, r\leq n$ and by means of elementary
    row and column transformation, $A$ can be transformed into the matrix,
    \begin{equation*}
	D = 
	\begin{pmatrix} 
	    I_r & 0  \\
	    0 & 0  
	\end{pmatrix}
    \end{equation*}
\end{Theorem}


\endinput
