\chapter{Basic Linear Algebra}

\section{Vector spaces}
Just as groups and rings were an abstraction of number systems, vector spaces are an abstraction of geometry.
In mordern geometry, a point in $3$ dimensional space is abstracted as a vector. Typically, we identify a
co-ordinate system and an origin to represent the $3D$ space around as. A \textbf{point} in space is
identified with a \textbf{vector} whose end is rooted in the origin and whose tip is placed at the point in
question. 
Given a vector in a $3D$ space we can multiply it by a real number $c$ called a \textbf{scalar} to stretch or
shrink the vector. See~\ref{fig:tikz:vector_scalar_mult}. 
When $c$ is negative we reverse the direction of the vector. 
\begin{figure}
  \includestandalone[width=0.5\textwidth]{tex/tikz_figures/scalar_multiplication}
  \caption{Scalar multiplying a vector}\label{fig:tikz:vector_scalar_mult}
\end{figure}
We can add two vectors
$\vv_{1}$ and $\vv_{2}$ to get a new vector $\vv_{1} + \vv_{2}$. Addition of two vectors is certainly
commutative as is shown in~\ref{fig:tikz:vector_add}. Note that whene we add $\vv_{1}$ to $-\vv_{1}$ we get
the zero vector which is the origin of our co-ordinate system. 

\begin{figure}
  \includestandalone[width=0.5\textwidth]{tex/tikz_figures/vector_addition}
  \caption{Vector addition}\label{fig:tikz:vector_add}
\end{figure}

We can abstract this notion by definining an algebraic structure that mimics the algebra of vectors in $3D$
space. Let $\F$ be any field.
\begin{Definition}
    A vector space over a field $\F$ is a triple $\measureS{\vsp{V}}{+}{\alpha}$ where,
    \begin{itemize}
	\item
	    $\vsp{V}$ is a set called the set of vectors.
	\item
	    $\map{+}{\vsp{V}\times\vsp{V}}{\F}$ is a binary operation.
	\item
	    $\map{\alpha}{\F\times\vsp{V}}{\F}$ is a scalar mulitiplication map such that
	    $\alpha(\lambda,\vect{v}) := \lambda\cdot\vect{v}$,
    \end{itemize}
    and,
    \begin{enumerate}
	\item
	    $\groupS{\vV}{+}{\zerv{\vV}}$ is a (commutative) Abelian group.
	\item For any $\beta,\gamma \in \F$ and $\vu,\vv \in \vV$
	    \begin{itemize}
		\item
		    $\beta\cdot(\vu + \vv) = \beta\cdot\vu + \beta\cdot\vv$.
		\item
		    $(\beta + \gamma)\cdot\vu = \beta\cdot\vu + \gamma\cdot\vu$.
		\item
		    $\beta\cdot(\gamma\cdot\vu) = (\beta\gamma)\cdot\vu$.
		\item
		    $1\cdot\vu = \vu$.
	    \end{itemize}
    \end{enumerate}
    We usually do not denote the scalar multiplication map and in short a vector space over a field is denoted
    by $\vspF{\vV}{F}$.
\end{Definition}
In this definition we have used $\textbf{boldface}$ letters for arbitrary vectors and greek letters for
scalars (field elements). We can immediately observe two useful properties from the definition,
\begin{Observation}
    Let $\vspF{\vV}{F}$ be a vector space. For any $\vv \in \vV$,
    \begin{enumerate}
	\item
	    $0\cdot\vv = \zerv{\vV}$
	    \begin{proof}
		We can write $0\cdot\vv$ as $(0+0)\cdot\vv$ which is equal to $0\cdot\vv + 0\cdot\vv$. There
		is an additive inverse $-0\cdot\vv$ such that $-0\cdot\vv + 0\cdot\vv = \zerv{\vV}$. Thus,
		\begin{align*}
		    &0\cdot\vv = 0\cdot\vv + 0\cdot\vv \\
		    &\iff 0\cdot\vv + -0\cdot\vv = 0\cdot\vv + 0\cdot\vv + -0\cdot\vv \\
		    &\iff \zerv{\vV} = 0\cdot\vv + \zerv{\vV} \\
		    &\iff \zerv{\vV} = 0\cdot\vv.
		\end{align*}	    
	    \end{proof}
	\item
	    $-1\cdot\vv = -\vv$.
	    \begin{proof}
		We can write $0\cdot\vv$ as $(1 + (-1))\cdot\vv$ and use the above observation along with the 
		fact that
		$1\cdot\vv = \vv$
	    \end{proof}
    \end{enumerate}
\end{Observation}
The two observations allow us to write $\beta\cdot\vv$ as $\beta\vv$ for any $\beta\in\F$ and $\vv\in\vV$.
\begin{Example}
    We list a few vector spaces.
    \begin{enumerate}
	\item
	    Let $\F = \R$ and $\vV = \Rn$. For any $\vu \in \vV$ we can write $\vu = \compV{u}{n}$. 
	    Define $\map{+}{\R^n\times\R^n}{\R}$ to be $\vu + \vv = \compSumV{u}{v}{n}$ and \break{}
	    $\map{\alpha}{\R\times\Rn}{\R}$ to be $\beta\cdot\vu = \compV{\beta u}{n}$.
	    Then $\vspF{\Rn}{\R}$ is a vector space.
	    We can think of $\R^1$ as a vector space over $\R$. When $n = 3$, we get our usual vector algebra
	    in the $3D$ space that motivated our discussion of vector spaces.
	\item
	    Let $X$ be a non-empty set and let $\F$ be a field.  We define,
	    \[\F(X) := \set{f}{\map{f}{X}{\F}},\]
	    with the following maps,
	    $\map{+}{\F(X)\times\F(X)}{\F}$ as follows,
	    \[\forEv{f,g\in\F(X)},\quad (f+g)(x) := f(x) + g(x),\]
	    for any $x\in X$ and, 
	    $\map{\alpha}{\F\times\F(X)}{\F}$ as
	    \[\forEv{\beta\in\F},\quad(\beta\cdot f)(x) := \beta f(x), \]
	    for any $x\in X$. Then $\vspF{\F(X)}{\F}$ is a vector space where the $\zerv{\F(X)}$ is the zero
	    function that maps all points in $X$ to $0 \in \F$.
    \end{enumerate}
\end{Example}

A particular vector space of the kind discussed in (2) above is the set of all polynomials. 
\begin{Definition}
    A polynomial defined on a set $X$ with coefficients from a field $\F$ is an expression of the form,
    \[f(x) = a_0 + a_1x + a_2x^2 + \cdots + a_{n}x^{n},\]
    where $n$ is a non-negative integer. The degree of a polynomial is the largest exponent $k$ of $x$ such
    that $a_k \neq 0$. We denote the space of polynomials as $\F\left[X\right]$ or sometimes when $X$ is
    obvious from the context as $P(\F)$.
\end{Definition}
\begin{Remark}
    Note that $\Fn = \F \times F \times \cdots \times \F$, where the product is taken $n$ times. 
    It is customary to express elements of $\Fn$ as column vectors. Thus if $\vv \in \F^3$, then we express
    $\vv = \colvec{3}{\alpha}{\beta}{\gamma}$ for some $\alpha,\beta,\gamma$ in $\F$ which are called the
    \textbf{co-ordinates} of $\vv$.
\end{Remark}
Just as in groups, we want to examine subsets of a given set that have the same algebraic structure.
\begin{Definition}
    Let $\vspF{\vV}{\F}$ be a vector space. A subset $\vW \subset \vV$ is called a subspace of $\vV$ if $\vW$
    is a vector space over $\F$.
\end{Definition}
The following theorem shows that we only need to check $3$ conditions on a subset to determine if it is a
subspace.
\begin{Theorem}\label{thm:crit_subspace}
    Let  $\vspF{\vV}{\F}$ be a vector space and let $\vW\subset \vV$ be a subset of $\vV$. Then
    $\vspF{\vW}{\F}$ is a subspace of $\vV$ if and only if,
    \begin{itemize}
	\item
	    $\zerv{\vV} \in \vW$.
	\item
	    (closed under addition) $\vx + \vy \in \vW$, whenever $\vx,\vy \in \vW$.
	\item
	    (closed under scalar multiplication) $\beta\cdot\vx \in \vW$, whenever $\vx\in\vW$ for any
	    $\beta\in\F$.
    \end{itemize}
\end{Theorem}
\begin{Example}
    Let $C(\F)$ be the set of all continuous functions defined on a set $X$. Clearly $C(\F) \subset \F(X)$ and
    it satisfies all the three properties above and so is a subspace of $\F(X)$.
\end{Example}
\begin{Example}
    Let $P_n(\F)$ be the space of polynomials of degree less than or equal to $n$. Then $P_n(\F)$ is a
    subspace of $P(\F)$.
\end{Example}
Given a collection of subspaces of a vector space, we can construct a new subspace. 
\begin{Theorem}\label{thm:intersect_subspaces}
    Any intersection of subspaces of a vector space $\vV$ is a subspace of $\vV$. It is the largest subspace
    that is contained in each subspace. 
\end{Theorem}
\begin{proof}
    Let $\famV$ be a collection of subspaces of $\vV$ and let
    \[\vW = \bigcap\limits_{\vU\in\famV}\vU = \set{\vx\in\vV}{\vx\in\vU,\quad\forEv{\vU\in\famV}}.\]
    To show that $\vW$ is a subspace of $\vV$ we need to show all the conditions in~\ref{thm:crit_subspace} are
    satisfied. Since $\zerv{\vV}$ is in all the $\vU$, it is also in $\vW$. Consider $\vx,\vy \in \vV$ such
    that $\vx,\vy$ are in $\vW$. Then by definition $\vx,\vy$ are in all $\vU$ and hence $\vx +\vy$ is in all
    $\vU$ and thus $\vx + \vy$ is in $\vW$. Similary, it is closed under scalar multiplication.
\end{proof}
We can ask what is the smallest subspace that contains a collection of subspaces of a vector space. We might
think the union of such a collection would give us an answer, but it is easy to check that given two subspaces
$\vU,\vW$, their union may fail to be a subspace. To answer this, let us start with a definition.
\begin{Definition}
    Suppose $\vV$ is a vector space and let $\listV{\vU}{m}$ be subspaces of $\vV$. Then sum of $\listV{\vU}{m}$
    is a subset of $\vV$ that contains all possible sums of elements of $\listV{\vU}{m}$. We denote it as,
    \[\sumV{\vU}{m} = \set{\sumV{\vu}{m}}{\vu_{i}\in\vU_{i},\,1\leq i \leq m}.\]
\end{Definition}
\begin{Example}
    Suppose $\vV$ is a subset of $\R^3$ containing all those vectors whose second and third co-ordinate are 
    zero and $\vW$ is a subset of $\R^3$ containing all those vectors whose first and third co-odinates are
    zero. Then \[\vV + \vW = \set{\colvec{3}{x}{y}{0}\in\R^3}{x,y\in\R}\] is a subset of $\R^3$.
\end{Example}
\begin{Theorem}
    Suppose $\listV{\vU}{m}$ are subspaces of $\vV$. Then \break{} 
    $\sumV{\vU}{m}$ is the smallest subspace of $\vV$
    containing $\listV{\vU}{m}$.
\end{Theorem}
\begin{proof}
    Since $\zerv{\vV}$ is in each $\vU_{i}$, the first condition of~\ref{thm:intersect_subspaces} is
    satisfied. Let $\beta\in\F$ be an arbitrary scalar and let $\vx,\vy$ be in the sum space. Then there are
    vectors $\listV{\vu}{m}$ and $\listV{\vv}{m}$ such that $\vu_{i},\vv_{i} \in \vU_{i}$ and,
    \begin{align*}
	&\vx = \sumV{\vu}{m},\\
	&\vy = \sumV{\vv}{m}.
    \end{align*}
    Thus, $\dotV{\beta}{\vx} + \vy$ is given by,
    \[\sumVecs{\dotV{\beta}{\vu}}{\vv}{m},\]
    which is in the sum space. Suppose $\vW$ is any subspace of $\vV$ such that $\vU_{i} \subset \vW$ for $1\leq i
    \leq m$. For any $\vx$ in the sum space $\vx = \sumV{\vu}{m}$, $\vu_{i}\in \vU_{i}$ and hence 
    $\vx \in \vW$ and thus the sum space is a subset of $\vW$.
\end{proof}
Thus the sum space is the smallest space that contains a collection of \textbf{finite} subspaces of a vector
space. The next example show that a vector in the sum space doesn't have a unique representation.
\begin{Example}
    Let,
    \begin{align*}
	&\vU_{1} = \set{\colvec{3}{x}{y}{0}\in\R^3}{x,y\in\R}\\
	&\vU_{2} = \set{\colvec{3}{0}{0}{z}\in\R^3}{z\in\R}\\
	&\vU_{3} = \set{\colvec{3}{0}{y}{y}\in\R^3}{y\in\R}\\
    \end{align*}
    Any $\vv = \colvec{3}{x}{y}{z} \in \R^3$ can be written as $\colvec{3}{x}{y}{z} = \colvec{3}{x}{y}{0} + 
    \colvec{3}{0}{0}{z} + \colvec{3}{0}{0}{0}$ and hence
    $\sumV{\vU}{m} = \R^3$. However, the representation is not unique. For example we can write
    \[\colvec{3}{0}{0}{0} = \colvec{3}{0}{0}{0} + \colvec{3}{0}{0}{0} + \colvec{3}{0}{0}{0},\]
    and
    \[\colvec{3}{0}{0}{0} = \colvec{3}{0}{1}{0} + \colvec{3}{0}{0}{1} + \colvec{3}{0}{-1}{-1}.\]
\end{Example}
When we can write a vector in the sum space uniquely as the sum of vectors in each subspace comprising the sum
space we call it a \textbf{Direct sum}.
\begin{Definition}[name=Direct Sum]
    Suppose $\vV$ is a vector space and let $\listV{\vU}{m}$ be subspaces of $\vV$. The sum $\sumV{\vU}{m}$ 
    space is called a direct sum if each element in the sum space can be written uniquely as a sum 
    $\sumV{\vu}{m}$, where each $\vu_{j}$ is in $\vU_{j}$. We denote the direct sum as,
    \[\vU_{1}\oplus\cdots\oplus\vU_{m}.\]
\end{Definition}
\begin{Example}
    Suppose $\vU$ is a the subspace of $\R^3$ consisting of those vectors whose last co-ordinate equals $0$
    and $\vW$ is the subspace of $\R^3$ of those vectors whose first two co-ordinate equals $0$. Then
    $\vU\oplus\vW = \R^3$. This is easily seen as any $\colvec{3}{x}{y}{z}$ in $\R^3$ is equal to
    $\colvec{3}{x}{y}{0} + \colvec{3}{0}{0}{z}$. Moreover, this is unique since we don't get any contribution
    from the first two co-ordinates from $\vW$ and from the last co-ordinate in $\vU$.
\end{Example}
\begin{Proposition}
    Suppose $\listV{\vU}{m}$ are subspaces of $\vV$. Then $\sumV{\vU}{m}$ is a direct sum if and only if the
    only way to write $\zerv{\vV}$ as a sum of vectors $\sumV{\vu}{m}$, where each $\vu_{j}$ is in $\vU_{j}$,
    is by taking each $\vu_{j}$ equal to $\zerv{\vV}$.
\end{Proposition}
\begin{proof}
    Note that $\zerv{\vV} = \zerv{\vV} + \cdots + \zerv{\vV}$, where we take the $m$ sums. If $\sumV{\vU}{m}$
    is a direct sum then this is the only way to write $\zerv{\vV}$ as the sum of $\listV{\vu}{m}$ vectors
    such that $\vu_{j}$ is in $\vU_{j}$.

    Now, let $\sumV{\vU}{m}$ be a direct such that if $\zerv{\vV} = \sumV{\vu}{m}$, then each $\vu_{j} =
    \zerv{\vV}$. Assume $\vv$ in the sum space can be written as,
    \[\vv = \sumV{\vu}{m},\]
    and
    \[\vv = \sumV{\vw}{m},\]
    where $\vu_{j},\vw_{j}$ is in $\vU_{j}$.
    Now $\vv - \vv = \zerv{\vV}$ is in the sum space and so,
    \[\zerv{\vV} = \diffVecs{\vu}{\vw}{m}.\]
    According to our hypothesis, each $(\vu_j - \vw_j) = \zerv{\vV}$ which means that the representation of
    $\vv$ is unique and hence the sum space is a direct sum.
\end{proof}
For a pair of subspaces, the condition to check if their sum space is a direct sum becomes simpler.
\begin{Proposition}
    Suppose $\vU,\vW$ are subspaces of $\vV$. Then $\vU + \vW$ is a direct sum if and only if $\vU\cap \vW =
    \setX{\zerv{\vV}}$.
\end{Proposition}
\begin{proof}
    First assume we have $\vU\oplus\vW$. Let $\vv$ be a vector in $\vU \cap \vW$. Then $-\vV$ is also in 
    $\vU \cap \vW$. Hence $\zerv{\vV} = \vv + -\vv$. This means that $\vv = \zerv{\vV}$ because in a direct
    sum the zero vector can only be written as sum of zero vector.

    Assume $\vU \cap \vW = \setX{\zerv{\vV}}$. 
    Let $\zerv{\vV} = \vu + \vw$. This means that $\vu = -\vw$ and hence $\vu$ is in $\vW$. Thus $\vu$ is in
    $\vU \cap \vW$. But this implies that $\vu = \zerv{\vV}$. Hence $\vw = \zerv{\vV}$.
\end{proof}
Our next definition is again motivated by lines and planes in $3D$ geometry.

\begin{Definition}[name=Linear combination]
    Let $\vspF{\vV}{\F}$ be a vector space over the field $\F$ and let $S$ be an non-empty set of $\vV$. A
    vector $\vv\in \vV$ is called a linear combination of the elements of $S$ if there is a \textbf{finite}
    number of elements $\listV{\vu}{n}$ in $S$ and $\listV{\beta}{n}$ in $\F$ such that,
    \[\vv = \lComb{\beta}{\vu}{n}.\]
    If $S$ has finite vectors $\listV{\vu}{n}$, then we say that $\vv$ is a linear combination of
    $\listV{\vu}{n}$.
\end{Definition}
Logically, a vector $\vv$ in $\vV$ is a linear combination of vectors in $S$ if,
\[\thereIs{\listV{\vu}{n};\,\vu_{i}\in S}\thereIs{\listV{\beta}{n};\,\beta_{i}\in\F}
    \suchThat{\vv}{=}{\lComb{\beta}{\vu}{n}}.\]
Thus a vector $\vv$ is \textbf{NOT} a linear combination of vectors in $S$ if no matter what finite set of
vectors we take in $S$ and no matter what corresponding finite set of scalars we take, we cannot write $\vv$
as a linear combination of those vectors and scalars. Note that the zero vector $\zerv{\vV}$ is a linear
combination of any non-empty subset of $\vV$ since $\zerv{\vV} = 0\vu$ for any $\vu \in \vV$.
\begin{Definition}[name=Span]
    Let $S$ be a non-empty subset of a vector space $\vV$. The span of $S$, denoted by $\Span{S}$, is the set
    of all the linear combinations of the elements of $S$. For convenience we define $\Span{S} = 
    \setX{\emptyset}$.
\end{Definition}
\begin{Example}
    Let $\vV = \R^3$ over the field $\R$. Let $S = \setX{\colvec{3}{1}{1}{0}}$. Then $\Span{S}$ is the set of
    all vectors of the form $\vu = \colvec{3}{\beta}{\beta}{0}$ which is seen as the line passing through
    origin and containing the point $\colvec{3}{1}{1}{0}$.

    If $S = \setX{\colvec{3}{1}{0}{0},\colvec{3}{0}{1}{0}}$, then the $\Span{S}$ is the set of all vectors of
    the form $\vu = \colvec{3}{\beta}{\gamma}{0}$ for some $\beta,\gamma \in \R$. This is seen as the $XY$
    plane.
\end{Example}
\begin{Proposition}
    The span of any subset $S$ of a vector space $\vV$ is a subspace of $\vV$. Moreover, any subspace $\vU$
    that contains $S$ must contain $\Span{S}$.
\end{Proposition}
\begin{proof}
    First note that $\zerv{\vV}$ is contained in $\Span{S}$. Let $\vv_1$ be a vector in $\Span{S}$. Then there
    are finite vectors $\listV{\vu}{n}$ in $S$ such that $\vv_1 = \lComb{\beta}{\vu}{n}$. Let $\vv_2$ be a
    vector in $\Span{S}$. Then ther are finite vectors $\listV{\vw}{m}$ in $S$ such that $\vv_2 =
    \lComb{\gamma}{\vw}{m}$.
    Then $\vv_{1} + \vv_{2} = \lComb{\beta}{\vu}{n} + \lComb{\gamma}{\vw}{m}$ is a linear combination of
    finite vectors in $S$. The case of scalar multiplication is similar.
\end{proof}
\begin{Definition}[name=Spanning set]
    A subset $S$ of vector space $\vV$ is called a spanning set of $\vV$ if $\Span{S} = \vV$. Thus every
    vector in $V$ can be written as a linear combination of vectors in $S$. We say that $S$ \textbf{generates}
    or \textbf{spans} $\vV$.
\end{Definition}
\begin{Definition}[name=Linear dependence]
    A subset $S$ of a vector space $\vV$ is called linearly dependent if there exists a finite number of
    \textbf{distinct} vectors $\listV{\vu}{n}$ in $S$ and scalars $\listV{\beta}{n}$ in $\F$ such that there
    is atleast one $\beta_{j} \neq 0$ and
    \[\lComb{\beta}{\vu}{n} = \zerv{\vV}.\]
\end{Definition}
Logically this means that $S$ is linearly dependent set if,
\[\thereIs{\listV{\vu}{n}}\thereIs{\listV{\beta}{n}}
    \suchThat{\thereIs{1\leq j \leq n};\beta_j\neq
	    0}{\text{and}}{\lComb{\beta}{\vu}{n} = \zerv{\vV}},\]
where each $u_i$ is in $S$. When is a set $S$ \textbf{NOT} linearly dependent? For that to happen we must
negate the logical statement above which says that if $S$ is NOT linearly dependent then,
\[\forEv{\listV{\vu}{n}}\forEv{\listV{\beta}{n}}
    \suchThat{\lComb{\beta}{\vu}{n} = \zerv{\vV}}{\implies}{\beta_{j}=0;\,\forEv{1\leq j\leq n}}.\]
This is an important notion and needs a definition,
\begin{Definition}[name=Linear independence]
    A subset $S$ of vector space $\vV$ is Linearly independent if it is not linearly dependent. Thus for any
    finite set of distinct vectors $\listV{\vu}{n}$ in $S$ and scalars $\listV{\beta}{n}$ in $\F$, if
    $\lComb{\beta}{\vu}{n} = \zerv{\vV}$, then each $\beta_{j} = 0$. 
\end{Definition}
A method to determine if a given set of finite vectors is linearly dependent will be shown later when we
discuss solution of linear systems of equation.
\begin{Proposition}
    Let $\vspF{\vV}{\F}$ be a vector field over the field $\F$ and let $S_1\subset S_2\subset\vV$.
    Then,
    \begin{enumerate}
	\item
	    If $S_1$ is linearly dependent, then $S_2$ is also linearly dependent.
	\item
	    If $S_2$ is linearly independent, then $S_1$ is also linearly independent.
    \end{enumerate}
    \begin{proof}
	We prove $(1)$ since $(2)$ is just the contrapositive of $(1)$.
	If $S_1$ is linearly dependent then there are vectors and scalars $\listV{\vu}{n}$ in $S_1$ and
	$\listV{\beta}{n}$ in $\F$ respectively, such that atleast one such $\beta_{j}$ is not $0$ and
	\[\lComb{\beta}{\vu}{n} = \zerv{\vV}.\]
	But since $S_1 \subset S_2$, the vectors $\listV{\vu}{n}$ are in $S_2$ and they satisfy the condition
	of linear dependence. Thus $S_2$ is linearly dependent.
    \end{proof}
\end{Proposition}

There is a crucial interplay between linear independence and span. We start with a few useful
observation.
\begin{Proposition}\label{prop:span_linear_ind_0}
    Let $S = \setV{\vu}{n}$ be a subset of a vector space $\vV$ over field $\F$, then 
    $S$ is linearly dependent if and only if there is a vector $\vu_{j} \in S$ such that
    $\vu_j$ is in $\Span{(S- \setX{\vu_{j}})}$. Moreover $\Span{S} = \Span{(S- \setX{\vu_{j}})}$
\end{Proposition}
\begin{proof}
    Let $S$ be linearly independent. Then there are scalars $\listV{\beta}{n}$ in $\F$ and vectors
    $\listV{\vu}{n}$ such that,
    \[\lComb{\beta}{\vu}{n} = \zerv{\vV},\]
    and there is atleast one $\beta_j$ such that $\beta_j \neq 0$.
    Since we have a \textbf{Field} $\F$, we know that
    every non-zero element of a field has a multiplicative inverse and so $\beta_j^{-1}$ exists 
    and hence, 
    \[\vu_j = \frac{1}{\beta_{j}}(\beta_1\vu_{1} + \cdots + \beta_{j-1}\vu_{j-1} + \beta_{j+1}\vu_{j+1} +
	\cdots + \beta_{n}\vu_{n}).\]
    Hence $\vu_{j}$ is in  $\Span{(\setV{\vu}{n} - \setX{\vu_{j}})}$.

    If there is a $\vu_{j}$ such that $\vu_j$ is in $\Span{(\setV{\vu}{n} - \setX{\vu_{j}})}$, then there are
    scalars such that,
    \[\vu_j = \beta_1\vu_{1} + \cdots + \beta_{j-1}\vu_{j-1} + \beta_{j+1}\vu_{j+1} +
	\cdots + \beta_{n}\vu_{n}.\]
    Thus, 
    \[\beta_1\vu_{1} + \cdots + \beta_{j-1}\vu_{j-1} + (-1)\vu_j + \beta_{j+1}\vu_{j+1} +
	\cdots + \beta_{n}\vu_{n} = \zerv{\vV}.\] 
    Hence $S$ is linearly independent. It is easy to check that $\Span{S} = \Span{(S- \setX{\vu_{j}})}$.
\end{proof}
\begin{Proposition}\label{prop:span_linear_ind_I}
    Let $S$ be a linearly independent subset of a vector space $\vV$, and let $\vv$ be an element of $V$ that
    is not in $S$. Then $S\cup\setX{\vv}$ is linearly dependent if and only if $\vv\in\Span{S}$.
\end{Proposition}
\begin{proof}
    Let $\vv$ be in $\Span{S}$. Then there are finite vectors $\listV{\vu}{n}$ in $S$ and scalars
    $\listV{\beta}{n}$ in $\F$ such that,
    \[\vv = \lComb{\beta}{\vu}{n}.\]
    Without loss of generality, we can assume $\vu$'s to be distinct. This means that,
    \[(-1)\vv + \lComb{\beta}{\vu}{n} = \zerv{\vV}.\]
    Since $\vv$ is not in $S$, none of the vectors $\vu_{j}$ is equal to $\vv$. Thus we have shown that the
    set $S\cup\setX{\vv}$ is linearly dependent.

    Assume $S\cup\setX{\vv}$ is a linearly dependent set. Then there is a set of finite distinct vectors
    $\listV{\vu}{n}$ in $S\cup\setX{\vv}$ and scalars $\listV{\beta}{n}$ such that,
    \[\lComb{\beta}{\vu}{n} = \zerv{\vv},\]
    and atleast one such $\beta_{j}$ is not equal to $0$. One of the $\vu_{j}$'s must be $\vv$. If not all the
    vectors are from $S$ and $S$ being linearly independent will contradict our assertion. Re-ordering, let
    $\vu_n$ be $\vv$, then
    \[\beta_{n}\vv + \lComb{\beta}{\vu}{n} = \zerv{\vV}.\]
    Again, with similar reasoning $\beta_{n}$ cannot be $0$. Since we have a \textbf{Field} $\F$, we know that
    every non-zero element of a field has a multiplicative inverse and so $\beta_n^{-1}$ exists and hence,
    \[\vv = \frac{1}{\beta_n^{-1}}(\lComb{\beta}{\vu}{n}).\]
    Thus $\vv$ is in $\Span{S}$.
\end{proof}
The proposition can be equivalently restated as follows:
\begin{Remark}
Let $S$ be a linearly independent subset of a vector space $\vV$, and let $\vv$ be an element of $V$ that
is not in $S$. Then $S\cup\setX{\vv}$ is linearly independent if and only if $\vv\not\in\Span{S}$.
\end{Remark}
\begin{Theorem}[name=Exchange Lemma]\label{thm:exch_lemma}
    Let $\vspF{\vV}{\F}$ be a vector space over a field $\F$. Consider the following sets:
    \begin{itemize}
	\item
	    $S = \setV{\vv}{m}$,
	\item
	    $L = \setV{\vu}{n}$,
    \end{itemize}
    such that $\Span{S} = V$ and $L$ is linearly independent. Then $m\geq n$.
\end{Theorem}
\begin{proof}
    We prove by contradiction. Assume $n > m$. The proof will use a construction of sets $S_{i}$ and $L_{i}$
    in each stage $i$,
    where, at the end of the stage we replace one element in $S_i$ by $\vu_i$. We will get a contradiction at
    stage $m$.

    \textbf{Stage 1}
    Since $\Span{S} = \vV$, we can write $\vu_{1}$ as,
    \[\vu_{1} = \lComb{\beta}{\vv}{m}.\]
    There must be one $\beta_j$ such that $\beta_j$ is not equal to $0$. Let us re-order so that the
    corresponding $\vv_{j}$ is $\vv_{m}$. Hence,
    \[\vu_{1} = \beta_{m}\vv_{m} + \lComb{\beta}{\vv}{m-1},\]
    and thus,
    \[\vv_m = \frac{1}{\beta_{m}}(\vu_{1} + \lComb{-\beta}{\vv}{m-1}).\]
    Replace $\vv_m$ by $\vu_1$.
    \begin{align*}
	&S_1 = \setX{\vv_{1},\ldots,\vv_{m-1},\vu_1},\\
	&L_1 = \setX{\vu_{2},\ldots,\vu_{m},\ldots,\vu_{n}}.
    \end{align*}
    Claim:
    \begin{itemize}
	\item
	    $\Span{S_1} = V$,
	\item
	    $L_1$ is linearly independent.
	\item
	    Number of $\vv_{j}$ in $S_{1}$ is $m-1$. 
    \end{itemize}
    Since $L_1 \subset L$ and $L$ was linearly independent, $L_1$ is linearly independent.
    For any $\vw$ in $\vV$, $\vw = \lComb{\gamma}{\vv}{m}$.
    Subsituting for $\vv_m$ we see that $\vw$ is in $\Span{S_1}$. The fact that we have $m-1$ $\vv$'s in $S_1$
    is by replacing one such $\vv$'s (namely $\vv_{m}$) from $S$ to get $S_1$.

    \textbf{Stage 2}
    Since $\Span{S_1} = \vV$, we can write $\vu_{2}$ as,
    \[\vu_{2} = \lComb{\beta}{\vv}{m-1} + \beta_m\vu_m.\]
    First note that $\beta_j$'s for $1\leq j \leq m-1$ 
    cannot be zero, since that would mean that $\vu_{2}$ is in $\Span{\setX{\vu_{1}}}$. Thus there exists a
    $j$ such that $1\leq j \leq m-1$ and $\beta_j$ is not equal to $0$. Let us re-order such that the
    corresponding $\vv_{j}$ is $\vv_{m-1}$. Hence,
    \[\vu_{2} = \lComb{\beta}{\vv}{m-2} + \beta_{m-1}\vv_{m-1} + \beta_m\vu_m.\]
    Replace $\vv_{m-1}$ by $\vu_{2}$ and set,
    \begin{align*}
	&S_2 = \setX{\vv_{1},\ldots,\vv_{m-2},\vu_2,\vu_1},\\
	&L_1 = \setX{\vu_{3},\ldots,\vu_{m},\ldots,\vu_{n}}.
    \end{align*}
    Again:
    \begin{itemize}
	\item
	    $\Span{S_2} = V$,
	\item
	    $L_2$ is linearly independent.
    \end{itemize}
    Since we have assumed that $n > m$, at the end of the $m^{th}$ stage, we would have exhausted 
    all of $\vv$'s, that is 
    \begin{align*}
	&S_m = \setX{\vu_{m},\vu_{m-1},\ldots,\vu_2,\vu_1},\\
	&L_m = \setX{\vu_{m+1},\ldots,\vu_{n}}.
    \end{align*}
    But note that $L_m$ is a subset of $L$ and hence should be linearly independent. But that is not the case,
    since $S_m$ spans $\vV$, we can write,
    \[\vu_{m+1} = \lComb{\beta}{\vu}{m},\]
    and this is not possible since all $\beta_j$ cannot be zero for that would mean $\vu_{m+1} = \zerv{\vV}$,
    but in that case $-\vu_{m+1} + \lComb{\beta}{\vu}{m} =\zerv{\vV}$ with non-zero coefficients.
\end{proof}
These two propositions are extremely useful. 
\begin{Definition}[name=Basis]
    Let $\vV$ be a vector space over field $\F$. A non-empty subset $\mcalB$ of $\vV$ is called a basis of
    $\vV$ if $\Span{\mcalB} = \vV$ and $\mcalB$ is linearly independent. If $\mcalB$ is \textbf{finite} then
    $\vV$ is called a \textbf{finite dimensional} vector space.
\end{Definition}
Note that, by~\ref{prop:span_linear_ind_I}, in a finite dimensional vector space the number of elements in a
spanning set is always greater than or equal to the number of elements in a linearly independent set. The
crucial property about basis in a finite dimensional vector space that we are going to see, is that every
vector will have a unique representation w.r.t~a given basis. 
First a few important observations.
\begin{Proposition}
    If $\vV$ is finite dimensional and $S$ is a subset of $\vV$ such that $\Span{S} = \vV$, then $S$ can be
    reduced to give a basis for $\vV$.
\end{Proposition}
\begin{proof}
    Let $S = \setV{\vu}{n}$. If $S$ is also linearly independent then $S$ is a basis. If not, then 
    by~\ref{prop:span_linear_ind_0}, there is a vector $\vu_j \in S$ such that $\vu_j$ is in the span of the
    rest of the vectors and the set formed by removing $\vu_j$ from $S$ doesn't change its span.  
    Let $S_1 = S - \setX{\vu_j}$. If $S_1$ is linearly independent the we have found a basis. IF not we repeat
    this process. Since $\vV$ is finite dimensional this process will stop eventually yielding a set of
    distinct finite vectors that is a basis for $\vV$.
\end{proof}
\begin{Proposition}
    If $\vV$ is finite dimensional and $L$ is a subset of $\vV$ such that $L$ is linearly independent, then
    $L$ can be extended to a basis for $\vV$.
\end{Proposition}
\begin{proof}
    Let $L = \setV{\vu}{n}$. Why does this set have to be finite? Since $\vV$ is finite dimensional there is a
    set $S = \setV{\vv}{m}$ such that $\Span{S} = \vV$ and by the Exchange lemme we know that $n \leq m$.
    If $L$ spans $\vV$ we have found our basis. If not, there is a $\vw \in \vV$ such that $\vw \not\in
    \Span{L}$. But by~\ref{prop:span_linear_ind_I}, $L\cup\setX{\vw}$ is also linearly independent. If this
    spans $\vV$ we are done, if not we continue. This process will stop eventually since we cannot have a
    linearly independent set of more than $m$ vectors.
\end{proof}
These two propositions enable us to give a unique number to a finite dimensional vector space called its
\textbf{dimension}
\begin{Proposition}
    If $\vV$ is a finite dimensional vector space, then $\vV$ admits a basis of finite vectors. Any two basis
    in $\vV$ must have the same number of elements. Moreover, any vector $\vv$ in $\vV$ can be written
    uniquely as a linear combination of vectors in the basis for $\vV$. 
\end{Proposition}
\begin{proof}
    If $\vV$ is finite dimensional then it has a finite spanning set $S$. Since $S$ can be reduced to a basis,
    every finite dimensional vector space admits a basis.
    Let $\mcalB_{1}$ be a basis for $\vV$ with $m$ elements and let $\mcalB_{2}$ be a basis for $\vV$ with $n$
    elements. Since $\mcalB_{1}$ is spanning and $\mcalB_{2}$ is linearly independent, $n \leq m$ by the
    exchange theorem. Reversing the roles we get $m \leq n$. Thus $m =n$.

    Let $\mcalB = \setV{\vu}{n}$ be a basis for $\vV$ and let
    \[\vv = \lComb{\beta}{\vu}{n},\]
    and
    \[\vv = \lComb{\gamma}{\vu}{n}\]
    be two representation of $\vv$ in $\vV$. Then $\zerv{\vV} = \lComb{(\beta-\gamma)}{\vu}{n}$. Since
    $\mcalB$ is linearly independent, this must mean that each $\beta_j = \gamma_j$ and hence the
    representation of $\vv$ in $\vV$ is unique w.r.t.~basis $\mcalB$.
\end{proof}
\begin{Definition}[name=Dimension]
    If $\vV$ is finite dimensional vector space then the number of vectors in a basis $\mcalB$ for $\vV$ is
    called the dimension of the vector space $\vV$.
\end{Definition}
\begin{Proposition}
    Let $\vV$ be a vector space over the field $\F$ such that $\dim{V} = n$. Then,
    \begin{enumerate}
	\item
	    If $S$ is a subset of $\vV$ such that $\Span{S} = \vV$, then the number of elements in $S$ is
	    greater than or equal to $n$. If number of elements in $S$ is equal to $n$, then $S$ is a basis
	    for $\vV$.
	\item
	    If $L$ is a linearly independent subset of $\vV$ that contains $n$ elements then $L$ is a basis
	    for $\vV$.
    \end{enumerate}
\end{Proposition}
\begin{proof}
    We prove in order.
    \begin{enumerate}
	\item
	    Let $\mcalB$ be a basis for $\vV$. Then $\mcalB$ has $n$ elements. Since a spanning set ($S$) cannot
	    have fewer elements than a linearly independent set ($\mcalB$) in a finite dimensional vector
	    space, the number of elements in $S$ is
	    greater than or equal to $n$. If $S$ has $n$ elements, $S$ can be reduced to a basis. But each
	    basis must have the same number of elements and so $S$ cannot be reduced any further. Hence $S$ is
	    a basis.
	\item
	    $L$ can be extended to a basis for $\vV$, but any basis must have the same number of elements
	    ($n$) and so $L$ must be a basis for $\vV$.
    \end{enumerate}
\end{proof}
\begin{Proposition}
    Let $\vW$ be a subspace of a finite dimensional vector space $\vV$. Then $\vW$ is also finite dimensional
    and $\dim{\vW} \leq \dim{\vV}$. Moreover, if $\dim{\vW} = \dim{\vV}$, then $V = W$.
\end{Proposition}
\begin{proof}
    Let $\dim{\vV} = n$ and let $\mcalB_{\vV}$ be a basis for $\vV$. If $\vW = \setX{\zerv{\vV}}$, then
    $\dim{\vW} = 0 \leq n$. Assume there is a non-zero element $\vv_{1}$ in $\vW$. If there is a $\vv_{2} \in
    \vV$ such that $\vv_{2}$ is not in the span of $\setX{\vv_{1}}$ and $\vv_{2}$ is in $\vW$, then
    $\setX{\vv_{1},\vv_{2}}$ is linearly independent subset of $\vW$. Continuing this way we can get a linearly
    independent set in $\vW$,
    \[\setV{\vv}{k}.\]
    However $k$ cannot be larger than $n$. Suppose $k = n$, 
    if $\setV{\vv}{k}$ doesn't span $\vW$, then there must be an
    element $\vv$ such that $\setX{\vv} \cup \setV{\vv}{k}$ is also linearly independent in $\vW$. But this
    cannot happen if $k = n$. Hence $\setV{\vv}{k}$ also spans $\vW$, with $k \leq n$, 
    and so is a basis for $\vW$. Thus
    dimension of $\vW$ is less than dimension of $\vV$.

    If $\dim{\vW} = n$, then the basis of $\vW$ is a linearly independent subset of $n$ vectors in $\vV$ and
    hence must be a basis for $\vV$. Thus $\vV \subset \vW$. Hence $\vV = \vW$.
\end{proof}
The~\ref{thm:exch_lemma} and the subsequent propositions are extremely useful. We give a few applications of them
for the case of sums and direct sums.
\begin{Proposition}
    If $\vW_{1},\vW_{2}$ are finite dimensional subspaces of vector space $\vV$, then the subspace 
    $\vW_{1} + \vW_{2}$ is finite dimensional, and
    \[\dim{(\vW_{1} + \vW_{2})} = \dim{\vW_{1}} + \dim{\vW_{2}} - \dim{(\vW_{1}\cap\vW_{2})}.\]
\end{Proposition}
\begin{proof}
    Let $\dim{V} = n$. 
    Since $\vW_{1} + \vW_{2}$ is the smallest subspace containing both $\vW_{1}$ and $\vW_{2}$, it is a
    subspace of $\vV$ and hence finite dimensional.
    Also, $(\vW_{1}\cap \vW_2)$ is the subspace contained in both $\vW_{1}$ and $\vW_{2}$ and hence it is
    finite dimensional. Let $\mcalB_{(\vW_{1}\cap \vW_2)} = \setV{\vv}{k}$ where $k \leq \dim{\vW_{1}}$ and
    $k \leq \dim{\vW_{2}}$.
    Then $\mcalB_{(\vW_{1}\cap \vW_2)}$ can be extended to a basis for $\vW_{1}$, given by,
    \[\mcalB_{\vW_{1}} = \setV{\vv}{k} \cup \setV{\vx}{p}.\]
    Similarly, 
    $\mcalB_{(\vW_{1}\cap \vW_2)}$ can be extended to a basis for $\vW_{2}$, given by,
    \[\mcalB_{\vW_{2}} = \setV{\vv}{k} \cup \setV{\vy}{q}.\]

    Claim:
    \[\setV{\vv}{k}\cup\setV{\vx}{p}\cup\setV{\vy}{q}\]
    is a basis of $\vW_{1} + \vW_{2}$.
    It is easy to see that the above spans $(\vW_{1} + \vW_{2})$. To check linear independence, 
    Let \[\lComb{\alpha}{\vv}{k} + \lComb{\beta}{\vx}{p} + \lComb{\gamma}{\vy}{q} = \zerv{\vV}.\]
    Then,
    \[\lComb{\alpha}{\vv}{k} + \lComb{\beta}{\vx}{p} = -\lComb{\gamma}{\vy}{q}. \]
    Since the vector on the left hand side is an element of $\vW_{1}$ and the vector on the right hand side is
    an element of $\vW_{2}$, they must be an element of both, i.e.~$(\vW_{1}\cap\vW_{2})$. Hence,
    \[\lComb{\gamma}{\vy}{q} = \lComb{\delta}{\vv}{k}.\]
    Thus,
    \[\lComb{(\alpha-\delta)}{\vv}{k} + \lComb{\beta}{\vx}{p} = \zerv{\vV}.\]
    These vectors are a basis for $\vW_{1}$ and so $\beta_j = 0$ for all $1 \leq j \leq p$. Thus we see that
    $\alpha_{i}$'s and $\gamma_{l}$'s are all zero since they are co-efficients for basis in $\vW_{2}$.
    Thus,
    \[\setV{\vv}{k}\cup\setV{\vx}{p}\cup\setV{\vy}{q}\]
    is a basis for $(\vW_{1} + \vW_{2})$.
    Hence,
    \begin{align*}
	\dim{(\vW_{1} + \vW_{2})} &= k + p + q \\
	& = (k + p) + (k + q) - q \\
	& = \dim{\vW_{1}} + \dim{\vW_{2}} - \dim{(\vW_{1}\cap\vW_{2})}.
    \end{align*}
\end{proof}
\begin{Remark}
    If $\vV = \vW_{1} + \vW_{2}$, then we get that,
    \[\dim{\vV} = \dim{\vW_{1}} + \dim{\vW_{2}} - \dim{(\vW_{1}\cap\vW_{2})}.\]
    If we have a direct sum, that is if $\vV = \vW_{1} \oplus \vW_{2}$, then the above formula becomes,
    \[\dim{\vV} = \dim{\vW_{1}} + \dim{\vW_{2}}.\]
\end{Remark}

\section{Homomorphic mappings between vector spaces: Linear Transformations}
Just as we did in the case of groups, where we first studied the group structure and then the functions that
preserve the structure, we now look at functions that preserve the vector space structure. Structure
preserving maps are called homomorphism. In the case of linear algebra, homomorphisms between vector spaces
are called linear transformation.

\begin{Definition}[name=Linear transformation]
    Let $\vspF{\vV}{\F},\vspF{\vW}{\F}$ be two vector spaces over the \textbf{same} field $\F$. A function,
    $\map{T}{\vV}{\vW}$ is called a linear transformation if:
    \begin{enumerate}
	\item
	    \[T(\underbrace{\vx + \vy}_{\text{op.~in $\vV$}}) = 
		\underbrace{T(\vx) + T(\vy)}_{\text{op.~in $\vW$}},\]
	\item
	    \[T(\underbrace{\beta\vx}_{\text{op.~in $\vV$}}) = 
		\underbrace{\beta T(\vx)}_{\text{op.~in $\vW$}},\]
    \end{enumerate}
    for any $\vx,\vy \in \vV$.
\end{Definition}
\begin{Remark}
    It is easily observable that $T(\zerv{\vV}) = \zerv{\vW}$ and that $T(\lComb{\beta}{\vx}{n}) =
    \lComb{\beta}{T(\vx)}{n}$.
\end{Remark}
\begin{Example}
    The following are all linear transformations.
    \begin{enumerate}
	\item
	    Let $\vx = \colvec{2}{x_1}{x_2}$ be an element of $\R^2$. Then $\map{T}{\R^2}{\R^2}$ defined by,
	    $T(\vx) = \colvec{2}{2x_1+x_2}{x_2}$ is a linear transformation.
	\item
	    Let $\vx = \colvec{2}{x_1}{x_2}$ be an element of $\R^2$. 
	    For any angle $\theta$ we define $\map{T_{\theta}}{\R^2}{\R^2}$ by, 
	    \[T(\vx) = \colvec{2}{x_1\cos{\theta} - x_2\sin{\theta}}{x_1\sin{\theta} + x_2\cos{\theta}},\]
	    is a linear transformation called the \textbf{rotation} by $\theta$.
	\item
	    Let $\vx = \colvec{2}{x_1}{x_2}$ be an element of $\R^2$.
	    Define $\map{T}{\R^2}{\R^2}$ by $T(\vx) = \colvec{2}{x_1}{-x_2}$. Then, $T$ is the linear
	    transformation called the \textbf{reflection} about the $x$ axis.
	\item
	    Let $\vx = \colvec{2}{x_1}{x_2}$ be an element of $\R^2$.
	    Define $\map{T}{\R^2}{\R^2}$ by $T(\vx) = \colvec{2}{x_1}{0}$. Then, $T$ is the linear
	    transformation called the \textbf{projection} on the $x$ axis.
    \end{enumerate}
\end{Example}
As in the case of groups, any linear transformation induces two important subspaces.
\begin{Definition}[name=Kernel]
    Let $\vV,\vW$ be vector spaces over the field $\F$ and let $\map{T}{\vV}{\vW}$ be a linear transformation. 
    We define the Kernel or \textbf{null space} as the subset $\ker{T}$ of $\vV$ as the inverse image
    $\invIm{T}{\zerv{\vW}}$.
\end{Definition}
\begin{Definition}[name=Image]
    Let $\vV,\vW$ be vector spaces over the field $\F$ and let $\map{T}{\vV}{\vW}$ be a linear transformation.
    We define the image or \textbf{range} as the subset $\Ima{T}$ of $\vW$ as the direct image
    $\dirIm{T}{\vV}$.
\end{Definition}
The following proposition is analogous to the case of groups.
\begin{Proposition}
    The kernel and image of a linear transformation $\map{T}{\vV}{\vW}$ are subspaces of $\vV$ and $\vW$
    respectively.
\end{Proposition}
The next proposition gives us a method to determine a spanning set for the image of a linear transformation.
\begin{Proposition}
    Let $\vV,\vW$ be vector spaces over the field $\F$, and let $\map{T}{\vV}{\vW}$ be a linear
    transformation. If $\mcalB_{\vV} = \setV{\vv}{n}$ is a basis for $\vV$, then
    \[\Ima{T} = \Span{\setT{\vv}{n}{T}}.\]
\end{Proposition}
\begin{proof}
    Let $\vw$ be in $\Ima{T}$. Then there is a vector $\vx \in \vV$ such that $T(\vx) = \vw$. But,
    $\vx = \lComb{\beta}{\vv}{n}$ for some scalars $\listV{\beta}{n}$ in $\F$. Thus,
    \[\vw = T(\lComb{\beta}{\vv}{n}) = \lCombT{\beta}{\vv}{n}{T},\]
    which means that $\vw$ in $\Span{\setT{\vv}{n}{T}}$. Thus, $\Ima{T} \subset \Span{\setT{\vv}{n}{T}}$.

    Let $\vw$ be in $\Span{\setT{\vv}{n}{T}}$ which means that,
    \[\vw = \lCombT{\beta}{\vv}{n}{T} = T(\lComb{\beta}{\vv}{n}).\]
    Let $\vx = \lComb{\beta}{\vv}{n}$. Then $\vx$ is a vector in $\vV$ such that $T(\vx) = \vw$ and hence
    $ \Span{\setT{\vv}{n}{T}} \subset \Ima{T}$.
\end{proof}
We measure the \emph{size} of a subspace by its dimension. The kernel and image are so important that we
attach special names to their dimensions.
\begin{Definition}[name=Nullity]
    The dimension of the kernel of a linear transformation $T$ is called its nullity and is denoted by
    nullity$(T)$.
\end{Definition}
\begin{Definition}[name=Rank]
    The dimension of the image of a linear transformation $T$ is called its rank and is denoted by rank$(T)$.
\end{Definition}
One of the most fundamental theorem of (finite dimensional) linear algebra is that of the relation between
rank and nullity. 
\begin{Theorem}[name=Dimension theorem]
    Let $\vV,\vW$ be vector spaces over the field $\F$ and let $\map{T}{\vV}{\vW}$ be a linear transformation.
    If $\vV$ is finite dimensional then,
    \[\dim{\vV} = \dim(\ker{T}) + \dim(\Ima{T}).\]
\end{Theorem}
\begin{proof}
    Since $\ker{T}$ is a subspace of $\vV$, it is finite dimensional. Let $\mcalB_{\ker{T}} = \setV{\vv}{k}$
    be a basis for $\ker{T}$. Then $\mcalB_{\ker{T}}$ can be extended to a basis for $\vV$ given by,
    \[\mcalB_{\vV} = \setV{\vv}{k} \cup \setV{\vu}{p},\]
    such that $k + p = n$.

    \textbf{Claim}: $\setT{\vu}{p}{T}$ is a basis for $\Ima{T}$.
    First note that \[\setT{\vv}{k}{T} \cup \setT{\vu}{p}{T}\] spans $\Ima{T}$. 
    Let $\vw$ be any vector in
    $\Ima{T}$. Then,
    \[\vw = \underbrace{\lCombT{\beta}{\vv}{k}{T}}_{\text{gives $\zerv{\vW}$}} + \lCombT{\gamma}{\vu}{p}{T},\]
    which means that 
    $\Ima{T} = \Span{\setT{\vu}{p}{T}}$ (we showed that $\Ima{T} \subset \setT{\vu}{p}{T}$. The other
    inclusion is similar.)
    We need to show that $\setT{\vu}{p}{T}$ is linearly independent subset of $\vW$.
    Let,
    \[\lCombT{\beta}{\vu}{p}{T} = \zerv{\vW}. \]
    This means that,
    \[\lComb{\beta}{\vu}{p} \in \ker{T},\]
    that is,
    \[\lComb{\beta}{\vu}{p} = \lComb{\alpha}{\vv}{k},\]
    which implies that,
    \[\lComb{\alpha}{\vv}{k} + \lComb{(-\beta)}{\vu}{p} = \zerv{\vV}.\]
    Since this a linear combination of basis vectors in $\vV$, each $\beta_j$ must be $0$. Thus,
    $\setT{\vu}{p}{T}$ is a basis for $\vW$. Thus, $\dim{\Ima{T}} = p$.
    Now,
    \begin{align*}
	\dim{\vV} &= k + p \\
	& = \dim{\ker{T}} + \dim{\Ima{T}}.
    \end{align*}
\end{proof}
For a linear transformation, being $1-1$ is intimately connected to being onto. The next proposition is
analogous to one for groups.
\begin{Proposition}
    A linear transformation $\map{T}{\vV}{\vW}$ is injective if and only if $\ker{T} = \setX{\zerv{\vV}}$.
\end{Proposition}
\begin{proof}
    Let $T$ be injective. Assume $\vv_{1}$ in $\ker{T}$. Then,
    $T(\vv_{1}) = \zerv{\vW}$. But since $\zerv{\vV}$ is also in $\ker{T}$, this implies $T(\vv_{1}) =
    T(\zerv{\vV})$. However, since $T$ is injective we get $\vv_{1} = \zerv{\vV}$. Thus $\ker{T} =
    \setX{\zerv{\vV}}$.

    Assume $\ker{T} = \setX{\zerv{\vV}}$. Let $\vv_{1},\vv_{2}$ be elements of $\vV$ such that $T(\vv_{1}) =
    T(\vv_{2})$. This means that $T(\vv_1 - \vv_2) = \zerv{\vW}$ and thus $(\vv_1 - \vv_2)$ is an element of
    $\ker{T}$. But only $\zerv{\vV}$ is an element of $\ker{T}$ and hence $\vv_1 = \vv_2$.

    Thus $T$ is $1-1$ if and only if $\dim{\ker{T}} = 0$.
\end{proof}
\begin{Proposition}
    Let $\vV,\vW$ be finite dimensional vector spaces over field $\F$ such that $\dim{\vV} = \dim{\vW}$. Let
    $\map{T}{\vV}{\vW}$ be a linear transformation. Then $T$ is one-to-one if and only if $T$ is onto.
\end{Proposition}
\begin{proof}
    Let $T$ be $1-1$. Then, by the above proposition, $\dim{\ker{T}} = 0$.
    To show that $T$ is onto, we need to show that $\dim{\Ima{T}} = \dim{\vW}$ since that would mean that
    $\Ima{T} = \vW$ (because $\Ima{T} \subset \vW$).
    By the dimension theorem,
    \begin{align*}
	\dim{\Ima{T}} &= \dim{\vV} - \dim{\ker{T}},\\
	&= \dim{\vV} - 0,\\
	&= \dim{\vW}.
    \end{align*}

    Let $T$ be onto. Hence $\dim{\Ima{T}} = \dim{\vW}$. By the dimension theorem,
    \begin{align*}
	\dim{\ker{T}} &= \dim{\vV} - \dim{\Ima{T}},\\
	&= \dim{\vV} - \dim{\vW},\\
	&= 0.
    \end{align*}
    Thus, $T$ is $1-1$.
\end{proof}
One of the most important properties of linear transformatios is that they are completely determined by their
action on a basis.
\begin{Proposition}
    Let $\vV,\vW$ be vector spaces over field $\F$. Let $\mcalB_{\vV} = \setV{\vv}{n}$ be a basis for $\vV$.
    If $\listV{\vw}{n}$ are vectors in $\vW$, then there is a unique linear transformation $\map{T}{\vV}{\vW}$
    such that $T(\vv_i) = \vw_{i}$ for each $i$.
\end{Proposition}
\begin{proof}
    We will first show existence. 
    Note that $\vv_i = \lComb{\delta_{i}}{\vv}{n}$ where $\delta_{ij} = 1$ if $i = j$ and $0$ if $i\neq j$.
    For any $\vu$ in $\vV$, we can write $\vu = \lComb{\alpha}{\vv}{n}$.
    Let $\map{T}{\vV}{\vW}$ be such that,
    \[T(\vu) = \lComb{\alpha}{\vw}{n}.\]
    Thus, we see that $T(\vv_i) = \vw_{i}$ for each $i$. 
    Let $\vx = \lComb{\alpha}{\vv}{n}$ and $\vy = \lComb{\beta}{\vv}{n}$. Then,
    $\vx + \vy = \lComb{(\alpha + \beta)}{\vv}{n}$.
    Thus,
    \[T(\vx + \vy) = \lComb{(\alpha + \beta)}{\vw}{n} = \lComb{\alpha}{\vw}{n} + \lComb{\beta}{\vw}{n} =
	T(\vx) + T(\vy).\]
    Similarly we can show $T(\beta\vx) = \beta T(\vx)$ for any $\beta \in \F$. Thus $T$ is a linear
    transformation.

    Let $\map{U}{\vV}{\vW}$ be another linear transformation such that $U(\vv_{i}) = \vw_{i}$ for each $i$.
    Then for any $\vx = \lComb{\alpha}{\vv}{n}$,
    \[U(\vx) = \lCombT{\alpha}{\vv}{n}{U} = \lComb{\alpha}{\vw}{n} = T(\vx).\]
\end{proof}
\section{Matrices}
In this section we will show that there is a very useful way to look at linear transformations between finite
dimensional vector spaces. In particular, we will show that such linear transformation are matrices. First we
define what an ordered basis means.
\begin{Definition}[name=Ordered Basis]
    Let $\vspF{\vV}{\F}$ be a finite dimensional vector space over the field $\F$. An ordered basis for $\vV$
    is a a basis endowed with a specific order, that is an ordered basis for $\vV$ is a finite sequence of
    linearly independent elements of $\vV$ that spans $\vV$.
\end{Definition}
\begin{Example}
    If $\mcalB_{\vV} = \setV{\vv}{n}$ is a basis for $\vV$, then there are $n$! \emph{different} ordered bases
    of $\vV$ from the given unordered basis $\mcalB_{\vV}$. We denote an ordered basis as
    $\left(\listV{\vv}{n}\right)$. Thus $\left(\vv_2,\vv_3,\ldots,\vv_{n},\vv_{1}\right)$ is an ordered basis
    for $\vV$ where the \textbf{first} basis is $\vv_2$. 
\end{Example}
\begin{Example}
    For any field $\F$, $\Fn$ is a vector space of $\F$. What is a basis for $\Fn$? Note that any $\alpha$ in
    $\F$ can be written as $\alpha\cdot 1$, where $1$ is the identity element of $\F$. Thus any
    $\vect{\alpha} = \colvec{4}{\alpha_{1}}{\alpha_{2}}{\vdots}{\alpha_{n}}$ can be written as,
    $\vect{\alpha} = \alpha_{1}\colvec{4}{1}{0}{\vdots}{0} + \cdots +
    \alpha_{n}\colvec{4}{0}{0}{\vdots}{1}$. The \textbf{vectors} $\ve_{i} = \colvec{4}{0}{\vdots}{1}{0}$ with $1$
    in the $i^{th}$ slots are called \textbf{standard} basis vectors for $\Fn$. Each field $\Fn$ is equipped
    with its standard basis vectors. Thus $\mcalB_{\Fn} = \setV{\ve}{n}$ is the standard unordered basis for
    $\Fn$. This is an important observation. To reiterate, any $\vx \in \Fn$ is represented by a column vector
    \[\vx = \colvec{4}{\alpha_{1}}{\alpha_{2}}{\vdots}{\alpha_{n}},\]
    if and only if
    \[\vx = \lComb{\alpha}{\ve}{n}.\]
\end{Example}
Let us explore the structure of linear transformation between fields (as vector spaces) in the following
examples. For any $\Fn$ we
write $\left(\listV{\ve}{n}\right)$ as the standard ordered basis for $\vspF{\Fn}{\F}$.
\begin{Example}
    Let $\map{T}{\F}{\F}$ be a linear transformation. How can $T$ be described? Any $\vect{\alpha} \in \F$ 
    can be written as $\alpha\cdot\vect{1}$.
    Thus $T(\vect{\alpha}) = \alpha T(\vect{1})$. Note that $\vect{1} =\ve_{1}$ is the standard basis for $\F$.
    Thus $T$ can be described as $T(\vx) = ax$ for any $\vx \in \F$ where $a = T(\ve_1)$.
\end{Example}
\begin{Example}
    Let $\map{T}{\F^2}{\F}$ be a linear transformation. How can $T$ be described? Any $\vx =
    \colvec{2}{x_1}{x_2}$ can be written as $\vx = x_1\ve_{1} + x_2\ve_{2}$. Thus $T$ can be 
    described as $T(\vx) = a_{11}x_1 + a_{12}x_2$, where $a_{11} = T(\ve_{1})$ and $a_{12} = T(\ve_{2})$.
\end{Example}
\begin{Example}
    We might be noticing a pattern. Let $\map{T}{\Fn}{\F}$ be a linear transformation. How can $T$ be
    described? Any $\vx = \colvec{5}{x_1}{x_2}{x_3}{\vdots}{x_n} \in \Fn$ can be written as 
    \[\vx = x_1\ve_1 + x_2\ve_2 + x_3\ve_3 + \cdots + x_n\ve_n.\]
    Thus,
    \[T(\vx) = a_{11}x_1 + a_{12}x_2 + \cdots + a_{1n}x_n,\]
    where $a_{1j} = T(\ve_{j})$.
\end{Example}
\begin{Example}
    Let $\map{T}{\F^2}{\F^2}$ be linear transformation. How can $T$ be described? We can think of $T$ as
    $\colvec{2}{T_1}{T_2}$ where $\map{T_i}{\F^2}{F}$ is a linear transformation. We have seen how these behave,
    that is for any $\vx = \colvec{2}{x_1}{x_2}$, 
    \[T_i(\vx) = a_{i1}x_1 + a_{i2}x_2.\]
    Thus, 
    \[T(\vx) = \colvec{2}{T_1(\vx)}{T_2(\vx)} = \colvec{2}{a_{11}x_1 + a_{12}x_2}{a_{21}x_1 + a_{22}{x_2}}.\]
\end{Example}
\begin{Example}\label{ex:LT_Mat_Fields}
    Let $\map{T}{\Fn}{\F^m}$ be linear transformation. How can $T$ be described? As seen above, we 
    can think of $T$ as $\colvec{4}{T_1}{T_2}{\vdots}{T_m}$ where each $\map{T_i}{\Fn}{F}$ is a linear
    transformation for $1 \leq i \leq m$. Let $\vx =  \colvec{3}{x_1}{\vdots}{x_n}$. For any $i$, 
    \[T_i(\vx) = a_{i1}x_1 + \cdots + a_{in}x_n.\]
    Thus,
    \[T(\vx) = \colvec{4}{T_1(\vx)}{T_2(\vx)}{\vdots}{T_m(\vx)} = 
	\colvec{4}{a_{11}x_1 + \cdots + a_{1n}x_n}{a_{21}x_1 + \cdots + a_{2n}x_n}{\vdots}{a_{m1}x_1 + \cdots +
	    a_{mn}{x_n}}.\]
\end{Example}
Now we are ready to describe a matrix.
\begin{Definition}[name=Matrix]
    A matrix $M$ is a table of size $m\times n$ where $m$ is the number of rows and $n$ is the number of columns.
    The elements of this table are elements of a field $\F$. We denote by $M[:,j]$, the $j^{th}$ column of the
    matrix $M$ and by $M[i,:]$ the $i^{th}$ row of the matrix. The set of all matrices of size $m\times n$
    with elements from field $\F$ is denoted by $\matmn{m}{n}$.

    By the notation ${[a_{ij}]}_{m\times n}$ we mean a matrix of size $m\times n$ whose elements are given by
    $a_{ij}$ for $1\leq i \leq m$ and $1 \leq j \leq n$.
\end{Definition}
Looking at our examples above, we see that any linear transformation $\map{T}{\Fn}{\F^{m}}$ can be described
by a matrix. For example the matrix of $T$ as given in~\ref{ex:LT_Mat_Fields} is given by,
\begin{equation*}
    M_{T} = 
    \begin{pmatrix} 
	a_{11}&a_{12}&\cdots&a_{1n}\\
	a_{21}&a_{22}&\cdots&a_{2n}\\
	\hdotsfor[2]{4}\\
	a_{m1}&a_{m2}&\cdots&a_{mn}
    \end{pmatrix}
\end{equation*}

To see the precise relation, let us consider the action of $T$ on a standard basis vector 
$\ve_{j}$. From~\ref{ex:LT_Mat_Fields}, we see that,

\[T_i(\ve_{j}) = a_{i1}0 + \cdots + a_{ij}1 + \cdots + a_{in}0 = a_{ij}.\]

Thus,
\[T(\ve_{j}) = \colvec{4}{T_1(\ve_{j})}{T_2(\ve_{j})}{\vdots}{T_m(\ve_{j})} = 
    \colvec{4}{a_{1j}}{a_{2j}}{\vdots}{a_{mj}},\]

which is the $j^{th}$ column of the matrix $M_{T}$. 
\begin{Definition}
    Given a linear transformation $\map{T}{\Fn}{\Fm}$, the matrix of $T$, denoted by $M_{T}$ is a table of
    size $m\times n$, with elements from field $\F$ such that,
    \[M_{T}[:,j] = T(\ve_{j}),\]
    where $\ve_{j}$ is the $j^{th}$ standard basis of the oderered basis,
    $\left(\setX{e_1,e_2,\ldots,e_n}\right)$ for $\Fn$.
\end{Definition}
We can impart a group structure on the set of matrices.
\begin{Proposition}
    Let $\F$ be a field. The set $\matmn{m}{n}$ is an additive Abelian group, where addition ($+$) is given
    by,
    \[{[a_{ij}]}_{m\times n} + {[b_{ij}]}_{m\times n} = {[a_{ij} + b_{ij}]}_{m\times n}, \]
    with the identity element ${[0]}_{m\times n}$ consiting of $0$ in each row and column.
\end{Proposition}
Thus, we can regard matrices as a separate entity without any relation to the underlying linear
transformation. However, it would be remiss to not discuss this deep connection.
\begin{Proposition}
    Let $\vV,\vW$ be vector spaces over the field $\F$. We denote by $\homv{\vV}{\vW}$ as
    the set of all linear transformations from $\vV$ to $\vW$, that is,
    \[\homv{\vV}{\vW} = \set{\map{T}{\vV}{\vW}}{\text{$T$ is a linear transformation}}.\]
    Then, $\groupS{\homv{\vV}{\vW}}{+}{\zerv{\homv{\vV}{\vW}}}$ is a vector space, where 
    \[(T_1 + T_2)(\vv) := T_1(\vv) + T_2(\vv),\]
    for any $T_1,T_2 \in \homv{\vV}{\vW}$ and any $\vv \in \vV$; and
    \[(\beta T_1)(\vv) := \beta(T_1(\vv)), \]
    for any $\beta \in \F$.
\end{Proposition}
We will not prove the proposition.

An immediate consequence of this proposition is that $\homv{\Fn}{\F^{m}}$ is a vector space.
In fact the group $\matmn{m}{n}$ can be given a vector space structure.

\begin{Proposition}
    $\matmn{m}{n}$ is a vector space over the field $\F$, with addition defined by,
    \[{[a_{ij}]}_{m\times n} + {[b_{ij}]}_{m\times n} = {[a_{ij} + b_{ij}]}_{m\times n}, \]
    and scalar multiplication by,
    \[\beta{[a_{ij}]}_{m\times n} = {[\beta a_{ij}]}_{m\times n}.\]
\end{Proposition}

What is the relation between $\homv{\Fn}{\F^{m}}$ and $\matmn{m}{n}$? Intiuition suggests that these two are
equivalent notions. 

Given two linear transformations $\map{T_1}{\Fn}{\F^m}$ and $\map{T_2}{\Fn}{\F^m}$, what is the 
matrix of $T_1 + T_2$?
From the definition,
\[M_{(T_1+T_2)}[:,j] = (T_1 + T_2)(\ve_{j}) = T_1(\ve_j) + T_2(\ve_j) = M_{T_1} + M_{T_2}.\]
In other words, if $M_{T_1} = {[a_{ij}]}_{m\times n}$ and $M_{T_2} = {[b_{ij}]}_{m\times n}$, then
$M_{(T_1+T_2)} = {[a_{ij} + b_{ij}]}_{m\times n}$. Similarly,
we can easily see that $M_{(\beta T)}[:,j] = (\beta T)(\ve_{j}) = \beta T(\ve_{j})$. Thus if $M_{T} =
{[a_{ij}]}_{m\times n}$, then $M_{(\beta T)} = {[\beta a_{ij}]}_{m\times n}$.  

Thus addition of two linear transformations amounts to addition of their matrices and similarly for scalar
multiplication. These two vector spaces seem to have the same algebraic structure. To make precise this notion
we need to show that they are isomorphic.

Recall that two algebraic structure are equivalent if they are isomorphic. 
In the context
of vector space, we define isomorphism as,
\begin{Definition}
    Two vector spaces $\vV$ and $\vW$ are isomorphic if there is a bijective linear transformation mapping
    $\vV$ onto $\vW$.
\end{Definition}
Before we show that $\homv{\Fn}{\F^{m}}$ and $\matmn{m}{n}$ are isomorphic, we will need to understand what it
means to mulitply a matrix with a vector. 

Note that if $T \in \homv{\Fn}{\F^{m}}$, then $T(\vx) = T(\lComb{\alpha}{\ve}{n}) =
\lCombT{\alpha}{\ve}{n}{T}$.
Thus,
\[T(\vx) = \finiteSum{\alpha_{i}M_{T}[:,i]}{i}{n}.\]

This gives us a way to define a product of a matrix with a vector.
\begin{Definition}
Let $A$ be any $m\times n$ matrix. 
For any $\vx \in \Fn$ such that \[\vx = \colvec{4}{\alpha_{1}}{\alpha_{2}}{\vdots}{\alpha_{n}},\] 
let us define \[A\vx := \finiteSum{\alpha_{i}M[:,i]}{i}{n},\]
as the \textbf{matrix-vector} product.
\end{Definition}

\begin{Proposition}
    Let $A$ be a $m\times n$ matrix and let \[\vx = \colvec{4}{\alpha_{1}}{\alpha_{2}}{\vdots}{\alpha_{n}},\]
    be a vector in $\Fn$.
    Then $L_{A}(\vx):= A\vx$ is a linear transformation from $\Fn$ to $\F^{m}$ and is called the linear
    transformation induced by $A$. 
\end{Proposition}
\begin{proof}
    Let $\vx = \lComb{\alpha}{\ve}{n}$ and $\vy = \lComb{\beta}{\ve}{n}$ be two vectors in $\Fn$.
    Then, $\vx + \vy = \finiteSum{(\alpha_{i} + \beta_{i})\ve_{i}}{i}{n}$ and thus,
    \[(\vx + \vy) = \colvec{4}{(\alpha_{1}+\beta_{1})}{(\alpha_{2}+\beta_{2})}{\vdots}{(\alpha_{n}+\beta_{n})}\]
    \begin{align*}
	L_{A}(\vx + \vy) &= A(\vx + \vy)\\
	& = \finiteSum{(\alpha_{j} + \beta_{j})A[:,j]}{j}{n} \\
	& = \finiteSum{\alpha_{j}A[:,j]}{j}{n} + \finiteSum{\beta_{j}A[:,j]}{j}{n}\\
	& = A\vx + A\vy \\
	& = L_{A}(\vx) + L_{A}(\vy),
    \end{align*}
    where the $3^{rd}$ equalilty is due to the fact that $A[:,j]$'s are vectors in $\F^m$.
    Similarly we can show scalar multiplication.
\end{proof}
\begin{Theorem}
    The vector spaces $\homv{\Fn}{\F^m}$ and $\matmn{m}{n}$ are isomorphic. Hence, any linear transformation
    from $\Fn$ to $\Fm$ is equivalent to a matrix of size $m\times n$.
\end{Theorem}
\begin{proof}
    WE need to find a linear transformation $\map{U}{\homv{\Fn}{\Fm}}{\matmn{m}{n}}$ such that $U$ is
    bijective.
    Let us define $U$ to be,
    \[U(T) = M_{T}.\]
    Clearly $U$ is a linear transformation. To show that $U$ is bijective we must find a linear transformation
    $\map{S}{\matmn{m}{n}}{\homv{\Fn}{\Fm}}$ such that,
    \begin{enumerate}
	\item
	    $(\fog{S}{U})(T) = T$, for any $T$ in $\homv{\Fn}{\Fm}$ and,
	\item
	    $(\fog{U}{S})(A) = A$, for any $A$ in $\matmn{m}{n}$. 
    \end{enumerate}
    Let $S(A) := L_{A}$.
    We will show the first one since the second is analogous.
    \begin{align*}
	(\fog{S}{U})(T) &= S(U(T)) \\
	&= S(M_{T})\\
	&= L_{M_{T}}
    \end{align*}
    To show that $T = L_{M_{T}}$, we need to just check if they agree on the basis vectors $(\listV{\ve}{n})$.
    For any $1\leq i \leq n$,
    \begin{align*}
	L_{M_{T}}(\ve_{i}) &= M_{T}\ve_{i}\\
	&= \finiteSum{\delta_{ij}M_{T}[:,j]}{j}{n}\\
	&= M_{T}[:,i]\\
	&= T(\ve_{i}).
    \end{align*}
    Here $\delta_{ij} = 1$ if $i =j$ and $0$ otherwise.
\end{proof}

Let us now look at the matrix of a composition. Let $\map{T_1}{\Fn}{\F^p}$ and $\map{T_2}{\F^p}{\F^m}$. Then
the composite $\map{\fog{T_2}{T_1}}{\Fn}{\Fm}$ is easily seen to be a linear transformation. Note that
$\fog{T_1}{T_2}$ may fail to be a function. What is the matrix of $\fog{T_2}{T_1}$?
Let $\left(\listV{\ve^{n}}{n}\right)$ be the standard ordered basis of $\Fn$,  
$\left(\listV{\ve^{p}}{p}\right)$ be the standard ordered basis for $\F^{p}$
and $\left(\listV{\ve^{m}}{m}\right)$ be the standard ordered basis for $\F^{m}$.
Let $M_{T_1} = {[a_{ij}]}_{p\times n}$ and let $M_{T_2} = {[b_{ij}]}_{m\times p}$.

By definition,
$M_{\fog{T_2}{T_1}}[:,j] = (\fog{T_2}{T_1})(\ve^{n}_{j})$.
Thus,
\begin{align*}
    M_{\fog{T_2}{T_1}}[:,j] &= (\fog{T_2}{T_1})(\ve^{n}_{j}). \\
    &= T_2(T_1(\ve^{n}_{j}))\\
    &= T_2(M_{T_1}[:,j])\\
    &= T_2(\colvec{4}{a_{1j}}{a_{2j}}{\vdots}{a_{pj}})\\
    &= T_2(\finiteSum{a_{kj}\ve^{p}_k}{k}{p})\\
    &= \finiteSum{a_{kj}T_2(\ve^{p}_k)}{k}{p}\\
    &= \finiteSum{a_{kj}M_{T_2}[:,k]}{k}{p}\\
    &= \finiteSum{a_{kj}\colvec{3}{b_{1k}}{\vdots}{b_{mk}}}{k}{p}\\
    &= \finiteSum{a_{kj}(\finiteSum{b_{ik}\ve^{m}_{i}}{i}{m})}{k}{p}\\
    &= \finiteSum{(\finiteSum{b_{ik}a_{kj}}{k}{p})\ve^{m}_{i}}{i}{m}.
\end{align*}
Hence, if $M_{\fog{T_2}{T_1}} = {[c_{ij}]}_{m\times n}$, then,
\[c_{ij} = \finiteSum{b_{ik}a_{kj}}{k}{p}.\]
We can use this to define a matrix multiplication operation.
\begin{Definition}
    Let $A = {[a_{ij}]}_{m\times p}$ be a matrix of size $m\times p$ and let $B = {[b_{ij}]}_{p\times n}$, then
    we define a matrix $C = {[c_{ij}]}_{m\times n}$ of size $m\times n$ as the \textbf{product} of the matrix
    $A$ and $B$ denoted by $C = AB$ whose elements are given by,
    \[c_{ij} = \finiteSum{a_{ik}b_{kj}}{k}{p}.\]
\end{Definition}
Now that we have seen that matrices can be identified with linear transformations, we will abstract this
notion to linear transformations between general finite dimensional vector spaces. The crucial idea that will
help us in this regard is the notion of a co-ordinate. A co-ordinate map makes a finite dimensional vector
space, of dimension $n$, look like $\Fn$.

\begin{Definition}[name=Co-ordinate]
    Let $\vspF{\vV}{F}$ be a finite dimensional vector space over $\F$ and let $\mcalB_{\vV} =
    \left(\listV{\vv}{n}\right)$ be an ordered basis for $\vV$. For any $\vx \in \vV$, we can write $\vx$
    uniquely as,
    \[\vx =  \lComb{\beta}{\vv}{n}.\]
    The co-ordinate of $\vx$ w.r.t~$\mcalB_{\vv}$, denoted by $\coord{\vx}{\mcalB_{\vv}}$, 
    is an element of $\Fn$ given by,
    \[\coord{\vx}{\mcalB_{\vv}} = \colvec{4}{\beta_{1}}{\beta_{2}}{\vdots}{\beta_{n}}.\]
\end{Definition}

\begin{Definition}[name=Co-ordinate map]
    Let $\vspF{\vV}{F}$ be a finite dimensional vector space over $\F$ and let $\mcalB_{\vV} =
    \left(\listV{\vv}{n}\right)$ be an ordered basis for $\vV$. For any $\vx \in \vV$, we can write $\vx$
    uniquely as,
    \[\vx =  \lComb{\beta}{\vv}{n}.\]
    The co-ordinate map $\map{\phi_{\mcalB_{\vV}}}{\vV}{\Fn}$, is given by:
    \[\varphi_{\mcalB_{\vV}}(\vx) = \coord{\vx}{\mcalB_{\vv}} = 
	\colvec{4}{\beta_{1}}{\beta_{2}}{\vdots}{\beta_{n}}.\]
\end{Definition}
\begin{Remark}
    Easy to see that a co-ordinate map is a linear transformation from $\vV$ to $\Fn$.
\end{Remark}
\begin{Remark}
    Note that for any $\vv_{j}$ in $\mcalB_{\vV}$, $\phi_{\mcalB_{\vv}}(\vv_{j}) =
    \coord{\vv_{j}}{\mcalB_{\vv}} = \ve_{j}$
    where $\ve_{j}$ is the $j^{th}$ standard basis for $\Fn$.
\end{Remark}

\begin{Example}
    Let $\vV = P_2(\R)$ be the vector space of polynomials in $\R$ of degree less than or
    equal to $2$. It is easy to check that $\mcalB_{\vV} = \setX{1,x,x^2}$
    is a basis for $\vV$. Let $f \in \vV$ be given by $4 + 6x - 7x^2$, then $f = 4(1) + 6(x) + (-7)x^2$ and
    hence $\coord{f}{\mcalB_{\vv}} = \colvec{3}{4}{6}{-7}$.
\end{Example}

We said that an ordered basis induces a co-ordinate map that makes a $n$ dimensional vector space look like
$\Fn$. We make this precise.
\begin{Proposition}
    Let $\vV$ be a finite dimensional vector space with dimension $n$.  Then $\vV$ is isomorphic to $\Fn$.
\end{Proposition}
\begin{proof}
    Let $\mcalB_{\vV} = \left(\listV{\vv}{n}\right)$ be an ordered basis for $\vV$.
    We will show that the co-ordinate map $\varphi_{\mcalB{\vV}}$ is an isomorphism from $\vV$ onto $\Fn$.
    First, observe that $\varphi_{\mcalB_{\vV}}$ is a linear transformation.
    Define $\map{\Phi_{\mcalB_{\vV}}}{\Fn}{\vV}$ as follows:
    \[\Phi_{\mcalB_{\vV}}(\colvec{4}{\beta_{1}}{\beta_{2}}{\vdots}{\beta_{n}}) = \lComb{\beta}{\vv}{n}. \]
    Clearly $\Phi_{\mcalB_{\vV}}$ is a linear transformation. It is easy to see that:
    \begin{enumerate}
	\item
	    $\left(\fog{\varphi_{\mcalB_{\vV}}}{\Phi_{\mcalB_{\vV}}}\right)
	    \left(\colvec{3}{a_1}{\vdots}{a_n}\right) = \colvec{3}{a_1}{\vdots}{a_n}$ 
	    for any $a_i \in \F$ and,
	\item
	    $(\fog{\Phi_{\mcalB_{\vV}}}{\varphi_{\mcalB_{\vV}}})(\vu) = \vu$ for any $\vu \in \vV$.
    \end{enumerate}
    Thus, $\Phi^{}_{\mcalB_{\vV}} = \varphi^{-1}_{\mcalB_{\vV}}$.
\end{proof}
The above Proposition is extremely useful. We will now see that given any linear transformation between abstract
finite dimensional vector spaces of dimension $n,m$ we can find the \textbf{equivalent} linear transformation
between $\Fn,\Fm$.
\begin{Definition}[name = Induced linear transformation]
    Let $\vV,\vW$ be finite dimensional vector spaces of dimension $n$ and $m$ respectively. Let 
    $\mcalB_{\vV} = \left(\listV{\vv}{n}\right)$ and $\mcalB_{\vW} = \left(\listV{\vw}{m}\right)$ 
    be ordered basis for $\vV$ and $\vW$ respectively.
    If $\map{T}{\vV}{\vW}$ is any linear transformation, we call $\map{\check{T}}{\Fn}{\Fm}$ the induced
    linear transformation of $T$ w.r.t.~bases $\mcalB_{\vV}$ and $\mcalB_{\vW}$ given by:
    \[\check{T} = \fog{\fog{\cMap{\mcalB_{\vW}}}{T}}{\cMapInv{\mcalB_{\vV}}}.\]
\end{Definition}
This makes it easier to define the matrix of a linear transformation between abstract finite dimensional
vector spaces.

%%%%%% Fig: Commutative diagram %%%%%

\begin{Definition}[name=Matrix]
    Let $\vV,\vW$ be finite dimensional vector spaces of dimension $n$ and $m$ respectively. Let 
    $\mcalB_{\vV} = \left(\listV{\vv}{n}\right)$ and $\mcalB_{\vW} = \left(\listV{\vw}{m}\right)$ 
    be ordered basis for $\vV$ and $\vW$ respectively.
    If $\map{T}{\vV}{\vW}$ is any linear transformation, we denote by $\matVW{T}{\mcalB_{\vV}}{\mcalB_{\vW}}$
    the matrix of $T$, given by,
    \[\matVW{T}{\mcalB_{\vV}}{\mcalB_{\vW}} := M_{\check{T}}.\]
\end{Definition}
Let $A = {[a_{ij}]}_{m\times n}$ be the matrix of $T$ i.e.~$A = \matVW{T}{\mcalB_{\vV}}{\mcalB_{\vW}}$. What
is the $j^{th}$ column of $A$, $A[:,j]$? By definition, $A[:,j] = M_{\check{T}}[:,j] =
\check{T}(\ve^{n}_{j})$, where $\mcalB_{\Fn} = \left(\listV{\ve^{n}}{n}\right)$ is the standard ordered basis
for $\Fn$.
Thus,
\begin{align*}
    A[:,j] &= \check{T}(\ve^{n}_{j})\\
    &= (\fog{\fog{\cMap{\mcalB_{\vW}}}{T}}{\cMapInv{\mcalB_{\vV}}})(\ve^{n}_{j}) \\
    &= \cMap{\mcalB_{\vW}}\left(T\left(\cMapInv{\mcalB_{\vV}}(\ve^{n}_{j})\right)\right)\\
    &= \cMap{\mcalB_{\vW}}\left(T(\vv_{j})\right) \\
    &= \coord{T(\vv_{j})}{\mcalB_{\vW}}
\end{align*}
Thus, $T(\vv_{j}) = \series{a_{ij}\vw_{i}}{i}{1}{n}$.

Let $\vu$ be a vector in $\vV$ such that $\vu = \lComb{\alpha}{\vv}{n}$. 
What is $T(\vu)$ in terms of its matrix 
$A = {[a_{ij}]}_{m\times n} = \matVW{T}{\mcalB_{\vV}}{\mcalB_{\vW}}$? Note that,
\[T(\vu) = \series{\beta_{i}\vw_{i}}{i}{1}{m}.\]
By definition,
\[\fog{\cMap{\mcalB_{\vW}}}{T} = \fog{\check{T}}{\cMap{\mcalB_{\vW}}}.\]
Thus,
\begin{align*}
    (\fog{\cMap{\mcalB_{\vW}}}{T})(\vu) & = \check{T}\left(\cMap{\mcalB_{\vV}}\left(\vu\right)\right)\\
    & = \check{T}\left(\coord{\vu}{\mcalB_{\vV}}\right)\\
    & = M_{\check{T}}\coord{\vu}{\mcalB_{\vV}}\\
    & = A\coord{\vu}{\mcalB_{\vV}}.
\end{align*}
Hence if $T(\vu) = \series{\beta_{i}\vw_{i}}{i}{1}{m}$, then each $\beta_{i}$ is given by,
\[\beta_i = \series{\alpha_{l}a_{il}}{l}{1}{n}.\]
To re-iterate if $T\in \homv{\vV}{\vW}$ where $\dim{\vV} = n$ and $\dim{\vW} = m$, and if $A$ is the matrix of
$T$, i.e.~$A = \matVW{T}{\mcalB_{\vV}}{\mcalB_{\vW}}$, then for any vector $\vu \in \vV$,
\[\coord{T(\vu)}{\mcalB_{\vW}} = A\coord{\vu}{\mcalB_{\vV}}.\]

Note that for any $\vx \in Fn$, $\coord{\vx}{\mcalB_{\Fn}} = \vx$, when $\mcalB_{\Fn}$ is the standard ordered
basis for $\Fn$. Now let us generalize the definition of a linear transformation induced by a matrix.

\begin{Definition}
    Let $A = {[a_{ij}]}_{m\times n}$ be a matrix of size $m\times n$. The linear transformation
    $L_{A}$ induced by $A$ is given as,
    \[L_{A}(\vx) := A\coord{\vx}{\mcalB_{\vv}},\]
    for any $\vx$ in $\Fn$ and any ordered basis $\mcalB_{\vv}$ of $\Fn$.
\end{Definition}
The next theorem states the equivalency of the vector space of matrices and linear transformation in finite
dimensional vector spaces. The proof is similar to the particular case of mappings between $\Fn$ and $\Fm$.
\begin{Theorem}
    Let $\vV,\vW$ be finite dimensional vector spaces (over a field $\F$) 
    of dimensions $n,m$ respectively. Then $\homv{\vV}{\vW}$
    and $\matmn{m}{n}$ are isomorphic.
\end{Theorem}
Next, we define the matrix of addition of two linear transformation and composition of two linear
transformations.
\begin{Theorem}
    Let $\vV,\vW$ be finite dimensional vector spaces over the field $\F$ and with dimensions $n,m$
    respectively. Let $\mcalB_{\vV}$ and $\mcalB_{\vW}$ be ordered bases for $\vV$ and $\vW$ respectively. Let
    $\map{T,U}{\vV}{\vW}$ be a linear transformation. Then
    \begin{enumerate}
	\item
	    \[\matVW{T + U}{\mcalB_{\vV}}{\mcalB_{\vW}} = 
		\matVW{T}{\mcalB_{\vV}}{\mcalB_{\vW}} + \matVW{U}{\mcalB_{\vV}}{\mcalB_{\vW}}.\] 
	\item
	    \[\matVW{\beta T}{\mcalB_{\vV}}{\mcalB_{\vW}} = \beta\matVW{T}{\mcalB_{\vV}}{\mcalB_{\vW}}.\]
    \end{enumerate}
\end{Theorem}
\begin{proof}
    We will show for $(1)$.
    Let $A = \matVW{T}{\mcalB_{\vV}}{\mcalB_{\vW}}$, $B = \matVW{U}{\mcalB_{\vV}}{\mcalB_{\vW}}$ and let 
    $C = \matVW{T + U}{\mcalB_{\vV}}{\mcalB_{\vW}}$.
    Let $\mcalB_{\vV} = \left(\listV{\vv}{n}\right)$. Then,
    \begin{align*}
	C[:,j] &= \cMap{\mcalB_{\vW}}{(T+U)(\vv_{j})}\\
	& = \cMap{\mcalB_{\vW}}{\left(T(\vv_{j}) + U(\vv_{j})\right)}\\
	& = \cMap{\mcalB_{\vW}}{T(\vv_j)} + \cMap{\mcalB_{\vW}}{U(\vv_j)}\\
	& = A[:,j] + B[:,j].
    \end{align*}
\end{proof}
\begin{Theorem}
    Let $\vV,\vW,\vZ$ be finite dimensional vector spaces over field $\F$ and of dimensions $n,p,m$
    respectively. Let $\mcalB_{\vV},\mcalB_{\vW}$ and $\mcalB_{\vZ}$ be ordered bases for $\vV,\vW$ and $\vZ$
    respectively. If $\map{T}{\vV}{\vW}$ and $\map{U}{\vW}{\vZ}$ are linear transformations, then
    \[\matVW{(\fog{U}{T})}{\mcalB_{\vV}}{\mcalB_{\vZ}} = 
	\matVW{U}{\mcalB_{\vW}}{\mcalB_{\vZ}}\matVW{T}{\mcalB_{\vV}}{\mcalB_{\vW}}. \]
\end{Theorem}
\begin{proof}
    Let $A = \matVW{T}{\mcalB_{\vV}}{\mcalB_{\vW}}$, $B = \matVW{U}{\mcalB_{\vW}}{\mcalB_{\vZ}}$ and let 
    $C = \matVW{(\fog{U}{T})}{\mcalB_{\vV}}{\mcalB_{\vZ}}$.
    Let $\mcalB_{\vV} = \left(\listV{\vv}{n}\right)$. 
    Let $\mcalB_{\vW} = \left(\listV{\vw}{p}\right)$. 
    Let $\mcalB_{\vZ} = \left(\listV{\vz}{m}\right)$. 
    Let us denote by ${\left[\cMap{\mcalB_{\vZ}}{(\cdot)}\right]}_{i}$ as the $i^{th}$ 
    co-ordinate (slot) of the co-ordinate
    map $\varphi^{}_{\mcalB_{\vZ}}$.

    Thus,
    \begin{align*}
	C[i,j] &= {\left[\cMap{\mcalB_{\vZ}}{(\fog{U}{T})(\vv_{j})}\right]}_{i}\\
	& = {\left[\cMap{\mcalB_{\vZ}}{U\left(T(\vv_{j})\right)}\right]}_{i}\\
	& = {\left[\cMap{\mcalB_{\vZ}}{U\left(\series{a_{lj}\vw_{l}}{l}{1}{p}\right)}\right]}_{i}\\
	& = {\left[\cMap{\mcalB_{\vZ}}{\left(\series{a_{lj}U(\vw_{l})}{l}{1}{p}\right)}\right]}_{i}\\
	& = {\left[\cMap{\mcalB_{\vZ}}{\left(\series{a_{lj}}{l}{1}{p}\series{b_{il}\vz_{i}}{i}{1}{m}\right)}\right]}_{i}\\
	& = {\left[\cMap{\mcalB_{\vZ}}{\left(\series{\left(\series{b_{il}a_{lj}}{l}{1}{p}\right)\vz_{i}}{i}{1}{m}\right)}\right]}_{i}\\
	& = \series{b_{il}a_{lj}}{l}{1}{p},\\
    \end{align*}
    which by definition means that $C = BA$.
\end{proof}
For any function $f$, it is important to ask if there is a $y$ such that $f(x) = y$ is meaningful. If $f$ were
invertible, then this would mean that $x = f^{-1}\circ f = f^{-1}(y)$. We will see shortly that when $f$ is a
linear transformation, such a question amounts to a solution of a system of linear equations. We have already
seen invertible transformations when dealing with isomorphisms. We make this notion precise and discuss a few
consequences in the following paragraphs. 

\begin{Definition}
    Let $\vV,\vW$ be vector spaces and let $\map{T}{\vV}{\vW}$ be a linear transformation. Let $I_{\vV}$ and
    $I_{\vW}$ be the identity linear transformations on $\vV$ and $\vW$ respectively. A function,
    $\map{U}{\vW}{\vV}$ is said to be an inverse of $T$ if,
    \begin{enumerate}
	\item
	    $\fog{U}{T} = I_{\vV}$ and,
	\item
	    $\fog{T}{U} = I_{\vW}$.
    \end{enumerate}
\end{Definition}
\begin{Remark}
    The definition for invertibility of $T$ described above is equivalent to stating that $T$ is one-one and
    onto. If $U$ is the inverse of $T$, then it is unique and we denote $U$ by $T^{-1}$.
\end{Remark}
\begin{Remark}
    If $T,U$ are invertible then ${(\fog{U}{T})}^{-1} = \fog{T^{-1}}{U^{-1}}$. Morevore, ${(T^{-1})}^{-1} =
    T$.
\end{Remark}
\begin{Theorem}
    Let $\vV,\vW$ be vector spaces and let $\map{T}{\vV}{\vW}$ be an invertible linear transformation. Then
    $T^{-1}$ is also a linear transformation.
\end{Theorem}
\begin{proof}
    First we will show that for any $\vw_1,\vw_2 \in \vW$, 
    \[T^{-1}(\vw_1 + \vw_2) = T^{-1}(\vw_1) + T^{-1}(\vw_2).\]
    Let $\vu_1 \in \vV $ be equal to $T^{-1}(\vw_1 + \vw_2)$ and let $\vu_2 \in \vV$ be equal to 
    $T^{-1}(\vw_1) + T^{-1}(\vw_2)$. Then,
    \[T(\vu_1) = \fog{T}{T^{-1}}(\vw_1 + \vw_2) = \vw_1 + \vw_2.\]
    And,
    \[T(\vu_2) = T(T^{-1}(\vw_1) + T^{-1}(\vw_2)) = T(T^{-1}(\vw_1)) + T(T^{-1}(\vw_2)) = \vw_1 + \vw_2.\]
    Since, $T$ is injective $\vu_1 = \vu_2$. Hence, 
    \[T^{-1}(\vw_1 + \vw_2) = T^{-1}(\vw_1) + T^{-1}(\vw_2).\]
    Similary, we can show scalar multiplication.
\end{proof}
Similarly, we can define invertible matrices.
\begin{Definition}
    The $n\times n$ identity matrix, denoted by $I_n$, is defined by \[I_n = {[\delta_{ij}]}_{n\times n},\]
    where $\delta_{ij} = 1$ if $i = j$ and $0$ otherwise.
\end{Definition}
\begin{Remark}
    It is easy to see that if $\vV$ is a vector space of dimension $n$ and if $\mcalB_{\vV}$ is an ordered
    basis for $\vV$, then $\matVW{I}{\mcalB_{\vV}}{\mcalB_{\vV}} = I_n$.
\end{Remark}
\begin{Definition}
    Let $A \in \matmn{m}{n}$. Then $A$ is invertible if there is a matrix $B \in \matmn{m}{n}$ such that
    \[AB = BA = I_n.\]
    We say that $B = A^{-1}$ and $A = B^{-1}$.
\end{Definition}
\begin{Proposition}
    Let $\map{T}{\vV}{\vW}$ be a linear transformation, where $\vV,\vW$ are finite dimensional vector spaces.
    If $T$ is invertible, then $\dim(\vV) = \dim(\vW)$.
\end{Proposition}
\begin{proof}
    By the rank-nullity theorem, $\dim(\vV) = \dim(\ker{T}) + \dim(\Ima{T})$. Since, $T$ is injective,
    $\dim(\ker{T}) = 0$ and since $T$ is surjective, $\dim{\Ima{T}} = \dim{\vW}$. Thus,
    $\dim(\vV) = \dim(\vW)$.
\end{proof}
\begin{Theorem}
    Let $\vV,\vW$ be finite dimensional vector spaces with ordered basis $\mcalB_{\vV},\mcalB_{\vW}$
    respectively. Let $\map{T}{\vV}{\vW}$ be a linear transformation. Then $T$ is invertible if and only if
    $\matVW{T}{\mcalB_{\vV}}{\mcalB_{\vW}}$ is invertible.
\end{Theorem}
\begin{proof}
    Suppose $T$ is invertible. Then $\dim{\vV} = \dim{\vW} = n$ say. Let,
    $\mcalB_{\vV} = \left(\listV{\vv}{n}\right)$ and $\mcalB_{\vW} = \left(\listV{\vw}{n}\right)$ be ordered
    bases for $\vV,\vW$ respectively. 
    Let $\matVW{T}{\mcalB_{\vV}}{\mcalB_{\vW}}$ be the matrix of $T$ and let 
    $\matVW{T^{-1}}{\mcalB_{\vw}}{\mcalB_{\vv}}$ be the matrix of $T^{-1}$. Then,
    \[I_n = \matVW{I_{\vV}}{\mcalB_{\vV}}{\mcalB_{\vV}} = 
	\matVW{\fog{T}{T}}{\mcalB_{\vV}}{\mcalB_{\vV}} =
	\matVW{T}{\mcalB_{\vV}}{\mcalB_{\vW}}\matVW{T^{-1}}{\mcalB_{\vW}}{\mcalB_{\vV}}.\]
    Thus, $\matVW{T}{\mcalB_{\vV}}{\mcalB_{\vW}}$ is ivertible.

    Suppose $A = \matVW{T}{\mcalB_{\vV}}{\mcalB_{\vW}}$ is invertible. Thus there is a matrix $B$ such that
    $BA = I_n = AB$. The goal is to show that $T$ is
    invertible i.e.~there is a $\map{U}{\vW}{\vV}$ such that $\fog{U}{T} = I_{\vV}$ and $\fog{T}{U} =
    I_{\vW}$. The idea is to use $B$ to construct $U$.
    Let $\check{U} = L_{B}$ i.e.~$\check{U}(\vx) = B\vx$ for any $\vx = \colvec{4}{x_1}{x_2}{\vdots}{x_n}$ in
    $\Fn$. Then we can construct $U$ as follows:
    \[U = \cMapInv{\mcalB_{\vV}}\circ\check{U}\circ\cMap{\mcalB_{\vW}}.\]
    Then for any $\vv_j \in \mcalB_{\vV}$,
    \begin{align*}
	\fog{U}{T} &= \left(\cMapInv{\mcalB_{\vV}}\circ\check{U}\circ\cMap{\mcalB_{\vW}}\right)(T(\vv_j))\\ 
	&= \cMapInv{\mcalB_{\vV}}\left(\check{U}\circ\cMap{\mcalB_{\vW}}\circ T(\vv_j)\right)\\
	& =  \cMapInv{\mcalB_{\vV}}\left(\check{U}(A[:,j])\right)\\
	& = \cMapInv{\mcalB_{\vV}}\left(BA[:,j]\right)\\
	& = \cMapInv{\mcalB_{\vV}}(\ve_{j})\\
	& = \vv_{j}.
    \end{align*}
    Thus, $\fog{U}{T} = I_{\vV}$. 
    Here, we used the fact that $\cMap{\mcalB_{\vW}}\circ T(\vv_j) = A[:,j]$ and that $BA = I_n$.
    Similarly, we can show $\fog{T}{U} = I_{\vW}$.
\end{proof}
We saw that $\homv{\vV}{\vW}$ and $\matmn{m}{n}$ are isomorphic vector spaces where $\dim{\vV} = n$ and
$\dim{\vW} = m$. What is the dimension of $\homv{\vV}{\vW}$? Note, that it is easy to see that the dimension
of $\matmn{m}{n}$ is $mn$. This is because we can find a basis $\set{A_{ij}}{1\leq i \leq m,1\leq j \leq n}$ 
of size $mn$ where $A_{ij}$ has all zeros except for the $ij$ entry which is $1$. We saw that any invertible
linear transformation must be a map from vector spaces of equal dimensions and hence $\dim{\homv{\vV}{\vW}}$
must be $mn$. We make this precise.
\begin{Proposition}
    Let $\vV,\vW$ be finite-dimensional vector spaces over a field $\F$. Then $\vV$ is isomorphic to $\vW$ if
    and only if $\dim{\vV} = \dim{\vW}$.
\end{Proposition}
\begin{proof}
    If $\vV,\vW$ are isomorphic, then there is an invertible linear transformation $\map{T}{\vV}{\vW}$. Thus,
    $\dim{\vV} = \dim{\vW}$.
    If $\dim{\vV} = \dim{\vW}$, then $\vV$ and $\vW$ have the same number of basis vectors. Let,
    $\mcalB_{\vV} = \left(\listV{\vv}{n}\right)$ and 
    $\mcalB_{\vW} = \left(\listV{\vw}{n}\right)$ be ordered bases for $\vV$ and $\vW$ respectively. Then there
    is a unique linear transformation $T$ such that $T(\vv_{j}) = \vw_{j}$. But this means that
    $\Ima{T} = span{T(\mcalB_{\vV})} = span{\vW} = \vW$ and so $T$ is onto. But since $\dim{\vV} = \dim{\vW}$, it
    means that $T$ is also injective. Hence $T$ is invertible.
\end{proof}

We have seen that linear transformations can be viewed as matrices when we specify an ordered bases. However,
we can have different ordered bases which will give rise to different matrices for the \textbf{same} linear
transformation. This concept leads to the definition of similar matrices. We will look into similar matrices
when we describe eigenvalues. For now, we will define a transition matrix or a change of co-ordinate basis
which arise when we want to see how the co-ordinates of a vector change when we change the basis.

%%%% Commutative diagram
\endinput

